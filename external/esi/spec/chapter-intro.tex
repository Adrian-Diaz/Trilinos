\chapter{Introduction}

\section{ESI Working Group Charter}

	The primary goal of the ESI effort is to develop an integral set of 
standards for scalable numerical equation-solver services and 
components. These standards 
are explicitly represented as an interoperable set of  interface 
specifications. 

	Motivated by the need to solve ASCI problems, the ESI working group 
is especially interested in scalable solution methods, components, and 
tools. Sparse linear equation solver services are subject to a variety of 
decompositions, and we do not claim that ours is "best" in a formal 
sense.  However, in producing this specificaiton we were guided by a 
desire to simplify the abstractions and a need to achieve maximal 
performance. 

	An auxillary goal of the working group is to promote the use and 
acceptance of these standards both within and external to the DOE. We 
openly invite interested parties to contribute to the technical 
discussion by joining the public technical forum 
(if-forum@z.ca.sandia.gov). 

\section{History of This Effort}

	The ESI working group was formed in September 1997 by a collection of 
ASCI code developers, and as such has particular interests in 
scalable computing solutions. The ESI effort is in part an extension 
of earlier efforts, and a coalescence of other proposed efforts. Our 
experience in interface development (see e.g., FEI) indicates that we 
can focus on ASCI problems, keep more general concerns in mind, and 
produce interface solutions of value to a wide class of users.  Our 
ability to consolidate services within common frameworks (i.e., under 
standard interfaces) provides benefits to users at all levels. 

	It bears mentioning that the ESI designers developed an object 
model characterization of the system, to be manifest in 
object-oriented codes.  We acknowledge our strong disposition toward 
object-oriented software engineering principles.  Throughout the 
development of this specification, we used both sIDL (cite CASC) and
C++ to define the core classes.  

\section{About This Document}

-purpose
-high-level document layout
-auxillary documentation sources
