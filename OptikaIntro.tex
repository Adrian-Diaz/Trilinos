\section{Introduction}
In the world of scientific computing there is a problem: most software 
developers are far more concerned with the functionality of their software 
rather than their user interface. This is understandable given
the limited time and pressures of scientific computing environments. And in 
cases where there are only a few users of a piece of software this type of 
development is tolerable. However, when a piece of software starts to be used 
by a wider audience, poor user interface design issues come to the forefront 
and can greatly hinder further adoption of a particular piece of software. 
Examples of such software include MPQC~\cite{mpqc}, DAKOTA~\cite{dakota}, and 
Tramonto~\cite{tramonto}.

Optika is an attempt to solve this
problem in a generic fashion for parameterized scientific applications.
Since developers of scientific applications don't really care about user 
interfaces, Optika aims to provide a minimal amount of hurdles for developers. 
At the same time, Optika trys to provide the user with an intuitive user
interface that can be easily navigated and utilized regardless of the 
underlying computations.

The purpose of this paper is to show in detail how Optika can be used to create
significantly better user intefaces for scientific applications. We will include
numerous examples throughout.
