\documentclass[12pt]{report}

%
% Set the title, author, and date
%
    \title{Trilinos Developer's Guide}

    \author{James Willenbring\\
       Computational Mathematics \& Algorithms \\
	  Sandia National Laboratories\\
	  P.O. Box 5800\\
	  Albuquerque, NM 87185-1110 \\
	  jmwille@sandia.gov \\
	 }

    % While this document is under active development, the date
    % field will reflect the current date, eventually a final date
    % will be assigned.
%    \today - Latex spits out errors when this is used here

%
% Start the document
%
    \begin{document}
	\maketitle

%
% Page style chosen that prints chapter heading and page number in the 
% header on each page.
%

    \pagestyle{headings}

    % -------------------------------------------------- %
    % Acknowledgement section
    % Use \section* since we don't want it in the table of context
    %
    %\clearpage - this is the beginning of the document
    \section*{Acknowledgement}
The author would like to acknowledge the support of the ASCI and LDRD programs 
that funded development of Trilinos and the talented group of Trilinos 
contributors: Michael Heroux, Robert Hoekstra, Alan Williams, Richard Lehoucq, 
Kevin Long, Tamara Kolda, Roger Pawlowski, David Day, Ray Tuminaro, 
Jonathan Hu, Mark Adams and Teri Barth.

    % --------------------------------------------------------- %
    % The table of contents and list of figures and tables
    % Comment out \listoffigures and \listoftables if there are no
    % figures or tables. Make sure this starts on an odd numbered page
    %
    \clearpage
    \tableofcontents
    \listoffigures
    \listoftables



% ------------------------------------------- %
% Introduction to Developer's Guide
%
    \chapter{Introduction}
	\section{Briew Overview}
	\begin{quote}
The Trilinos Project is an effort to develop parallel solver algorithms and 
libraries within 
an object-oriented software framework for the solution of large-scale, complex
multi-physics engineering and scientific applications.   Our emphasis is on 
developing robust, scalable algorithms in a software framework, using abstract 
interfaces for flexible interoperability of components and providing a 
full-featured set of concrete classes that implement all abstract interfaces. 
Trilinos components are primarily written in C++, but provide essential C and 
Fortran user interface support.  We provide an open architecture that allows 
easy integration with other solver packages and we deliver our software to 
the outside community via the Gnu Lesser General Public License
(LGPL)~\cite{gnu-license-site}.
	\end{quote}
% Figure out how to cite Mike's overview doc
	\section{Motivation}
	\begin{quote}
Research efforts in advanced solution algorithms and parallel solver
libraries have historically had a large impact on engineering and
scientific computing.  Algorithmic advances increase the range
of tractable problems and reduce the cost of solving existing
problems.  Well-designed solver libraries provide a mechanism for
leveraging solver development across a broad set of applications and
minimize the cost of solver integration.  Emphasis is
required in both new algorithms and new software in order
to achieve the maximum impact of efforts.

The Trilinos project encompasses a variety of efforts that are to some
extent self-contained but at the same time inter-related.  The
Trilinos design allows individual packages to grow and mature
autonomously to the extent the algorithms and package developers
dictate. 
	\end{quote}
%85

	\section{What is Covered in this Guide}
**(see lines 230-243)

The Trilinos Developer's Guide is meant to assist existing, as well as new and
potential Trilinos developers.  Topics covered include what is required to
integrate an existing package into Trilinos and examples of how those 
requirements can be met, as well as what tools are available as a Trilinos
package.  **(rewrite this section)**

For a higher-level view of the Trilinos project, please see **(Cite An Overview
of the Trilinos Project by Mike Heroux). 

% start reading Mike's stuff at 243

	\chapter{Requirements \& Guildlines}
	**(Req's and how to satisfy them - break into sections)**



	\chapter{Suggested Practices}
	**(suggested coding guidelines, and other non-manditory practices that are encouraged - break up.  See Overview Doc pg 11,14-15.)**
	


	\chapter{Services Available to Trilinos Packages}
	A number of services exist for Trilinos packages.
	\section{Configuration Management}
	**(See Overview Doc pg 12)**
Autoconf~\cite{Autoconf},  Automake~\cite{Automake} and
Libtool~\cite{Libtool} provide a robust, full-featured set of tools for
building software across a broad set of platforms (see also the ``Goat
Book''~\cite{GoatBook}).  Although these
tools are not official standards, they are commonly used in many
packages.  Many existing
Trilinos packages use Autoconf and Automake (and will use
Libtool in the future).  However, use of these tools is not required.

In order to minimize the amount of redudant effort, Trilinos provides a set of
M4~\cite{M4} macros that can be used by any other
package that wants to use Autoconf and Automake for configuring and
building libraries.  These macros perform common configuration tasks such as
locating a valid LAPACK~\cite{lapack} library, or checking for a user-defined 
MPI C compiler.  These macros can be found ...
**(Add something about looking at the configure.ac files, and who to contact)**

	\section{Regression Testing}
	**(See Overview Doc pg 12)**
	
	\section{Source Code Repository and Build Tools}
	**(See Overview Doc pg 12)**
Source code repository and build tools: Trilinos source code is
maintained in a CVS~\cite{CVS} repository that is accessible via a
web-based interface package called Bonsai~\cite{Bonsai}.  Features and bug reports
are tracked using Bugzilla~\cite{Bugzilla}, and email lists are
maintained for Trilinos as a whole and for each package.  Support for new
packages can easily be added.  All tools are accessible from the main
Trilinos website~\cite{Trilinos-home-page}.
**(Break this down and talk about how to use the various tools)**

	\chapter{Tools Available to Trilinos Packages}
	**(Maybe break up by category and then by tool - reword title, this section for software that doesn't need to be duplicated)**
	\section{Portable Interface to BLAS and LAPACK}
	**(See Overview Doc pg 12)**

Portable interface to BLAS and LAPACK: The Basic Linear Algebra
Subprograms (BLAS)~\cite{BLAS1,BLAS2,BLAS3} and LAPACK~\cite{lapack}
provide a large repository of robust, high-performance mathematical
software for serial and shared memory parallel dense linear algebra
computations.  However, the BLAS and LAPACK interfaces are Fortran
specifications, and the mechanism for calling Fortran interfaces from
C and C++ varies across computing platforms.  Epetra (and Tpetra)
provide a set of simple, portable interfaces to the BLAS and LAPACK
that provide uniform access to the BLAS and LAPACK across a broad
set of platforms.  These interfaces are accessible to
other packages.
**(Expand on this)**


	\chapter{Integrating a Package into Trilinos}
	**(list formal? process, meet above requirements Overview Doc pg 11,14-15)**




	\chapter{Interoperability Status for Existing Trilinos Packages}


    % ---------------------------------------------------------------------- %
    % References
    %
    \clearpage
    \bibliographystyle{plain}
    \bibliography{TrilinosDevGuide}
    \addcontentsline{toc}{section}{References}

% Include Appendices??  If they are needed, they go here.


\end{document}
