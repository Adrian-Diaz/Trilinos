%@HEADER
% ************************************************************************
% 
%          Trilinos: An Object-Oriented Solver Framework
%              Copyright (2001) Sandia Corporation
% 
% Under terms of Contract DE-AC04-94AL85000, there is a non-exclusive
% license for use of this work by or on behalf of the U.S. Government.
% 
% This program is free software; you can redistribute it and/or modify
% it under the terms of the GNU General Public License as published by
% the Free Software Foundation; either version 2, or (at your option)
% any later version.
%   
% This program is distributed in the hope that it will be useful, but
% WITHOUT ANY WARRANTY; without even the implied warranty of
% MERCHANTABILITY or FITNESS FOR A PARTICULAR PURPOSE.  See the GNU
% General Public License for more details.
%   
% You should have received a copy of the GNU General Public License
% along with this program; if not, write to the Free Software
% Foundation, Inc., 675 Mass Ave, Cambridge, MA 02139, USA.
% 
% Questions? Contact Michael A. Heroux (maherou@sandia.gov)
% 
% ************************************************************************
%@HEADER

\section{Introduction}

The Trilinos Project is an effort to facilitate the design, development,
integration and ongoing support of mathematical software libraries. In
particular, the goal of the Trilinos Project is develop parallel solver
algorithms and libraries within an object-oriented software framework
for the solution of large-scale, complex multiphysics engineering and
scientific applications. The emphasis is on developing robust, scalable
algorithm in a software framework, using abstract interfaces for
flexible interoperability of components while providing a full-featured
set of concrete classes that implement all abstract interfaces.

%%%
%%%
%%%

\subsection{Getting Started with Trilinos}
\label{sec:getting}

The Trilinos Project uses a two-level software structure designed around
collections of packages. A Trilinos package is an integral unit, usually
developed to solve a specific task, by a (relatively) small group of
expert of the field.  Packages exist underneath the Trilinos top level,
which provides a common look-and-feel. Each package has its own
structure, documentation and set of examples. In principle, Trilinos
packages can live independently. However, each package is even more
valuable when combined with other Trilinos packages.

\smallskip

Trilinos is a large software project, and currently about twenty
packages are included. Fully understanding all the functionalities of
the Trilinos packages requires time. The entire set of packages covers a
wide range of numerical methods for large scale computing. Some packages
are focused on the development of computational schemes, like for
instance the solution of linear and nonlinear systems, to the definition
of parallel preconditioners for Krylov methods, eigenvalue computation.
Other packages are more focused on implementation issues (like
definition of matrices and vectors, abstract classes for linear
operators). The first Chapters of this tutorial will be focused on
implementation issues, while the last Chapters will have a more
``mathematical'' taste.

Each package offers sophisticated features, that cannot be ``unleashed''
at a very first usage. For each package, wew will outline only the basic
features, and we refer to the documentation of each package for a more
involved usage. Our goal is to present enough material so that the
reader can successfully use the described packages.  In fact, for new
users, it is neither easy, nor necessary, to manage all the Trilinos
functionalities. At the beginning, it is more important for them to
understand how to manage the basic classes, such as vector, matrix and
linear system classes. However, it is clear that for a fine-tuning, the
reader will have to look through each package's documentation and
examples.

\medskip

Although all packages have the same importance in the Trilinos
structure, a typical user will probably --- at least at the beginning
--- make use of the following packages:
\begin{itemize} 
\item {\bf Epetra}. This package defines the basic classes for
  distributed matrices and vectors, linear operators and linear
  problems. Epetra classes are the common language spoken by all the
  Trilinos packages (even if some of them can ``speak'' other
  languages). Each Trilinos package is able to accept in input Epetra
  objects. This allows powerful combinations among the various Trilinos
  functionalities.
\item {\bf AztecOO}. This is a linear solve package based on
  preconditioned Krylov methods. It supports all the Aztec interfaces
  and functionality, but also provides significant new functionality.
\item {\bf IFPACK}. This is a package to perform various incomplete
  factorizations, and it is here used in conjunction with AztecOO.
\item {\bf ML}. This is an algebraic multilevel preconditioned package, which
  provided scalable preconditioning capabilities for a variety of
  problem classes. It is here used in conjunction with AztecOO.
\item {\bf Amesos}. This package provides a common interface to various
  direct solvers (generally available outside the Trilinos framework),
  both sequential and parallel.
\item {\bf NOX}. This is a collection of nonlinear solvers, designed to
  be easily integrated into an application and used with many different
  linear solvers.
\item {\bf Triutils}. This is a collection of various utilities, that
  can be extremely useful in some phases of software development.
\end{itemize}

Table~\ref{tab:tripackages} gives a partial overview of what can be
accomplished using Trilinos.
\begin{table}[htbp]
  \centering
  \begin{tabular}{| p{10cm} | p{3cm} |}
    \hline
    {\bf Task} & {\bf Package} \\
    \hline
    Light-weight interface to BLAS and LAPACK: & Epetra, Teuchos$^\star$ \\\hline
    Definition of serial dense or sparse matrices: & Epetra \\\hline
    Definition of distributed sparse matrices:& Epetra \\\hline
    solve a linear system with preconditioned Krylov accelerators, like
    CG, GMRES, Bi-CGSTAB, TFQMR:& AztecOO, Belos$^\star$ \\\hline
    Definition of incomplete factorizations:& AztecOO, \newline IFPACK \\\hline
    Definition of a multilevel preconditioner:& ML \\\hline
    Definition of a one-level Schwarz preconditioner (overlapping domain
    decomposition):& AztecOO, \newline IFPACK \\\hline
    Definition a two-level Schwarz preconditioner, with coarse grid based on
    aggregation:& AztecOO+ML \\\hline
    Solution of  systems of nonlinear equations:& NOX \\\hline
    interface with various direct solvers, as UMFPACK, MUMPS, SuperLU
    and others :& Amesos \\\hline
    Computation of eigenvalue of large, sparse matrices:& Anasazi$^\star$
    \\\hline
    Solution of complex linear equations (using equivalent real formulation):&
    Komplex$^\star$ \\\hline
    Definition of segregated preconditioners and block preconditioners (for
    instance, for the incompressible Navier-Stokes equations):&
    Meros$^\star$ \\\hline
    Templated interface to BLAS and LAPACK, arbitrary-precision
    arithmetic, parameter lists:& Teuchos$^\star$ \\\hline
    Define abstract interfaces to vectors, linear operators, and solvers:& TSF$^\star$, TSFCore$^\star$, TSFExtended$^\star$    \\
    \hline
  \end{tabular}
  \caption{Partial overview of what can be done with Trilinos. $\star$:
    not covered in this tutorial.}
  \label{tab:tripackages}
\end{table}

This tutorial is divided into 7 chapters:
\begin{enumerate}
\item Chapter \ref{chap:epetra_vec} describes the Epetra\_Vector class;
\item Chapter \ref{chap:epetra_mat} introduces the Epetra\_Matrix
  class; 
\item Chapter \ref{chap:epetra_others} briefly describes some other
  Epetra classes;
\item Chapter \ref{chap:aztecoo} shows how to solve linear systems with
  AztecOO;
\item Chapter \ref{chap:ifpack} presents the basic usage of IFPACK;
\item Chapter \ref{chap:ml} introduces multilevel preconditioners based
  on ML;
\item Chapter \ref{chap:amesos} introduces the Amesos package;
\item Chapter \ref{chap:nox} outlines the main features of the Trilinos
  nonlinear solver package, NOX.
\item Chapter \ref{chap:triutils} presents some tools provided with the
  Triutils package. 
\end{enumerate}

\begin{remark}
  As already pointed out, Epetra objects are meant to be the ``common
  language'' spoken by all the Trilinos packages, and therefore the new
  user must become familiar with those objects. Therefore we suggest to
  read Chapters \ref{chap:epetra_vec}-\ref{chap:epetra_others} before
  considering other Trilinos packages. Also, Chapter~\ref{chap:aztecoo}
  should be read before Chapters~\ref{chap:ifpack} and~\ref{chap:ml}
  (even if both IFPACK and ML can be compiled and run without AztecOO).
\end{remark}

This tutorial assume a basic background in numerical methods for PDEs,
and in iterative linear and nonlinear solvers. Although not strictly
necessary, the reader is suppose to have a certain familiarity with
distributed memory computing and, to a minor extent, with MPI.

\smallskip

Note that this tutorial is not a substitute ofr individual packages
documentation. Also, for an overview of all the Trilinos packages, the
Trilinos philosophy, and a description of the packages provided by
Trilinos, the reader is referred to \cite{Trilinos-Overview}.
Developers should also consider the Trilinos Developers' Guide, which
addresses many topics, including the development tools used by Trilinos'
developers, and how to include a new package\footnote{ Trilinos provides
  a variety of services to a developer wanting to integrate a package
  into Trilinos.  They include Autoconf~\cite{Autoconf},
  Automake~\cite{Automake} and Libtool~\cite{Libtool} provide a robust,
  full-featured set of tools for building software across a broad set of
  platforms.  Although these tools are not official standards, they are
  widely used.  All existing Trilinos packages use Autoconf and
  Automake.  Libtool support will be added in future releases.}.

%%%
%%%
%%%

\subsection{Installing Trilinos}
\label{sec:installing}

To obtain Trilinos, please refers to the instructions reported at the
following web site:
\begin{verbatim}
http://software.sandia.gov/Trilinos
\end{verbatim}

Trilinos has been compiled on a variety of architectures, including
Linux, Sun Solaris, SGI Irix, DEC, and many others. Trilinos has been
designed to support parallel applications. However, it can be compiled
and run on serial computer.  Detailed comments on the installation, and
an exhaustive list of FAQs, can be found at the web pages:
\begin{verbatim}
http://software.sandia.gov/Trilinos/installing_manual.html
http://software.sandia.gov/Trilinos/faq.html
\end{verbatim}


Before using Trilinos, users might decide to set the environmental
variables \verb!TRILINOS_HOME!, indicating the full path of the Trilinos
directory, \verb!TRILINOS_LIB!, indicating the location of the compiled
Trilinos library, and \verb!TRILINOS_ARCH!, containing the architecture
and the communicator currently used.  For example, using the BASH shell,
command lines of the form
\begin{verbatim}
export TRILINOS_HOME=/home/msala/Trilinos
export TRILINOS_ARCH=LINUX.MPI
export TRILINOS_LIB=${TRILINOS_HOME}/${TRILINOS_ARCH}
\end{verbatim}
can be places in the users' \verb!.bashrc! file.

\smallskip

Here, we briefly report the procedure one should follow in order to
compile Trilinos as required by the examples reported in the following
chapters \ref{chap:epetra_vec}-\ref{chap:triutils}\footnote{Amesos can
  be more difficult to compile for the unexperienced user, as it
  required some information about the packages to interface. Suggestions
  about the configuration of Amesos are reported in
  Chapter~\ref{chap:amesos}. More details about the installation of
  Trilinos can be found in \cite{Trilinos-Users-Guide}.}.  Suppose we
want to compile under LINUX with MPI. The installation procedure can be
are reported below. (\verb!$! indicates the shell prompt.)
\begin{verbatim}
$ cd ${TRILINOS_HOME}
$ mkdir ${TRILINOS_ARCH}
$ cd ${TRILINOS_ARCH}
$ ../configure --prefix="${TRILINOS_HOME}/${TRILINOS_ARCH}" \
  --enable-mpi --with-mpi-compilers \
  --enable-triutils --enable-aztecoo \
  --enable-ifpack \
  --enable-ml --enable-nox | tee configure_${TRILINOS_ARCH}.log
$ make | tee make_${TRILINOS_ARCH}.log
$ make install | tee make_install_${TRILINOS_ARCH}.log
\end{verbatim}

\begin{remark}
  All Trilinos packages can be build to run with or without MPI. If MPI
  is enabled (using \verb!--enable-mpi!), the users must know the
  procedure for beginning MPI jobs on their computer system(s). In some
  cases, options must be set on the configure line to specify the
  location of MPI include files and libraries.
\end{remark}

%%%
%%%
%%%

\subsection{Compiling and Linking a program using Trilinos}
\label{sec:intro_compiling}

In order to compile and link (part of) the Trilinos library, the use can
decide to use a Makefile as reported below. This Makefile refers to one
of the examples, reported in the NOX subdirectory of this tutorial.
\begin{verbatim}
 1: TRILINOS_HOME = /home/msala/Trilinos/
 2: TRILINOS_ARCH - LINUX_MPI
 3: TRILINOS_LIB = $(TRILINOS_HOME)$(TRILINOS_ARCH)
 4: 
 5: include $(TRILINOS_HOME)/build/makefile.$(TRILINOS_ARCH)
 6: 
 7: MY_COMPILER_FLAGS = -DHAVE_CONFIG_H $(CXXFLAGS) -c -g\
 8:                    -I$(TRILINOS_LIB)/include/
 9:
10: MY_LINKER_FLAGS = $(LDFLAGS) $(TEST_C_OBJ) \
11:         -L$(TRILINOS_LIB)/lib/ \
12:         -lnoxepetra -lnox -lifpack \
13:         -laztecoo -lepetra -llapack -lblas $(ARCH_LIBS)
14:
15: ex1: ex1.cpp
16:         $(CXX)     ex1.cpp $(MY_COMPILER_FLAGS)
17:         $(LINKER)  ex1.o   $(MY_LINKER_FLAGS)    -o ex1.exe
\end{verbatim}

Line number have been reported for  reader's convenience. 

The lines 1-3 can be omitted, see Section \ref{sec:installing}.  Line 5
includes basic definitions of Trilinos. (Note that, on some
architectures, one may need to use \verb!gmake! instead of \verb!make!.)
In line 7, the variable \verb!HAVE_CONFIG_H! is defined. Linker flags of
lines 10-13 defines the library to link (location of BLAS and LAPACK can
change on different platforms). The variable \verb!ARCH_LIBS! is defined
in line 5.

To run the compiled example in a sequential environment, simply type
\begin{verbatim}
$ ./ex2.exe
\end{verbatim}
In a MPI environment, the user might have to
use an instruction of type
\begin{verbatim}
$ mpirun -np 2 ./ex1.exe
\end{verbatim}
Please check the local MPI documentation for more details. 

%%%
%%%
%%%

\subsection{Copyright and Licensing of Trilinos}
\label{sec:copyright}

Trilinos is released under the Lesser GPL GNU Licence.

Trilinos is copyrighted by Sandia Corporation. Under the terms of
Contract DE-AC04-94AL85000, there is a non-exclusive license for use of
this work by or on behalf of the U.S. Government.  Export of this
program may require a license from the United States Government.

NOTICE: The United States Government is granted for itself and others
acting on its behalf a paid-up, nonexclusive, irrevocable worldwide
license in ths data to reproduce, prepare derivative works, and perform
publicly and display publicly.  Beginning five (5) years from July 25,
2001, the United States Government is granted for itself and others
acting on its behalf a paid-up, nonexclusive, irrevocable worldwide
license in this data to reproduce, prepare derivative works, distribute
copies to the public, perform publicly and display publicly, and to
permit others to do so.

NEITHER THE UNITED STATES GOVERNMENT, NOR THE UNITED STATES DEPARTMENT
OF ENERGY, NOR SANDIA CORPORATION, NOR ANY OF THEIR EMPLOYEES, MAKES ANY
WARRANTY, EXPRESS OR IMPLIED, OR ASSUMES ANY LEGAL LIABILITY OR
RESPONSIBILITY FOR THE ACCURACY, COMPLETENESS, OR USEFULNESS OF ANY
INFORMATION, APPARATUS, PRODUCT, OR PROCESS DISCLOSED, OR REPRESENTS
THAT ITS USE WOULD NOT INFRINGE PRIVATELY OWNED RIGHTS.

\medskip

Some parts of Trilinos are dependent on a third party code. Each third
party code comes with its own copyright and/or licensing requirements.
It is responsibility of the user to understand these requirements.

%%%
%%%
%%%

\subsection{Programming Language Used in this Tutorial}
\label{sec:language}

Trilinos is written in C++ (for most packages), and in C. Some
interfaces are provided to FORTRAN code (mainly BLAS and LAPACK
routines). Even if a limited support is included for C programs (and a
more limited for FORTRAN code), to unleashed the full power of Trilinos
we suggest to use C++. All the example programs contained in this
tutorial will be in C++; some packages contains examples in C.

%%%
%%%
%%%

\subsection{Referencing Trilinos}
\label{sec:referencing}

The Trilinos project can be referenced by using the following BiBTeX
citation information:
\begin{verbatim}
@techreport{Trilinos-Overview,
title = "{An Overview of Trilinos}",
author = "Michael Heroux and Roscoe Bartlett and Vicki Howle
Robert Hoekstra and Jonathan Hu and Tamara Kolda and
Richard Lehoucq and Kevin Long and Roger Pawlowski and
Eric Phipps and Andrew Salinger and Heidi Thornquist and
Ray Tuminaro and James Willenbring and Alan Williams ",
institution = "Sandia National Laboratories",
number = "SAND2003-2927",
year = 2003}

@techreport{Trilinos-Dev-Guide,
title = "{Trilinos Developers Guide}",
author = "Michael A. Heroux and James M. Willenbring and Robert Heaphy",
institution = "Sandia National Laboratories",
number = "SAND2003-1898",
year = 2003}

@techreport{Trilinos-Dev-Guide-II,
title = "{Trilinos Developers Guide Part II: ASCI Software Quality
Engineering Practices Version 1.0}",
author = "Michael A. Heroux and James M. Willenbring and Robert Heaphy",
institution = "Sandia National Laboratories",
number = "SAND2003-1899",
year = 2003}

@techreport{Trilinos-Users-Guide,
title = "{Trilinos Users Guide}",
author = "Michael A. Heroux and James M. Willenbring",
institution = "Sandia National Laboratories",
number = "SAND2003-2952",
year = 2003}
\end{verbatim}
These BiBTeX information can be downloaded from the web page

\begin{verb}
http://software.sandia.gov/Trilinos/citing.html
\end{verb}

%%%
%%%
%%%

\subsection{A Note on Directory Structure}
\label{sec:into_note}

Each Trilinos package in contained in the subdirectory
\begin{verbatim}
${TRILINOS_HOME}/packages
\end{verbatim}
The structure of all packages is quite similar (although not exactly
equal). As a general line, source files are in
\begin{verbatim}
${TRILINOS_HOME}/packages/<package-name>/src
\end{verbatim}
Example files are reported in \begin{verbatim}
${TRILINOS_HOME}/packages/<package-name>/examples
\end{verbatim}
and test files in
\begin{verbatim}
${TRILINOS_HOME}/packages/<package-name>/test
\end{verbatim}
The documentation is reported
\begin{verbatim}
${TRILINOS_HOME}/packages/<package-name>/doc
\end{verbatim}
Often, Trilinos developers use Doxygen\footnote{Copyright \copyright
  1997-2003 by Dimitri van Heesch. More information can by found at the
  web address {\tt http://www.stack.nl/~dimitri/doxygen/}.}. For
instance, to create the documentation for Epetra, we use can type
\begin{verbatim}
$ cd ${TRILINOS_HOME}/packages/epetra/doc
$ doxygen Doxyfile
\end{verbatim}
and then browse it using an HTML reader, or compiling the \LaTeX file
using
\begin{verbatim}
$ cd ${TRILINOS_HOME}/packages/epetra/doc/latex
$ make
\end{verbatim}

%%%
%%%
%%%

\subsection{List of Trilinos Developers}
\label{sec:intro_incomplete}

A list of the Trilinos' developers, updated to December 2003, would
include the following names (in alphabetical order):

Roscoe A. Bartlett,
Jason A. Cross,
David M. Day,
Robert Heaphy,
Michael A. Heroux (project leader),
Russell Hooper,
Vicki E. Howle,
Robert J. Hoekstra,
Jonathan J. Hu,
Tamara G. Kolda,
Richard B. Lehoucq,
Paul Lin,
Kevin R. Long,
Roger P. Pawlowski,
Michael N. Phenow,
Eric T. Phipps,
Andrew J. Rothfuss,
Marzio Sala,
Andrew G. Salinger,
Paul M. Sexton,
Kendall S. Stanley,
Heidi K. Thornquist,
Ray S. Tuminaro,
James M. Willenbring,
Alan Williams.

