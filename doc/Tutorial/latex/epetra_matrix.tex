% @HEADER
% ***********************************************************************
% 
%            Trilinos: An Object-Oriented Solver Framework
%                 Copyright (2001) Sandia Corporation
% 
% Under terms of Contract DE-AC04-94AL85000, there is a non-exclusive
% license for use of this work by or on behalf of the U.S. Government.
% 
% This library is free software; you can redistribute it and/or modify
% it under the terms of the GNU Lesser General Public License as
% published by the Free Software Foundation; either version 2.1 of the
% License, or (at your option) any later version.
%  
% This library is distributed in the hope that it will be useful, but
% WITHOUT ANY WARRANTY; without even the implied warranty of
% MERCHANTABILITY or FITNESS FOR A PARTICULAR PURPOSE.  See the GNU
% Lesser General Public License for more details.
%  
% You should have received a copy of the GNU Lesser General Public
% License along with this library; if not, write to the Free Software
% Foundation, Inc., 59 Temple Place, Suite 330, Boston, MA 02111-1307
% USA
% Questions? Contact Michael A. Heroux (maherou@sandia.gov) 
% 
% ***********************************************************************
% @HEADER

\section{Working with Epetra Matrices}
\label{chap:epetra_mat}

Epetra contains several matrix classes.  Epetra matrices can be defined
to be either {\em serial} or {\em distributed}. A serial matrix could be
the matrix corresponding to a given element in a finite-element
discretization, or the Hessemberg matrix in the GMRES method. Those
matrices are of (relatively) small size, so that it is not convenient to
distribute them across the processes.

Other matrices, e.g.~the linear system matrices, must be distributed to
obtain scalability.  For distributed sparse matrices, the basic Epetra
class is Epetra\_RowMatrix, meant for double-precision matrices with row
access.  Epetra\_RowMatrix is a pure virtual class.  The classes that
are derived from \verb!Epetra_RowMatrix! include:
\begin{itemize}
\item \verb!Epetra_CrsMatrix! for point matrices;
\item \verb!Epetra_VbrMatrix! for block matrices (that is, for
  matrices which have a block structure, for example the ones deriving
  from the discretization of a PDE problem with multiple unknowns for
  node);
\item \verb!Epetra_FECrsMatrix! and \verb!Epetra_FEVbrMatrix! for
  matrices arising from FE discretizations.
\end{itemize}

The purpose of the Chapter is to review the allocation and assembling of
different types of matrices as follows:
\begin{itemize}
\item The creation of (serial) dense matrices (in Section~\ref{sec:dense_mat});
\item The creation of sparse point matrices (in Section~\ref{sec:sparse_mat});
\item The creation of sparse block matrices (in Section~\ref{sec:sparse_vbr});
\item The insertion of non-local elements using finite-element matrices
  (in Section~\ref{sec:fematrix}).
\end{itemize}

%%%
%%%
%%%

\subsection{Serial Dense Matrices}
\label{sec:dense_mat}

Epetra supports sequential dense matrices with the class
Epetra\_SerialDenseMatrix.  A possible way to create a serial dense
matrix \verb!D! of dimension \verb!n!  by \verb!m! is
\begin{verbatim}
#include "Epetra_SerialDenseMatrix.h"
Epetra_SerialDenseMatrix D(n,m);
\end{verbatim}
One could also create a zero-size object, 
\begin{verbatim}
Epetra_SerialDenseMatrix D();
\end{verbatim}
and then shape this object:
\begin{verbatim}
D.Shape(n,m);
\end{verbatim}
({\tt D} could be reshaped using \verb!ReShape()!.)

An Epetra\_SerialDenseMatrix is stored in a column-major order in the
usual FORTRAN style. This class is built on the top of the BLAS library,
and is derived from Epetra\_Blas (not covered in this tutorial).
Epetra\_SerialDenseMatrix supports dense rectangular matrices.

\smallskip

To access the matrix element at the i-th row and the j-th column, it is
possible to use the parenthesis operator (\verb!A(i,j)!), or the bracket
operator (\verb!A[j][i]!, note that i and j are reversed)\footnote{The
  bracket approach is in general faster, as the compiler can inline the
  corresponding function. Instead, some compiler have problems to inline
  the parenthesis operator.}.

As an example of the use of this class, in the following code we
consider a matrix-matrix product between two rectangular matrices
\verb!A! and \verb!B!. 
\begin{verbatim}
int NumRowsA = 2, NumColsA = 2;
int NumRowsB = 2, NumColsB = 1;
Epetra_SerialDenseMatrix A, B;
A.Shape(NumRowsA, NumColsA);
B.Shape(NumRowsB, NumColsB);
// ... here set the elements of A and B
Epetra_SerialDenseMatrix AtimesB;
AtimesB.Shape(NumRowsA,NumColsB);  
double alpha = 1.0, beta = 1.0;
AtimesB.Multiply('N','N',alpha, A, B, beta);
cout << AtimesB;
\end{verbatim}
\verb!Multiply()! performs the operation $C = \alpha A + \beta B$, where
$A$ replaced by $A^T$ if the first input parameter is \verb!T!, and $B$
replaced by $B^T$ if the second input parameter is \verb!T!.  The
corresponding source code file is \TriExe{epetra/ex10.cpp}.

\smallskip

To solve a linear system with a dense matrix, one has to create an
Epetra\_SerialDenseSolver. This class uses the most robust techniques
available in the LAPACK library. The class is built on the top of BLAS
and LAPACK, and thus has excellent performance and numerical
stability\footnote{Another package, Teuchos, covered in
  Chapter~\ref{chap:teuchos}, allows a templated access to LAPACK.
  ScaLAPACK is supported through Amesos, see
  Chapter~\ref{chap:amesos}.}.

The primary difference is that Epetra\_LAPACK is a ``thin'' layer on the
top of LAPACK, while Epetra\_SerialDenseSolver attempts to provide easy
access to the more robust dense linear solvers.

Epetra\_LAPACK is preferable if the user seeks a convenient wrapper
around the FORTRAN LAPACK routines, and the problem at hand is
well-conditioned. Instead, when the user wants (or potentially wants to)
solve ill-conditioned problems or favors a more object-oriented
interface, then we suggest Epetra\_SerialDenseMatrix..

\smallskip

Given an Epetra\_SerialDenseMatrix and two Epetra\_SerialDenseVectors
{\tt x} and {\tt b}, the general approach is as follows:
\begin{verbatim}
Epetra_SerialDenseSolver Solver();
Solver.SetMatrix(D);
Solver.SetVectors(x,b);
\end{verbatim}
Then, it is possible to invert the matrix with \verb!Invert()!, solve
the linear system with \verb!Solve()!, apply iterative refinement with
\verb!ApplyRefinement()!. Other methods are available; for instance,
\begin{verbatim}
double rcond=Solve.RCOND();
\end{verbatim}
returns the reciprocal of the condition number of matrix {\tt D} (or -1
if not computed).

\TriExe{epetra/ex11.cpp} outlines some of the capabilities of the
Epetra\_SerialDenseSolver class.

%%%
%%%
%%%

\subsection{Distributed Sparse Matrices}
\label{sec:sparse_mat}

Epetra provides an extensive set of classes to create and fill
distributed sparse matrices. These classes allow row-by-row or
element-by-element constructions. Support is provided for common matrix
operations, including scaling, norm, matrix-vector multiplication and
matrix-multivector multiplication\footnote{Methods for matrix-matrix
  products are avaiable through the EpetraExt package. Another
  alternative is to use the efficient matrix-matrix product of ML, which
  requires ML\_Operator objects. One may use light-weight conversions to
  ML\_Operator, perform the ML matrix-matrix product, then convert the
  result to Epetra Matrix.}.

Using Epetra objects, applications do not need to know about the
particular storage format, and other implementation details such as data
layout, the number and location of ghost nodes. Epetra furnishes two
basic formats, one suited for point matrices, the other for block
matrices.  The former is presented in this Section; the latter is
introduced in Section~\ref{sec:sparse_vbr}. Other matrix formats can be
introduced by deriving the Epetra\_RowMatrix virtual class as needed.

\begin{table}
\begin{center}
\begin{tabular}{ | p{15cm} | }
\hline
\tt virtual int 
Multiply (bool TransA, const Epetra\_MultiVector \&X, Epetra\_MultiVector
\&Y) const=0 \\
Returns the result of a Epetra\_RowMatrix multiplied by a
Epetra\_MultiVector X in Y.  \\
\\
\tt virtual int 
Solve (bool Upper, bool Trans, bool UnitDiagonal, const
Epetra\_MultiVector \&X, Epetra\_MultiVector \&Y) const=0 \\
Returns result of a local-only solve using a triangular Epetra\_RowMatrix with Epetra\_MultiVectors X and Y. \\
\\
\tt virtual int 
InvRowSums (Epetra\_Vector \&x) const=0 \\
Computes the sum of absolute values of the rows of the Epetra\_RowMatrix,
results returned in x.  \\
\\
\tt virtual int 
LeftScale (const Epetra\_Vector \&x)=0 \\
Scales the Epetra\_RowMatrix on the left with a Epetra\_Vector x.  \\
\\
\tt virtual int 
InvColSums (Epetra\_Vector \&x) const=0 \\
Computes the sum of absolute values of the cols of the Epetra\_RowMatrix,
results returned in x.  \\
\\
\tt virtual int 
RightScale (const Epetra\_Vector \&x)=0 \\
Scales the Epetra\_RowMatrix on the right with a Epetra\_Vector x.  \\
\hline
\end{tabular}
\caption{Mathematical methods of {\tt Epetra\_RowMatrix}}
\label{tab:row_matrix_math}
\end{center}
\end{table}

\begin{table}
\begin{center}
\begin{tabular}{ | p{15cm} | }
\hline
\tt virtual bool 
Filled () const=0 \\
If FillComplete() has been called, this query returns true, otherwise it
returns false. \\
\tt virtual double 
NormInf () const=0 \\
Returns the infinity norm of the global matrix. \\
\tt virtual double 
NormOne () const=0 \\
Returns the one norm of the global matrix. \\
\tt virtual int 
NumGlobalNonzeros () const=0 \\
Returns the number of nonzero entries in the global matrix. \\
\tt virtual int 
NumGlobalRows () const=0 \\
Returns the number of global matrix rows. \\
\tt virtual int 
NumGlobalCols () const=0 \\
Returns the number of global matrix columns. \\
\tt virtual int 
NumGlobalDiagonals () const=0 \\
Returns the number of global nonzero diagonal entries, based on global
row/column index comparisons. \\
\tt virtual int 
NumMyNonzeros () const=0 \\
Returns the number of nonzero entries in the calling processor's portion
of the matrix. \\
\tt virtual int 
NumMyRows () const=0 \\
Returns the number of matrix rows owned by the calling processor. \\
\tt virtual int 
NumMyCols () const=0 \\
Returns the number of matrix columns owned by the calling processor. \\
\tt virtual int 
NumMyDiagonals () const=0 \\
Returns the number of local nonzero diagonal entries, based on global
row/column index comparisons. \\
\tt virtual bool 
LowerTriangular () const=0 \\
If matrix is lower triangular in local index space, this query returns
true, otherwise it returns false. \\
\tt virtual bool 
UpperTriangular () const=0 \\
If matrix is upper triangular in local index space, this query returns
true, otherwise it returns false. \\
\tt virtual const Epetra\_Map \& 
RowMatrixRowMap () const=0 \\
Returns the Epetra\_Map object associated with the rows of this matrix. \\
\tt virtual const Epetra\_Map \& 
RowMatrixColMap () const=0 \\
Returns the Epetra\_Map object associated with the columns of this
matrix. \\
\tt virtual const Epetra\_Import * 
RowMatrixImporter () const=0 \\
Returns the Epetra\_Import object that contains the import operations for
distributed operations. \\
\hline
\end{tabular}
\caption{Atribute access methods of {\tt Epetra\_RowMatrix}}
\label{tab:row_matrix_atr}
\end{center}
\end{table}



\begin{remark}
  Some numerical algorithms require the application of the linear
  operator only. For this reason, some applications choose not to 
  store a given matrix. Epetra can handle this situation using with the
  Epetra\_Operator class; see Section~\ref{sec:operator}.
\end{remark}

The process of creating a sparse matrix is more involved than the
process for dense matrices. This is because, in order to obtain
excellent numerical performance, one has to provide an estimation of
the nonzero elements on each row of the sparse matrix. (Recall that
dynamic allocation of new memory and copying the old storage into the
new one is an expensive operation.)

As a general rule, the process of constructing a (distributed) sparse
matrix is as follows:
\begin{itemize}
\item allocate an integer array \verb!Nnz!, whose length equals the
  number of local rows;
\item loop over the local rows, and estimate the number of nonzero
  elements of that row;
\item create the sparse matrix using \verb!Nnz!;
\item fill the sparse matrix.
\end{itemize}

As an example, in this Section we will present how to construct a
distributed (sparse) matrix, arising from a finite-difference solution
of a one-dimensional Laplace problem. This matrix looks like:
\begin{equation*}
A = \begin{pmatrix}
 2 & -1 &     &   &    \\
-1 &  2     & -1     &        &    \\
   & \ldots & \ldots & \ldots & -1 \\
   &        &        & -1     & 2
\end{pmatrix}.
\end{equation*}
The example illustrates how to construct the matrix,
and how to perform matrix-vector operations.
The code can be found in \TriExe{epetra/ex12.cpp}.

We start by specifying the global dimension (here is 5, but can be any
number):
\begin{verbatim}
int NumGlobalElements = 5;
\end{verbatim}
We create a map (for the sake of simplicity linear), and define the
local number of rows and the global numbering for each local row:
\begin{verbatim}
Epetra_Map Map(NumGlobalElements,0,Comm);
int NumMyElements = Map.NumMyElements();
int * MyGlobalElements = Map.MyGlobalElements( );
\end{verbatim}
In particular, we have that \verb!j=MyGlobalElements[i]! is the global
numbering for local node \verb!i!.  Then, we have to specify the number
of nonzeros per row. In general, this can be done in two ways:
\begin{itemize}
\item Furnish an integer value, representing the number of nonzero
  element on each row (the same value for all the rows);
\item Furnish an integer vector \verb!NumNz!, of length
  \verb!NumMyElements()!, containing the nonzero elements of each row.
\end{itemize}

The first approach is trivial: the matrix is create with the simple
instruction
\begin{verbatim}
Epetra_CrsMatrix A(Copy,Map,3);
\end{verbatim}
(The \verb!Copy! keyword is explained in Section~\ref{sec:distr_vec}.)
In this case, Epetra considers the number 3 as a ``suggestion,'' in the
sense that the user can still add more than 3 elements per row (at the
price of a possible performance decay).  The second approach is as
follows:
\begin{verbatim}
int * NumNz = new int[NumMyElements];
for( int i=0 ; i<NumMyElements ; i++ )
if( MyGlobalElements[i]==0 || 
    MyGlobalElements[i] == NumGlobalElements-1)
  NumNz[i] = 2;
else
  NumNz[i] = 3;
\end{verbatim}
We are building a tridiagonal matrix where each row has (-1 2 -1).  Here
\verb!NumNz[i]! is the number of nonzero terms in the i-th global
equation on this process (2 off-diagonal terms, except for the first and
last equation).

Now, the command to create an Epetra\_CsrMatrix is
\begin{verbatim}
Epetra_CrsMatrix A(Copy,Map,NumNz);
\end{verbatim}
We add rows one at a time. The matrix \verb!A! has been created in Copy
mode, in a way that relies on the specified map. To fill its values, we
need some additional variables: let us call them \verb!Indexes! and
\verb!Values!. For each row, \verb!Indices! contains global column
indices, and \verb!Values! the correspondingly values.
\begin{verbatim}
double * Values = new double[2];
Values[0] = -1.0; Values[1] = -1.0;
int * Indices = new int[2];
double two = 2.0;
int NumEntries;

for( int i=0 ; i<NumMyElements; ++i ) {
  if (MyGlobalElements[i]==0) {
      Indices[0] = 1;
      NumEntries = 1;
  } else if (MyGlobalElements[i] == NumGlobalElements-1) {
    Indices[0] = NumGlobalElements-2;
    NumEntries = 1;
  } else {
    Indices[0] = MyGlobalElements[i]-1;
    Indices[1] = MyGlobalElements[i]+1;
    NumEntries = 2;
  }
  A.InsertGlobalValues(MyGlobalElements[i], NumEntries, 
                       Values, Indices);
  // Put in the diagonal entry
  A.InsertGlobalValues(MyGlobalElements[i], 1, &two, 
                       MyGlobalElements+i);
}
\end{verbatim}
Note that column indices have been inserted using global indices (but a
method called \verb!InsertMyValues! can be used as well) .  Finally, we
transform the matrix representation into one based on local indexes. The
transformation in required in order to perform efficient parallel
matrix-vector products and other matrix operations.
\begin{verbatim}
A.FillComplete();
\end{verbatim}
This call to \verb!FillComplete()! will reorganize the internally stored
data so that each process knows the set of internal, border and external
elements for a matrix-vector product of the form $B = AX$. Also, the
communication pattern is established. As we have specified just one map,
Epetra considers that the the rows of $A$ are distributed among the
processes in the same way of the elements of $X$ and $B$.  Although
standard, this approach is only a particular case.  Epetra allows the
user to handle the more general case of a matrix whose Map differs from
that of $X$ and that of $B$. In fact, each Epetra matrix is defined by
{\sl four} maps:
\begin{itemize}
\item Two maps, called RowMap and ColumnMap, define the sets of rows and
  columns of the elements assigned to a given processor. In general, one
  processor cannot set elements assigned to other
  processors\footnote{Some classes, derived from the Epetra\_RowMatrix,
    can perform data exchange; see for instance Epetra\_FECrsMatrix or
    Epetra\_FEVbrMatrix.}. RowMap and ColumnMap define the pattern of
  the matrix, as it is used during the construction. They can be
  obtained using the methods \verb!RowMatrixRowMap()!  and
  \verb!RowMatrixColMap()! of the Epetra\_RowMatrix class. Usually, as a
  ColumnMap is not specified, it is automatically created by Epetra.  In
  general RowMap and ColumnMap can differ.
\item DomainMap and RangeMap define, instead, the parallel data layout
  of $X$ and $B$, respectively. Note that those two maps can be
  completely different from RowMap and ColumnMap, meaning that a matrix
  can be constructed using a certain data distribution, then used on
  vectors with another data distribution. DomainMap and RangeMap can
  differ.  Maps can be obtained using the methods \verb!DomainMap()! and
  \verb!RangeMap()!.
\end{itemize}
The potential of the approach are illustated by the example file\\
\TriExe{epetra/ex24.cpp}. In this example, to be run using two
processors, we build two maps: \verb!MapA! will be used to construct the
matrix, while \verb!MapB! to define the parallel layout of the vectors
$X$ and $B$. For the sake of simplicity, $A$ is diagonal.
\begin{verbatim}
Epetra_CrsMatrix A(Copy,MapA,MapA,1);
\end{verbatim}
As usual in this Tutorial, the integer vector \verb!MyGlobalElementsA!
contains the global ID of local nodes. To assemble $A$, we cycle over
all the local rows (defined by \verb!MapA!):
\begin{verbatim}
for( int i=0 ; i<NumElementsA ; ++i ) {
  double one = 2.0;
  int indices = MyGlobalElementsA[i];
  A.InsertGlobalValues(MyGlobalElementsA[i], 1, &one, &indices );
}
\end{verbatim}
Now, as both $X$ and $B$ are defined using \verb!MapB!, instead of
calling \verb!FillComplete()!, invoke
\begin{verbatim}
A.FillComplete(MapB,MapB);
\end{verbatim}
Now, we can create $X$ and $B$ as vectors based on \verb!MapB!, and
perform the matrix-vector product:
\begin{verbatim}
Epetra_Vector X(MapB);   Epetra_Vector B(MapB);  
A.Multiply(false,X,B);
\end{verbatim}  

\begin{remark}
Although presented for Epetra\_CrsMatrix objects, the distinction
between RowMap, ColMap, DomainMap, and RangeMap holds for all classed
derived from Epetra\_RowMatrix. 
\end{remark}

%%%
%%%
%%%

\medskip

Example \TriExe{epetra/ex14.cpp} shows the use of some of the methods of
the Epetra\_CrsMatrix class. The code prints out information about the
structure of the matrix and its properties.  The output will be
approximatively as
reported here:
\begin{verbatim}
[msala:epetra]> mpirun -np 2 ./ex14
*** general Information about the matrix
Number of Global Rows = 5
Number of Global Cols = 5
is the matrix square  = yes
||A||_\infty          = 4
||A||_1               = 4
||A||_F               = 5.2915
Number of nonzero diagonal entries = 5( 100 %)
Nonzero per row : min = 2 average = 2.6 max = 3
Maximum number of nonzero elements/row = 3
min( a_{i,j} )      = -1
max( a_{i,j} )      = 2
min( abs(a_{i,j}) ) = 1
max( abs(a_{i,j}) ) = 2
Number of diagonal dominant rows        = 2 (40 % of total)
Number of weakly diagonal dominant rows = 3 (60 % of total)
*** Information about the Trilinos storage
Base Index                 = 0
is storage optimized       = no
are indices global         = no
is matrix lower triangular = no
is matrix upper triangular = no
are there diagonal entries = yes
is matrix sorted           = yes
\end{verbatim}

Other examples for Epetra\_CrsMatrix include:
\begin{itemize}
\item Example \TriExe{epetra/ex13.cpp} implements a simple distributed
  finite-element solver.  The code solves a 2D Laplace problem with
  unstructured triangular grids. In this example, the information about
  the grid is hardwired.  The interested user can easily modify those
  lines in order to read the grid information from a file.
\item Example \TriExe{epetra/ex15.cpp} explains how to export an
  Epetra\_CrsMatrix to file in a MATLAB format.  The output of this
  example will be approximatively  as follows:
\begin{verbatim}
[msala:epetra]> mpirun -np 2 ./ex15
A = spalloc(5,5,13);
% On proc 0: 3 rows and 8 nonzeros
A(1,1) = 2;
A(1,2) = -1;
A(2,1) = -1;
A(2,2) = 2;
A(2,3) = -1;
A(3,2) = -1;
A(3,3) = 2;
A(3,4) = -1;
% On proc 1: 2 rows and 5 nonzeros
A(4,4) = 2;
A(4,5) = -1;
A(4,3) = -1;
A(5,4) = -1;
A(5,5) = 2;
\end{verbatim}
  A companion to this example is \newline \TriExe{epetra/ex16.cpp},
  which exports an Epetra\_Vector to MATLAB format. Note also that the
  package EpetraExt contains several purpose tools to read and write
  matrices in various formats.
\end{itemize}
%%%
%%%
%%%

\subsection{Creating Block Matrices}
\label{sec:sparse_vbr}

This section reviews how to work with block matrices (where each block
is a dense matrix)\footnote{Trilinos offers capabilities to deal with
  matrices composed by few sparse blocks, like for instance matrices
  arising from the discretization of the incompressible Navier-Stokes
  equations, through the Meros package (not covered in this tutorial).}.
This format has been designed for PDE problems with more than one
unknown per grid node.  The resulting matrix has a sparse block
structure, and the size of each dense block equals the number of PDE
equations defined on that block.  This format is quite general, and can
handle matrices with variable block size, as is done is the following
example.

First, we create a map, containing the distribution of the blocks:
\begin{verbatim}
Epetra_Map Map(NumGlobalElements,0,Comm);
\end{verbatim}
Here, a linear decomposition is used for the sake of simplicity, but any
map may be used as well.
Now, we obtain some information about the map:
\begin{verbatim}
// local number of elements
int NumMyElements = Map.NumMyElements();
// global numbering of local elements
int * MyGlobalElements = new int [NumMyElements];
Map.MyGlobalElements( MyGlobalElements );
\end{verbatim}
A block matrix can have blocks of different size.  Here, we suppose that
the dimension of diagonal block row $i$ is $i+1$.  The integer vector
\verb!ElementSizeList! will contain the block size of local element
\verb!i!.
\begin{verbatim}
Epetra_IntSerialDenseVector ElementSizeList(NumMyElements);
 for( int i=0 ; i<NumMyElements ; ++i ) 
   ElementSizeList[i] = 1+MyGlobalElements[i];
\end{verbatim}
Here \verb!ElementSizeList! is declared as Epetra\_IntSerialDenseVector,
but an int array is fine as well.

Now we can create a map for the block distribution:
\begin{verbatim}
Epetra_BlockMap BlockMap(NumGlobalElements,NumMyElements,
                         MyGlobalElements, 
                         ElementSizeList.Values(),0,Comm);
\end{verbatim}
and finally we can create the VBR matrix based on \verb!BlockMap!. In
this case, nonzero elements are located in the diagonal and the
sub-diagonal above the diagonal.
\begin{verbatim}
Epetra_VbrMatrix A(Copy, BlockMap, 2);

int Indices[2];
double Values[MaxBlockSize];

for( int i=0 ; i<NumMyElements ; ++i ) {
  int GlobalNode = MyGlobalElements[i];
  Indices[0] = GlobalNode;
  int NumEntries = 1;
  if( GlobalNode != NumGlobalElements-1 ) {
    Indices[1] = GlobalNode+1;
    NumEntries++;
  }
  A.BeginInsertGlobalValues(GlobalNode, NumEntries, Indices);
  // insert diagonal
  int BlockRows = ElementSizeList[i];
  for( int k=0 ; k<BlockRows * BlockRows ; ++k )
    Values[k] = 1.0*i;
  B.SubmitBlockEntry(Values,BlockRows,BlockRows,BlockRows);

  // insert off diagonal if any
  if( GlobalNode != NumGlobalElements-1 ) {
    int BlockCols = ElementSizeList[i+1];
    for( int k=0 ; k<BlockRows * BlockCols ; ++k )
      Values[k] = 1.0*i;
    B.SubmitBlockEntry(Values,BlockRows,BlockRows,BlockCols);
  }
  B.EndSubmitEntries();
}
\end{verbatim}
Note that, with VBR matrices, we have to insert one block at time.  This
required two more instructions, one to start this process
(\verb!BeginInsertGlobalValues!), and another one to commit the end of
submissions (\verb!EndSubmitEntries!). Similar functions to sum and
replace elements exist as well.
 
\smallskip

Please refer to \TriExe{epetra/ex17.cpp} for the entire source.

%%%
%%%
%%%

\subsection{Insert non-local Elements Using FE Matrices}
\label{sec:fematrix}

The most important additional feature provided by the
Epetra\_FECrsMatrix with respect to Epetra\_CrsMatrix, is the capability
to set non-local matrix elements. We will illustrate this using the
following example, reported in \newline \TriExe{epetra/ex23.cpp}. In the
example, we will set all the entries of a distributed matrix from
process 0. For the sake of simplicity, this matrix is diagonal, but more
complex cases can be handled as well.

First, we define the Epetra\_FECrsMatrix in Copy mode as
\begin{verbatim}
Epetra_FECrsMatrix A(Copy,Map,1);
\end{verbatim}
Now, we will set all the diagonal entries from process 0:
\begin{verbatim}
if( Comm.MyPID() == 0 ) {
  for( int i=0 ; i<NumGlobalElements ; ++i ) {
    int indices[2];
    indices[0] = i; indices[1] = i;
    double value = 1.0*i;
    A.SumIntoGlobalValues(1,indices,&value);
  }
}
\end{verbatim}
The Function \verb!SumIntoGlobalValues! adds the coefficients specified
in \verb!indices! (as pair row-column) to the matrix, adding them to any
coefficient that may exist at the specified location. In a finite
element code, the user will probably insert more than one coefficient
at time (typically, all the matrix entries corresponding to an elemental
matrix).

Next, we need to exchange data, to that each matrix element not owned by
process 0 could be send to the owner, as specified by \verb!Map!. This
is accomplished by calling, on all processes,
\begin{verbatim}
A.GlobalAssemble();
\end{verbatim}
A simple 
\begin{verbatim}
cout << A;
\end{verbatim}
can be used to verify the data exchange.

%%%
%%%
%%%



