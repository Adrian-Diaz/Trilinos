%
\section{Design of \texttt{Ref\-Count\-Ptr<>}}
\label{rcp:apdx:design}
%

{\bsinglespace
\begin{figure}
\begin{center}
%\fbox{
\includegraphics*[bb= 0.0in 0.0in 4.5in 5.3in,scale=1.0
]{RefCountPtrClassDiagram}
%}%fbox
\end{center}
{}\caption{ {}\label{rcp:fig:rcp-class-diagram} UML class diagram :
This shows the basic design of {}\texttt{Ref\-Count\-Ptr<>} with its
reference-count node classes.  }
\end{figure}
\esinglespace}

Here we describe the basics of the C++ design of
{}\texttt{Ref\-Count\-Ptr<>} and discuss what goes on under the hood.
It is not necessary that developers using {}\texttt{Ref\-Count\-Ptr<>}
know these details in order to successfully use the class but C++
experts and the otherwise curious will want to know this information.
This information will also be very useful for those who need to debug
through code that uses {}\texttt{Ref\-Count\-Ptr<>}.  We, however,
only provide an outline of the implementation and the interested
reader is encouraged to study the actual C++ implementation and its
test program to fill in the gaps left in this discussion.  Note, the
best way to browse the C++ code is through the doxygen generated HTML
files that are part of the Teuchos package in the Trilinos
documentation on the web (just do a web search for Teuchos).

Figure {}\ref{rcp:fig:rcp-class-diagram} shows a UML
{}\cite{ref:uml_distilled_2nd_ed} class diagram for the basic design.
The templated class {}\texttt{Ref\-Count\-Ptr<T>} is the user-level
type that this document describes.  There are also other classes that
are used behind the scenes to manage the reference count information,
manage extra data and control deallocation.  Reference-count
information is managed primarily though an abstract non-templated
shared {}\texttt{Ref\-Count\-Ptr\-\_node} class object.  This abstract
class has only one subclass
{}\texttt{Ref\-Count\-Ptr\-\_node<\-ConcreteT\-,Dealloc\_T>} but this
subclass is templated on the actual concrete type {}\texttt{ConcreteT}
of the underly reference-counted object and the deallocation policy
(i.e.~function) object type {}\texttt{Dealloc\_T}.  The purpose of
these private utility classes will be make clear in the following
discussion.

Every {}\texttt{Ref\-Count\-Ptr<T>} object maintains a typed pointer
to the reference-counted object through a private data member
{}\texttt{ptr} (of type {}\texttt{T*}) which is shown in the UML
diagram as the association between {}\texttt{Ref\-Count\-Ptr<T>} and
{}\texttt{ConcreteT} (the actual concrete type of the object being
reference counted).  Here, the type {}\texttt{T} must be a supported
interface for the concrete type {}\texttt{ConcreteT}
(i.e.~{}\texttt{T} is base class of {}\texttt{ConcreteT} or
{}\texttt{T} are {}\texttt{ConcreteT} are the same) and this is shown
by the lollipop interface {}\texttt{T} connected to the type
{}\texttt{ConcreteT}.  It is through the private data member
{}\texttt{ptr} that the functions {}\texttt{get()},
{}\texttt{operator*()} and {}\texttt{operator->()} are implemented.

As mentioned above, every {}\texttt{Ref\-Count\-Ptr<T>} object also
maintains a pointer to a shared non-templated
{}\texttt{Ref\-Count\-Ptr\-\_node} object which is called
{}\texttt{node}.  This is shown using the association from
{}\texttt{Ref\-Count\-Ptr<T>} to {}\texttt{Ref\-Count\-Ptr\-\_node}
with the role name {}\texttt{node}.  The
{}\texttt{Ref\-Count\-Ptr\-\_node} {}\texttt{node} object maintains a
reference count ({}\texttt{count}) for the number of
{}\texttt{Ref\-Count\-Ptr<>} objects pointing to itself.  The
reference-count modification functions {}\texttt{incr\_count()} and
{}\texttt{deincr\_count()} on the {}\texttt{Ref\-Count\-Ptr\-\_node}
object {}\texttt{node} are called by {}\texttt{Ref\-Count\-Ptr}'s
constructors, destructors and assignment operators in order to
maintain the proper count.  The {}\texttt{Ref\-Count\-Ptr\-\_node}
{}\texttt{node} object also maintains an ownership flag
{}\texttt{has\-\_ownership} that determines if the deallocator will be
used to deallocate the shared object when the reference count goes to
zero (this is described later).  This boolean is used to primarily
support the case where objects allocated on the stack or statically
must be wrapped by a {}\texttt{Ref\-Count\-Ptr<>} object in order to
be passed to some piece of code.

When a {}\texttt{Ref\-Count\-Ptr<T>} object is constructed to
{}\texttt{NULL}, its {}\texttt{ptr} and {}\texttt{node} private
pointer data members are set to {}\texttt{NULL}.  This case is
indicated in the UML class diagram by the multiplicity indicators 0..1
(where the 0 multiplicity is for the case where the object is in the
{}\texttt{NULL} state).  This is stated explicitly by the constraint
{}\texttt{\{(node==NULL)==(ptr==NULL)\}} which will only be true if
both {}\texttt{node} and {}\texttt{ptr} are {}\texttt{NULL} or
non-{}\texttt{NULL} .

%
\subsection{Construction, conversion and assignment of {}\texttt{Ref\-Count\-Ptr<>} objects}
%

Some more of the details of the C++ design of
{}\texttt{Ref\-Count\-Ptr<>} will be discussed in the context of the
following snippet of user code that uses the classes from the class
hierarchy first presented in Section {}\ref{rcp:sec:intro}.

{\scriptsize\begin{verbatim}
if( ... ) {
  RefCountPtr<A>   a_ptr1 = rcp(new C);
  RefCountPtr<A>   a_ptr2 = a_ptr1;
  RefCountPtr<B1>  b1_ptr = rcp_dynamic_cast<B1>(a_ptr1);
  RefCountPtr<B2>  b2_ptr = rcp_dynamic_cast<B2>(a_ptr1);
  ...
}
\end{verbatim}}

{}\noindent{}In the above user code, a single concrete object of type
{}\texttt{C} is dynamically allocated and put into a
{}\texttt{Ref\-Count\-Ptr<A>} object.  Three other
{}\texttt{Ref\-Count\-Ptr<>} objects (of different interface types)
are then created that reference this same object.

In the first statement

{\scriptsize\begin{verbatim}
  RefCountPtr<A>   a_ptr1 = rcp(new C);
\end{verbatim}}

{}\noindent{}the first {}\texttt{Ref\-Count\-Ptr<>} object is created
by the expression {}\texttt{rcp(new C)} which calls the following
templated function

{\scriptsize\begin{verbatim}
template<class T> RefCountPtr<T> rcp( T* p )
{
  return RefCountPtr<T>(p,true);
}
\end{verbatim}}

{}\noindent{}which calls the constructor

{\scriptsize\begin{verbatim}
template<class T> RefCountPtr<T>::RefCountPtr( T* p, bool has_ownership )
  :ptr_(p)
  ,node_( p
     ? new PrivateUtilityPack::RefCountPtr_node_tmpl<T,DeallocDelete<T> >(p,DeallocDelete<T>(),has_ownership)
     : NULL )
{}
\end{verbatim}}

{}\noindent{}which is instantiated with the type {}\texttt{T = C}.  As
shown in the above constructor, the pointer {}\texttt{T* p} is passed
into the constructor of the concrete
{}\texttt{Ref\-Count\-Ptr\-\_node\-\_tmpl<\-C\-,Dealloc\-Delete<C> >}
node.  This process results in the exact same typed address that was
returned from {}\texttt{new C} being stored in the underlying
{}\texttt{Ref\-Count\-Ptr\-\_node\-\_tmpl<\-C\-,Dealloc\-Delete<C> >}
node object and it is this pointer and address that will be used to
delete the object later.  It is this process involving the use of the
templated function {}\texttt{rcp()} that guarantees that the same
address returned from {}\texttt{new} is passed back to
{}\texttt{delete} which addresses the problem described in Appendix
{}\ref{rcp:apdx:mivbc} that occurs on some platforms.  Figure
{}\ref{rcp:fig:rcp-object-diagram_m_1} shows a UML object diagram for
the first {}\texttt{Ref\-Count\-Ptr<C>} object created by the
statement {}\texttt{rcp(new C)}.  This is a temporary object which is
maintained by the compiler and we give it the name {}\texttt{t1} even
through it really has no name.  Note that the initial reference count
on the
{}\texttt{Ref\-Count\-Ptr\-\_node\-\_tmpl<\-C,\-Dealloc\-Delete<C> >}
is initialized by default as {}\texttt{count=1}.

{\bsinglespace
\begin{figure}
\begin{center}
%\fbox{
\includegraphics*[bb= 0.0in 0.0in 2.6in 1.8in,scale=1.0
]{RefCountPtrObjectDiagram_1}
%}%fbox
\end{center}
{}\caption{ {}\label{rcp:fig:rcp-object-diagram_m_1} Object diagram
for the temporary returned from the expression {}\texttt{rcp(new
C)}. }
\end{figure}
\esinglespace}

{\bsinglespace
\begin{figure}
\begin{center}
%\fbox{
\includegraphics*[bb= 0.0in 0.0in 4.4in 1.8in,scale=1.0
]{RefCountPtrObjectDiagram0}
%}%fbox
\end{center}
{}\caption{ {}\label{rcp:fig:rcp-object-diagram_0} Object diagram for
the state just before the completion of the statement
{}\texttt{RefCountPtr<A> a\_ptr1 = rcp(new C);}. }
\end{figure}
\esinglespace}

{\bsinglespace
\begin{figure}
\begin{center}
%\fbox{
\includegraphics*[bb= 0.0in 0.0in 4.4in 1.8in,scale=1.0
]{RefCountPtrObjectDiagram1}
%}%fbox
\end{center}
{}\caption{ {}\label{rcp:fig:rcp-object-diagram_1} Object diagram for
the state just after the completion of the statement
{}\texttt{RefCountPtr<A> a\_ptr1 = rcp(new C);}.  }
\end{figure}
\esinglespace}

{\bsinglespace
\begin{figure}
\begin{center}
%\fbox{
\includegraphics*[bb= 0.0in 0.0in 4.8in 1.8in,scale=1.0
]{RefCountPtrObjectDiagram4}
%}%fbox
\end{center}
{}\caption{ {}\label{rcp:fig:rcp-object-diagram_4} Object diagram for
the state after the creation of the four {}\texttt{Ref\-Count\-Ptr<>}
objects {}\texttt{a\_ptr1}, {}\texttt{a\_ptr2}, {}\texttt{b1\_ptr} and
{}\texttt{b2\_ptr}.  }
\end{figure}
\esinglespace}

After the first {}\texttt{Ref\-Count\-Ptr<C}} object is created from
{}\texttt{rcp(new C)}, as shown in Figure
{}\ref{rcp:fig:rcp-object-diagram_m_1}, the rest of the statement

{\scriptsize\begin{verbatim}
  RefCountPtr<A>   a_ptr1 = rcp(new C);
\end{verbatim}}

{}\noindent{}is executed.  This statement is executed in one of two
ways based on how the compiler generates the code.  The straightforward
(non-optimized) approach is to default construct the
{}\texttt{Ref\-Count\-Ptr<A>} object
\texttt{a\_ptr1} first and then call the assignment operator (see [???]).
However, the assignment operator for {}\texttt{Ref\-Count\-Ptr<>} is
not doubly templated so to assign a {}\texttt{Ref\-Count\-Ptr<C>}
object to a {}\texttt{Ref\-Count\-Ptr<A>} object, the doubly templated
copy constructor must be invoked to perform a conversion from a
{}\texttt{Ref\-Count\-Ptr<C>} object to a
{}\texttt{Ref\-Count\-Ptr<A>} object.  The copy constructor invoked is
as follows.

{\scriptsize\begin{verbatim}
template<class T> template <class T2> RefCountPtr<T>::RefCountPtr(const RefCountPtr<T2>& r_ptr)
  : ptr_(r_ptr.get()) // will not compile if T2* is not convertible to T*
  , node_(r_ptr.access_node())
{
  if(node_) node_->incr_count();
}
\end{verbatim}}

{}\noindent{}Note how the reference count is explicitly incremented on
the node.

This conversion generates another temporary that we will call
{}\texttt{t2}.  With this temporary {}\texttt{Ref\-Count\-Ptr<A> t2}
created, the following assignment operator is then invoked.

{\scriptsize\begin{verbatim}
template<class T>  RefCountPtr<T>& RefCountPtr<T>::operator=(const RefCountPtr<T>& r_ptr)
{
  if(node_) {
    if( r_ptr.node_ == node_ )
      return *this; // Assignment to self!
    if( !node_->deincr_count() )
      delete node_;
  }
  ptr_  = r_ptr.ptr_;
  node_ = r_ptr.node_;
  if(node_) node_->incr_count();
  return *this;
}
\end{verbatim}}

{}\noindent{}The above implementation of the assignment operator is a
little more complex since it has to look out for assignment-to-self
and has to remove its reference to an existing reference-counted
object (if it has one).  After the assignment operator is invoked and the statement

{\scriptsize\begin{verbatim}
  RefCountPtr<A>   a_ptr1 = rcp(new C);
\end{verbatim}}

{}\noindent{}has completed and just before the compiler-generated
temporaries are destroyed, the state of the objects are as shown in
Figure {}\ref{rcp:fig:rcp-object-diagram_0}.

After the statement

{\scriptsize\begin{verbatim}
  RefCountPtr<A>   a_ptr1 = rcp(new C);
\end{verbatim}}

{}\noindent{}finishes executing and the compiler deletes the
temporaries the object diagram looks like Figure
{}\ref{rcp:fig:rcp-object-diagram_1}.  At this point, only one
{}\texttt{Ref\-Count\-Ptr<>} object is in existence and the reference
count is back to one (i.e.~{}\texttt{count=1}).

Note that through return value optimization [???] and the conversion
of

{\scriptsize\begin{verbatim}
  RefCountPtr<A>   a_ptr1 = rcp(new C);
\end{verbatim}}

{}\noindent{}to

{\scriptsize\begin{verbatim}
  RefCountPtr<A>   a_ptr1(rcp(new C));
\end{verbatim}}

{}\noindent{}(which is allowed by the C++ standard [???]) that this
statement will result in no temporary objects being created when
optimizations are turned on at compile time.

After describing the details of the first statement above, we could
also describe the remaining statements

{\scriptsize\begin{verbatim}
  RefCountPtr<A>   a_ptr2 = a_ptr1;
  RefCountPtr<B1>  b1_ptr = rcp_dynamic_cast<B1>(a_ptr1);
  RefCountPtr<B2>  b2_ptr = rcp_dynamic_cast<B2>(a_ptr1);
\end{verbatim}}

{}\noindent{}in same level of detail but the issues are largely the
same so we do not discuss them.  Interested readers are encouraged to
look the implementation of the template function
{}\texttt{rcp\-\_dynamic\-\_cast()} for instance to see how it works.
After all of the above four {}\texttt{Ref\-Count\-Ptr<>} objects are
constructed, the object diagram looks Figure
{}\ref{rcp:fig:rcp-object-diagram_4}.

%
\subsection{Destruction of reference-counted objects}
%

Up to this point we have described how {}\texttt{Ref\-Count\-Ptr<>}
objects are constructed and assigned and how reference counts are
manipulated.  Now it is equality important to describe how
reference-counted objects maintained through
{}\texttt{Ref\-Count\-Ptr<>} are deleted when all of the
{}\texttt{Ref\-Count\-Ptr<>} objects referring to the underlying
object are removed.  Again, considering the snippet of user code

{\scriptsize\begin{verbatim}
if( ... ) {
  RefCountPtr<A>   a_ptr1 = rcp(new C);
  RefCountPtr<A>   a_ptr2 = a_ptr1;
  RefCountPtr<B1>  b1_ptr = rcp_dynamic_cast<B1>(a_ptr1);
  RefCountPtr<B2>  b2_ptr = rcp_dynamic_cast<B2>(a_ptr1);
  ...
}
\end{verbatim}}

When the end of the {}\texttt{if(...) \{ ... \}} block is reached, the
automatic {}\texttt{RefCountPtr<>} objects {}\texttt{a\_ptr1},
{}\texttt{a\_ptr2}, {}\texttt{b1\_ptr} and {}\texttt{b2\_ptr}
allocated on the stack by the compiler are popped off of the stack and
destroyed in the opposite order that they where created.  When each of
these {}\texttt{RefCountPtr<>} objects is destroyed, the following
destructor is called

{\scriptsize\begin{verbatim}
template<class T> RefCountPtr<T>::~RefCountPtr()
{
  if(node_ && node_->deincr_count() == 0 ) delete node_;
}
\end{verbatim}}

{}\noindent{}The above destructor is quite simple.  If the
{}\texttt{RefCountPtr<>} object being destroyed is not {}\texttt{NULL}
(i.e.~{}\texttt{node\_ != 0}) then the reference count is
deincremented and if the reference count reaches zero, the node object
is deleted.  Therefore, in the above snippet of user code, the
reference count drops from four to three when {}\texttt{b2\_ptr} is
destroyed, and from three to two when {}\texttt{b2\_ptr} is destroyed
and so on until {}\texttt{a\_ptr1} is destroyed.  When
{}\texttt{a\_ptr1} is destroyed the reference count goes to zero and
the statement {}\texttt{delete node\_;} is executed.  This then calls
the destructor

{\scriptsize\begin{verbatim}
template<class T, class Dealloc_T>
class RefCountPtr_node_tmpl : public RefCountPtr_node {
public:
  ...
  ~RefCountPtr_node_tmpl()
  { if( has_ownership() ) dealloc_.free(ptr_); }
  ...
};
\end{verbatim}}

{}\noindent{}which will call on the template deallocator policy object
to free the maneged object if it has ownership to do so.  As
mentioned earlier, because of the way that the very first
{}\texttt{RefCountPtr<>} object was created, the pointer
{}\texttt{ptr\_} is exactly the same pointer returned from
{}\texttt{new C} which addresses the problem discussed in Appendix
{}\ref{rcp:apdx:mivbc}.  In this case, the default deallocator

{\scriptsize\begin{verbatim}
template<class T>
class DeallocDelete { public: void free( T* ptr ) { if(ptr) delete ptr; }  };
\end{verbatim}}

{}\noindent{}is used which calls {}\texttt{delete ptr;} to finally
free the managed object.

%
\subsection{Management of extra data}
%

ToDo: Update this for new specification for set\_extra\_data(...)!

The last issue to discuss is how extra data is managed as described in
Section {}\ref{rcp:sec:extra-data}.  As shown in Figure
{}\ref{rcp:fig:rcp-class-diagram}, each
{}\texttt{Ref\-Count\-Ptr\-\_node} object contains an array of
{}\texttt{any} objects.  The {}\texttt{any} class used here is an
adaptation of the {}\texttt{boos\-::any} class [???] that is also
contained in the Teuchos package and exists in the {}\texttt{Teuchos}
namespace.

The class {}\texttt{any} uses RTTI which allows it to store any type
of built-in or user-defined type that has value semantics (i.e.~has a
copy constructor and an assignment operators defined) and then to
retrieve it upon request (assuming the client knows the correct type).

The templated extra-date manipulation functions

{\scriptsize\begin{verbatim}
template<class T1, class T2>
int Teuchos::set_extra_data( const T1 &extra_data, Teuchos::RefCountPtr<T2> *p, int ctx )
{
  return p->access_node()->set_extra_data( extra_data, ctx );
}

template<class T1, class T2>
T1& Teuchos::get_extra_data( RefCountPtr<T2>& p, int ctx )
{
  return any_cast<T1>(p.access_node()->get_extra_data(ctx));
}
\end{verbatim}}

{}\noindent{}first described in Section {}\ref{rcp:sec:extra-data}
call on the non-templated functions

{\scriptsize\begin{verbatim}
class RefCountPtr_node {
public:
  ...
  int set_extra_data( const any &extra_data, int ctx );
  any& get_extra_data( int ctx );
  const any& get_extra_data( int ctx ) const;
  ...
};
\end{verbatim}}

{}\noindent{}that actually manipulate the array of {}\texttt{any}
objects.

Note that since {}\texttt{Ref\-Count\-Ptr<>} is itself a class that
uses value semantics that it can be stored in an {}\texttt{any} object
and therefore one or more {}\texttt{Ref\-Count\-Ptr<>} objects can be
associated with an existing {}\texttt{Ref\-Count\-Ptr<>} object using
{}\texttt{set\-\_extra\-\_data()} (this was described in Section
{}\ref{rcp:sec:extra-data}).
