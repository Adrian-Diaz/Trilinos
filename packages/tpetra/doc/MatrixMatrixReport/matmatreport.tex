\documentclass[pdf,12pt]{SANDreport}
\usepackage{verbatim}
\usepackage{url}
\usepackage{graphicx}
\usepackage{xspace}
\usepackage{fancyvrb}

\newcommand{\Aprime}{\ensuremath{A'}\xspace}
\newcommand{\Bprime}{\ensuremath{B'}\xspace}
\title{Evaluation of Tpetra's Sparse Matrix-Matrix Multiplication Routine}
\author{Kurtis L. Nusbaum}
\date{}

\SANDnum{SANDXXXX-XXXX}
\SANDprintDate{August 2011}
\SANDauthor{Kurtis L. Nusbaum}
\begin{document}
\VerbatimFootnotes

\maketitle

\begin{abstract}
Over the course of the last year, a sparse matrix-matrix multiplication routine has been developed for the Tpetra package.
This routine is based the same algorithm that is used in EpetraExt with heavy modifications. Since it 
achieved a working state, several major optimizations have been made in an effort to speed up the routine. This report will
discuss the optimizations made to the routine, its current state, and where future work needs to be done.
\end{abstract}
\clearpage
\section*{Acknowledgments}
Thanks to all of those at Sandia National Labs who helped provide the tools needed to compile this report. 
A special thanks to Dr.\ Chris 
Siefert, Dr.\ Mark Hoemmen, and Dr.\ Johnathan Hu whose guidance was invaluable. Thanks to Dr.\ Mike Heroux for funding this 
project. Thank you to the entire Trilinos community whose help is always appreciated. Thanks to St.\ John's University
for providing resources which helped create the material that this report evaluates. 

This research used resources of the National Energy Research Scientific Computing Center, which is supported by the
Office of Science of the U.S.\ Department of Energy under Contract No.\ DE-AC02-05CH11231.

The format of this report is based on information found in~\cite{Sand98-0730}.
\cleardoublepage
\tableofcontents
\listoffigures
\listoftables
\clearpage

\SANDmain
\section{Introduction}
Over the course of the last year, a sparse matrix-matrix multiplication routine has been developed for the 
Tpetra~\cite{TpetraHomePage}
package. This routine is based on the same algorithm that is used in EpetraExt~\cite{EpetraExtHomePage} 
with heavy modifications. Since it achieved a working state, several major optimizations have been made in an 
effort to speed up the routine. This report will
discuss the optimizations made to the routine, its current state, and where future work needs to be done.


\section{Basic Outline of the Algorithm}
The Tpetra sparse matrix-matrix multiply algorithm allows two matrices ($A$ and $B$) to be multiplied. The result of this 
multiplication is then placed in a third matrix ($C$).
The basic algorithm is as follows:
\begin{enumerate}
  \item \Aprime and \Bprime are created from $A$ and $B$, respectively. If it has been specified that $A$ should be 
  transposed, an actual 
  transpose of the matrix is created and assigned to \Aprime. Otherwise \Aprime is simply set equal to $A$. The same is 
  done for creating \Bprime.
  \item A ``view'' of \Aprime and a ``view'' of \Bprime are created. These views provide fast access to information that 
  will be needed later in the algorithm. In addition, any imports of off-processor elements are done. Namely, all the 
  rows in \Bprime that contain columns needed by the local copy of \Aprime are imported.
  \item The sparsity pattern of $C$ is determined by doing a symbolic multiplication of \Aprime and \Bprime, i.e., no actual
  value computations are done. In this step, column indices for $C$ are computed and used to construct a graph.
  \item The actual multiplication of \Aprime and \Bprime is done by iterating through each row of \Aprime. For each row in 
  \Aprime, every row in \Bprime is looped through and the appropriate calculations are done.
  \item Unless indicated otherwise by the user, \verb!fillComplete! is called on matrix $C$.
\end{enumerate}

\section{Optimizations}

\subsection{Fixing FillComplete's Sort}
As part of its algorithm, the \verb!fillComplete! function relies on a function called \verb!sort2!. This function performs 
a sort on two arrays by sorting the first array and concurrently doing the same permutations on the second array, 
i.e., both arrays are sorted according to the ordering of the first array. The main use case for this function is sorting 
an indices array and moving the values in a values array so that they stay matched up with their associated index.

Up until recently, the \verb!sort2! function relied on an insertion sort algorithm. We modified the function so that it 
first checks to see if the arrays are already sorted (which happens quite often) and returns right away if they are. 
If the arrays are not sorted, a quicksort is performed on the arrays. This means that for the section of our sparse
matrix-matrix multiply routine where we call \verb!fillComplete! we went from running on average $O(n^2)$ operations to 
running on average $O(n \log(n))$ operations.
It's also worth noting that since our \verb!sort2! function checks first to see if the arrays are already in order, we 
avoid the worst case runtime for the quicksort routine.

\subsection{Removal of Specific Transpose Mode Kernels}
In the original ExpetraExt algorithm there were separate multiplication kernels for each possible transpose combination 
(e.g., $A^TB$, $AB^T$, and $A^TB^T$).
Some of these kernels relied on a function called \verb!find_rows_containing_columns!. The relevant trait to know about this 
function is that during execution it created an array that was of size 
$N_c+ 2N_p + N_p N_r$, where
$N_c$, $N_p$, and $N_r$ are the number of local columns, processors, and local rows, respectively.
Obviously this is not scalable. At anything but the lowest processor counts, this array quickly balloons to a size that 
won't fit in memory. We decided to remove this function and the specific transpose kernels. Instead, we explicitly 
transpose the matrices using Tpetra's \verb!RowMatrixTransposer! if needed
and use the $A \times B$ kernel on those matrices. This has \emph{significant}~\footnote{See section~\ref{perfSect} for 
details} performance benefits when doing operations like $A^TB$.

\subsection{Streamlining the Graph Building Routine}
The original algorithm from EpetraExt used the same function for both building the graph of matrix $C$ and calculating its 
values by doing two passes through the same function with different data. 
As a result, on the first pass through the function when the graph was being calculated, the values for matrix $C$ were also 
calculated, but thrown out. It's not until the second pass through the function 
that the values calculated are actually inserted into matrix $C$. We modified this function so that it
takes an argument which indicates whether or not we're just calculating the graph for matrix $C$. If we're just calculating 
the graph, all the value calculation is skipped. This saves the routine 
$O(Numrows \times nnzMax(A) \times nnzMax(B))$ operations where $nnzMax$ is the maximum number of non zeros in the rows of a 
given matrix.

\section{Performance}
\label{perfSect}
We conducted a series of weak-scaling studies on the Hopper NERSC machine. Hopper is a Cray XE6 with 24 cores
per node~\footnote{For more machine information regarding Hopper, please visit 
\url{http://www.nersc.gov/systems/hopper-cray-xe6}}. We define weak scaling as increasing the problem size while holding 
the amount of work done on each node constant. We define weak scaling efficiency as $Time_{serial}/Time_{parallel}$.
It is worth noting that none of the testing done below involving Tpetra utilizes any Kokkos~\cite{KokkosHomePage} kernels. 

\subsection{Comparison of Development Stages}
Figure~\ref{tpetracomptime} and Figure~\ref{tpetracompeff} show the test results of comparing the Tpetra sparse 
matrix-matrix multiply routine at various stages of its 
development. The tests were as such: Two matrices were constructed, each using a 3D Laplace stencil. Each node was assigned 
$110^3$ number of rows from each matrix (this means each core has approximately 55500 rows). 
We then timed the cost of multiplying these two matrices together. We ran these experiments three times for each node 
count, and then averaged the results. The red line represents the routine as it was when we first got the routine working. 
The blue line is from after the \verb!sort2! algorithm problem had been fixed and we removed the specific transpose mode 
kernels. And the purple line represents the routine after all optimizations had been applied.
As can be clearly seen, the optimizations we applied helped the algorithm substantially. After all of our optimizations 
were applied, the sparse matrix-matrix multiply routine was running in roughly half the time it was originally.

\begin{figure}
\centering
\includegraphics[scale=.5]{tpetraDevData/time.pdf}
\caption[Tpetra Development Time Comparison]{A comparison of the Tpetra sparse matrix-matrix multiply routine's runtime 
throughout various stages of its development. Note this is a simple $C=A \times B$; so we are running in non-transpose mode.}
\label{tpetracomptime}
\end{figure}

\begin{figure}
\centering
\includegraphics[scale=.5]{tpetraDevData/eff.pdf}
\caption[Tpetra Development Efficiency Comparison]{A comparison of the Tpetra sparse matrix-matrix multiply routine's 
scaling efficiency
throughout various stages of its development. Note this is a simple $C=A \times B$; so we are running in non-transpose mode.}
\label{tpetracompeff}
\end{figure}


\subsection{Tpetra vs. EpetraExt vs. ML}
Figure~\ref{alltime} and Figure~\ref{alleff} show the test results of comparing the Tpetra sparse matrix-matrix multiply 
routine (with all of its optimizations) to other sparse matrix-matrix multiply routines in Trilinos~\cite{1089021} 
(namely the original 
EpetraExt routine and a slightly modified version ML's Epetra sparse matrix-matrix multiply 
routine~\footnote{The only difference 
between the ML's \verb!Epetra_MatMat_Mult! and the one we used in our test is that we modified the routine to use ML's
matrix storage when assembling matrix $C$ rather than Epetra's}). 
These tests were conducted in the same manner as above. 
The encouraging thing about these results is that the Tpetra algorithm is as good as if not better than the EpetraExt 
algorithm at scale. It should be noted that the ML lines are cut short because at the next problem size in our testing, 
they would fail due to a memory error. Consequently, we could not run tests for greater problem sizes with ML. 

\begin{figure}
\centering
\includegraphics[scale=.5]{allMatsData/time.pdf}
\caption[All Algorithms Time Comparison]{A comparison of various Trilinos sparse matrix-matrix multiply routines' runtime. The
ML line is cut short due to a memory error that prevented testing at higher node counts.}
\label{alltime}
\end{figure}

\begin{figure}
\centering
\includegraphics[scale=.5]{allMatsData/eff.pdf}
\caption[All Algorithms Efficiency Comparison]{A comparison of various Trilinos sparse matrix-matrix multiply routines' scaling 
efficiency. The ML line is cut short due to a memory error that prevented testing at higher node counts.}
\label{alleff}
\end{figure}

\subsection{Transpose Mode Tests}
Since Tpetra and EpetraExt differ wildly when it comes to transpose modes, we decided to compare the two. We did the same
testing procedure as above, except that we requested A be transposed~\footnote{One could first explicitly transpose a 
matrix and then give it to the ML algorithm which does not have a transpose mode. We did not do this because the point of 
this test was to compare EpetraExt and Tpetra.}. Figure~\ref{transtime}
and Figure~\ref{transeff} show the results. To say that the Tpetra algorithm scales better than the EpetraExt algorithm
would be a gross understatement. The Tpetra algorithm levels off at about a 5 second runtime, where as EpetraExt 
algorithm's runtime seems to grow exponentially. We attempted to conduct a similar experiment where we transposed B 
instead of A, 
but the EpetraExt algorithm was taking even longer (over ten minutes on a single node of Hopper) and we didn't want to 
waste our limited computing resources. 

\begin{figure}
\centering
\includegraphics[scale=.5]{atransData/time.pdf}
\caption[Transpose Time Comparison]{A comparison of sparse matrix-matrix multiple runtime in Tpetra and EpetraExt using transpose mode}
\label{transtime}
\end{figure}

\begin{figure}
\centering
\includegraphics[scale=.5]{atransData/eff.pdf}
\caption[Transpose Efficiency Comparison]{A comparison of sparse matrix-matrix multiple scaling efficiencies in Tpetra and EpetraExt using transpose mode}
\label{transeff}
\end{figure}

\section{Areas for Future Improvement}

\subsection{Serial Test Motivations}
Table~\ref{serialTable} shows the results of running three Trilinos-based sparse matrix-matrix multiply algorithms in 
serial. These tests were conducted identically to the ones in section~\ref{perfSect} with the exception of being run on a 
single \emph{core} of Hopper. Since this test was in serial, all run times should be void of any time that might be 
attributed to communication. The discrepancy between the EpetraExt and ML has been known for quite some time and 
is believed by several scientists at Sandia National Labs to be attributed to a hashing schemed that the ML algorithm 
employs. We believe the discrepancy between Tpetra's and EpetraExt's time has do to with Tpetra's 
\verb!fillComplete! function since most everything else in this simple test 
is relatively the same between the Tpetra algorithm and the EpetraExt algorithm. This leads us to believe that
by improving Tpetra's underlying architecture (i.e. refining \verb!fillComplete!) and implementing ML's hashing
scheme, we should be able to make Tpetra's sparse matrix-matrix multiplication routine run at the same level as ML.

\begin{table}
\centering
\begin{tabular}{ | l | l | l | l | }
\hline
 & Tpetra & EpetraExt & ML \\ \hline
Time (Seconds) & 1.4 & 0.9 & 0.3 \\ \hline
\end{tabular}
\caption{Serial Matrix-Matrix Multiplication Run Times}
\label{serialTable}
\end{table}

\subsection{Improvement of Underlying Tpetra Architecture}
The most important thing that will help the sparse matrix-matrix multiply routine is to improve the implementation of the
underlying Tpetra architecture. 
Undoubtedly the inefficient \verb!sort2! routine we found is not the only problem with Tpetra at large. There have been 
reports from other scientists at Sandia National Labs that things like the Tpetra Import/Export classes are not running as 
nearly as fast as their Epetra counterparts. In addition, Tpetra has no performance tests. Consquently, the performance of 
almost all of tpetra is unknown. These things need to be investigated, and fixed if necessary.

\subsection{Possible Implementation of ML's Algorithm}
ML's sparse matrix-matrix multiply routine uses a complex hashing scheme in order to speed lookup times for 
column indices. Overall, ML's sparse matrix-matrix multiplication algorithm is fast, and several scientist's at Sandia National 
Labs believe this is mainly due to the hashing scheme. Figure~\ref{hashalgo} outlines, in pseudo code, the 
algorithm as current Tpetra developers understand it. Implementing this algorithm in the Tpetra sparse matrix-matrix multiply 
routine should not be all that difficult and could potentially provide great speed improvments.

\begin{figure}
\centering
{\footnotesize
\begin{verbatim}
1. For matrix B only, create a hashtable where given a globalid, 
we get a unique hashtag. i.e. hash[gid] = hashtag
Note that in Serial we don't actually need a "hash", local id's should suffice
2.Create a "reverse" map, one where we can do rmap[hastag] = gid
3. Allocate an array called acc_index with the same size as the hash table
4. Allocate two arrays called acc_col and acc_val whose size is 
equal to the maximum number of row entries in matrix A.
5. For each row i in A{
  acc_index.fill(-1)
  ArrayView cur_A_cols;
  ArrayView cur_A_vals;
  A->getRowView(i, cur_A_cols, cur_A_vals);
  curr_acc_ptr=0;
  for each column k in row A[i]{
    ArrayView cur_B_cols;
    ArrayView cur_B_vals;
    (B or Bimport)->getRowView(k, cur_B_cols, cur_B_vals);
    for(j=0; j< cur_B_cols.size(); ++j){
      cur_acc_index = hashtable(getGlobalElement(cur_B_cols[j])) //precomputing this might be useful
      if(acc_index(cur_acc_index) == -1){
        acc_col[cur_acc_prt] = cur_acc_index  //Probably should just put in gid actually
        acc_val[cur_acc_ptr] = cur_B_vals[j]*cur_A_vals[k]
        acc_index[curr_acc_index]=cur_acc_ptr++
      }
      else{
        acc_val[acc_index(cur_acc_index)] += cur_B_vals[j]*cur_A_vals[k]
      }
    }
  }
  c.insertGlobalVals(i, (Global_ids_of_hashes(acc_col))(0, curr_acc_ptr), acc_val(0,cur_acc_ptr))
}

\end{verbatim}
}
\caption[Hash based algorithm]{ML's hash based algorithm for sparse matrix-matrix multiply}
\label{hashalgo}
\end{figure}

\subsection{Kokkos Kernel}
While not necessarily indicated by the serial tests, writing a Kokkos kernel for the matrix-multiply routine could improve
performance. The implementation of such a kernel should be fairly trivial and provide significant speedups by taking
advantage of node-level parallelism.

\section{Contributions}
All matrix multiplication code was adapted from EpetraExt for use in Tpetra by Kurtis Nusbaum. Test code was written by 
Jonathan Hu, Chris Siefert, Jeremie Gaidamour, and Kurtis Nusbaum.

\clearpage
\providecommand*{\phantomsection}{}
\phantomsection
\addcontentsline{toc}{section}{References}
\bibliographystyle{plain}
\bibliography{matmatreport}

\appendix
\section{Trilinos Configure Script}
Figure~\ref{triConf} shows the script that was used to configure the build of Trilinos on Hopper that was used for testing.
\begin{figure}
\centering
{\footnotesize
\VerbatimInput{../../../../sampleScripts/linux-hopper-tpetra-cmake}
}
\caption{Configure script used for Trilinos}
\label{triConf}
\end{figure}
\begin{SANDdistribution}[NM]

\SANDdistInternal{1}{1318}{Kurtis Nusbaum}{01426}
\SANDdistInternal{1}{1320}{Dr. Michael Heroux}{01426}
\SANDdistInternal{1}{0378}{Dr. Christopher Siefert}{01443}
\SANDdistInternal{1}{9159}{Dr. Jonathan Hu}{01426}
\SANDdistInternal{1}{1320}{Dr. Jeremie Gaidamour}{01426}
\end{SANDdistribution}


\end{document}
