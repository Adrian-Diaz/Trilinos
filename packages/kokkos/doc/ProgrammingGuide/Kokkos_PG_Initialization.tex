
\chapter{Initialization}

In order to use Kokkos an initialization call is required. 
That call is responsible for aquiring hardware resources such as threads. 
Typically this call should be placed right at the start of a program similar to 'MPI\_Init'.
If both MPI and Kokkos are used, the Kokkos initialization should happen right after the MPI initialization, 
so that potential process binding masks are set up.

\section{Default Initialization}
The most simple way to initialize Kokkos is calling the generic:

\begin{lstlisting}
Kokkos::initialize(int argc, char* argv[]); 
\end{lstlisting}
function, which interprets command line arguments to determine the requested settings.
Using this function will initialize the default execution space
\lstinline|Kokkos::DefaultExecutionSpace|
 and its default host execution space 
\lstinline|Kokkos::DefaultHostExecutionSpace|
if applicable.
What the default execution spaces are depends on the Kokkos configuration. 
The current order from from low to high ranking is 
\lstinline|Kokkos::Serial|, 
\lstinline|Kokkos::Threads|, 
\lstinline|Kokkos::OpenMP|, 
\lstinline|Kokkos::Cuda| with the highest enabled one being the respective defaults.
For example: if  \lstinline|Kokkos::Cuda|, \lstinline|Kokkos::Threads|, and 
\lstinline|Kokkos::Serial| are enabled, than \lstinline|Kokkos::Cuda| is the
default execution space and \lstinline|Kokkos::Threads| is the default host execution space.

Command line arguments come in a 'prefixed' and 'non-prefixed' version. 
The former will be removed from the list of arguments while the latter are not.
Argument options are given with a '=' sign.
If the same argument occures more than once, the last one counts. 
Furthermore, prefixed versions of the command line arguments take precedence over the non-prefixed ones.
For example \lstinline|--kokkos-threads=4 --threads=2| results in number of threads to be set to 4,
 while \newline\lstinline|--kokkos-threads=4 --threads=2 --kokkos-threads=3| results in threads being set to 3.
A full list of command line options is given in table \ref{TBL:CommandLineOptions}.

\begin{table}
\caption{Table of command line options for \lstinline|Kokkos::initialize|}
\label{TBL:CommandLineOptions}
\begin{small}
\begin{tabular}[t]{lp{0.5\textwidth}}
\hline
Argument & Description \\\hline
\lstinline|--kokkos-help| & print this message \\
\lstinline|--kokkos-threads=INT| & 
specify total number of threads or number of threads per NUMA region if used in conjunction with '--numa' option. \\
\lstinline|--kokkos-numa=INT| & specify number of NUMA regions used by process. \\
\lstinline|--kokkos-device=INT| & specify device id to be used by Kokkos. \\
\lstinline|--kokkos-ndevices=INT[,INT]| & used when running MPI jobs. Specify number of
devices per node to be used. Process to device
mapping happens by obtaining the local MPI rank
and assigning devices round-robin. The optional
second argument allows for an existing device
to be ignored. This is most useful on workstations
with multiple GPUs of which one is used to drive
screen output.\\
\hline
\end{tabular}
\end{small}
\end{table}

Instead of giving \lstinline|Kokkos::initialize()| command line arguments, one can directly provide the necessary parameters for intialization.
Used that way the call accepts one to three integer arguments. The meaning of those arguments depends on the default execution space. 
If the default execution space is a host space the first argument is the number of threads per numa region and the second (optional) argument is the number of numa regions to be used by this process.
When \lstinline|Kokkos::Cuda| is the default execution space, the last given argument is the device id to be used by this process. 

At the end of each program Kokkos needs to be shut down in order to free resources. 
This is achieved by calling:
\begin{lstlisting}
Kokkos::finalize(); 
\end{lstlisting}

\section{Execution Space Specific Initialization}

Instead of calling the generic initialization, one can call initialization for each execution space on its own. 
If the associated host execution space of an execution space is not identical to the latter, it has to be initialized first.
For example when compiling with support for pthreads and Cuda, \lstinline|Kokkos::Threads| has to be initialized before \lstinline|Kokkos::Cuda|.
The initialization calls take different arguments for each of the execution spaces.
\lstinline|Kokkos::Cuda| 



