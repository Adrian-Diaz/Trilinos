\chapter{INTRODUCTION}
The Software Utilities Package for the Engineering
Sciences (SUPES) is a collection of subprograms which perform frequently
used non-numerical services for the engineering applications programmer.
The
three functional categories of SUPES are: (1) input command parsing, (2) dynamic
memory management, and (3) system dependent utilities.  The subprograms in
categories one and two are written in standard FORTRAN-77~\cite{ansi},
while the subprograms
in category three are written in the C programming language.
Thus providing a standardized FORTRAN interface to
several system dependent features across a variety of hardware configurations
while using a single set of source files.
This feature can be viewed as a maintenance aid from
several perspectives.
Among these are:
there is only one set of source files to account for,
it allows one to standardize the build procedure,
and it provides a clearer starting point for any future ports.
In fact,
a build procedure is now part of the standard SUPES distribution and
is documented in Chapter~\ref{sec:install}.
Further,
the system dependent modules set an appropriate template for the
porting of SUPES to other hardware and/or software configurations.

Applications programmers face many similar user and system interface problems
during code development.  Because ANSI standard FORTRAN does not address many of
these problems, each programmer solves these problems for his/her own code.
SUPES aids the programmer by:
\begin{enumerate}

\item Providing a library of useful subprograms.

\item Defining a standard interface format for common utilities.

\item Providing a single point for debugging of common utilities.  That
is, SUPES has to be debugged only once and then is ready for use
by any code.
\end{enumerate}

Use of SUPES by the applications programmer can expand a code's capability,
reduce errors, minimize support effort and reduce development time.  Because
SUPES was designed to be reliable and supportable, there are some features
that are not included.  (1) It is not extremely sophisticated, rather it is
reliable and maintainable.  (2) Except for the extension library (Chapter
4), it is not system dependent.  (3) It does not take advantage of extended
system capabilities since they may not be available on a wide range of
operating systems.  (4) It is not written to maximize cpu speed.

It is the intention of the authors to maintain SUPES on all scientific
computer systems commonly used by Engineering Sciences Directorate (1500)
staff.
Currently these systems include:
\begin{enumerate}
\item Sun 3 and Sun 4 running SunOS operating system version 4.0.3 and later,

\item VAXen running VMS version 4.5 and later,

\item Cray X/MP and Y/MP running UNICOS version 5.0 and later, and

\item Alliant F/X 8 running Concentrix 5.0.0.
\end{enumerate}
A notable omission to the above list is the Cray running either CTSS
or the COS operating systems.
These configurations still require the FORTRAN source code for the extension
library that was provided in previous
implementations of SUPES~\cite{SUPES}.
This code continues to be included in the current standard SUPES distribution,
though a build procedure designed for these systems is not.
Specific ports of the SUPES utilities to new machines and/or operating systems will be
added to the original source files as the need arises.
Other Sandia personnel may obtain copies of SUPES from the
authors.  SUPES will also be available to non-Sandia personnel through the
National Energy Software Center.
