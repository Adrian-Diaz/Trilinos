\chapter{Site Supplements} \label{a:site}

\chapter{SITE SUPPLEMENTS} \label{sec:site}
This appendix contains a supplement for each site at which SUPES is
currently installed.  Changes to the current systems and the addition of new
sites will require that this appendix be amended; the information contained
here should be considered just a starting point.

All system independent source code for SUPES is stored on the SNLA Central
File System (CFS) with those files having a file type of ``.STX''
being stored in Standard Text format.
The SNLA installation of SUPES contains both the previous
and new versions.
This is done for two reasons,
first, to provide the necessary compatibility during an interim
migration period,
and,
second,
to assure that current users of Cray/CTSS continue to have a point of reference
for the SUPES library.

The previous version is stored under the root directory ``/SUPES''.
The table below documents the files stored in this directory. 

\begin{tabular}{|ll|} \hline \hline
\multicolumn{1}{|c}{Node} & \multicolumn{1}{c|}{Contents} \\ \hline
FRE\_FLD.STX   &  Free field reader source code\\
MEM\_MGR.STX   &  Memory manager source code\\
EXT\_LIB.STX   &  Skeleton FORTRAN extension library source code\\
FRR\_TEST.STX  &  Free field reader test program source code\\
MEM\_TEST.STX  &  Memory manager test program source code\\
EXT\_TEST.STX  &  FORTRAN extension library test program source code\\ \hline
\hline
\end{tabular}

The current version is stored under the CFS root directory
``/SUPES2\_1'' in the following files (the last two are {\em not} in Standard Text
Format):

\begin{tabular}{|ll|} \hline \hline
\multicolumn{1}{|c}{Node} & \multicolumn{1}{c|}{Contents} \\ \hline
FRE\_FLD.STX   &  Free field reader source code\\
MEM\_MGR.STX   &  Memory manager source code\\
EXT\_LIB.STX   &  Portable C extension library source code\\
FRR\_TEST.STX  &  Free field reader test program source code\\
MEM\_TEST.STX  &  Memory manager test program source code\\
EXT\_TEST.STX  &  Extension library test program source code\\
SUPES2\_1.BCK  &  The version 2.1 distribution in VMS BACKUP format\footnotemark%
\addtocounter{footnote}{-1}\\
SUPES2\_1.TAR  &  The version 2.1 distribution in UN*X TAR format\footnotemark%
\\ \hline
\hline
\end{tabular}
\footnotetext{Note that these files are {\em not\/}
stored in Standard Text}
The current extension library has been ported
to run on the following machine/operating system combinations:
\begin{enumerate}
\item Sun 3 and Sun 4 running SunOS operating system version 4.0.3 and later,

\item VAXen running VMS version 4.5 and later,

\item Cray X/MP and Y/MP running UNICOS version 5.0 and later, and

\item Alliant F/X 8 running Concentrix 5.0.0.
\end{enumerate}
A notable exception to the above list is the Cray using the CTSS operating system.
This configuration still requires the FORTRAN source code for the extension
library that was provided in previous
implementations of SUPES.
This code continues to be included in the current standard SUPES distribution,
though a build procedure designed for this system is not.

These files may be retrieved via the MASS utility and converted to Native
Text Format via the NTEXT utility.  Sandia personnel may consult the
Computer Consulting and Training Division (2614) for details on these
utilities.

\section{Site Supplement For 1500 VAX Cluster (VAX/VMS 5.1)}

\subsection{Linking}
The SUPES package is accessed on the 1500 VAX CLUSTER (SAV01, SAV03,
SAV07 and SAV08) as an object library located via either of two system logical
names.
Which one that the user uses depends on which version that he or she
wants to use.
The older
SUPES routines are linked to an application program as follows:

\begin{verbatim}
$ LINK your_program,SUPES/LIB,etc.
\end{verbatim}
While the newer version can be accessed at link time via:
\begin{verbatim}
$ LINK your_program,SUPES2_1/LIB,etc.
\end{verbatim}
The last of the above commands assumes that the SUPES2\_1 library has
been installed by someone using the \verb+VMSINSTALL.COM+
command procedure.
If that is not the case,
then the user will be informed by
the \verb+LINK+er that there is an abundance of unsatisfied
external references that have been made.
To avoid this scenario,
one should be sure to provide the \verb+VAX C+ Run Time Library
to the \verb+LINK+ command.
One way to do this appropriately is to modify the above
link command as:
\begin{verbatim}
$ LINK your_program,SUPES2_1/LIB,SYS$LIBRARY:VAXCRTL/LIB
\end{verbatim}
The alternative is to define the logical \verb+LNK$LIBRARY+
to be \verb+SYS$LIBRARY:VAXCRTL+.
For systems which already have this logical assigned,
define the logical \verb+LNK$LIBRARY_n+,
where n is the smallest integer for which the corresponding
logical has not been assigned.
(Hints about how to go about this are provided in the
file \verb+[.BUILD]VMSINSTALL.COM+.)

\subsection{Defining Unit/File or Symbol/Value for EXNAME:}
Both versions of SUPES use this extension library call in the
same manner.
A file name is connected to a unit number via a logical name of the form
FORnnn,  where ``nnn'' is a three digit integer indicating the FORTRAN unit
number. For example:
\begin{verbatim}

$ ASSIGN CARDS.INP FOR007

\end{verbatim}
causes the following FORTRAN statements to open ``CARDS.INP'' on unit 7.
\begin{verbatim}
CALL EXNAME( 7, NAME, LN )
OPEN( 7, FILE=NAME(1:LN) )
\end{verbatim}

One caveat to note regarding the above sequence is that if the \verb+ASSIGN+
statement is not performed,
the user program will abort with an error in the \verb+OPEN+ statement.
A possible,
or preferred code sequence is:
\begin{verbatim}
CALL EXNAME( 7, FILENM, LN )
IF( LN .EQ. 0 ) THEN      ! EXNAME returns a zero for LN if no ASSIGN
                          ! has been performed.  Use the system default.
  OPEN( 7 )
ELSE                      ! I've found an ASSIGN'd filename, use it.
  OPEN( 7, FILE=FILENM )
ENDIF
\end{verbatim}
where the system default mentioned in the above FORTRAN comment
is a file named ``\verb+FOR007.DAT+'' in the current
default directory.

EXNAME looks for a DCL symbol of the form EXTnn, where ``nn'' is a two digit
integer which defines a symbol number.  For example:
\begin{verbatim}

$ EXT01 = "HELLO"

\end{verbatim}
will cause the following call to return NAME=``HELLO'' and LN=5.
\begin{verbatim}

CALL EXNAME( -1, NAME, LN )

\end{verbatim}

\subsection{Interface to EXREAD} 
EXREAD will prompt to,
and read from,
SYS\$INPUT.
It will automatically echo to SYS\$OUTPUT if that device is a terminal.
However,
if a program is run in a context where SYS\$OUTPUT is {\em not} a
terminal,
such as from within a command procedure,
the input is not echoed---the user will have to control this
himself with an appropriate call parameter to the routine FREFLD.
EXREAD supports all the VMS command line editing features (e.g., CTRL/U,
$<$up-arrow$>$, etc.).  An end-of-file from the terminal keyboard is
indicated by CTRL/Z. 

\subsection{Additional Comments Regarding SUPES2\_1}
When attempting to redefine the logical SYS\$OUTPUT,
the user should note that under VMS,
the mixed language environment has a minor side effect:
two versions of the output file are created by default.
To avoid this scenario,
he or she,
will have to explicitly open the file.
The following code segment demonstrates the required command sequence:
\begin{verbatim}

$ OPEN/WRITE SYSOUT OUTPUT.DAT
$ ASSIGN/USER_MODE SYSOUT SYS$OUTPUT
$ RUN PROG
$ CLOSE SYSOUT

\end{verbatim}

Finally,
the user should be aware that his or her program is being
linked with the VAX C Run-time Library.
Consequently,
certain function,
subroutine
or more generally,
external symbol names\footnote{This does {\em not}
include FORTRAN keywords, for example, \verb+READ+ and
\verb+WRITE+ statements.}
might be in conflict with some of these
run time library functions.
These include the names:
\begin{enumerate}

\item \verb+SPRINTF+

\item \verb+GETENV+

\item \verb+READ+

\item \verb+WRITE+

\item \verb+STRNCPY+

\item \verb+STRCPY+

\item \verb+STRLEN+

\item \verb+SBRK+

\item \verb+BRK+

\item \verb+ISATTY+

\item \verb+PERROR+

\item \verb+ISASCII+

\item \verb+ISCNTRL+

\item \verb+ISALPHA+

\item \verb+ISLOWER+

\item \verb+TOUPPER+
\end{enumerate}
The remedy is to redefine any user-supplied conflict
when warned by the linker of multiply defined symbol names.

\subsection{Source Code}
The source code for the old FORTRAN extension library for the VAX/VMS operating
system is stored in the SNLA Central File System under node
``/SUPES/VMS/EXT\_LIB.STX'' in SNLA Standard Text format.
Conversely,
the new version is in
``/SUPES2\_1/EXT\_LIB.STX''.

\cleardoublepage

\section{Site Supplement for SNLA CRAY-1/S (COS 1.11)}

\subsection{Linking}
The newer version of SUPES is not available for this system.
However,
the older package can still be accessed on the SNLA CRAY-1/S
using COS 1.11 as an object library.
The permanent dataset containing SUPES is accessed as follows:
\begin{verbatim}

ACCESS,DN=SUPES,ID=ACCLIB.

\end{verbatim}
SUPES routines are then linked to an application program as follows:

\begin{verbatim}

LDR,other_options,LIB=SUPES:other_libraries.

\end{verbatim}

\subsection{Defining Unit/File or Symbol/Value for EXNAME}
A file name is connected to a unit number via an alias of the form FTnn,
where ``nn'' is a two digit integer indicating the FORTRAN unit number. For
example:
\begin{verbatim}

ASSIGN,DN=CARDS,A=FT07.

\end{verbatim}
causes the following FORTRAN statements to open 'CARDS' on unit 7.
\begin{verbatim}
          CALL EXNAME( 7, NAME, LN )
          OPEN( 7, FILE=NAME(1:LN) )
\end{verbatim}
Again,
the more suitable code sequence is
\begin{verbatim}
CALL EXNAME( 7, FILENM, LN )
IF( LN .EQ. 0 ) THEN       ! EXNAME returns a zero for LN if no ASSIGN
                           ! has been performed.  Use the system default.
  OPEN( 7 )
ELSE                       ! I've found an ASSIGN'd filename, use it.
  OPEN( 7, FILE=FILENM )
ENDIF
\end{verbatim}

If no file has been assigned the alias for a particular unit, EXNAME will
return a file name of the form TAPEnn, where ``nn'' is a one (if less than
ten) or two digit integer indicating the FORTRAN unit number---this is also the
system default.

EXNAME looks for a JCL symbol of the form Jn, where ``n'' is a one digit
integer which defines a symbol number.  For example:
\begin{verbatim}

SET(J1='HELLO')

\end{verbatim}
will cause the following call to return NAME=``HELLO'' and LN=5.
\begin{verbatim}

CALL EXNAME( -1, NAME, LN )
\end{verbatim}

\subsection{Interface to EXREAD}
EXREAD will read from \$IN and automatically echo to \$OUT.  COS at SNLA has
no interactive capability.

\subsection{Known Problems}
The CFT 1.11 support routines contain a bug which may cause FREFLD to
function improperly.  FREFLD was modified for this installation such that
application programs which call FREFLD should not notice any problem.

The problem is that the CFT 1.11 support routines do not return an error in
the IOSTAT argument for invalid real formats; a zero value and a zero
(success) status are returned in such a case.  The symptom observed from
FREFLD is that KVALUE will indicate that a valid REAL value was specified
for a data field which contains an invalid REAL format; the value returned
in RVALUE for this field will be set correctly to zero.  To work around this
problem FREFLD was modified to downgrade KVALUE from one (valid REAL value)
to zero (invalid REAL value) under the following conditions:
\begin{enumerate}
\item The field does not contain a valid INTEGER value.
\item The REAL value translated for the field is zero.
\item The field does not begin with '0.' nor '.0'.
\end{enumerate}

\subsection{Source Code}
The source code for the FORTRAN extension library for the COS 1.11 operating
system is stored in the SNLA Central File System under node
``/SUPES/COS/EXT\_LIB.STX''
in SNLA Standard Text format.  The source code for
the modified version of FREFLD described above is stored under node
``/SUPES/COS/FRE\_BUG.STX'' in SNLA Standard Text format.

\cleardoublepage

\section{Site Supplement for SNLA CRAY-1/S (UNICOS)}

\subsection{Linking}
The new version of SUPES is the only one that is available for this system.
It resides in the directory \verb+/usr/local/lib+
with the file name \verb+libsupes.a+

In what follows,
an example of how the SUPES routines can be linked to an application program
is given:
\begin{verbatim}

% cf77 -o your-executable your-source.f -lsupes.
\end{verbatim}

\subsection{Defining Unit/File or Symbol/Value for EXNAME}
A file name is connected to a unit number via an environment
variable of the form FOR0nn,
where ``nn'' is a two digit integer indicating the FORTRAN unit number. For
example, if the user is currently running under the shell program
\verb+/bin/csh+, the required sequence is:
\begin{verbatim}

% setenv FOR007 cards.dat

\end{verbatim}
This causes the following FORTRAN statements to open '\verb+cards.dat+' on unit 7.
\begin{verbatim}
CALL EXNAME( 7, FILENM, LN )
IF( LN .EQ. 0 ) THEN       ! EXNAME returns a zero for LN if no ASSIGN
                           ! has been performed.  Use the system default.
  OPEN( 7 )
ELSE                       ! I've found an ASSIGN'd filename, use it.
  OPEN( 7, FILE=FILENM )
ENDIF
\end{verbatim}

From the Bourne Shell,
\verb+/bin/sh+,
the following sequence is required:
\begin{verbatim}

$ FOR007=cards.dat
$ export TERM

\end{verbatim}
If no file has been assigned,
a system default file name of the form \verb+fort.nn+, where ``\verb+nn+''
is a one (if less than
ten) or two digit integer indicating the FORTRAN unit number that will be written.

Similarly,
EXNAME looks for an environment variable of the form
EXTnn.
So that
\begin{verbatim}

% setenv EXT05 hello

\end{verbatim}
will cause the following call to return NAME=``hello'' and LN=5.
\begin{verbatim}

CALL EXNAME( -1, NAME, LN )
\end{verbatim}

\subsection{Interface to EXREAD}
EXREAD will read from \verb+stdin+ and automatically echo to \verb+stdout+.

\subsection{Known Problems}
The Cray running UNICOS appears to have some compiler specific problems
when linking programs of differing levels of optimization.
To alleviate this situation,
two versions of the SUPES library are maintained in \verb+/usr/local/lib+,
\verb+libsupes.a+ and
\verb+libsupesnopt.a+.
The user will be responsible for linking to the appropriate library.

\subsection{Source Code}
The source code for the extension library for the UNICOS operating
system is stored in the SNLA Central File System under node
``/SUPES2\_1/EXT\_LIB.STX''
in SNLA Standard Text format.

\cleardoublepage

\section{Site Supplement For SNLA CRAY X-MP/24 (CTSS/CFTLIB 1.11 or 1.14)}

\subsection{Linking}
The old SUPES package is all that is currently available on this system.
It is accessed on the SNLA CRAY X-MP/24 as an object library
which is stored in a public library file.  Two versions of this object
library exists: one for the CFT 1.11 compiler, and one for the CFT 1.14
compiler.  The CFT 1.11 object library is obtained interactively as follows:
\begin{verbatim}
lib acclib
ok. x supes11
ok. end
switch supes11 supes
\end{verbatim}

Either compiler version can also be obtained within a CCL procedure.  For
example, the CFT 1.14 object library can be extracted by:
\begin{verbatim}
lib acclib
-x supes14
-end
switch supes14 supes
\end{verbatim}

The SUPES routines are then linked to an application program as follows:
\begin{verbatim}

ldr other_options,lib=(supes,other_libraries)

\end{verbatim}
Note that CFTLIB is a dependent library of SUPES, so there is no need to
specify cftlib in the above lib list.


\subsection{Defining Unit/File or Symbol/Value for EXNAME}
A file name is connected to a unit number via a name of the form tapenn,
where ``nn'' is a one (if less than ten) or two digit integer indicating the
FORTRAN unit number. This name can be replaced via the execution line as
shown in the following example:

\verb+myprog tape7=cards+

The above command would cause the following FORTRAN statements within
``myprog'' to open ``cards'' on unit 7:
\begin{verbatim}
CALL EXNAME( 7, NAME, LN )
OPEN( 7, FILE=NAME(1:LN) )
\end{verbatim}

EXNAME looks for a symbol on the execution line of the form extn, where
``n'' is a one digit integer which defines a symbol number.  For example: 
\begin{verbatim}

myprog ext1=HELLO

\end{verbatim}
will cause the following call within 'myprog' to return NAME=``HELLO'' and
LN=5.
\begin{verbatim}

CALL EXNAME( -1, NAME, LN )
\end{verbatim}

\subsection{Interface to EXREAD}
EXREAD will read from ``input'' and automatically echo to ``output''.  By
default, EXREAD connects both ``input'' and ``output'' to ``tty''.  CTSS defines
``tty'' as the next higher level controller, which is normally the terminal
keyboard / screen for an interactive job, or the JCI / log files for a batch
job.  An end-of-file from the terminal keyboard is indicated by a null
response (just a carriage return).

The default connections for either ``input'' or ``output'' can be overridden on
the execution line as follows:

\verb+myprog input=deck output=list+


\subsection{Known Problems}
Contrary to the ANSI FORTRAN standard, CTSS does not automatically open the
standard input and output devices.  This causes reading from or writing to
UNIT=\last to fail unless you add some CTSS-specific code, such as a PROGRAM
statement argument list.  EXNAME and EXPARM, as well as EXREAD, explicitly
open the standard input and output devices according to the rules described
above.  This is an advantage to the applications programmer since it avoids
nonstandard code, but it places the following restrictions on any program
which calls EXNAME, EXPARM, or EXREAD under CTSS:
\begin{enumerate}
\item Do not use a PROGRAM statement argument list.

\item Do not read from nor write to UNIT=* before a call to either EXNAME,
        EXPARM, or EXREAD.
\end{enumerate}

\subsection{Source Code}
The source code for the FORTRAN extension library for the CTSS/CFTLIB/SNLA
operating system is stored in the SNLA Central File System under nodes
``/SUPES/VMS/EXT\_111.STX'' and ``/SUPES/VMS/EXT\_114.STX'' in SNLA Standard Text
format for the CFT 1.11 and 1.14 compilers, respectively.

\section{Site Supplement for SNLA Alliant FX/8 (Concentrix 5.0.0)}

\subsection{Linking}
The new version of SUPES is the only one that is available for this system.
It resides in the directory \verb+/usr/local/lib+
with the file name \verb+libsupes.a+

In what follows,
an example of how the SUPES routines can be linked to an application program
is given:
\begin{verbatim}

% fortran -o your-executable your-source.f -lsupes.
\end{verbatim}

\subsection{Defining Unit/File or Symbol/Value for EXNAME}
A file name is connected to a unit number via an environment
variable of the form FOR0nn,
where ``nn'' is a two digit integer indicating the FORTRAN unit number. For
example, if the user is currently running under the shell program
\verb+/bin/csh+, the required sequence is:
\begin{verbatim}

% setenv FOR007 cards.dat

\end{verbatim}
This causes the following FORTRAN statements to open '\verb+cards.dat+' on unit 7.
\begin{verbatim}

CALL EXNAME( 7, FILENM, LN )
IF( LN .EQ. 0 ) THEN             ! EXNAME returns a zero for LN if no ASSIGN
                                 ! has been performed.  Use the system default.
  OPEN( 7 )
ELSE                             ! I've found an ASSIGN'd filename, use it.
  OPEN( 7, FILE=FILENM )
ENDIF

\end{verbatim}

From the Bourne Shell,
\verb+/bin/sh+,
the following sequence is required:
\begin{verbatim}

$ FOR007=cards.dat
$ export TERM

\end{verbatim}

If no file has been assigned,
a system default file name of the form \verb+fort.nn+, where ``\verb+nn+''
is a one (if less than
ten) or two digit integer indicating the FORTRAN unit number that will be written.

Similarly,
EXNAME looks for an environment variable of the form
EXTnn.
So that
\begin{verbatim}

% setenv EXT05 hello

\end{verbatim}
will cause the following call to return NAME=``hello'' and LN=5.
\begin{verbatim}

CALL EXNAME( -1, NAME, LN )
\end{verbatim}

\subsection{Interface to EXREAD}
EXREAD will read from \verb+stdin+ and automatically echo to \verb+stdout+.

\subsection{Source Code}
The source code for the extension library for the Alliant
is stored in the SNLA Central File System under node
``/SUPES2\_1/EXT\_LIB.STX''
in SNLA Standard Text format.

\cleardoublepage
\section{Site Supplement for SNLA Sun Workstations (SunOS version 4)}

\subsection{Linking}
The new version of SUPES is also the only one that is available for this system.
Note that the SUPES installation must have been performed according to 
the installation instructions~\ref{sec:install}.
If so,
then it resides in the directory \verb+/usr/local/lib+
with the file name \verb+libsupes.a+

In what follows,
an example of how the SUPES routines can be linked to an application program
is given:
\begin{verbatim}

% f77 -o your-executable your-source.f -lsupes.
\end{verbatim}

\subsection{Defining Unit/File or Symbol/Value for EXNAME}
A file name is connected to a unit number via an environment
variable of the form FOR0nn,
where ``nn'' is a two digit integer indicating the FORTRAN unit number. For
example, if the user is currently running under the shell program
\verb+/bin/csh+, the required sequence is:
\begin{verbatim}

% setenv FOR007 cards.dat

\end{verbatim}
This causes the following FORTRAN statements to open '\verb+cards.dat+' on unit 7.
\begin{verbatim}

CALL EXNAME( 7, FILENM, LN )
IF( LN .EQ. 0 ) THEN             ! EXNAME returns a zero for LN if no ASSIGN
                                 ! has been performed.  Use the system default.
  OPEN( 7 )
ELSE                             ! I've found an ASSIGN'd filename, use it.
  OPEN( 7, FILE=FILENM )
ENDIF

\end{verbatim}

From the Bourne Shell,
\verb+/bin/sh+,
the following sequence is required:
\begin{verbatim}

$ FOR007=cards.dat
$ export TERM

\end{verbatim}
If no file has been assigned,
a system default file name of the form \verb+fort.nn+, where ``\verb+nn+''
is a one (if less than
ten) or two digit integer indicating the FORTRAN unit number that will be written.

Similarly,
EXNAME looks for an environment variable of the form
EXTnn.
So that
\begin{verbatim}

% setenv EXT05 hello

\end{verbatim}
will cause the following call to return NAME=``hello'' and LN=5.
\begin{verbatim}

CALL EXNAME( -1, NAME, LN )
\end{verbatim}

\subsection{Interface to EXREAD}
EXREAD will read from \verb+stdin+ and automatically echo to \verb+stdout+.

\subsection{Source Code}
The source code for the extension library for the Sun
is stored in the SNLA Central File System under node
``/SUPES2\_1/EXT\_LIB.STX''
in SNLA Standard Text format.

\cleardoublepage


\section{Execution Files} \label{exec:files}

The table below summarizes the files used by \numbers.

\begin{center} \begin{tabular}{||l|c|c|c||}
\hline
\multicolumn{1}{||c}{Description} &
\multicolumn{1}{|c}{Unit Number} &
\multicolumn{1}{|c}{Type} &
\multicolumn{1}{|c||}{File Format} \\
\hline
     User input & standard input  & input  & Section~\ref{c:commands} \\
    User output & standard output & output & ASCII \\
Output file     &               7 & output & ASCII \\
EXODUS database &               9 & input  & \\
\hline
\end{tabular} \end{center}

All files must be connected to the appropriate unit before
\numbers\ is run. Each file (except standard input and output) is
opened with the name retrieved by the \caps{EXNAME} routine of the
\caps{SUPES} library \cite{supes}.

\section{Special Software}

\numbers\ is written in ANSI \caps{FORTRAN}-77.
\numbers\ uses the following software package:
\setlength{\itemsep}{\medskipamount} \begin{itemize}
\item the \caps{SUPES} package \cite{supes} which includes dynamic
memory allocation, a free-field reader, and \caps{\caps{FORTRAN}}
extensions.
\end{itemize}
