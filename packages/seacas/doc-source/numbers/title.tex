%This turns on the SAND report covers
\newif\ifdraft\draftfalse
\newif\ifsand\sandtrue

\SANDnum{SAND88-0737}
\SANDauthor{Gregory D. Sjaardema}
\SANDprintDate{\today}
\newcommand{\theTitle}{{\bold\sf NUMBERS}:\\
       A Collection of Utilities \\ 
       for Pre- and Postprocessing \\
       Two- and Three-Dimensional \\ 
       EXODUS Finite Element Models}
\title{\theTitle}
\ifsand
\pdfbookmark[1]{Cover}{cover}
\doCover
\newpage
\else
\SANDmarks{cover}
\setcounter{page}{3}
\fi

\pdfbookmark[1]{Title}{title}

%\begin{titlepage}
\begin{center}
\SANDnumVar\\
\SANDreleaseTypeVar\\
\ifdraft
Draft Date: \SANDprintDateVar\\
\else
Printed \SANDprintDateVar\\
\fi

\vspace{0.65in}
\CoverFont{m}{24}{28pt}
\theTitle\\
\vspace{0.65in}
\CoverFont{m}{12}{14pt}
\SANDauthorVar\\
Simulation Modeling Sciences Department\\
Sandia National Laboratories\\
Albuquerque, NM 87185-0380\\
\vspace*{.5in}
\textbf{Abstract}
\end{center}
\numbers\ is an application which reads and stores data from a finite
element model described in the \exo\ database format. Within this shell
program are several utility routines which generate information about
the finite element model.  The utilities currently implemented in
\numbers\ allow the analyst to determine information such as (1)~the
volume and coordinate limits of each of the materials in the model;
(2)~the mass properties of the model; (3)~the minimum, maximum, and
average element volumes for each material; (4)~the volume and change in
volume of a cavity; (5)~the nodes or elements that are within a
specified distance from a user-defined point, line, or plane; (6)~an
estimate of the explicit central-difference timestep for each material;
(7)~the validity of contact surfaces or slidelines, that is, whether two
surfaces overlap at any point; and (8)~the distance between two
surfaces. 
