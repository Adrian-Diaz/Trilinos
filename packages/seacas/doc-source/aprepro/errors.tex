\chapter{Error, Warning, and Informational Messages}\label{ch:errors}

Several error, warning, and informational messages will be printed by
\aprepro{} if certain conditions are encountered during the parsing of an
input file. The messages are of the form:

\begin{apinp}
Aprepro:  Type:  Message  (file,  line  line\#)
\end{apinp}

Where \cmd{Type} is \cmd{ERROR} for an error message, \cmd{WARN} for a
warning message, or \cmd{INFO} for an informational message; \cmd{Message}
is an explanation of the problem, \cmd{file} is the filename being processed
at the time of the message, and \cmd{line\#} is the number of the line
within that file. Error messages are always output, Warning messages
are output by default and can be turned off by using the \cmd{-W} or
\cmd{--warning} command option, and Informational messages are turned off by
default and can be turned on by using the \cmd{-M} or \cmd{--message} command
option. (See Chapter~ref{ch:execution}.)

\section{Error Messages}

\begin{description}
\item[Aprepro:  ERROR:  parse  error  (file,  line  line\#)]    
An unrecognized or ill-formed expression has been entered. Parsing of
the file continues following this expression.

\item[Aprepro: ERROR: Can't open 'file': No such file or directory] The file specified 
in the include command cannot be found or does not exist. Aprepro will terminate 
processing following this error message.

\item[Aprepro: ERROR: Can't open 'file': Permission denied] The file specified in the 
include or output command could not be opened due to insufficient permission. 
Aprepro will terminate processing following this error message.

\item[Aprepro: ERROR: Improperly Nested ifdef/ifndef statements (file, line line\#)] 
An invalid ifdef/ifndef block has been detected. Typically this is caused by an 
extra endif or else statement.

\item[Aprepro:  ERROR:    Zero  divisor  (file,  line  line\#)]    An expression tried
to divide by zero. The expression is given the value of the dividend and parsing continues.

\item[Aprepro: ERROR: Invalid units system type. Valid types are: 'si', 'cgs', 'cgs-ev', 'shock', 'swap', 'ft-lbf-s', 'ft-lbm-s', 'in-lbf-s'] 
The units system specified in the command could not be found. This
is most likely due to a misspelling of the units system name.

\item[Aprepro: ERROR: function (file, line line\#) DOMAIN error: Argument out of domain]
The arithmetic function function has been passed an invalid
argument. For example, the above error would be printed for each of
the expressions:
\begin{apinp}
\{sqrt(-1.0)\}  \{log(0.0)\}  \{asin(1.1)\}
\end{apinp}

since the arguments are out of the valid domain for the function. The
value returned by the function following an error is
system-dependent. See the function's man page on your system for more
information.


\end{description}

\section{Warning Messages}

\begin{description}
\item[Aprepro:  WARN:  Undefined  variable  'variable'  (file,  line  line\#)]
A variable is used in an expression before it has been defined. The variable is 
set equal to zero or the null string (\texttt{"}\texttt{"}) and parsing continues.

\item[Aprepro:  WARN:  Variable  'variable'  redefined  (file,  line  line\#)]
A previously defined variable is being set equal to a new value.

\item[Aprepro: (IMMUTABLE) Variable 'variable' is immutable and cannot be modified (file, line line\#)]
The value of a variable that was created as an immutable variable was
modified. No value will be returned by the expression. See page~\pageref{immutable} and
page~\ref{immutable_block} for a description of immutable variables.
\end{description}


\section{Informational Messages}

\begin{description}
\item[Aprepro: INFO: Included File: 'filename' (file, line line\#)] The file filename 
is being included at line line\# of file file. This message will also be printed 
during the execution of a loop block since temporary files are used to implement 
the looping function, and during the execution of the units conversion and material 
database access routines.
\end{description}
