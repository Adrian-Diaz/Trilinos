\documentclass[10pt]{amsart}

\usepackage{fullpage}
%\usepackage{amsmath}
%\usepackage{amsthm}
%\usepackage{graphicx}
%\usepackage{url}
%\usepackage{subfigure}
%\usepackage[foot]{amsaddr}

\date{}
\title{CSAR Proposal: Robust Hybrid Solvers for Sparse Linear Systems with Application to Structural Mechanics and Circuit Simulation}

\pagestyle{empty}
\setlength{\topmargin}{-0.25in}
\setlength{\headheight}{0in}
\setlength{\headsep}{0in}
\advance\textheight by 1.0in 
%\advance\textwidth by 0.6in 
%\advance\oddsidemargin by -0.3in
%\advance\evensidemargin by -0.3in


\begin{document}

\maketitle

%\vspace{-8mm}
\paragraph{\bf Introduction:}
We propose to develop parallel algorithms and software for robust solvers
of sparse linear systems. We seek to address two issues that
are not adequately addressed by current solver packages:
(i) high performance on manycore and future exascale architechtectures, and
(ii) robustness for ill-conditioned problems.

As parallelism in a single node increases, NW applications
have to adapt to a hybrid system where each compute node by itself
is a shared memory system (likely NUMA) with dozens of cores. 
In order to address the new challenges and opportunities 
for robust sparse linear solvers on the node,
we have developed a hybrid sparse solver ShyLU (initially funded by 
Office of Science). ShyLU is hybrid in the
mathematical sense (direct and iterative methods) making it more
robust than other preconditioners, while being less expensive
than a direct factorization.  ShyLU is also hybrid in the parallel
computing sense (MPI and threads). We envision ShyLU as a scalable
subdomain solver for large problems and/or as a standalone black-box solver
for medium sized problems.

ShyLU is based on a general Schur complement framework, with different options
for partitioning the matrices, block-diagonal solves and Schur complement
approximations.  Our hybrid implementation scales better than
a flat MPI implementation as the problem size per subdomain gets smaller, 
which is important for strong scaling in applications.
%The MPI + threads implementation helps ShyLU to scale well for up to $384$ cores.
Our first target application is circuit simulation (Xyce, POC: H. Thornquist). 
There is an acute lack of robust iterative methods in this area, so sparse
direct methods are often used. The most successful iterative methods
for circuits are based on Schur complements but no such parallel solver
is currently available in Trilinos or other DOE solver libraries.
%
Other potential applications are structural mechanics, where ShyLU
may be used as a subdomain solver within FETI (POC: K. Pierson) or
domain decomposition methods like GDSW (POC: G. Reese, C. Dohrmann). 
We also anticipate that ShyLU
will be useful as a smoother within ML/MueLu (POC: J. Hu, C. Siefert).
Tramonto?

\paragraph{\bf Proposed Work:}
ShyLU is currently based on Trilinos, in particular Epetra.
We will use Trilinos as a development platform and delivery vehicle.
ShyLU's robustness and scalability depends on good
approximations to the Schur complement. Currently, ShyLU uses 
structure-based probing
or a values-based dropping method. The dropping approximation, for
example, was added for circuit simulation problems. However, for problems
with a nice stucture to the Schur complement, from applications like Tramanto,
a probing based method will be more appropriate. We will develop better
approximation to Schur complement based on the needs of our applications, for
example we know indefinite matrices require a better approximation than either
of the two approaches in ShyLU.  

\begin{description}
\item[Year 1:] 
Release of ShyLU as stable code in Trilinos.
Make available as smoother for ML.
Evaluate performance on solid mechanics test problems from 
1500 collaborators.
Integrate as (optional) solver within Xyce. 
\item[Year 2:] 
Improve ShyLU algorithm for structurally nonsymmetric problems.
Improve robustness for circuit and mechanics problems, including 
indefinite problems.
\item[Year 3:] 
Tpetra-based version of ShyLU, compatible with 2nd generation Trilinos.
Support for complex arithmetic (to help GDSW).
Make available as smoother for MueLu.
\end{description}


\paragraph{\bf Personnel and Budget:}
Erik Boman, PI (1426): 0.3 FTE\\
David Day (1442): 0.2 FTE\\
Siva Rajamanickam (1426): 0.3 FTE\\
Heidi Thornquist (1445): 0.3 FTE\\

Total budget request: XXX FY12, YYY FY13, ZZZ FY14.

\end{document}
