%
% Command to put backslashes in front of underscores
%             :1,$s/\([^\\]\)\_/\1\\_/g
%    actually this changes a couple of lines within math symbols
%    like $P_t$.
%
\section{\ML\ Functions \label{subroutines}}

%%%%%%%%%%%%%%%%%%%%%%%%%%%%%%%%%%%%%%%%%%%%%%%%%%%%%%%%%%%%%%%%%%%%%%%%%%%%%%%

\addcontentsline{toc}{subsection}{AZ\_ML\_Set\_Amat}
\protobox{int AZ\_ML\_Set\_Amat(ML *ml\_object, int k, int isize, int osize, AZ\_MATRIX *Amat,\\
 \phantom{int AZ\_ML\_Set\_Amat(}int *proc\_config)}

\vspace{2em}
{\flushleft{\bf Description} \hrulefill}
\vspace{1em}

Create an \ML\  matrix view of an existing \Aztec matrix and store it within the `ml\_object'
context.


\vspace{2em}
{\flushleft{\bf Parameters} \hrulefill}
\vspace{1em}

\optionbox{ml\_object}{On input, \ML\  object pointer (see ML\_Create). On output, the
                      discretization matrix of level k is the same as given by Amat.}

\optionbox{k}{On input, indicates level within ml\_object hierarchy (should be between
                  0 and Nlevels${}^\dagger$-1).}

\optionbox{isize}{On input, the number of local rows in the submatrix stored on this
                  processor.}

\optionbox{osize}{On input, the number of columns in the local submatrix stored on this
                       processor not including any columns associated with ghost unknowns.}

\optionbox{Amat}{On input, an \Aztec data structure representing a matrix.
                       See the \Aztec User's Guide.}

\optionbox{proc\_config}{On input, an \Aztec data structure representing processor information.
                       See the \Aztec User's Guide.}


%%%%%%%%%%%%%%%%%%%%%%%%%%%%%%%%%%%%%%%%%%%%%%%%%%%%%%%%%%%%%%%%%%%%%%%%%%%%%%%

\addcontentsline{toc}{subsection}{AZ\_set\_ML\_preconditioner}
\protobox{void AZ\_set\_ML\_preconditioner(AZ\_PRECOND **Precond, AZ\_MATRIX *Amat, \\
 \phantom{void AZ\_set\_ML\_preconditioner(}ML *ml\_object, int options[])}

\vspace{2em}
{\flushleft{\bf Description} \hrulefill}
\vspace{1em}

Associate the multigrid V cycle method defined in ml\_object with an \Aztec preconditioner.
Thus, when Precond and options are passed into the \Aztec iterative solver, it will
invoke the V cycle multigrid algorithm described by ml\_object.

\vspace{2em}
{\flushleft{\bf Parameters} \hrulefill}
\vspace{1em}

\optionbox{Precond}{On input, an \Aztec data structure representing a preconditioner.
                    On output, the multigrid V cycle method described by ml\_object will
                    be associated with this preconditioner.  See the \Aztec User's Guide.}

\optionbox{Amat}{On input, an \Aztec data structure representing a matrix.
                 See the \Aztec User's Guide.}

\optionbox{ml\_object}{On input, \ML\  object pointer (see ML\_Create) representing a V cycle
                      multigrid method.}

\optionbox{options}{On input, an \Aztec data structure representing user chosen options.
                    On output, set appropriately for multigrid V cycle preconditioner.}

%%%%%%%%%%%%%%%%%%%%%%%%%%%%%%%%%%%%%%%%%%%%%%%%%%%%%%%%%%%%%%%%%%%%%%%%%%%%%%%

\addcontentsline{toc}{subsection}{ML\_Aggregate\_Create}
\protobox{int ML\_Aggregate\_Create(ML\_Aggregate **agg\_object)}

\vspace{2em}
{\flushleft{\bf Description} \hrulefill}
\vspace{1em}

Create an aggregate context (or handle). This instance will be used in all
subsequent function invocations that set aggregation options.

\vspace{2em}
{\flushleft{\bf Parameters} \hrulefill}
\vspace{1em}

\optionbox{agg\_object}{On input, a pointer to a noninitialized ML\_Aggregate object pointer.
               On output, points to an initialized ML\_Aggregate object pointer.}


%%%%%%%%%%%%%%%%%%%%%%%%%%%%%%%%%%%%%%%%%%%%%%%%%%%%%%%%%%%%%%%%%%%%%%%%%%%%%%%

\addcontentsline{toc}{subsection}{ML\_Aggregate\_Destroy}
\protobox{int ML\_Aggregate\_Destroy(ML\_Aggregate **agg\_object)}

\vspace{2em}
{\flushleft{\bf Description} \hrulefill}
\vspace{1em}

Destroy the aggregate context, agg\_object, and delete all memory allocated by \ML\  in
building and setting the aggregation options.

\vspace{2em}
{\flushleft{\bf Parameters} \hrulefill}
\vspace{1em}

\optionbox{agg\_object}{On input, aggregate object pointer (see ML\_Aggregate\_Create). On output,
               all memory allocated by \ML\  and associated with this context is freed.}

%%%%%%%%%%%%%%%%%%%%%%%%%%%%%%%%%%%%%%%%%%%%%%%%%%%%%%%%%%%%%%%%%%%%%%%%%%%%%%%

\addcontentsline{toc}{subsection}{ML\_Aggregate\_Set\_CoarsenScheme\_Coupled}
\protobox{int ML\_Aggregate\_Set\_CoarsenScheme\_Coupled(ML\_Aggregate *agg\_object)}

\vspace{2em}
{\flushleft{\bf Description} \hrulefill}
\vspace{1em}

Set the aggregate coarsening scheme to be used as `coupled' (see Section \ref{aggregation options}).

\vspace{2em}
{\flushleft{\bf Parameters} \hrulefill}
\vspace{1em}

\optionbox{agg\_object}{On input, aggregate object pointer (see ML\_Aggregate\_Create). On output,
               the `coupled' aggregation will be used for automatic coarsening.}

%%%%%%%%%%%%%%%%%%%%%%%%%%%%%%%%%%%%%%%%%%%%%%%%%%%%%%%%%%%%%%%%%%%%%%%%%%%%%%%

\addcontentsline{toc}{subsection}{ML\_Aggregate\_Set\_CoarsenScheme\_MIS}
\protobox{int ML\_Aggregate\_Set\_CoarsenScheme\_MIS(ML\_Aggregate *agg\_object)}

\vspace{2em}
{\flushleft{\bf Description} \hrulefill}
\vspace{1em}

Set the aggregate coarsening scheme to be used as `MIS' (see Section \ref{aggregation options}).

\vspace{2em}
{\flushleft{\bf Parameters} \hrulefill}
\vspace{1em}

\optionbox{agg\_object}{On input, aggregate object pointer (see ML\_Aggregate\_Create). On output,
               the `MIS' aggregation will be used for automatic coarsening.}


%%%%%%%%%%%%%%%%%%%%%%%%%%%%%%%%%%%%%%%%%%%%%%%%%%%%%%%%%%%%%%%%%%%%%%%%%%%%%%%

\addcontentsline{toc}{subsection}{ML\_Aggregate\_Set\_CoarsenScheme\_Uncoupled}
\protobox{int ML\_Aggregate\_Set\_CoarsenScheme\_Uncoupled(ML\_Aggregate *agg\_object)}


\vspace{2em}
{\flushleft{\bf Description} \hrulefill}
\vspace{1em}

Set the aggregate coarsening scheme to be used as `uncoupled' (see Section \ref{aggregation options}).

\vspace{2em}
{\flushleft{\bf Parameters} \hrulefill}
\vspace{1em}

\optionbox{agg\_object}{On input, aggregate object pointer (see ML\_Aggregate\_Create). On output,
               the `uncoupled' aggregation will be used for automatic coarsening.}


%%%%%%%%%%%%%%%%%%%%%%%%%%%%%%%%%%%%%%%%%%%%%%%%%%%%%%%%%%%%%%%%%%%%%%%%%%%%%%%

\addcontentsline{toc}{subsection}{ML\_Aggregate\_Set\_CoarsenScheme\_METIS}
\protobox{int ML\_Aggregate\_Set\_CoarsenScheme\_METIS(ML\_Aggregate *agg\_object)}

\vspace{2em}
{\flushleft{\bf Description} \hrulefill}
\vspace{1em}

Set the aggregate coarsening scheme to be used as `METIS (see Section \ref{aggregation options}).

\vspace{2em}
{\flushleft{\bf Parameters} \hrulefill}
\vspace{1em}

\optionbox{agg\_object}{On input, aggregate object pointer (see ML\_Aggregate\_Create). On output,
               the `METIS' aggregation will be used for automatic coarsening.}

%%%%%%%%%%%%%%%%%%%%%%%%%%%%%%%%%%%%%%%%%%%%%%%%%%%%%%%%%%%%%%%%%%%%%%%%%%%%%%%

\addcontentsline{toc}{subsection}{ML\_Aggregate\_Set\_CoarsenScheme\_ParMETIS}
\protobox{int ML\_Aggregate\_Set\_CoarsenScheme\_ParMETIS(ML\_Aggregate *agg\_object)}

\vspace{2em}
{\flushleft{\bf Description} \hrulefill}
\vspace{1em}

Set the aggregate coarsening scheme to be used as `ParMETIS (see Section \ref{aggregation options}).

\vspace{2em}
{\flushleft{\bf Parameters} \hrulefill}
\vspace{1em}

\optionbox{agg\_object}{On input, aggregate object pointer (see ML\_Aggregate\_Create). On output,
               the `ParMETIS' aggregation will be used for automatic coarsening.}

%%%%%%%%%%%%%%%%%%%%%%%%%%%%%%%%%%%%%%%%%%%%%%%%%%%%%%%%%%%%%%%%%%%%%%%%%%%%%%%

\addcontentsline{toc}{subsection}{ML\_Aggregate\_Set\_DampingFactor}
\protobox{int ML\_Aggregate\_Set\_DampingFactor( ML\_Aggregate *ag, double factor)}

\vspace{2em}
{\flushleft{\bf Description} \hrulefill}
\vspace{1em}

Set the damping factor used within smoothed aggregation. In particular,
the interpolation operator will be generated by 
$$
P = (I - \frac{\omega}{\tilde{\rho}} A ) P_t
$$
where $A$ is the discretation matrix, $\omega$ is the damping factor (default 
is $\frac{4}{3}$), $\rho$ is an estimate of the spectral radius of $A$, and 
$P_t$ are the seed vectors (tentative prolongator).

\vspace{2em}
{\flushleft{\bf Parameters} \hrulefill}
\vspace{1em}

\optionbox{agg\_object}{On input, aggregate object pointer (see 
                        ML\_Aggregate\_Create). On output,
                        the damping factor is set to factor.}

\optionbox{factor}{On input, damping factor that will be associated with
                   this aggregation object.}

%%%%%%%%%%%%%%%%%%%%%%%%%%%%%%%%%%%%%%%%%%%%%%%%%%%%%%%%%%%%%%%%%%%%%%%%%%%%%%%

\addcontentsline{toc}{subsection}{ML\_Aggregate\_Set\_MaxCoarseSize}


\protobox{int ML\_Aggregate\_Set\_MaxCoarseSize( ML\_Aggregate *agg\_object, int size  )}

\vspace{2em}
{\flushleft{\bf Description} \hrulefill}
\vspace{1em}

Set the maximum coarsest mesh to `size'. No further coarsening is performed if the total 
number of matrix equations is less than this `size' (see Section \ref{multigrid options}).

\vspace{2em}
{\flushleft{\bf Parameters} \hrulefill}
\vspace{1em}

\optionbox{agg\_object}{On input, aggregate object pointer (see ML\_Aggregate\_Create). On output,
               the coarsest mesh size will be set.}

\optionbox{size}{On input, size indicating the maximum coarsest mesh size.}

%%%%%%%%%%%%%%%%%%%%%%%%%%%%%%%%%%%%%%%%%%%%%%%%%%%%%%%%%%%%%%%%%%%%%%%%%%%%%%%
%
%\addcontentsline{toc}{subsection}{ML\_Aggregate\_Set\_MaxLevels}
%
%
%\protobox{int ML\_Aggregate\_Set\_MaxLevels( ML\_Aggregate *agg\_object, int level ) }
%
%\vspace{2em}
%{\flushleft{\bf Description} \hrulefill}
%\vspace{1em}
%
%Set the total number of mesh levels to `level'. This means that the multigrid method will
%not have more then `level' levels.
%
%\vspace{2em}
%{\flushleft{\bf Parameters} \hrulefill}
%\vspace{1em}
%
%\optionbox{agg\_object}{On input, aggregate object pointer (see ML\_Aggregate\_Create). On output,
%               the maximum number of levels will be set.}
%
%\optionbox{level}{On input, size indicating the maximum number of multigrid levels.}
%
%%%%%%%%%%%%%%%%%%%%%%%%%%%%%%%%%%%%%%%%%%%%%%%%%%%%%%%%%%%%%%%%%%%%%%%%%%%%%%%

\addcontentsline{toc}{subsection}{ML\_Aggregate\_Set\_NullSpace}
\protobox{int ML\_Aggregate\_Set\_NullSpace(ML\_Aggregate *agg\_object, int num\_PDE\_eqns,
                               int null\_dim,\\
 \phantom{int ML\_Aggregate\_Set\_NullSpace(}double *null\_vect, int leng)}

\vspace{2em}
{\flushleft{\bf Description} \hrulefill}
\vspace{1em}

Set the seed vectors (rigid body mode vectors) to be used in smoothed aggregation.
Also indicate the number of degrees of freedom (DOF) per node so that the
aggregation algorithm can group them together.

\vspace{2em}
{\flushleft{\bf Parameters} \hrulefill}
\vspace{1em}

\optionbox{agg\_object}{On input, an ML\_Aggregate object pointer created by invoking ML\_Aggregate\_Create. On
               output, the seed vectors and DOFs per node are set to null\_vect and
               num\_PDE\_eqns respectively.}

\optionbox{num\_PDE\_eqns}{On input, indicates number of equations that should be grouped in blocks
                       when performing the aggregation. This guarantees that different DOFs
                       at a grid point remain within the same aggregate.}

\optionbox{null\_dim}{On input, number of seed vectors that will be used when creating the
                       smoothed aggregation grid transfer operator.}

\optionbox{null\_vect}{On input, the seed vectors are given in sequence. Each processor
                       gives only the local components residing on the processor. If null,
                       default seed vectors are used.}

\optionbox{leng}{On input, the length of each seed vector.}


%%%%%%%%%%%%%%%%%%%%%%%%%%%%%%%%%%%%%%%%%%%%%%%%%%%%%%%%%%%%%%%%%%%%%%%%%%%%%%%

\addcontentsline{toc}{subsection}{ML\_Aggregate\_Set\_SpectralNormScheme\_Calc}
\protobox{int ML\_Aggregate\_Set\_SpectralNormScheme\_Calc( ML\_Aggregate *ag )}

\vspace{2em}
{\flushleft{\bf Description} \hrulefill}
\vspace{1em}

Set the method to be used for estimating the spectral radius of $A$
(the discretization matrix) to be
conjugate gradient. This spectral radius estimate is used when smoothing
the initial prolongation operator  (see ML\_Aggregate\_Set\_DampingFactor).

\vspace{2em}
{\flushleft{\bf Parameters} \hrulefill}
\vspace{1em}

\optionbox{agg\_object}{On input, aggregate object pointer (see 
                        ML\_Aggregate\_Create). On output,
                        the spectral radius estimate will be determined by a 
                        conjugate gradient routine.}

%%%%%%%%%%%%%%%%%%%%%%%%%%%%%%%%%%%%%%%%%%%%%%%%%%%%%%%%%%%%%%%%%%%%%%%%%%%%%%%

\addcontentsline{toc}{subsection}{ML\_Aggregate\_Set\_SpectralNormScheme\_Anorm}
\protobox{int ML\_Aggregate\_Set\_SpectralNormScheme\_Anorm( ML\_Aggregate *ag)}

\vspace{2em}
{\flushleft{\bf Description} \hrulefill}
\vspace{1em}

Set the method to be used for estimating the spectral radius of $A$ 
(the discretization matrix) to be
the infinity norm. This spectral radius estimate is used when smoothing
the initial prolongation operator  (see ML\_Aggregate\_Set\_DampingFactor).

\vspace{2em}
{\flushleft{\bf Parameters} \hrulefill}
\vspace{1em}

\optionbox{agg\_object}{On input, aggregate object pointer (see 
                        ML\_Aggregate\_Create). On output,
                        the spectral radius estimate will be taken as
                        the infinity norm of the matrix.}

%%%%%%%%%%%%%%%%%%%%%%%%%%%%%%%%%%%%%%%%%%%%%%%%%%%%%%%%%%%%%%%%%%%%%%%%%%%%%%%

\addcontentsline{toc}{subsection}{ML\_Aggregate\_Set\_Threshold}
\protobox{int ML\_Aggregate\_Set\_Threshold(ML\_Aggregate *agg\_object, double tolerance)}

\vspace{2em}
{\flushleft{\bf Description} \hrulefill}
\vspace{1em}

Set the drop tolerance used when creating the matrix graph for aggregation. Entries in the
matrix $A$ are dropped when $ | A(i,j) | \le tol\_d * \sqrt{ | A(i,i) A(j,j) | } $. See
Section \ref{aggregation options} for more details.

\vspace{2em}
{\flushleft{\bf Parameters} \hrulefill}
\vspace{1em}

\optionbox{agg\_object}{On input, an ML\_Aggregate object pointer created by invoking
               ML\_Aggregate\_Create. On
               output, drop tolerance for creating the matrix graph is set.}

\optionbox{tolerance}{On input, value to be used for dropping matrix entries.}

%%%%%%%%%%%%%%%%%%%%%%%%%%%%%%%%%%%%%%%%%%%%%%%%%%%%%%%%%%%%%%%%%%%%%%%%%%%%%%%

\addcontentsline{toc}{subsection}{ML\_Create}
\protobox{int ML\_Create(ML **ml\_object, int Nlevels)}

\vspace{2em}
{\flushleft{\bf Description} \hrulefill}
\vspace{1em}

Create an \ML\  solver context (or handle). This \ML\  instance will be used in all
subsequent \ML\  function invocations. The \ML\  object has a notation of levels where
different multigrid operators corresponding to different grid levels are stored.

\vspace{2em}
{\flushleft{\bf Parameters} \hrulefill}
\vspace{1em}


\optionbox{ml\_object}{On input, a pointer to a noninitialized \ML\  object pointer. On output,
               points to an initialized \ML\  object pointer.}
\optionbox{Nlevels}{Maximum number of multigrid levels within this \ML\  object.}

%%%%%%%%%%%%%%%%%%%%%%%%%%%%%%%%%%%%%%%%%%%%%%%%%%%%%%%%%%%%%%%%%%%%%%%%%%%%%%%

\addcontentsline{toc}{subsection}{ML\_Destroy}
\protobox{int ML\_Destroy(ML **ml\_object)}

\vspace{2em}
{\flushleft{\bf Description} \hrulefill}
\vspace{1em}

Destroy the \ML\  solver context, ml\_object, and delete all memory allocated by \ML\  in
building and setting options.

\vspace{2em}
{\flushleft{\bf Parameters} \hrulefill}
\vspace{1em}

\optionbox{ml\_object}{On input, \ML\  object pointer (see ML\_Create). On output, all memory
                   allocated by \ML\  and associated with this context is freed.}

%%%%%%%%%%%%%%%%%%%%%%%%%%%%%%%%%%%%%%%%%%%%%%%%%%%%%%%%%%%%%%%%%%%%%%%%%%%%%%%

\addcontentsline{toc}{subsection}{ML\_Gen\_Blocks\_Aggregates}
\protobox{int ML\_Gen\_Blocks\_Aggregates(ML\_Aggregate *agg\_object, 
          int k, int *nblocks, int **block\_list)}

\vspace{2em}
{\flushleft{\bf Description} \hrulefill}
\vspace{1em}

Use aggregates to partition submatrix residing on local processor into 
blocks. These blocks can then be used within smoothers (see for example
ML\_Gen\_Smoother\_VBlockJacobi or ML\_Gen\_Smoother\_VBlockSymGaussSeidel).

\vspace{2em}
{\flushleft{\bf Parameters} \hrulefill}
\vspace{1em}

\optionbox{ml\_object}{On input, \ML\  object pointer (see ML\_Create).}

\optionbox{k}{On input, indicates level within ml\_object hierarchy where 
              the aggregate information is found that defines partitioning.}

\optionbox{nblocks}{On output, indicates the number of partitions.}

\optionbox{block\_list}{On output, equation i resides in the
                        block\_list[i]th partition.}

%%%%%%%%%%%%%%%%%%%%%%%%%%%%%%%%%%%%%%%%%%%%%%%%%%%%%%%%%%%%%%%%%%%%%%%%%%%%%%%

\addcontentsline{toc}{subsection}{ML\_Gen\_Blocks\_Metis}
\protobox{int ML\_Gen\_Blocks\_Metis(ML *ml\_object, int k, int *nblocks,
          int **block\_list)}

\vspace{2em}
{\flushleft{\bf Description} \hrulefill}
\vspace{1em}

Use Metis to partition submatrix residing on local processor into 
blocks. These blocks can then be used within smoothers (see for example
ML\_Gen\_Smoother\_VBlockJacobi or ML\_Gen\_Smoother\_VBlockSymGaussSeidel).

\vspace{2em}
{\flushleft{\bf Parameters} \hrulefill}
\vspace{1em}

\optionbox{ml\_object}{On input, \ML\  object pointer (see ML\_Create).}

\optionbox{k}{On input, indicates level within ml\_object hierarchy where 
              the discretization matrix is found that will be partitioned.}

\optionbox{nblocks}{On input, indicates number of partitions desired on each
                    processor. On output, indicates the number of partitions
                    obtained.}

\optionbox{block\_list}{On output, equation i resides in the
                        block\_list[i]th partition.}

%%%%%%%%%%%%%%%%%%%%%%%%%%%%%%%%%%%%%%%%%%%%%%%%%%%%%%%%%%%%%%%%%%%%%%%%%%%%%%%

\addcontentsline{toc}{subsection}{ML\_Gen\_CoarseSolverSuperLU}
\protobox{int ML\_Gen\_CoarseSolverSuperLU(ML *ml\_object, int k)}

\vspace{2em}
{\flushleft{\bf Description} \hrulefill}
\vspace{1em}

Use SuperLU for the multigrid coarse grid solver on level k within ml\_object and
perform any initialization that is necessary.

\vspace{2em}
{\flushleft{\bf Parameters} \hrulefill}
\vspace{1em}

\optionbox{ml\_object}{On input, \ML\  object pointer (see ML\_Create). On output, the coarse
                      grid solver of level k is set to use SuperLU.}

\optionbox{k}{On input, indicates level within ml\_object hierarchy (must be the
                  coarsest level in the multigrid hierarchy).}

%%%%%%%%%%%%%%%%%%%%%%%%%%%%%%%%%%%%%%%%%%%%%%%%%%%%%%%%%%%%%%%%%%%%%%%%%%%%%%%

\addcontentsline{toc}{subsection}{ML\_Gen\_MGHierarchy\_UsingAggregation}
\protobox{int ML\_Gen\_MGHierarchy\_UsingAggregation(ML *ml\_object, int start, 
          int inc\_or\_dec,\\
 \phantom{int ML\_Gen\_MGHierarchy\_UsingAggregation(}ML\_Aggregate *agg\_object)}

\vspace{2em}
{\flushleft{\bf Description} \hrulefill}
\vspace{1em}

Generate a multigrid hierarchy via the method of smoothed aggregation.
This hierarchy includes a series of grid transfer operators as well as
coarse grid approximations to the fine grid discretization operator.
On completion, return the total number of multigrid levels in the newly
created hiearchy.

\vspace{2em}
{\flushleft{\bf Parameters} \hrulefill}
\vspace{1em}

\optionbox{ml\_object}{On input, \ML\  object pointer (see ML\_Create). On output,
               coarse levels are filled with grid transfer operators and
               coarse grid discretizations corresponding to a multigrid hierarchy.}
\optionbox{start}{On input, indicates multigrid level within ml\_object where the fine grid
                  discretization is stored.}
\optionbox{inc\_or\_dec}{On input, ML\_INCREMENT or ML\_DECREMENT. Normally, set to 
                ML\_INCREMENT
                       meaning that the newly created multigrid operators should be stored in
                       the multigrid levels: start, start+1, start+2, start+3, etc. If Set to
                       ML\_DECREMENT, multigrid operators are stored in start, start-1, 
                       start-2, etc.}
\optionbox{agg\_object}{                   On input, an initialized aggregation object defining options to the
                       generation of grid transfer operators. If set to NULL, default
                       values are used for all aggregation options. See ML\_Aggregate\_Create.}


%%%%%%%%%%%%%%%%%%%%%%%%%%%%%%%%%%%%%%%%%%%%%%%%%%%%%%%%%%%%%%%%%%%%%%%%%%%%%%%

\addcontentsline{toc}{subsection}{ML\_Gen\_SmootherAmesos}
\protobox{int ML\_Gen\_SmootherAmesos(ML *ml\_object, int k, int AmesosSolver,\\
 \phantom{int ML\_Gen\_SmootherAmesos(} int MaxProcs)}

\vspace{2em}
{\flushleft{\bf Description} \hrulefill}
\vspace{1em}

Use Amesos interface to direct solvers for the multigrid coarse grid
solver on level k within ml\_object and perform any initialization that
is necessary.

\vspace{2em}
{\flushleft{\bf Parameters} \hrulefill}
\vspace{1em}

\optionbox{ml\_object}{On input, \ML\  object pointer (see ML\_Create). On output, the coarse
                      grid solver of level k is set to use Amesos.}

\optionbox{k}{On input, indicates level within ml\_object hierarchy (must be the
                  coarsest level in the multigrid hierarchy).}
                
                \optionbox{AmesosSolver}{On input, indicates the direct
                  solver library to use in the coarse solution. It can
                  be: ML\_AMESOS\_UMFPACK, ML\_AMESOS\_KLU,
                  ML\_AMESOS\_SUPERLUDIST, ML\_AMESOS\_MUMPS,
                  ML\_AMESOS\_SCALAPACK.}

\optionbox{MaxProcs}{On input, indicates maximum number of processors to
  use in the coarse solution (only for ML\_AMESOS\_SUPERLUDIST).}

%%%%%%%%%%%%%%%%%%%%%%%%%%%%%%%%%%%%%%%%%%%%%%%%%%%%%%%%%%%%%%%%%%%%%%%%%%%%%%%

\addcontentsline{toc}{subsection}{ML\_Gen\_SmootherAztec}
\protobox{void ML\_Gen\_SmootherAztec(ML *ml\_object, int k, int options[], double params[],\\
 \phantom{void ML\_Gen\_SmootherAztec(}int proc\_config[], double status[], int N\_iterations,\\
 \phantom{void ML\_Gen\_SmootherAztec(}int pre\_or\_post, 
                                       void (*prec\_fun)(double *, int *, int *,\\
 \phantom{void ML\_Gen\_SmootherAztec(}double *, AZ\_MATRIX  *,
                              AZ\_PRECOND *))}


\vspace{2em}
{\flushleft{\bf Description} \hrulefill}
\vspace{1em}

Set the smoother (either pre or post as indicated by pre\_or\_post) at level k within
the multigrid solver context to invoke \Aztec. The specific \Aztec scheme is given by the
\Aztec arrays: options, params, proc\_config, and status and \Aztec preconditioning function:
prec\_function.

\vspace{2em}
{\flushleft{\bf Parameters} \hrulefill}
\vspace{1em}

\optionbox{ml\_object}{On input, \ML\  object pointer (see ML\_Create). On output, a smoother
                      function is associated within ml\_object at level k.}

\optionbox{k}{On input, indicates where the smoother function pointer will be
                  stored within the multigrid hierarchy.}

\optionbox{options, params\\ proc\_config, status}{On input, \Aztec arrays that determine the \Aztec scheme and are used
                       for \Aztec to return information. See the \Aztec User's Guide.}

\optionbox{N\_iterations}{On input, maximum \Aztec iterations
                       within a single smoother invocation. When set to
                       {\tt AZ\_ONLY\_PRECONDITIONER},
                       only one iteration of the preconditioner is used without 
                       an outer Krylov method.}
\optionbox{pre\_or\_post}{On input, ML\_PRESMOOTHER or ML\_POSTSMOOTHER indicating whether
                       the smoother should be performed before or after the coarse grid
                       correction.}

\optionbox{prec\_fun}{On input, \Aztec preconditioning function indicating what preconditioner
                       will be used within \Aztec. Normally, this is set to AZ\_precondition.
                       See the \Aztec User's Guide.}


%%%%%%%%%%%%%%%%%%%%%%%%%%%%%%%%%%%%%%%%%%%%%%%%%%%%%%%%%%%%%%%%%%%%%%%%%%%%%%%

\addcontentsline{toc}{subsection}{ML\_Gen\_Smoother\_BlockGaussSeidel}
\protobox{int ML\_Gen\_Smoother\_BlockGaussSeidel(ML *ml\_object, int k, int pre\_or\_post,
                                     int ntimes,\\
 \phantom{int ML\_Gen\_Smoother\_BlockGaussSeidel(}double omega, int blocksize)
}

\vspace{2em}
{\flushleft{\bf Description} \hrulefill}
\vspace{1em}

Set the multigrid smoother for level k of ml\_object and perform any initialization
that is necessary. When using block Gauss Seidel, the total number of equations must be
a multiple of blocksize. Each consecutive group of blocksize unknowns is grouped into
a block and a block Gauss Seidel algorithm is applied.

\vspace{2em}
{\flushleft{\bf Parameters} \hrulefill}
\vspace{1em}

\optionbox{ml\_object}{On input, \ML\  object pointer (see ML\_Create). On output, the pre or
               post smoother of level k is set to block Gauss Seidel.}

\optionbox{k}{On input, indicates level within ml\_object hierarchy (should be between
               0 and Nlevels${}^\dagger$-1). ML\_ALL\_LEVELS sets the smoothing on all levels in ml\_object.}

\optionbox{pre\_or\_post}{On input, ML\_PRESMOOTHER or ML\_POSTSMOOTHER indicating whether the
                        pre or post smoother is to be set.}

\optionbox{ntimes}{On input, sets the number of block Gauss Seidel iterations that
                   will be performed.}

\optionbox{omega}{On input, sets the damping parameter to be used during this block
                  Gauss Seidel smoothing.}

\optionbox{blocksize}{On input, sets the size of the blocks to be used during block
                      Gauss Seidel smoothing.}

%%%%%%%%%%%%%%%%%%%%%%%%%%%%%%%%%%%%%%%%%%%%%%%%%%%%%%%%%%%%%%%%%%%%%%%%%%%%%%%

\addcontentsline{toc}{subsection}{ML\_Gen\_Smoother\_GaussSeidel}
\protobox{int ML\_Gen\_Smoother\_GaussSeidel(ML *ml\_object, int k, int pre\_or\_post, 
          int ntimes,\\
 \phantom{int ML\_Gen\_Smoother\_GaussSeidel(}double omega)
}

\vspace{2em}
{\flushleft{\bf Description} \hrulefill}
\vspace{1em}

Set the multigrid smoother for level k of ml\_object and perform any initialization
that is necessary.

\vspace{2em}
{\flushleft{\bf Parameters} \hrulefill}
\vspace{1em}

\optionbox{ml\_object}{On input, \ML\  object pointer (see ML\_Create). On output, the pre or
               post smoother of level k is set to Gauss Seidel.}

\optionbox{k}{On input, indicates level within ml\_object hierarchy (should be between
               0 and Nlevels${}^\dagger$-1). ML\_ALL\_LEVELS sets the smoothing on all levels in ml\_object.}

\optionbox{pre\_or\_post}{On input, ML\_PRESMOOTHER or ML\_POSTSMOOTHER indicating whether the
                        pre or post smoother is to be set.}

\optionbox{ntimes}{On input, sets the number of Gauss Seidel iterations that will be performed.}

\optionbox{omega}{On input, sets the damping parameter to be used during this Gauss
                  Seidel smoothing.}



%%%%%%%%%%%%%%%%%%%%%%%%%%%%%%%%%%%%%%%%%%%%%%%%%%%%%%%%%%%%%%%%%%%%%%%%%%%%%%%

\addcontentsline{toc}{subsection}{ML\_Gen\_Smoother\_Jacobi}
\protobox{int ML\_Gen\_Smoother\_Jacobi(ML *ml\_object, int k, int pre\_or\_post, int ntimes,\\
 \phantom{int ML\_Gen\_Smoother\_Jacobi(}double omega)}


\vspace{2em}
{\flushleft{\bf Description} \hrulefill}
\vspace{1em}

Set the multigrid smoother for level k of ml\_object and perform any initialization
that is necessary.

\vspace{2em}
{\flushleft{\bf Parameters} \hrulefill}
\vspace{1em}

\optionbox{ml\_object}{   On input, \ML\  object pointer (see ML\_Create). On output, the pre or
                  post smoother of level k is set to Jacobi.}

\optionbox{k}{   On input, indicates level within ml\_object hierarchy (should be between
                  0 and Nlevels${}^\dagger$-1). ML\_ALL\_LEVELS sets the smoothing on all levels in ml\_object.}

\optionbox{pre\_or\_post}{On input, ML\_PRESMOOTHER or ML\_POSTSMOOTHER indicating whether the
                        pre or post smoother is to be set.}

\optionbox{ntimes}{    On input, sets the number of Jacobi iterations that will be performed.}

\optionbox{omega}{     On input, sets the damping parameter to be used during this Jacobi
                       smoothing. ML\_DEFAULT sets it to .5}

%%%%%%%%%%%%%%%%%%%%%%%%%%%%%%%%%%%%%%%%%%%%%%%%%%%%%%%%%%%%%%%%%%%%%%%%%%%%%%%

\addcontentsline{toc}{subsection}{ML\_Gen\_Smoother\_SymGaussSeidel}
\protobox{int ML\_Gen\_Smoother\_SymGaussSeidel(ML *ml\_object, int k, int pre\_or\_post,
          int ntimes,\\
 \phantom{int ML\_Gen\_Smoother\_SymGaussSeidel(}double omega)}

\vspace{2em}
{\flushleft{\bf Description} \hrulefill}
\vspace{1em}

Set the multigrid smoother for level k of ml\_object and perform any initialization
that is necessary.

\vspace{2em}
{\flushleft{\bf Parameters} \hrulefill}
\vspace{1em}

\optionbox{ml\_object}{On input, \ML\  object pointer (see ML\_Create). On output, the pre or
               post smoother of level k is set to symmetric Gauss Seidel.}

\optionbox{k}{On input, indicates level within ml\_object hierarchy (should be between
               0 and Nlevels${}^\dagger$-1). ML\_ALL\_LEVELS sets the smoothing on all levels in ml\_object.}

\optionbox{pre\_or\_post}{On input, ML\_PRESMOOTHER or ML\_POSTSMOOTHER indicating whether the
                        pre or post smoother is to be set.}

\optionbox{ntimes}{On input, sets the number of symmetric Gauss Seidel iterations that
                   will be performed.}

\optionbox{omega}{On input, sets the damping parameter to be used during this symmetric
                  Gauss Seidel smoothing.}

%%%%%%%%%%%%%%%%%%%%%%%%%%%%%%%%%%%%%%%%%%%%%%%%%%%%%%%%%%%%%%%%%%%%%%%%%%%%%%%

\addcontentsline{toc}{subsection}{ML\_Gen\_Smoother\_VBlockJacobi}
\protobox{int ML\_Gen\_Smoother\_VBlockJacobi(ML *ml\_object, int k, int pre\_or\_post,
                 int ntimes,\\
 \phantom{int ML\_Gen\_Smoother\_VBlockJacobi(}double omega, int nBlocks, int *blockIndices)}

\vspace{2em}
{\flushleft{\bf Description} \hrulefill}
\vspace{1em}

Set the multigrid smoother for level k of ml\_object and perform any initialization
that is necessary. A block Jacobi smoothing algorithm will be used where the size of the
blocks can vary and is given by nBlocks and blockIndices
(see ML\_Gen\_Blocks\_Aggregates and ML\_Gen\_Blocks\_Metis).


\vspace{2em}
{\flushleft{\bf Parameters} \hrulefill}
\vspace{1em}

\optionbox{ml\_object}{On input, \ML\  object pointer (see ML\_Create). On output, the pre or
               post smoother of level k is set to variable block Jacobi.}

\optionbox{k}{On input, indicates level within ml\_object hierarchy (should be between
               0 and Nlevels${}^\dagger$-1). ML\_ALL\_LEVELS sets the smoothing on all levels in ml\_object.}

\optionbox{pre\_or\_post}{On input, ML\_PRESMOOTHER or ML\_POSTSMOOTHER indicating whether the
                        pre or post smoother is to be set.}

\optionbox{ntimes}{On input, sets the number of block Jacobi iterations that
                   will be performed.}

\optionbox{omega}{On input, sets the damping parameter to be used during this block
                  Jacobi smoothing.}

\optionbox{nBlocks}{On input, indicates the total number of block equations in matrix.}

\optionbox{blockIndices}{On input, blockIndices[i] indicates block to which ith element belongs. }


%%%%%%%%%%%%%%%%%%%%%%%%%%%%%%%%%%%%%%%%%%%%%%%%%%%%%%%%%%%%%%%%%%%%%%%%%%%%%%%

\addcontentsline{toc}{subsection}{ML\_Gen\_Smoother\_VBlockSymGaussSeidel}
\protobox{int ML\_Gen\_Smoother\_VBlockSymGaussSeidel(ML *ml\_object, int k, 
           int pre\_or\_post,\\
 \phantom{int ML\_Gen\_Smoother\_VBlockSymGaussSeidel(}int ntimes, double omega, 
           int nBlocks,\\
 \phantom{int ML\_Gen\_Smoother\_VBlockSymGaussSeidel(}int *blockIndices)
}

\vspace{2em}
{\flushleft{\bf Description} \hrulefill}
\vspace{1em}

Set the multigrid smoother for level k of ml\_object and perform any initialization
that is necessary. A block Gauss Seidel smoothing algorithm will be used where the size of the
blocks can vary and is given by nBlocks and blockIndices
(see ML\_Gen\_Blocks\_Aggregates and ML\_Gen\_Blocks\_Metis).

\vspace{2em}
{\flushleft{\bf Parameters} \hrulefill}
\vspace{1em}

\optionbox{ml\_object}{On input, \ML\  object pointer (see ML\_Create). On output, the pre or
               post smoother of level k is set to variable block symmetric Gauss
               Seidel.}

\optionbox{k}{On input, indicates level within ml\_object hierarchy (should be between
               0 and Nlevels${}^\dagger$-1). ML\_ALL\_LEVELS sets the smoothing on all levels in ml\_object.}

\optionbox{pre\_or\_post}{On input, ML\_PRESMOOTHER or ML\_POSTSMOOTHER indicating whether the
                        pre or post smoother is to be set.}

\optionbox{ntimes}{On input, sets the number of block symmetric Gauss Seidel iterations that
                   will be performed.}

\optionbox{omega}{On input, sets the damping parameter to be used during this block
                  symmetric Gauss Seidel smoothing.}

\optionbox{nBlocks}{On input, indicates the total number of block equations in matrix.}

\optionbox{blockIndices}{On input, blockIndices[i] indicates block to which ith element belongs. }


%%%%%%%%%%%%%%%%%%%%%%%%%%%%%%%%%%%%%%%%%%%%%%%%%%%%%%%%%%%%%%%%%%%%%%%%%%%%%%%

\addcontentsline{toc}{subsection}{ML\_Gen\_Solver}
\protobox{int ML\_Gen\_Solver(ML *ml\_object, int scheme, int finest\_level, int coarsest\_level)}


\vspace{2em}
{\flushleft{\bf Description} \hrulefill}
\vspace{1em}

Initialize the \ML\  solver context, ml\_object, so that it is ready to be used in a solve.
ML\_Gen\_Solver should be called after the multigrid cycle is fully specified but before
ML\_Iterate or ML\_Solve\_MGV is invoked.

\vspace{2em}
{\flushleft{\bf Parameters} \hrulefill}
\vspace{1em}

\optionbox{ml\_object}{On input, \ML\  object pointer (see ML\_Create). On output, all necessary
                       initialization is completed.}

\optionbox{scheme}{On input, must be set to {ML\_MGV} indicating a multigrid V cycle is used.}

\optionbox{finest\_level}{On input, indicates the location within ml\_object where the
                       finest level is stored. Normally, this is `0'.}

\optionbox{coarsest\_level}{On input, indicates location within ml\_object where the
                       coarsest grid is stored. When doing smoothed aggregation, this can
                       be determined using the total number of multigrid levels returned by
                       ML\_Gen\_MGHierarchy\_UsingAggregation.}

%%%%%%%%%%%%%%%%%%%%%%%%%%%%%%%%%%%%%%%%%%%%%%%%%%%%%%%%%%%%%%%%%%%%%%%%%%%%%%%

\addcontentsline{toc}{subsection}{ML\_Get\_Amatrix}
\protobox{int ML\_Get\_Amatrix(ML *ml\_object, int k, ML\_Operator **matrix)}

\vspace{2em}
{\flushleft{\bf Description} \hrulefill}
\vspace{1em}

Set *matrix to point to the discretization matrix associated at level 
k within the multigrid solver context ml\_object. This pointer can then
be passed into functions like: ML\_Operator\_Apply, ML\_Operator\_Get\_Diag, and
ML\_Operator\_Getrow.

\vspace{2em}
{\flushleft{\bf Parameters} \hrulefill}
\vspace{1em}

\optionbox{ml\_object}{On input, \ML\  object pointer (see ML\_Create).}

\optionbox{k}{On input, indicates which level within the multigrid hierarchy should be 
              accessed.}

\optionbox{matrix}{On output, *matrix points to the discretization matrix at level k
                   within the multigrid hierarchy. This pointer can then be passed into
                   the functions
                   ML\_Operator\_Apply, ML\_Operator\_Get\_Diag, and ML\_Operator\_Getrow.}


%%%%%%%%%%%%%%%%%%%%%%%%%%%%%%%%%%%%%%%%%%%%%%%%%%%%%%%%%%%%%%%%%%%%%%%%%%%%%%%

\addcontentsline{toc}{subsection}{ML\_Get\_MyGetrowData}
\protobox{void * ML\_Get\_MyGetrowData(ML\_Operator *Amat)}


\vspace{2em}
{\flushleft{\bf Description} \hrulefill}
\vspace{1em}

Returns the user specific data pointer associated with the ML\_Operator
given by Amat. This function is normally employed when users write
their own matrix getrow function and they need to get back the 
pointer that was given with ML\_Init\_Amatrix.

\vspace{2em}
{\flushleft{\bf Parameters} \hrulefill}
\vspace{1em}

\optionbox{Amat}{On input, points to matrix for which we seek the internal data pointer.}


%%%%%%%%%%%%%%%%%%%%%%%%%%%%%%%%%%%%%%%%%%%%%%%%%%%%%%%%%%%%%%%%%%%%%%%%%%%%%%%

\addcontentsline{toc}{subsection}{ML\_Get\_MyMatvecData}
\protobox{void * ML\_Get\_MyMatvecData(ML\_Operator *Amat)}


\vspace{2em}
{\flushleft{\bf Description} \hrulefill}
\vspace{1em}

Returns the user specific data pointer associated with the ML\_Operator
given by Amat. This function is normally employed when users write
their own matrix-vector product function and they need to get back the 
pointer that was given with ML\_Init\_Amatrix.

\vspace{2em}
{\flushleft{\bf Parameters} \hrulefill}
\vspace{1em}

\optionbox{Amat}{On input, points to matrix for which we seek the internal data pointer.}


%%%%%%%%%%%%%%%%%%%%%%%%%%%%%%%%%%%%%%%%%%%%%%%%%%%%%%%%%%%%%%%%%%%%%%%%%%%%%%%

\addcontentsline{toc}{subsection}{ML\_Get\_MySmootherData}
\protobox{void * ML\_Get\_MySmootherData(ML\_Smoother *Smoother)}


\vspace{2em}
{\flushleft{\bf Description} \hrulefill}
\vspace{1em}

Returns the user specific data pointer associated with the ML\_Smoother
object given by Smoother.
This function is normally employed when users write
their own smoother function and they need to get back the 
pointer that was given with ML\_Set\_Smoother.

\vspace{2em}
{\flushleft{\bf Parameters} \hrulefill}
\vspace{1em}

\optionbox{Smoother}{On input, points to the smoother for which we seek the 
internal data pointer.}


%%%%%%%%%%%%%%%%%%%%%%%%%%%%%%%%%%%%%%%%%%%%%%%%%%%%%%%%%%%%%%%%%%%%%%%%%%%%%%%

\addcontentsline{toc}{subsection}{ML\_Init\_Amatrix}
\protobox{int ML\_Init\_Amatrix(ML *ml\_object, int k, int ilen, int olen, void *data)
}

\vspace{2em}
{\flushleft{\bf Description} \hrulefill}
\vspace{1em}

Set the size information for the discretization matrix associated at level k within
ml\_object.  Additionally, associate a data pointer that can be used
when writing matrix-vector product and matrix getrow functions.

\vspace{2em}
{\flushleft{\bf Parameters} \hrulefill}
\vspace{1em}

\optionbox{ml\_object}{On input, \ML\  object pointer (see ML\_Create). On output, size information
                       is associated with the discretization matrix at level k.}

\optionbox{k}{On input, indicates where discretization size information will be
                       stored within the multigrid hierarchy.}

\optionbox{ilen}{On input, the number of local rows in the submatrix stored on this
                       processor.}

\optionbox{olen}{On input, the number of columns in the local submatrix stored on this
                       processor not including any columns associated with ghost unknowns.}

\optionbox{data}{On input, a data pointer that will be associated with the discretization
                       matrix and could be used for matrix-vector product and matrix getrow
                       functions.}



%%%%%%%%%%%%%%%%%%%%%%%%%%%%%%%%%%%%%%%%%%%%%%%%%%%%%%%%%%%%%%%%%%%%%%%%%%%%%%%

\addcontentsline{toc}{subsection}{ML\_Iterate}
\protobox{int ML\_Iterate(ML *ml\_object, double *sol, double *rhs)
}

\vspace{2em}
{\flushleft{\bf Description} \hrulefill}
\vspace{1em}

Iterate until convergence to solve the linear system using the multigrid V cycle
defined within ml\_object.

\vspace{2em}
{\flushleft{\bf Parameters} \hrulefill}
\vspace{1em}

\optionbox{ml\_object}{On input, \ML\  object pointer (see ML\_Create).}

\optionbox{sol}{On input, a vector containing the initial guess for the linear
                system contained in ml\_object. On output, the solution obtained by
                performing repeated multigrid V cycles.}

\optionbox{rhs}{On input, a vector contain the right hand side for the linear
                system contained in ml\_object.}




%%%%%%%%%%%%%%%%%%%%%%%%%%%%%%%%%%%%%%%%%%%%%%%%%%%%%%%%%%%%%%%%%%%%%%%%%%%%%%%

\addcontentsline{toc}{subsection}{ML\_Operator\_Apply}
\protobox{int ML\_Operator\_Apply(ML\_Operator *A, int in\_length, double p[], int out\_length,\\
\phantom{int ML\_Operator\_Apply(}double ap[])}

\vspace{2em}
{\flushleft{\bf Description} \hrulefill}
\vspace{1em}

Invoke a matrix-vector product using the ML\_Operator {\tt A}. That is perform
$ ap = A * p $. Any communication or ghost variables work needed for this
operation is also performed.

\vspace{2em}
{\flushleft{\bf Parameters} \hrulefill}
\vspace{1em}

\optionbox{A}{On input, an ML\_Operator (see ML\_Get\_Amatrix).}

\optionbox{in\_length}{On input, length of vector $p$ (not including ghost variable space).}

\optionbox{p}{On input, vector which will be multiplied by $A$.}

\optionbox{out\_length}{On input, length of vector $ap$.}

\optionbox{ap}{On output, vector containing result of $A * p$.}


%%%%%%%%%%%%%%%%%%%%%%%%%%%%%%%%%%%%%%%%%%%%%%%%%%%%%%%%%%%%%%%%%%%%%%%%%%%%%%%

\addcontentsline{toc}{subsection}{ML\_Operator\_Get\_Diag}
\protobox{int ML\_Operator\_Get\_Diag(ML\_Operator *A, int length, double **diag)}

\vspace{2em}
{\flushleft{\bf Description} \hrulefill}
\vspace{1em}

Get the diagonal of the ML\_Operator {\tt A} (which is assumed to be square).

\vspace{2em}
{\flushleft{\bf Parameters} \hrulefill}
\vspace{1em}

\optionbox{A}{On input, an ML\_Operator (see ML\_Get\_Amatrix).}

\optionbox{length}{On input, number of diagonal elements wanted.}

\optionbox{diag}{On output, sets a pointer to an array containing the diagonal elements.
                 NOTE: this is not a copy but in fact a pointer into an ML data structure.
                 Thus, this array should not be freed.}

%%%%%%%%%%%%%%%%%%%%%%%%%%%%%%%%%%%%%%%%%%%%%%%%%%%%%%%%%%%%%%%%%%%%%%%%%%%%%%%

\addcontentsline{toc}{subsection}{ML\_Operator\_Getrow}
\protobox{int ML\_Operator\_Getrow(ML\_Operator *A, int N\_requested\_rows,
                int requested\_rows[],\\
\phantom{int ML\_Operator\_Getrow(}int allocated\_space, int columns[], double values[],\\
\phantom{int ML\_Operator\_Getrow(}int row\_lengths[])}
%\phantom{int ML\_Operator\_Apply(}double ap[])}

\vspace{2em}
{\flushleft{\bf Description} \hrulefill}
\vspace{1em}

Get a row (or several rows) from the ML\_Operator {\tt A}. If there is not enough space
in {\tt columns} and {\tt values} to hold the nonzero information, this routine returns
a `0'. Otherwise, a `1' is returned.

\vspace{2em}
{\flushleft{\bf Parameters} \hrulefill}
\vspace{1em}

\optionbox{A}{On input, an ML\_Operator (see ML\_Get\_Amatrix).}

\optionbox{N\_requested\_rows}{On input, number of matrix rows for which information is returned.}

\optionbox{requested\_rows}{On input, specific rows for which information will be returned.}

\optionbox{allocated\_space}{On input, length of {\tt columns} and {\tt values}.}

\optionbox{columns}{On output, the column numbers of each nonzero within each row requested
                    in {\tt requested\_rows} (where column numbers associated with 
                    {\tt requested\_rows[i]} appear before column numbers associated 
                    with {\tt requested\_rows[j]} with {\tt i} $<$ {\tt j}).}

\optionbox{values}{On output, the nonzero values of each nonzero within each row requested
                    in {\tt requested\_rows} (where nonzero values associated with 
                    {\tt requested\_rows[i]} appear before nonzero values associated 
                    with {\tt requested\_rows[j]} with {\tt i} $<$ {\tt j}).}

\optionbox{row\_lengths}{On output, {\tt row\_lengths[i]} indicates the number of nonzeros
                        in row {\tt i}.}


%%%%%%%%%%%%%%%%%%%%%%%%%%%%%%%%%%%%%%%%%%%%%%%%%%%%%%%%%%%%%%%%%%%%%%%%%%%%%%%

\addcontentsline{toc}{subsection}{ML\_Set\_Amatrix\_Getrow}
\protobox{int ML\_Set\_Amatrix\_Getrow(ML *ml\_object, int k,
        int (*getrow)(ML\_Operator *, int , int* , int,\\
 \phantom{int ML\_Set\_Amatrix\_Getrow(}int*, double* , int*),
        int (*comm  )(double *vec, void *data),\\
 \phantom{int ML\_Set\_Amatrix\_Getrow(}int comm\_vec\_leng)
}

\vspace{2em}
{\flushleft{\bf Description} \hrulefill}
\vspace{1em}

Set the matrix getrow function for the discretization matrix associated at level
k within the multigrid solver context ml\_object.

\vspace{2em}
{\flushleft{\bf Parameters} \hrulefill}
\vspace{1em}

\optionbox{ml\_object}{On input, \ML\  object pointer (see ML\_Create). On output, matrix getrow
               function is associated with the discretization matrix at level k.}

\optionbox{k}{On input, indicates where the matrix getrow function pointer will be
               stored within the multigrid hierarchy.}

\optionbox{getrow}{On input, a function pointer to the user-defined matrix getrow
                   function. See Section \ref{single}.}

\optionbox{comm}{On input, a function pointer to the user-defined communication
                 function. See Section \ref{sec:multiple}.}

%%%%%%%%%%%%%%%%%%%%%%%%%%%%%%%%%%%%%%%%%%%%%%%%%%%%%%%%%%%%%%%%%%%%%%%%%%%%%%%

\addcontentsline{toc}{subsection}{ML\_Set\_Amatrix\_Matvec}
\protobox{int ML\_Set\_Amatrix\_Matvec(ML *ml\_object, int k,
                      int (*matvec)(ML\_Operator *, int, double *,\\
 \phantom{int ML\_Set\_Amatrix\_Matvec(}int, double *))}

\vspace{2em}
{\flushleft{\bf Description} \hrulefill}
\vspace{1em}

Set the matrix-vector product function for the discretization matrix associated at level 
k within the multigrid solver context ml\_object.

\vspace{2em}
{\flushleft{\bf Parameters} \hrulefill}
\vspace{1em}

\optionbox{ml\_object}{On input, \ML\  object pointer (see ML\_Create). On output, matrix-vector
               product function is associated with the discretization matrix at level k.}

\optionbox{k}{On input, indicates where the matrix-vector product function pointer
                  is stored within the multigrid hierarchy.}

\optionbox{matvec}{On input, a function pointer to the user-defined matrix-vector
                   product function. See Section \ref{single}}.


%%%%%%%%%%%%%%%%%%%%%%%%%%%%%%%%%%%%%%%%%%%%%%%%%%%%%%%%%%%%%%%%%%%%%%%%%%%%%%%

\addcontentsline{toc}{subsection}{ML\_Set\_ResidualOutputFrequency}
\protobox{int ML\_Set\_ResidualOutputFrequency(ML *ml\_object, int output\_freq)}


\vspace{2em}
{\flushleft{\bf Description} \hrulefill}
\vspace{1em}

Set the output frequency of residual information. ML\_Iterate prints the two norm of the
residual every output\_freq iterations.

\vspace{2em}
{\flushleft{\bf Parameters} \hrulefill}
\vspace{1em}

\optionbox{ml\_object}{On input, \ML\  object pointer (see ML\_Create). On output, residual
               printing frequency is set.}

\optionbox{output\_freq}{On input, value to use for printing frequency.}






%%%%%%%%%%%%%%%%%%%%%%%%%%%%%%%%%%%%%%%%%%%%%%%%%%%%%%%%%%%%%%%%%%%%%%%%%%%%%%%

\addcontentsline{toc}{subsection}{ML\_Set\_Smoother}
\protobox{int ML\_Set\_Smoother(ML *ml\_object, int k , int pre\_or\_post, void *data,\\
 \phantom{int ML\_Set\_Smoother(}int (*func)(ML\_Smoother *, int, double *, int, double *), char *label)}

\vspace{2em}
{\flushleft{\bf Description} \hrulefill}
\vspace{1em}

Set the smoother (either pre or post as indicated by pre\_or\_post) at level k within
the multigrid solver context to invoke the user-defined function `func' and 
pass in the data pointer `data' via ML\_Get\_MySmootherData.

\vspace{2em}
{\flushleft{\bf Parameters} \hrulefill}
\vspace{1em}

\optionbox{ml\_object}{On input, \ML\  object pointer (see ML\_Create). On output, a smoother
               function is associated within ml\_object at level k.}

\optionbox{k}{On input, indicates where the smoother function pointer will be
               stored within the multigrid hierarchy.}

\optionbox{pre\_or\_post}{On input, ML\_PRESMOOTHER or ML\_POSTSMOOTHER indicating whether
                       the smoother should be performed before or after the coarse grid
                       correction.}

\optionbox{data}{On input, a data pointer that will be passed into the user-defined
                 function `func'.}

\optionbox{func}{On input, smoothing function to be used at level k when performing
                 a multigrid V cycle. The specific signature and details of this
                 function are given in Section \ref{multigrid options}.}

\optionbox{label}{On input, a character string to be associated with Smoother. This string
                  is printed by some routines when identifying the method.}



%%%%%%%%%%%%%%%%%%%%%%%%%%%%%%%%%%%%%%%%%%%%%%%%%%%%%%%%%%%%%%%%%%%%%%%%%%%%%%%

\addcontentsline{toc}{subsection}{ML\_Set\_Tolerance}
\protobox{int ML\_Set\_Tolerance(ML *ml\_object, double tolerance)}

\vspace{2em}
{\flushleft{\bf Description} \hrulefill}
\vspace{1em}

Set the convergence criteria for ML\_Iterate. Convergence is declared when the 2-norm
of the residual is reduced by `tolerance' over the initial residual. This means that
if the initial residual is quite small (i.e. the initial guess corresponds quite
closely with the true solution), ML\_Iterate might continue to iterate without
recognizing that the solution can not be improved due to round-off error.
Note: the residual is always computed after performing presmoothing on the finest level
(as opposed to at the beginning or end of the iteration). Thus, the true residual should be a
little bit better than the one used by ML.

\vspace{2em}
{\flushleft{\bf Parameters} \hrulefill}
\vspace{1em}

\optionbox{ml\_object}{On input, \ML\  object pointer (see ML\_Create). On output, tolerance
               is set for convergence of ML\_Iterate.}

\optionbox{tolerance}{On input, value to use for convergence tolerance.}

%%%%%%%%%%%%%%%%%%%%%%%%%%%%%%%%%%%%%%%%%%%%%%%%%%%%%%%%%%%%%%%%%%%%%%%%%%%%%%%

\addcontentsline{toc}{subsection}{ML\_Solve\_MGV}
\protobox{int ML\_Solve\_MGV(ML *ml\_object, double *din, double *dout)}

\vspace{2em}
{\flushleft{\bf Description} \hrulefill}
\vspace{1em}

Perform one multigrid V cycle iteration to the solve linear system defined within
ml\_object.

\vspace{2em}
{\flushleft{\bf Parameters} \hrulefill}
\vspace{1em}

\optionbox{ml\_object}{On input, \ML\  object pointer (see ML\_Create).}

\optionbox{din}{On input, the right hand side vector to be used when performing
                       multigrid.}

\optionbox{dout}{On output, an approximate solution obtained after one multigrid V cycle.}

\renewcommand{\thefootnote}{$\dagger$}
\footnotetext{Nlevels refers to the argument given with {\tt ML\_Create}.}

%\section{Do not correspond to current code}
%\begin{itemize}
%\item No hybrid method
%\item Not all parameters are given in the manual.
%\item Dropping terms when doing the prolongator smoother. There is no set function and it
%      is not discussed in the manual.
%\item What exactly does ML\_Aggregate\_Set\_DampingFactor do?
%\item There is no ML\_Aggregate\_UseInftyNorm.
%\item There is no ML\_Aggregate\_Set\_MaxLambdaEstimate.
%\end{itemize}
