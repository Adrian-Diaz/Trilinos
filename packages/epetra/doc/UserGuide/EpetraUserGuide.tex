 \documentclass[12pt,relax]{EpetraUserGuide}
 \usepackage{fancyvrb}
\SANDsubtitle{}
\title{\EpetraTM{} User Guide}

\author{Michael A. Heroux, Robert J. Hoekstra and Alan Williams \\
    \\
        Sandia National Laboratories\\
        P.O. Box 5800\\
        Albuquerque, NM 87185-1110
     }

    % There is a "Printed" date on the title page of a SAND report, so
    % the generic \date should generally be empty.
    \date{\today} % Remove ``\today'' in final version


\SANDnum{SAND2004-xxxx}
\SANDprintDate{Printed July 2004}
\SANDauthor{Michael A. Heroux \\
    Computational Mathematics and Algorithms Department \\
    \\
    Robert J. Hoekstra \\
    Computational Sciences Department \\
    \\
    Alan Williams \\
    Alan's Group Name \\
    \\
     Sandia National Laboratories \\
    P.~O.~Box 5800 \\
    Albuquerque, NM 87185-1110}


\SANDreleaseType{Unlimited Release}


\SANDdistcategory{UC-999}
% New commands

\newcommand{\comm}{Epetra\_Comm}
\newcommand{\serialcomm}{Epetra\_SerialComm}
\newcommand{\mpicomm}{Epetra\_MpiComm}
\newcommand{\mpismpcomm}{Epetra\_MpiSmpComm}
\newcommand{\lbcomm}{Epetra\_LbComm}
\newcommand{\map}{Epetra\_Map}
\newcommand{\blockmap}{Epetra\_BlockMap}
\newcommand{\serialdensevector}{Epetra\_SerialDenseVector}
\newcommand{\intserialdensevector}{Epetra\_IntSerialDenseVector}
\renewcommand{\vector}{Epetra\_Vector}
\newcommand{\multivector}{Epetra\_MultiVector}
\newcommand{\crsgraph}{Epetra\_CrsGraph}
\newcommand{\crsmatrix}{Epetra\_CrsMatrix}
\newcommand{\vbrmatrix}{Epetra\_VbrMatrix}
\newcommand{\rowmatrix}{Epetra\_RowMatrix}
\newcommand{\distributor}{Epetra\_Distributor}
\newcommand{\directory}{Epetra\_Directory}
\newcommand{\distobject}{Epetra\_DistObject}
\newcommand{\linearproblem}{Epetra\_LinearProblem}
\newcommand{\operator}{Epetra\_Operator}
\newcommand{\fecrsmatrix}{Epetra\_FECrsMatrix}
\newcommand{\fevbrmatrix}{Epetra\_FEVbrMatrix}
\newcommand{\import}{Epetra\_Import}
\newcommand{\export}{Epetra\_Export}

\newtheorem{remark}{Remark}

\begin{document}
\maketitle

\begin{abstract}

The \TrilinosTM{} Project is an effort to facilitate the design, development,
integration and ongoing support of mathematical software libraries.
One special Trilinos package is \EpetraTM{}.  Epetra provides a collection of
vector, graph and matrix objects that can be used on serial
and parallel computers.  It is designed to make construction, use and redistribution
of these objects as efficient and easy as possible.  Epetra objects are compatible with all
other Trilinos packages, including all linear, nonlinear and eigen solvers and all
preconditioners packages.  It also implements all Trilinos abstract
interfaces.  Finally, Epetra
provides C++ programmers easy access to the Zoltan~\cite{zoltan-ug}
load balancing package, the BLAS~\cite{BLAS1,BLAS2,BLAS3} and LAPACK~\cite{lapack}.

This user guide is designed to address two types of users:
(i) those who, in addition to using Epetra
objects, are also constructing Epetra vector, graphs and matrices.  This group
includes application developers who are constructing these objects for the purpose
of solving linear, nonlinear and eigensystems.
(ii) those who
are primarily interested in using already-constructed Epetra objects, such as
numerical algorithm developers who want to implement algorithms via
operations on Epetra vector, graph and matrix objects.
\end{abstract}


\section*{Acknowledgement}
The authors would like to acknowledge the support of the ASCI and LDRD
programs that funded development of Epetra.

\clearpage
\tableofcontents
\listoffigures
\listoftables

\clearpage

\section*{Nomenclature}
\addcontentsline{toc}{section}{Nomenclature}
\begin{itemize}
\item[Trilinos]
The name of the project of which Epetra is one of the packages.  Also a Greek term which,
loosely translated means ``a string of pearls,''
meant to evoke an image that each Trilinos package is a pearl in its
own right, but is even more valuable when combined with other
packages.
\item[Petra]
A Greek term meaning ``foundation.''  Trilinos has three Petra
libraries: Epetra, Tpetra and Jpetra that provide basic classes
for constructing and manipulating matrix, graph and vector
objects.  Epetra is the current production version that is
split into two packages, one core and one extensions.
\item[Comm Object]
An instance of one of the Epetra Comm classes.  Presently we support
an MPI, Serial and SMP/MPI implementation of the base \comm{} interface.
\item[Map Object]
Object-oriented algebraic preconditioner, compatible with
Epetra and AztecOO.
\end{itemize}

NOTE: Add section on repeated use of Epetra objects.


\section{Introduction}
\label{Section:Introduction}

Epetra is a collection of C++ classes that support the construction
and use of common numerical linear algebra objects such as vectors,
graphs and matrices.  It is primarily designed for irregular, parallel
distributed memory computations, but can also be used on serial
systems, and
with structured data.  This user guide is intended to (i) introduce a new
user to the basic features of Epetra, (ii) discuss the details of Epetra
constructors, and (iii) illustrate the use of
Epetra's advanced parallel data repartitioning capabilities to
experienced users.  This document is not intended as a reference
manual.  Detailed descriptions of Epetra classes and methods can be
found online at the Trilinos Project home page~\cite{Trilinos-home-page}.


\subsection{Overview of Major Epetra Classes and Features}
\label{Section:Introduction:Overview}

Epetra contains many classes and provides a wide spectrum of
functionality within these classes.  However, there is a relatively
short list of classes and features that are commonly used.  We discuss
these here.

\subsubsection{Communication Support}
Epetra is written to run on parallel, distributed memory computers.  It accomplishes
this via its abstract \comm{} class.  \comm{} is a pure virtual class,
specifying the methods that Epetra needs in order to execute on a
distributed memory computer.  However, \comm{} provides no
implementation of these methods.  Instead, Epetra contains
several derived classes that implement the \comm{} interface,
providing the needed functionality.  In this way, Epetra is not
explicitly dependent on any one communication library or interface.
Presently we have four implementations of \comm{}:
\begin{enumerate}
\item {\bf \serialcomm{}: } Implements the \comm{}
methods to run on a serial computer.  In this case, the majority of
the methods have trivial implementations.  However, having the
\serialcomm{} class allows the rest of Epetra classes to be used
independently from the type of parallel computer being used and allows
serial use of Epetra within a parallel application.
\item {\bf \mpicomm{}: } Uses a user-provided MPI
communicator to implement the \comm{} methods.  In fact, many of the
\comm{} methods have a name similar to an existing MPI function.
Since MPI is by far the most common communication interface and
library used with Epetra, this bias to MPI function names is
intentional.
\item {\bf \mpismpcomm{}:} An experimental implementation that
incorporates information about shared memory threads.  This
implementation of \comm{} is intended for situations where threads,
e.g.~pthreads, are created and retained as part of the call to the
\mpismpcomm{} constructor.  These threads can then be used by any
kernels in Epetra that have been instrumented to run with threads.  At
this point, \mpismpcomm{} is purely for research purposes.
\item {\bf \lbcomm{}: } A new version of \comm{} built using an MPI
communicator and special tuned implementations of basic communication
operations coming from the LB\_Comm library that is part of the Zoltan
load balancing library~\cite{zoltan-ug}.
\end{enumerate}

By using an abstract interface to provide distributed memory support,
Epetra allows for future development of novel parallel communication.
Also, although MPI is by far the most commonly used implementation of
\comm{}, Epetra has no explicit dependence on MPI and therefore MPI is
not required to build Epetra.

\subsubsection{Distributed Data Support}

Coupled with distributed memory communications, Epetra also provides
support for distributed data objects.  Users can constructed,
manipulate and redistribute Epetra objects in a flexible and powerful
way.  The primary classes that are used to describe the layout of
distributed objects are the {\bf \blockmap{}} and {\bf \map{}} classes.  Both of
these classes encode information about the distribution of {\it global
IDs (GIDs)}. GIDs are integer labels that can be associated
with elements of  Epetra distributed objects.  For example, the
elements of an \vector{} are the vector values and each
value of an \vector{} object is associated with a GID.
Similarly, each row of an \crsgraph{}, an \crsmatrix{} and an
\multivector{} is an element and is associated with a GID.  Multiple
elements can be associated with the same GID, but each element has a
unique GID.

\subsection{Primary Linear Algebra Classes}
Although there are many classes in Epetra, the core linear algebra classes are the
vector, graph and matrix classes.  All of these classes implement the
\distobject{} interface, and can therefore be redistributed across the
parallel machine in a variety of ways.  However, a simple use of Epetra usually
involves the construction of a single \comm{} object, which is passed
in to the constructor of a single \map{} object and the resulting \map{}
is then used to construct any number of vector and matrix objects,
something we illustrate in Section~\ref{Section:FirstExample}.
The primary Epetra linear algebra classes are as follows:
\begin{enumerate}
\item {\bf \vector{}: } Supports construction and use of distributed
vectors of double-precision numbers.  Once constructed, \vector{}
objects can be used in multiple ways.  Common operations such as
norms, dot products and vector updates are supported by methods in
this class.  Additional functionality is available via several
extension techniques discussed later.
\item{\bf \multivector{}: } An \multivector{} object is a collection
of \vector{} objects (although \vector{} is actually implemented as a
specialization of \multivector{}).  Specifically, an \multivector{}
object is a collection of vectors with the same size and
distribution.  This kind of object is useful for block algorithms and
applications that manage multiple vectors simultaneously.  \vector{}
and \multivector{} objects are understood by all Trilinos packages.
\item{\bf \crsgraph{}:} Supports the construction and use of adjacency
graphs.  These graphs are used to describe the pattern of Epetra
sparse matrix classes and provide pattern-based information to
load-balancing interfaces.  The graphs are also used to implement
overlapping subdomain algorithms and a variety of other parallel
algorithms.
\item{\bf \crsmatrix{}:} Supports construction and use of distributed
sparse matrix objects.  Once constructed, an \crsmatrix{} object can
be used with any Trilinos solvers or preconditioners.  This class also
supports common matrix and matrix-vector operations such as matrix
scaling, matrix norms and matrix-vector multiplication.
\item{\bf \vbrmatrix{}:} Although less frequently used than the
\crsmatrix{} class, this class supports the construction of sparse
matrices whose entries are dense matrices.  This type of matrix is
frequently found in applications where multiple degrees of freedom are
tracked at each mesh point.  When properly used, the \vbrmatrix{}
class can offer substantial machine performance and algorithmic
robustness improvements.

\end{enumerate}

\section{A First Example}
\label{Section:FirstExample}
Before proceeding with additional descriptive information, we
introduce a simple example in this section in order to explicitly
illustrate a straight-forward use of Epetra.  Although this example only
starts to tap the capabilities of Epetra, we hope the reader finds it
useful.

The example code, listed in Figures~\ref{Figure:FirstExample}
and~\ref{Figure:PowerMethod}, is the main program driver and
a simple implementation of the Power Method~\cite{GoluVanL96} for computing the dominant
eigenvalue and associated eigenvector of a given matrix,
respectively.  In the remainder of this section we proceed with a
description of the code in these figures.
\begin{figure}
\begin{center}
\begin{Verbatim}[frame=single,numbers=left,fontfamily=courier,fontsize=\tiny]
// @HEADER
// ***********************************************************************
// 
//        AztecOO: An Object-Oriented Aztec Linear Solver Package 
//                 Copyright (2002) Sandia Corporation
// 
// Under terms of Contract DE-AC04-94AL85000, there is a non-exclusive
// license for use of this work by or on behalf of the U.S. Government.
// 
// This library is free software; you can redistribute it and/or modify
// it under the terms of the GNU Lesser General Public License as
// published by the Free Software Foundation; either version 2.1 of the
// License, or (at your option) any later version.
//  
// This library is distributed in the hope that it will be useful, but
// WITHOUT ANY WARRANTY; without even the implied warranty of
// MERCHANTABILITY or FITNESS FOR A PARTICULAR PURPOSE.  See the GNU
// Lesser General Public License for more details.
//  
// You should have received a copy of the GNU Lesser General Public
// License along with this library; if not, write to the Free Software
// Foundation, Inc., 59 Temple Place, Suite 330, Boston, MA 02111-1307
// USA
// Questions? Contact Michael A. Heroux (maherou@sandia.gov) 
// 
// ***********************************************************************
// @HEADER

#include "AztecOO_config.h"
#include "AztecOO_Version.h"
#ifdef HAVE_MPI
#include "mpi.h"
#include "Epetra_MpiComm.h"
#else
#include "Epetra_SerialComm.h"
#endif
#include "Epetra_Map.h"
#include "Epetra_Vector.h"
#include "Epetra_CrsMatrix.h"
#include "AztecOO.h"

int main(int argc, char *argv[]) {

#ifdef HAVE_MPI
  MPI_Init(&argc,&argv);
  Epetra_MpiComm Comm( MPI_COMM_WORLD );
#else
  Epetra_SerialComm Comm;
#endif
  if (Comm.MyPID()==0)
    cout << AztecOO_Version() << endl << endl;

  cout << Comm <<endl;

  int NumMyElements = 100;
  // Construct a Map that puts same number of equations on each processor
  Epetra_Map Map(-1, NumMyElements, 0, Comm);
  int NumGlobalElements = Map.NumGlobalElements();

  // Create a Epetra_Matrix
  Epetra_CrsMatrix A(Copy, Map, 3);
  
  // Add  rows one-at-a-time
  double negOne = -1.0;
  double posTwo = 2.0;
  for (int i=0; i<NumMyElements; i++) {
    int GlobalRow = A.GRID(i); int RowLess1 = GlobalRow - 1; int RowPlus1 = GlobalRow + 1;

    if (RowLess1!=-1) A.InsertGlobalValues(GlobalRow, 1, &negOne, &RowLess1);
    if (RowPlus1!=NumGlobalElements) A.InsertGlobalValues(GlobalRow, 1, &negOne, &RowPlus1);
    A.InsertGlobalValues(GlobalRow, 1, &posTwo, &GlobalRow);
  }
  
  // Finish up
  A.FillComplete();

  // Create x and b vectors
  Epetra_Vector x(Map);
  Epetra_Vector b(Map);
  b.Random(); // Fill b with random values

  // Create Linear Problem
  Epetra_LinearProblem problem(&A, &x, &b);
  // Create AztecOO instance
  AztecOO solver(problem);

  solver.SetAztecOption(AZ_precond, AZ_Jacobi);
  solver.Iterate(100, 1.0E-8);

  cout << "Solver performed " << solver.NumIters() << " iterations." << endl
       << "Norm of true residual = " << solver.TrueResidual() << endl;
#ifdef HAVE_MPI
  MPI_Finalize() ;
#endif
  return 0;
}

\end{Verbatim}

\end{center}
\caption{\label{Figure:FirstExample} Simple Driver for Power Method Example}
\end{figure}

\subsection{Explanation of Figure~\ref{Figure:FirstExample}}

\paragraph{Lines 1--2}
Include system header files needed for screen output and system
functions.
\paragraph{Line 3}
Include the Epetra\_config.h file.  This file contains macros
definitions that were defined during the configuration process.  In
particular, the macro HAVE\_MPI will be defined or undefined in this
file, depending on whether Epetra was built in MPI mode or not.  We
will use HAVE\_MPI below to determine if our example code should be
compiled with MPI support or not.
\paragraph{Lines 4--9}
Include the appropriate implementation of the \comm{} class.  If
Epetra was built in MPI mode, the macro ``HAVE\_MPI'' will be defined
and this example will be built with MPI support.  If not, then the
example will be built in serial mode.  Note that these lines of code
 and lines 18--23 are the only difference between a serial and distributed memory
version of the example.
\paragraph{Lines 10--14}
Include the other Epetra header files needed to use Epetra classes.
It is a good practice to explicitly include header files for all
Epetra classes you explicitly use and only those header files.
\paragraph{Line 16}
Declare prototype for power\_method function.
\paragraph{Line 17}
Start of main program.  Note that argv[1] will be used to pass in the
problem size.
\paragraph{Lines 18--24}
Depending on whether or not Epetra was built in MPI
mode\footnote{Whether or not Trilinos, and Epetra as a Trilinos package, is built
in MPI mode is determined by how the Trilinos (or Epetra) {\tt
configure} script in invoked.  If no MPI-related arguments are passed
to the configure script, then packages are built in serial mode only.
If one or more MPI options are invoked, then packages are built with
MPI support (in addition to serial support).}, MPI will be initialized
and an \mpicomm{} object will be constructed, or an \serialcomm{}
object will be constructed.  Please note that, in principle, the serial
version of this example would work, even if Epetra were built in MPI
mode.  Serial mode is always available.

Line 24 prints the Comm object to cout.  Most Epetra classes have
implemented the ostream << operator so that an instance of an Epetra
object can be viewed.  Sometime the volume of output can be very
large, so some care is needed when using this feature.
\paragraph{Lines 25--30}
This example accepts a single integer argument
specifying the number of equations that should be used.  These lines
of code capture that number and put it into NumGlobalElements.
\paragraph{Lines 31--33}
Constructs an \map{} object that has NumGlobalElements elements spread
across the parallel (or serial) machine.  The second argument (a
``0'') indicates that our global indices are zero-based.  Fortran
users would typically pass in a ``1'' here.  The third argument is the
Comm object we just built.
\paragraph{Line 35}
Once an \map{} object is constructed, we can query it for how many
elements of the map are on the calling process\footnote{For readers
who are not familiar with a single-program, multiple data (SPMD)
programming model, it may be useful to read a bit about it.  Typing
``SPMD tutorial'' into a web search engine should be a sufficient
starting point.}.  In this way, the remainder of our code can operate
independent of how data is distributed.
\paragraph{Lines 37--57: Preview}
The majority of executable code in this sample driver is devoted to
constructing our sparse matrix.  At the end of this code segment we
will have constructed a diagonal with each processor owning roughly
the same number of rows and columns.  All diagonal values will be set
to 2.0, except for the very first diagonal value, which will be set to
4.0.  Since our matrix is diagonal, its eigenvalues are the diagonal
values and the dominant eigenvalue is 4.0.
\paragraph{Line 37}
Instantiates (creates) an \crsmatrix{}.  The first argument tells the
constructor whether or not data passed in to this object should be
copied (user values and indices will be copied to internal storage) or
viewed (user values and indices will be pointed to by this object and
the user {\it must} guarantee the integrity of that data).  View mode
is available across many Epetra classes.  In general, this is a very
dangerous practice.  However, in certain very important situations, it
is essential to have this mode.  This is especially true when using
Epetra with Fortran, or when accepting matrix data from other parts of
application where it is too expensive to replicate the data storage.

The second argument is the \map{} object we just constructed.  The
third argument is an advisory value telling the constructor
approximately how many nonzero values will be defined for each row of
the matrix\footnote{getting this value wrong does not
affect the correctness of results, but may affect performance and
efficient use of memory}.  We are constructing a diagonal matrix, so the value ``1''
is appropriate.

At this point the matrix is an empty ``bucket'' ready to receive
matrix values and indices.  Also, at this point, most of the methods
in the \crsmatrix{} cannot be called successfully for this object.
\paragraph{Lines 39--41}
Here we create two helper objects that not really necessary for such a
simple example, but which illustrate how examples that are more
complex might
use the \serialdensevector{} and \intserialdensevector{} classes
effectively.
Line 39 allocates a vector of length one to hold the diagonal value
we will insert into each row.  Line 40 sets the value to 2.0, which is
the value we will assign to all the diagonals of our matrix.  We will
later replace the first diagonal value with a value of 4.0.
\paragraph{Lines 42--48}
These lines insert values and indices into the matrix we just
instantiated.  Our matrix is very simple--a single diagonal entry per
row--so these lines may seem too complicated, but we are illustrating
a process that is effective for more realistic settings.
The \crsmatrix{} class is
designed to accept matrix entries in a variety of ways.  However,
there is a strong bias in the interface to accept multiple entries in
a specified row.  Values for any matrix entry can be submitted one or more times,
with subsequent values being added to any existing value.
Section~\ref{Section:ConstructingCrsMatrices} discusses these issues
in detail.

Line 42 loops over all elements
owned by the calling processor.  Line 43 get the global ID associated
with the current loop index.  This value will be passed in as the
first argument to the
InsertGlobalValues() method to indicate the global row into which the calling
processor is inserting values.  Line 44 sets the column index to the
same as the row index because we have only a diagonal value.

Lines 46--47 is a call the the InsertGlobalValues() method on our matrix
object A.  The first argument is the ID of the global row that the
calling processor will add values to.  The second argument indicates
how many values we are passing in on this call.  We are using the
Length() method of the \serialdensevector{} class to provide this
information.  The third argument is a pointer to an array of doubles
that are the values we are passing in,
provided by the Values() method from \serialdensevector{}.  Finally
the last argument is a pointer to an array of ints that are the column
indices we are passing in.  Because we are only passing in a single
value per row, both of the last arguments to InsertGlobalValues() are trivial
uses of arrays, but illustrate a general process.
\paragraph{Lines 49--55}
The last modification we want to make to our matrix is to put a value
of 4.0 in the first diagonal entry.  We accomplish this by having each
processor query the map to see the processor owns GID 0.  Only the
processor that does own that GID will execute the remaining lines of
code in this segment.  Lines 50--52 set up the value and index
information.  Lines 53--54 are a call to ReplaceGlobalValues(), which
will replace existing values in our matrix with the new values we submit.
\paragraph{Line 57}
After all values are submitted to our matrix, we finally call the
method FillComplete().  This method performs all of the analysis
on our matrix required for efficient execution on a parallel
computer.  Part of the procces involves changing the index values of
the matrix to a local index system on each processor.  This will be
described in greater detail in
Section~\ref{Section:ConstructingCrsMatrices}.
\paragraph{Lines 59--60}
After constructing our matrix, we call the power method function,
passing in our matrix.  Then we print out the eigenvalue estimate that
was returned.
\paragraph{Lines 61--63}
If our code was compile in MPI mode, we need to call MPI\_Finalize()
for proper clean up.
\paragraph{Line 64} Program exit.

\subsection{Explanation of Figure~\ref{Figure:PowerMethod}}

\begin{figure}
\begin{center}
\input{Ex1a}
\end{center}
\caption{\label{Figure:PowerMethod} Simple Power Method Routine}
\end{figure}

\paragraph{Lines 1--2}
Include system header files needed for screen output and system
functions.
\paragraph{Lines 3--6}
Include Epetra header files needed to use Epetra classes.
It is a good practice to explicitly include header files for all
Epetra classes you explicitly use and only those header files.
\paragraph{Line 8} Function declaration.  All information required for
efficient parallel execution is associated with the matrix A that we
pass in.
\paragraph{Lines 10--12}
Define parameter values needed for the power method iterations.  On
exit, lambda will be returned as the approximate eigenvalue.  We
arbitrarily define the maximum number of iterations to be ten times
the dimension of our problem.  We arbitrarily define the tolerance to
be 1.0E-10.
\paragraph{Lines 14--16}
Here we define vectors needed to execute the power method.  The
\vector{} constructor requires an \map{}.  We use the same map that
was used to construct the matrix.  Note that it is not necessary that
the distribution of vectors matches the distribution of matrices.  In
other words, we could use a different map to construct these vectors
as long as all of the GIDs required for matrix operations were present
in the vector map.  This topic is discussed in greater detail in
Section~\ref{Section:MapsAndMatrices}.  By constructing the vectors using
the row map of our matrix, we are forcing these vectors to be
distributed on the parallel machine conformally with the rows of the
matrix.  In many common situations, this is often the best
distribution.
\paragraph{Line 18}
The Random() method fills the associated object with pseudo-random
numbers.  This method is actually part of the \multivector{} class,
but since \vector{} is a specialization of \multivector{}, this method
is also available for \vector{} objects.
\paragraph{Lines 20--22}
Define variables needed for the iteration.
\paragraph{Line 23} While loop requiring at least one iteration and
iterating until niters is reach or the residual is less than the
tolerance.
\paragraph{Lines 24--27}
These four lines use some of the methods available in the \vector{}
and \crsmatrix{} classes to implement the power method.  Line 24
computes the 2-norm of the vector z and returns it in the variable
normz.  Note that the Norm2() method is actually defined in the
\multivector{} class.  Norm2() returns an array of doubles containing
the 2-norm of each vector in the \multivector{} object.  An \vector{} is
an \multivector{} with one vector.  Therefore we can use the Norm2()
interface by passing the address of a double (in this case
``\&normz'') as an array of doubles of length one.

Line 25 computes q, which is z normalized in the 2-norm.  Line 26
computes z = A*q.  The first argument is a bool set to false, asserting
that we are not multiplying by the transpose of A.  Line 27 computes
the dot-product of q and z and puts the result in the variable
lambda.  Like the Norm2() and Scale() methods, Dot() is a method in
the \multivector{} class
\paragraph{Lines 28--36}
This segment of code is not essential to the power method.  It
periodically (every 10 iterations, and on the final iteration) checks
the 2-norm of the difference between A*q and lambda*q.  If lambda and
q are an eigenpair, the difference will be zero, so the norm of the
residual is a reasonable measure of how close we are to convergence.
The only new issue worth mentioning here is to note that the if statement in
line 32 is only around the output stream.  It makes sure that only one
processor generatest the output message.  A user might be tempted to
put the if statement on line 28 since the computations in this segment
do affect the overal iteration.  However, the calls to Update() and
Norm2() on lines 30 and 31 require all processors to participate.  If
processor 0 were the only processor to execute these methods, Update()
would produce incorrect results and Norm2() would cause the user
program to stall.  More details on these topics can be found in
Section~\ref{Section:Troubleshooting} on troubleshooting.
\paragraph{Lines 37--39}
Line 37 increments the iteration count.  Line 39 our approximation to
the dominant eigenvalue.  Note that the vector q is our associated
approximate eigenvector, but for this example we do not return it.

\section{Epetra Communication Classes}

In Section~\ref{Section:Introduction:Overview} we briefly discusses
the Epetra communication classes.  In this section we present more
details.  The class \comm{} is a {\it pure virtual class} sometimes
called an {\it interface} that
specifies how most of Epetra's classes interact with a parallel
machine.  The advantages of using an interface are:
\begin{enumerate}
\item We can have mulitple implementations of the interface, including future
implementations that we did not anticipate.
\item Explicit dependencies on outside libraries, e.g., MPI, are
restricted to \comm{} implementation classes.
\item A trivial serial implementation of \comm{} allows Epetra to be
built without any third-party communication library, which is
sometimes very important.
\end{enumerate}

\subsection{\comm{} Overview}

The \comm{} interface specifies the complete set of functions that Epetra need
for parallel {\it single program multiple data} (SPMD) execution.
Table~\ref{Table:CommMethods} lists the methods we need.
\begin{table}
\scriptsize
\begin{center}
\begin{tabular}{ | p{15cm} | }
\hline {\bf Barrier Methods}
\verb!void  Barrier()!\\
  \comm{} Barrier function.\\
\\
{\bf Broadcast Methods }
\verb!int  Broadcast (double *MyVals, int Count, int Root)!\\
  \comm{} Broadcast function. \\
\\
\verb!int  Broadcast (int *MyVals, int Count, int Root)!\\
  \comm{} Broadcast function. \\
\\
{\bf Gather Methods }
\verb!int  GatherAll (double *MyVals, double *AllVals, int Count)!\\
  \comm{} All Gather function. \\
\\
\verb!int  GatherAll (int *MyVals, int *AllVals, int Count)!\\
  \comm{} All Gather function. \\
\\
{\bf Sum Methods }
\verb!int  SumAll (double *PartialSums, double *GlobalSums, int Count)!\\
  \comm{} Global Sum function. \\
 \\
\verb!int  SumAll (int *PartialSums, int *GlobalSums, int Count)!\\
  \comm{} Global Sum function. \\
 \\
{\bf Max/Min Methods }
\verb!int  MaxAll (double *PartialMaxs, double *GlobalMaxs, int Count)!\\
  \comm{} Global Max function. \\
 \\
\verb!int  MaxAll (int *PartialMaxs, int *GlobalMaxs, int Count)!\\
  \comm{} Global Max function. \\
 \\
\verb!int  MinAll (double *PartialMins, double *GlobalMins, int Count)!\\
  \comm{} Global Min function. \\
 \\
\verb!int  MinAll (int *PartialMins, int *GlobalMins, int Count)!\\
  \comm{} Global Min function. \\
 \\
{\bf Parallel Prefix Methods }
\verb!int  ScanSum (double *MyVals, double *ScanSums, int Count)!\\
  \comm{} Scan Sum function. \\
 \\
\verb!int  ScanSum (int *MyVals, int *ScanSums, int Count)!\\
  \comm{} Scan Sum function. \\
 \\
{\bf Attribute Accessor Methods}
\verb!int  MyPID ()!\\
  Return my process ID. \\
 \\
\verb!int  NumProc ()!\\
  Returns total number of processes. \\
 \\
{\bf Gather/Scatter and Directory Constructors }
\verb!Epetra_Distributor *  CreateDistributor ()!\\
  Create a distributor object. \\
 \\
\verb!Epetra_Directory *  CreateDirectory (const Epetra_BlockMap &Map)!\\
  Create a directory object for the given \blockmap{}. \\
 \\
\hline
\end{tabular}
\caption{\comm{} Methods}
\label{Table:CommMethods}
\end{center}
\end{table}
Most of the methods are essentially equivalent to standard MPI functions.
This is no accident, since MPI is the most commonly used implementation of
\comm{}.  The most notable additions to common MPI functions are the two
''create'' methods which, when called must return a distributor or directory
object that is compatible with the implementation of \comm{}.  We discuss this
in more detail now.

Although \comm{} is the primary abstract parallel machine class, there are two
other very important classes that Epetra uses for SPMD computations:
\distributor{} and \directory{}.

\distributor{} is used to support sparse all-to-all
communications that are required for unstructured parallel sparse matrix calculations.
For example, with a row-based distribution of a sparse matrix $A$,
a standard sparse matrix vector multiplication kernel to compute $y=Ax$
will typically require elements of $x$ from other processors be sent to a processor
that has nonzeros in a corresponding column of $A$.  However, for most applications
only a handful of $x$ values needs to be sent to any one processor, so each processor
only communicates with a subset of all processors.  For many applications, the
communication pattern for these operations is used many times.  Therefore,
the \distributor{} object pre-computes the communication pattern required for
processor to communicate with its required neighbors, and then, as often as needed,
the transfer operation is done later, using the pre-computed pattern.  Epetra provides
user-level support for these operations via the \import{} and \export{} classes, discussed
in Section~\ref{Section:ImportExport}.

\directory{} is used to support the arbitrary index labeling features of Epetra.  Epetra
distributed objects are partitioned across the parallel machine using arbitrary integer
indices.  For example, a 100-element vector could be distributed across a 4-processor machine
putting element 0 through 24 on PE 0, 25 through 49 on PE 1, 50 through 74 on PE 2 and
75 through 99 on PE 3.  However, if the vector is associated with a grid that was automatically
generated by a meshing tool, then the mesh node IDs might not be labeled contiguously or
may not be assigned contiguously to a processor.  In this case, it is very convenient to use
the mesh node IDs for the vector IDs.  Furthermore, arbitrary index labeling plays an important
role in parallel data repartitioning capabilities provided by the \import{} and \export{} classes.
The \directory{} class provides the support needed for arbitrary index labeling as follows:
\begin{enumerate}
\item \directory{} Setup: At construction time, each processor registers each of the
index labels (referred to as global IDs or GIDs in Section~\ref{Section:maps})
that it owns with the directory.  The directory
itself is a distributed object with a known distribution.
\item \directory{} Query: After construction, if a processor wants to know which other processor
owns a particular GID, it queries the \directory{} object for the owner's processor ID.
\end{enumerate}
In Section~\ref{Section:maps} we discuss directories in relationship to \map{} objects, since
\directory{} objects are generated by \map{} objects.

\begin{remark}
To provide an alternative implementation of Epetra's parallel machine interface,
three abstract interfaces must be addressed: \comm{}, \distributor{} and \directory.
\end{remark}

\begin{remark}
To provide an improved implementation of one or a handful of \comm{} methods,
it is possible to create a new class, say \verb!MyFastMpiComm!, that derives
from \mpicomm{} and implements only the methods that you want to replace.
\end{remark}

\subsection{\comm{} Implementations}

Presently the two most commonly used \comm{} implementations are
\serialcomm{} and \mpicomm{}.  \serialcomm{} is used on serial
machines, for debugging and for creating and using serial object
within a parallel task.  \mpicomm{} is the most commonly used
implementation for production applications.  Experimental versions
of \comm{} have been developed for using threads with MPI
(\mpismpcomm) and using an optimized library LBCOMM from Zoltan.
Future work may include a UPC~\cite{UPC} version and a version to
support fault-tolerant computations.
\begin{figure}
  % Requires \usepackage{graphicx}
  \includegraphics{class_Epetra_Comm_inherit_graph.pdf}\\
  \caption{Epetra Comm Class Inheritance Graph}\label{class_Epetra_Comm_inherit_graph}
\end{figure}

\section{Epetra Maps and BlockMaps}
\label{Section:maps}

\map{} objects are used to describe the layout of distributed
objects.  The \map{} class is actually a specialization of an
\blockmap{}.  Specifically, an \map{} object {\it isa} \blockmap{}
with a constant block size of one.  However, to simplify discussion,
we first introduce the \map{} class.

\subsection{\map{} Overview}
An \map{} object contains all information needed to work with
distributed objects.  Its primary data are a collection of {\it
elements}.  Each element of a map has a global ID (GID) and a local ID
(LID).  Elements are ordered on each processor by their LIDs.  On each
processor, LID values range from 0 to the number of elements on the
processor.  The GIDs
of an \map{} are a set of signed integer values.  GIDs are used as
labels and need not be contiguous, ordered or uniquely associated with
a processor.  There are several map constructors that support varying
complexity in assigning GIDs to elements on a processor.  The example
code in Figure~\ref{Figure:MapExamples} illustrate use of the three
\map{} constructors (there is also a copy constructor).  We also
summarize them as follows:
\begin{enumerate}
\item {\bf Uniform linear map:}  The simplest form of map is
constructed by specifying the total number of elements that will be in
the map.  Given this value the \map{} constructor will assign
approximately the same number of elements to each processor and assign
GIDs to each element in a contiguous, increasing order starting with
processor 0.  Figure~\ref{Figure:MapExamples} shows the details for a
map with 100 elements on 3 processors.
\item {\bf Non-uniform linear map:}  This form of map is a small
generalization of the previous case,
where the calling processor specifies the number of
elements it want assigned to itself.  The constructor will determine
the total number of elements in the map and will assign GIDs in a
contiguous, increasing order.
\item {\bf General map:} The most general map is constructed by each
processor passing in a list of GIDs to the \map{} constructor.  The
map will be constructed with the number of specified elements and the
exact set of GIDs passed in.  In this case, the calling processor may
define the GIDs to be any integer values.  No order is assumed, and
repeated values are allowed.  Also, no range of integer values is
assumed.  This most general form of map allows us to support very
flexible data communication and redistribution capabilities.
\end{enumerate}
\begin{figure}
\begin{center}
\input{ExMaps}
\end{center}
\caption{\label{Figure:MapExamples} Three \map{} constructor examples.}
\end{figure}

\map{} (or more precisely \blockmap{}) provides the foundation for Epetra
distributed objects.  In particular, \map{} (or \blockmap{}) is a required
input argument to the \distobject{} class, which is a base class for all Epetra
distributed objects.  When an Epetra distributed object is constructed, each \map{}
element is associated with a packet of data for the distributed object.
Table~\ref{Table:PacketDefinitions} lists the packet definitions for the most
common Epetra distributed objects.
\begin{table}
\begin{center}
\begin{tabular}{ | p{7.5cm} | p{7.5cm} | }
\hline\hline
Type of Distributed Object & Packet Definition \\\hline
\vector{} & Single vector value \\\hline
\multivector{} & Row of vector values \\\hline
\crsgraph{} & List of row indices for one graph row. \\\hline
\crsmatrix{} & List of row indices and values for one matrix row. \\\hline
\hline
\end{tabular}
\caption{Packet Definitions for Various Epetra Distributed Objects}
\label{Table:PacketDefinitions}
\end{center}
\end{table}


\subsection{Global and Local IDs}

On each processor a \blockmap{} object contains a list of elements.  Each
processor can have zero or more elements in the \blockmap.  Each \blockmap{}
element has three attributes:
\begin{enumerate}
\item Local identifier (LID): An integer value in the range 0 to number of elements
on the processor. Elements are order on the processor by LID in increasing order.
\item Global identifier (GID): A signed integer value that is assigned by the user via
one of several constructors.  GIDs need not be unique, ordered or contiguous.
\item Size: A positive integer value that reflects the relative size of an element.  \map{}
objects are \blockmap{} objects where all elements have size 1.
\end{enumerate}

%\subsection{Constructing Maps}
%\subsubsection{Uniform Maps}
%\subsubsection{Linear Maps}
%\subsubsection{General Maps}

\subsection{\blockmap{} Objects}

Most of our discussion so far has focused on \map{} objects for simplicity.  \blockmap{}
provide another level of structure by allowing the association of a {\it size} with each
element.  The interpretation of the size depends on the type of \distobject{} that is
being considered.  Table~\ref{Table:SizeDefinitions} shows the interpretation for several
common \distobject{} classes.
\begin{table}
\begin{center}
\begin{tabular}{ | p{7.5cm} | p{7.5cm} | }
\hline\hline
Type of Distributed Object & Size Definition \\\hline
\vector{} & Number of vector values associated with the element \\\hline
\multivector{} & Number of multivector rows associated with the element \\\hline
\crsgraph{} & Row (column) map: Number of rows (columns associated with the element. \\\hline
\crsmatrix{} & Row (column) map: Number of rows (columns associated with the element. \\\hline
\hline
\end{tabular}
\caption{Size Definitions for Various Epetra Distributed Objects}
\label{Table:SizeDefinitions}
\end{center}
\end{table}

As mentioned above, \map{} is a specialization of \blockmap{}, a \blockmap{} where the
element sizes are all 1.  \blockmap{} becomes important in situations where multiple data values
are closely associated with each other.  The classic example involves tracking multiple degrees of
freedom (DOFs) at a single mesh point in a problem domain.  In this situation, the size attribute of an
\blockmap{} can be used to record how many DOFs are associated with each mesh point.  In some situations
the number of DOFs associated with each mesh point is constant across the whole mesh.  \blockmap{} has
a constructor for this situation.  In the most general situation, the DOFs vary across the mesh, so that
a variable element size is required.  This is the most general case.
\section{Epetra Vectors}
\subsection{Constructing \vector{} Objects}
\subsection{Using \vector{} Objects}
\section{Epetra MultiVectors}
\subsection{Constructing \multivector{} Objects}
\subsubsection{Using \multivector{} Objects}
\section{All About Epetra Maps, Matrices and Graphs}
\label{Section:MapsAndMatrices}

\section{Constructing \crsmatrix{} Objects}
\label{Section:ConstructingCrsMatrices}


\section{ Advanced topics}
\subsection {Using \crsmatrix{} Objects as Column Matrices}

\clearpage
\bibliographystyle{plain}
\bibliography{EpetraUserGuide}
\addcontentsline{toc}{section}{References}

\appendix
\section{Troubleshooting}
\label{Section:Troubleshooting}
\subsection{Conditional Code, Incorrect Results and Stalled Programs}
Many methods in Epetra distributed classes (those derive from the
\distobject{} class) require all processors to
participate in the method call.  For example, to compute the update of
an \vector{}, all processors that own a portion of the vector must
call the Update() method.  When calling a Norm2() method or something
similar, all processors, regardless of whether they own any portion of
a vector, must participate in the Norm2() call.
Figure~\ref{Figure:HungCode} illustrates
several versions of a code segment that computes the 2-norm of a residual
and prints it from processor 0.  Only the final version is correct.
\begin{figure}
\begin{center}
\input{HungCode}
\end{center}
\caption{\label{Figure:HungCode} Common errors with conditional code}
\end{figure}

\section{Common Bugzilla Tasks}
\label{Section:Bugzilla}


\end{document}
