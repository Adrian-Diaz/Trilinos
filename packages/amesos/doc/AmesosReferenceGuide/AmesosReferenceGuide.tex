%@HEADER
% ************************************************************************
% 
%          Trilinos: An Object-Oriented Solver Framework
%              Copyright (2001) Sandia Corporation
% 
% Under terms of Contract DE-AC04-94AL85000, there is a non-exclusive
% license for use of this work by or on behalf of the U.S. Government.
% 
% This program is free software; you can redistribute it and/or modify
% it under the terms of the GNU General Public License as published by
% the Free Software Foundation; either version 2, or (at your option)
% any later version.
%   
% This program is distributed in the hope that it will be useful, but
% WITHOUT ANY WARRANTY; without even the implied warranty of
% MERCHANTABILITY or FITNESS FOR A PARTICULAR PURPOSE.  See the GNU
% General Public License for more details.
%   
% You should have received a copy of the GNU General Public License
% along with this program; if not, write to the Free Software
% Foundation, Inc., 675 Mass Ave, Cambridge, MA 02139, USA.
% 
% Questions? Contact Michael A. Heroux (maherou@sandia.gov)
% 
% ************************************************************************
%@HEADER

\documentclass[11pt]{SANDreport}
\usepackage{graphicx}
\usepackage{amsmath,amsfonts,amsthm}
\usepackage{amssymb}

\usepackage{times}

\def\choicebox#1#2{\noindent$\hphantom{th}$\parbox[t]{2.10in}{\sf
#1}\parbox[t]{3.35in}{#2}\\[0.8em]}

\setlength\oddsidemargin {-0pt}\setlength\evensidemargin{-0pt}
\setlength{\textwidth}{150mm}\setlength{\textheight}{210mm}
\headheight=12.45pt


\title{Amesos Reference Guide}
\SANDnum{SAND2004-XXX}
\SANDauthor{
Marzio Sala, Ken Stanley}

\SANDprintDate{May 2004}
\SANDreleaseType{Unlimited Release}

\newcommand{\Amesos}{Amesos}
\newcommand{\Trilinos}{Trilinos}
\newcommand{\TrilinosTM}{Trilinos \copyright}

\newtheorem{remark}{Remark}

\begin{document}

\maketitle

\begin{abstract}
  
  This document introduces the use of \Amesos{}.
  
  The Amesos package provides an object-oriented interface to several
  direct sparse solvers. Amesos will solve (using a direct factorization
  method) the linear systems of equations
  \begin{equation}
    \label{eq:linear_system}
    A X = B
  \end{equation}
  where $A$ is stored as an Epetra\_RowMatrix object, and $X$ and $B$ are
  Epetra\_MultiVector objects.

  The Amesos package has been designed to face some of the challenges of
  direct solution of linear systems. In fact, many solvers have been
  proposed in the last years, and often each of them requires different
  input formats for the linear system matrix. Moreover, it is not uncommon
  that the interface changes between revisions. Amesos aims to solve those
  problems, furnishing a clean, consistent interface to many direct
  solvers.
  
\end{abstract}

\clearpage

\SANDmain

\tableofcontents

\clearpage

\section*{Acknowledgments}
The authors would like to acknowledge the support of the ASCI and LDRD programs
that funded development of Trilinos.

\smallskip

%Tim Davis is acknowledged for providing KLU and UMFPACK.

\smallskip

%Xiaoye Sherry Li is thanked for providing all the SuperLU libraries, and
%for the suggestions and supports about the Amesos\_Superludist
%interface.

\smallskip

%The MUMPS developers' team is acknowledged for providing MUMPS.

\clearpage

\section{Introduction}

Aim of the Amesos package is to provide an object-oriented interface to
several direct sparse solvers. All the interfaces will have the same
look-and-feel, so that the user can easily switch from one package to
another. Using Amesos, users can interface their codes with a (large)
variety of direct linear solvers, sequential or parallel, simply by a
code instruction of type
\begin{verbatim}
AmesosObject.Solve();
\end{verbatim}
or, more generally, by
\begin{verbatim}
AmesosObject.SymbolicFactorization();
AmesosObject.NumericFactorization();
AmesosObject.Solve();
\end{verbatim}
Amesos will take care of redistributing data among the
processors, if necessary.

Amesos is an interface to other packages, mainly developed outside the
Trilinos framework\footnote{Currently, serial SuperLU is included in the
  Trilinos framework.}. In order to use those packages, the user should
carefully check copyright and licensing of those third party codes.
Please refer to the web page or the documentation of each particular
package for details.

Amesos contains several classes: 
\begin{itemize}
\item \verb!Amesos_KLU!: Interface to Amesos's internal solver
  KLU~\cite{KLU}, described in Section~\ref{sec:klu}.
\item \verb!Amesos_Umfpack!: Interface to Tim Davis's
  UMFPACK~\cite{umfpack-home-page}. Section~\ref{sec:umfpack} presents the basic
  functionalities of this class.
\item \verb!Amesos_Dscpack!: Interface to Padma Raghavan's
  DSCPACK~\cite{dscpack-home-page}, presented in Section~\ref{sec:DSCPACK}.
\item \verb!Amesos_Superludist!: Interface to Xiaoye S.~Li's SuperLU
  solver suite, including SuperLU, SuperLU\_dist 1.0 and SuperLU\_dist
  2.0~\cite{superlu-home-page}. The SuperLU\_dist 2.0 interface is
  presented in Section~\ref{sec:superludist}.
\item \verb!Amesos_Mumps!: Interface to MUMPS 4.3.1~\cite{mumps-home-page}, see
  Section~\ref{sec:mumps}.
\item \verb!Amesos_Scalapack!: Interface to ScaLAPACK~\cite{scalapack}, 
  presented in Section~\ref{sec:scalapack}.
\end{itemize}


Note that Amesos does {\sl not} contain interfaces to LAPACK routines.
Other Trilinos packages already offer those routines (Epetra and
Kokkos).

All the Amesos classes are derived from a base class mode,
\verb!Amesos_BaseSolver!. This abstract interface provides the basic
functionalities for all Amesos solvers, and allows users to choose
different direct solvers very easily -- by changing an input scalar
parameter. See Section~\ref{sec:amesos_generic} for more details.

In this document, we will suppose that matrix $A$ in
equation~(\ref{eq:linear_system}) is defined as an Epetra\_RowMatrix, in
principle with nonzero entries on all the processes defined in the
Epetra\_Comm communicator in use. $X$ and $B$, instead, are
Epetra\_MultiVector, defined on the same communicator.  Some of the
supported packages are serial solvers: in this case, if solving with
more than one processor, the linear problem is shipped to processor 0,
solved, then the solution is broadcasted in the solution vector $X$. For
parallel solvers, instead, various options are supported, depending on
the package at hand:
\begin{itemize}
\item The Amesos\_Superludist interface can be used over all the
  processes, as well as on a subset of them. The matrix is kept in
  distributed form over the processes of interest;
\item Amesos\_Mumps can keep the matrix in a distributed form over all
  the processes, or the matrix can be shipped to processor 0. In both
  cases, all the processes in the MPI communicator will be used.
\end{itemize}


%%%
%%%
%%%

\section{Configuring and Installation Amesos}
\label{sec:3pl}

Amesos is distributed through the Trilinos project, and can be
downloaded from the web site
\begin{verbatim}
http://software.sandia.gov/Trilinos
\end{verbatim}

\smallskip

Each of the Amesos classes provides an interface to a third party direct
sparse solver code\footnote{Exception to this rule is KLU, which is
  distributed with Amesos.}. In order to install a particular class, you
must first install the underlying direct sparse solver code. Generally,
the Amesos installation requires four steps:
\begin{enumerate}
\item Finding MPI for your machine;
\item Finding optimized BLAS for your machine;
\item Installing the third party code needed by the Amesos class that
  you intend to use;
\item Configuring Trilinos with Amesos. 
\end{enumerate}

Amesos is configured and built using the GNU auatoconf~\cite{Autoconf}
and automake~\cite{Automake} tools. This should help for the first two
points. As regards the third point, we refer to the documentation of
each package. The fourth point is addresses here.

Let \verb!$TRILINOS_HOME! be a shell variable representing the location
of the Trilinos source directory.  In order to configure Trilinos with
Amesos, for instance on a LINUX machine with MPI, one may do the following:
\begin{verbatim}
% cd $TRILINOS_HOME
% mkdir LINUX_MPI
% cd LINUX_MPI
% ../configure --with-mpi-compilers \
               --prefix=$TRILINOS_HOME/LINUX_MPI \
               --enable-amesos \
               FLAGS \
               AMESOS_FLAGS
% make
% make install
\end{verbatim}
Here, \verb!FLAGS! represents the set of configure options for other
Trilinos packages, and \verb!AMESOS_FLAGS! the configure options
specific to Amesos. 

The configure options required to enable a specific interface are
reported in each 3-part package's section. A complete list of them can
be obtained by typing
\begin{verbatim}
$TRILINOS_HOME/packages/amesos/configure --help
\end{verbatim}
Here, we {\sl suggest} to copy all the libraries and the include files
for the 3-part libraries in a directory. Let the shell variable
\verb!$TRILINOS_3PL! represents this directory. Note that the user can
adopt other strategies as well.

%%%
%%%
%%%

\section{Amesos\_BaseSolver: A Generic Interface to Direct Solvers}
\label{sec:amesos_generic}

All Amesos objects are constructed from the function class
\verb!Amesos_Factory!.  Amesos\_Factory allows a code to delay the
decision about which concrete class to use to implement the
Amesos\_BaseSolver interface. The main goal of this class is to allow
the user to select any supported (and enabled at configuration time)
direct solver, simply changing an input parameter. Another remarkable
advantage of Amesos\_BaseSolver is that, using this class, users does
not have to include the header files of the 3-part libraries in their
code\footnote{Using Amesos\_BaseSolver, 3-part libraries header files
  are required in the compilation of Amesos only.}.

An example of use of this class is as follows. First, the following
header files must be included:
\begin{verbatim}
  #include "Amesos_Factory.h" 
  #include "AmesosClassType.h"
\end{verbatim}
Then, let \verb!A! be an Epetra\_RowMatrix object (for instance, and
Epetra\_CrsMatrix). We need to define a linear problem,
\begin{verbatim}
  Epetra_LinearProblem * Amesos_LinearProblem = 
                         new Epetra_LinearProblem;
  Amesos_LinearProblem->SetOperator( A ) ; 
\end{verbatim}
and to create an Amesos parameter list (which can be empty):
\begin{verbatim}
  Teuchos::ParameterList ParamList ;
\end{verbatim}
Now, let \verb!Choice! be an AmesosClassType variable, with one of the
following values: 
\begin{itemize}
\item AMESOS\_KLU;
\item AMESOS\_UMFPACK;
\item AMESOS\_MUMPS;
\item AMESOS\_SUPERLUDIST;
\item AMESOS\_DSCPACK.
\end{itemize}
We can construct an \verb!Amesos_BaseSolver! object as follows:
\begin{verbatim}
  Amesos_BaseSolver * A_Base;
  Amesos_Factory A_Factory;

  A_Base = A_Factory.Create(Choice, *Amesos_LinearProblem, 
                            ParamList );
  assert(A_Base!=0);
\end{verbatim}
Symbolic and numeric factorizations are computed using methods
\begin{verbatim}
  A_Base->SymbolicFactorization();
  A_Base->NumericFactorization();
\end{verbatim}
The numeric factorization phase will checl whether a symbolic
factorization exists or not. If not, method
\verb!SymbolicFactorization()! is invoked.  Solution is computed (after
setting of LHS and RHS in the linear problem), using
\begin{verbatim}
  A_Base->Solve();
\end{verbatim}
The solution phase will checl whether a numeric factorization exists or
not. If not, method \verb!SymbolicFactorization()! is called.

Users must provide the nonzero structure of the matrix for the symbolic
phase, and the actual nonzero values for the numeric
factorization. Right-hand side and solution vectors must be set before
the solution phase, for instance using
\begin{verbatim}
  Amesos_LinearProblem->SetLHS(x);
  Amesos_LinearProblem->SetRHS(b);
\end{verbatim}

A common ingredient to all the Amesos classes is the Teuchos parameters'
list. This object, whose definition requires the input file
\verb!Teuchos_ParameterList.hpp!, is used to specify the parameters that
affect the 3-part libraries. For a detailed presentation of Teuchos, we
refer to~\cite{Teuchos-home-page}. Here, we simply recall that the
parameters' list can be created as
\begin{verbatim}
  Teuchos::ParameterList AmesosList;
\end{verbatim}
and parameters can be set as
\begin{verbatim}
  AmesosList.setParameter(ParameterName,ParameterValue);
\end{verbatim}
Here, \verb!ParameterName! is a string containing the parameter name,
and \verb!ParameterValue! is any valid C++ object that specifies the
parameter value (for instance, an integer, a pointer to an array or to
an object).

Amesos has to level of parameters: 
\begin{enumerate}
\item a first level refers to parameters that affect all solvers;
\item a second level refers to parameters that are specific to a
  particular solver.
\end{enumerate}

We now list all the parameters that affect all the Amesos solvers.

\vskip .1in \choicebox{\tt UseTranspose}{[bool] If false, solve linear
  system~(\ref{eq:linear_system}). Otherwise, solve the linear system
  with the transpose matrix $A^T$.. Default: false.}

\choicebox{\tt Threshold}{[double] Drop all elements whose absolute
  value is below the specified threshold. Default: 0.0.}

\choicebox{\tt AddZeroToDiag}{[bool] If true, a zero element will be
  added to the diagonal if not present. Default: false.}

\choicebox{\tt PrintTiming}{[bool] Print timing information. Default: true.}

\choicebox{\tt ComputeVectorNorms}{[bool] After solution, compute the
  2-norm of $B$ and $X$.  Default: false.}

\choicebox{\tt ComputeTrueResidual}{[bool] After solution, compute the
  real residual $\| B - A X \|$. Default: false.}

\choicebox{\tt PrintStatistics}{[bool] Print statistics. Default: true.}

%\choicebox{\tt SolveLocally}{[bool] If true, Amesos will solve on each
%  process, independently from the others. Only local rows of
%  system~(\ref{eq:linear_system}) will be converted to the 3-part
%  package. Users must take care of providing compatible solution and
%  right-hand side vectors.}

\choicebox{\tt Redistribute}{[bool] If true, ..... Default: true.}

\smallskip

Solver-specific parameters are reported in each package's
subsection. The general procedure is as follows: the user creates a
sublist with a given name (for instance, the sublist for MUMPS is
"mumps"), then sticks all the solver's specific parameters in this
sublist. An example is as follows:
\begin{verbatim}
  int ictnl[40];
  // defines here the entries of ictnl
  Teuchos::ParameterList & AmesosMumpsList = 
    AmesosList.sublist("mumps");
  AmesosMumpsList.setParemeter("ICTNL", ictnl);
\end{verbatim}
Parameters and sublists not recognized are simply ignored. Recall that
spaces are important, and that parameters' list is case sensitive!

%%%
%%%
%%%

\section{Amesos Interface to KLU}
\label{sec:klu}

KLU is a serial, unblocked code ideal for getting started and for very
sparse matrices, such as circuit matrces.
KLU computes 
\[
A^T X = B
\]
more efficiently than system~(\ref{eq:linear_system}). For the KLU
package, the latter linear system requires a matrix transpose.

KLU is Tim Davis' implementation of Gilbert-Peierl's left-looking sparse
partial pivoting algorithm, with Eisenstat and Liu's symmetric pruning.
Gilbert's version appears as [L,U,P]=lu(A) in MATLAB. It doesn't exploit
dense matrix kernels, but it is the only sparse LU factorization
algorithm known to be asymptotically optimal, in the sense that it takes
time proportional to the number of floating-point operations. It is the
precursor to SuperLU, thus the name ("clark Kent LU"). For very sparse
matrices that do not suffer much fill-in (such as most circuit matrices
when permuted properly) dense matrix kernels do not help, and the
asymptotic run-time is of practical importance.

\smallskip

In order to use KLU, Amesos must be configured with the options
\begin{verbatim}
--enable-amesos-klu
\end{verbatim}
(KLU is not an external solver.)

KLU is a serial solver. Amesos will take care of moving matrix,
solution and right-hand side to processor 0 (using Epetra\_Import
objects), solve the linear system on processor 0, them broadcast the
solution as required.

The following options in the ParameterList will affect Amesos\_Klu:
\begin{itemize}
\item ????
\end{itemize}

%%%
%%%
%%%

\section{Amesos Interface to UMFPACK}
\label{sec:umfpack}

UMFPACK is a C package copyrighted by Timothy A.~Davis. More information
can be obtained at the web page
\begin{verbatim}
  http://www.cise.ufl.edu/research/sparse/umfpack
\end{verbatim}

In order to use UMFPACK, Amesos must be configured with the options
\begin{verbatim}
--enable-amesos-umfpack 
--with-amesos-umfpacklib=<UMFPACK library> 
--with-amesos-umfpackincdir=<UMFPACK include files>
--with-amesos-umfpackamdlib=<AMD library>
--with-amesos-umfpackamdincdir=<AMD include files>
\end{verbatim}

UMFPACK is a serial solver. Amesos will take care of moving matrix,
solution and right-hand side to processor 0 (using Epetra\_Import
objects), solve the linear system on processor 0, them broadcast the
solution as required.

The following options in the ParameterList will affect Amesos\_Umfpack:
\begin{itemize}
\item ????
\end{itemize}

%%%
%%%
%%%

\section{Amesos Interface to DSCPACK}
\label{sec:dscpack}

%%%
%%%
%%%

\section{Amesos Interface to SuperLU\_dist 2.0}
\label{sec:superludist}

SuperLU\_DIST, written by Xiaoye S. Li, is a parallel extension to the
serial SuperLU library.  SuperLU\_DIST is written in ANSI C, using MPI
for communication, and it is targeted for the distributed memory
parallel machines. It is copyrighted by The Regents of the University of
California, through Lawrence Berkeley National Laboratory.  We refer to
the web site
\begin{verbatim}
  http://www.nersc.gov/~xiaoye/SuperLU
\end{verbatim}
and to the SuperLU\_DIST manual~\cite{superlu-manual} for more
information.

SuperLU\_DIST includes routines to handle both real and complex matrices
in double precision. However, as Amesos is currently based on the Epetra
package (that does not handle complex matrices), only double precision
matrices can be considered.

SuperLU\_DIST provides two input interfaces for matrices $A$ and $B$ in
equation~(\ref{eq:linear_system}). The Amesos\_Superludist class
supports the interface for a distributed matrix $A$, given in input as
an Epetra\_RowMatrix, and distributed vectors $B$, given in input as
Epetra\_MultiVector. Both $A$ and $B$ will not be modifies by
Amesos\_Superludist. The solution of the linear system is copied in the
distributed Epetra\_MultiVector X.

Amesos\_Superludist can solve the linear system on a subset of the
processes, as specified in the parameters' list. This is done by
creating a new process group derived from the MPI group of the
Epetra\_Comm object, with function \verb!superlu_gridinit()!. Users can
specify rows and columns of the grid (see later description of
SuperLU\_dist input parameters). This can be useful to obtain better
scalability. Amesos\_Superludist insultates the users from the creation
of new MPI groups.

\medskip

In order to interface with SuperLU\_dist 2.0, Amesos must be configured with
the options
\begin{verbatim}
--enable-amesos-superludist 
--with-amesos-superludistlib=<SuperLU_dist library>
--with-amesos-superludistincdir=<SuperLUdist include files>
\end{verbatim}

SuperLU\_dist is a parallel solver, that can accept matrices in
distributed forms. Given a distributed Epetra\_RowMatrix, Amesos will
convert each local matrix into the SuperLU\_dist format.  Alternatively,
users can specify the maximum number of processes to include in the
computation. (This can be justified when running on more than about 32
processes -- depending on matrix size and nonzero entries.) 

The SuperLU\_dist constructor will look for a sublist, called
\verb!Superludist!. If this sublist is present, the following parameters
will be considered:

\vskip .1in

\choicebox{\tt MaxProcesses}{Linear system matrix will be distributed on
  MaxProcesses only (if this number is available to the system). The
  processes will be defined on an approximate square grid. If set to -1,
  all the processes in the MPI communicator will be used. Default: -1.}

\choicebox{\tt nprow}{Distributes processes on a square grid with
  specified number of rows. Default: 0.}

\choicebox{\tt npcol}{Distributes processes on a square grid with
  specified number of cols. Note: If the user specifies a positive
  nprow, it {\sl must} specify npcol (different from 0). Those two
  parameters will overwrite {\tt MaxProcesses}. Default: 0.}

\choicebox{\tt ReuseSymbolic}{... Default: false}

The following parameters reflect the behavior of SuperLU\_dist options
argument, as specified in the SuperLU\_dist
manual~\cite{superlu-manual}[pages 55--56]. The user is referred to this
manual for a detailed explanation of the reported parameters. Default
values are taken as reported in the SuperLU\_dist manual.

\vskip .1in

\choicebox{\tt Fact}{[string] Specifies whether or not the factored form
  of the matrix $A$ is supplied onentry and, if not, how the matrix will
  be factored. It can be: {\tt DOFACT, SamePattern,
    SmapPattern\_SameRowPerm, FACTORED}. Default: {\tt
    SamePattern\_SameRowPerm}.}

\choicebox{\tt Equil}{[bool] Specifies whether to equilibrate the system of
  not. Default: true.}

\choicebox{\tt ColPerm}{[string] Specifies whether to column ordering.
  It can be: {\tt NATURAL, MMD\_AT\_PLUS\_A, MMD\_ATA, COLAMD, MY\_PERMC}. 
  Default: {\tt MMD\_AT\_PLUS\_A}.}

\choicebox{\tt perm\_c}{[int *] Specifies the ordering to use when {\tt
    ColPerm = MY\_PERMC}.}

\choicebox{\tt RowPerm}{[string] Specifies whether to row ordering.
  It can be: {\tt NATURAL, LargeDiag, MY\_PERMR}. 
  Default: {\tt LargeDiag}.}

\choicebox{\tt perm\_r}{[int *] Specifies the ordering to use when {\tt
    RowPerm = MY\_PERMR}.}

\choicebox{\tt ReplaceTinyPivot}{[bool] Specifies whether to replace the
  tiny diagonals with $\varepsilon \| A \|$ during LU
  factorization. Default: true.}

\choicebox{\tt IterRefine}{[string] Specifies how to perform iterative
  refinement. It can be: {\tt NO, DOUBLE, EXTRA}. Default: {\tt
    DOUBLE}.}


%%%
%%%
%%%

\section{Amesos Interface to MUMPS 4.3.1}
\label{sec:mumps}

MUMPS (``MUltifrontal Massively Parallel Solver'') is a parallel direct
solver, written in FORTRAN 90 with C interface, copyrighted by P. R.
Amestoy, I. S.  Duff, J. Koster, J.-Y.  L'Excellent. Up-to-date copies
of the MUMPS package can be obtained from the Web page
\begin{verbatim}
http://www.enseeiht.fr/apo/MUMPS/
\end{verbatim}
Here, for the sake of completeness, we brefly present a broad view of
the MUMPS package, so that the reader can better understand the
Amesos\_Mumps interface. For details about the algorithms and the
implementation, as well as of the input parameters, we refer
to~\cite{mumps-manual}

MUMPS can solve the original system~(\ref{eq:linear_system}), as well as
the transposed system, given an assembled or elemental matrix. Note that
only the assembled format is supported by Amesos\_Mumps. Mumps offers,
among other features, error analysis, iterative refinement, scaling of
the original matrix, Schur complement with respect to a prescribed
subset of rows. Reordering techniques can take advantage of PORD
(distributed within MUMPS), or METIS~\cite{METIS}. 

Amesos\_Mumps is based on the distributed double-precision version of
MUMPS (which requires MPI, BLAS, BLACS
and ScaLAPACK~\cite{scalapack}).  Complex linear system are not
supported, and linear system in single precision are under development.

The default implementation of Amesos\_Mumps handles distributed
matrices. It is also possible to ask Amesos\_Mumps to move the matrix to
processor 0.

\medskip

In order to interface with MUMPS 4.3.1, Amesos must be configured with
the options
\begin{verbatim}
--enable-amesos-mumps 
--with-amesos-mumpslib=<MUMPS library>
--with-amesos-mumpsincdir=<MUMPS include files>
\end{verbatim}

The MUMPS constructor will look for a sublist, called
\verb!mumps!. 
The user can set all the MUMPS's parameters, by sticking pointers to the
integer array ICNTL and the double array CNTL to the parameters' list,
or by using the functions reported at the end of this section.

\vskip .1in
\choicebox{\tt ICTNL}{[int[40]] Pointer to an integer array, containing
  the integer parameters (see~\cite{mumps-manual}[pages
  13--17].}

\choicebox{\tt CTNL}{[double[5]] Pointer to an double array, containing
  the double parameters (see~\cite{mumps-manual}[page 17].}

\choicebox{\tt PermIn}{[int *] Use integer vectors of size
  NumGlobalElements (global dimension of the matrix) as given ordering.
  {\tt PermIn} must be defined on the host only, and allocated by the
  user, if the user sets ICNTL(7) = 1.}
  
\choicebox{\tt Maxis}{[int] Sets Maxis value.}

\choicebox{\tt Maxs}{[int] Sets Maxis value.}

\choicebox{\tt ColPrecScaling}{[double *] Uses double precision vectors
  of size NumGlobalElements (global dimension of the matrix) as scaling
  for columns and rows. The double vector must be defined on the host
  only, and allocated by the user, if the user sets ICNTL(8) = -1.}

\choicebox{\tt RowPrecScaling}{[double *] Uses double precision vectors
  of size NumGlobalElements (global dimension of the matrix) as scaling
  for columns and rows. The double vector must be defined on the host
  only, and allocated by the user, if the user sets ICNTL(8) = -1.}


\bigskip

Other functions are avaiable to retrive the output values. The following
Amesos\_Mumps methods are {\sl not} supported by the Amesos\_BaseSolver
class; hence, the user must create an Amesos\_Mumps object in order to
take advantage of them.

\medskip

\begin{verbatim}
double * GetRINFO() 
\end{verbatim}
Gets the pointer to the RINFO array (defined on all processes).

\medskip

\begin{verbatim}
int * GetINFO() 
\end{verbatim}
Gets the pointer to the INFO array (defined on all processes).

\medskip

\begin{verbatim}
double * GetRINFOG() 
\end{verbatim}
Gets the pointer to the RINFOG array (defined on host only).

\medskip

\begin{verbatim}
int * GetINFOG() 
\end{verbatim}
Gets the pointer to the INFOG array (defined on host only).

\bigskip
  
A functionality that is peculiar to MUMPS, is the ability to return
the Schur complement matrix, with respect to a specified set of nodes.

\begin{verbatim}
int ComputeSchurComplement(bool flag,
                           int NumSchurComplementRows, 
                           int * SchurComplementRows);
\end{verbatim}
This method computes (if \verb!flag! is true) the Schue complement with
respect to the set of indices included in the integer array {\tt
  SchurComplementRows} , of size {\tt NumSchurComplementRows}. This is a
{\sl global} Schur complement, and it is formed (as a dense matrix) on
processor 0 only.

\medskip

\begin{verbatim}
Epetra_CrsMatrix * GetCrsSchurComplement();
\end{verbatim}
This method returns the Schur complement in an Epetra\_CrsMatrix, on host
only. No checks are performed to see whether this action is legal or not
(that is, if the call comes after the solver has been invocated).
The returned Epetra\_CrsMatrix must be freed by the user.

\medskip

\begin{verbatim}
Epetra_SerialDenseMatrix * GetDenseSchurComplement();
\end{verbatim}
This method returns the Schur complement as a Epetra\_SerialDenseMatrix
(on host only).

\medskip


As an example, the following frament of code shows how to use MUMPS to
obtain the Schur complement matrix with respect to a given subsets of
nodes.  First, we need to create an parameter list, and an Amesos\_Mumps
object.
\begin{verbatim}
  Teuchos:::ParameterList params;
  Amesos_Mumps * Solver;
  Solver = new Amesos_Mumps(*Problem,params);
\end{verbatim}
Then, we define the set of nodes that will constitute the Schur
complement matrix. This must be defined on processor 0 only. For
instance, one may have:
\begin{verbatim}
  int NumSchurComplementRows = 0;
  int * SchurComplementRows = NULL;
  if( Comm.MyPID() == 0 ) {
    NumSchurComplementRows = 4;
    SchurComplementRows = new int[NumSchurComplementRows];
    SchurComplementRows[0] = 0;
    SchurComplementRows[1] = 1; 
    SchurComplementRows[2] = 2;
    SchurComplementRows[3] = 3;
  }
\end{verbatim}
Now, we can ask for the Schur complement using
\begin{verbatim}  
  Solver->ComputeSchurComplement(true, NumSchurComplementRows,
                                       SchurComplementRows);
\end{verbatim}
The Schur complement matrix can be obtain after the solver phase:  
\begin{verbatim}  
  Solver->Solve();
  Epetra_CrsMatrix * SC;
  SC = Solver->GetCrsSchurComplement();
  Epetra_SerialDenseMatrix * SC_Dense;
  SC_Dense = Solver->GetDenseSchurComplement();
\end{verbatim}

%%%
%%%
%%%

\section{Amesos Interface to ScaLAPACK}
\label{sec:scalapack}

%%%
%%%
%%%

\section{Example Code}
\label{sec:comparison}

In this section we report a complete code, whose aim is to compare the
performances of various direct solvers, using the Amesos\_BaseSolver
interface. 

The source of the entire code is reported below. Several comments are
reported to detail all phase. The source file is
\begin{verbatim}
$TRILINOS_HOME/packages/amesos/examples/Amesos_ReferenceGuide.cpp
\end{verbatim}

First, we need to include the appropriate headers. The variable
\verb!HAVE_CONFIG_H! must have been defined -- in the file, or at
compilation time. We include the configuration file of Epetra and
Amesos, for the Teuchos parameter list, for the Amesos\_Factory, and of
a matrix gallery class defined in Triutils.
\begin{verbatim}
#define HAVE_CONFIG_H
#include "Epetra_config.h"
#ifdef HAVE_MPI
#include "mpi.h"
#include "Epetra_MpiComm.h"
#else
#include "Epetra_SerialComm.h"
#endif
#include "Epetra_Vector.h"
#include "Epetra_Time.h"
#include "Amesos_config.h"
#include "Teuchos_ParameterList.hpp"
#include "Amesos_Factory.h"
#include "Trilinos_Util_MatrixGallery.h"
\end{verbatim}
The code can be run with or without MPI; however, the supported versions
of MUMPS and SuperLU\_dist requires MPI.
\begin{verbatim}
int main(int argc, char *argv[]) {

#ifdef HAVE_MPI
  MPI_Init(&argc, &argv);
  Epetra_MpiComm Comm(MPI_COMM_WORLD);
#else
  Epetra_SerialComm Comm;
#endif
\end{verbatim}
Here we use the class Trilinos\_Util\_MatrixGallery to read and
Harwell/Boeing matrix from file, whose name is hardwired in the code for
simplicity. The name can be read from the input line using class
Trilinos\_Util\_ShellOptions (see \cite{Trilinos-tutorial}), or the
Teuchos package.
\begin{verbatim}
  Trilinos_Util_MatrixGallery G("hb", Comm);
  G.Set("matrix_name","662_bus.rsa");
\end{verbatim}
Class Trilinos\_Util\_MatrixGallery automatically defines an
Epetra\_LinearProblem, that can be obtained as follows:
\begin{verbatim}  
  Epetra_LinearProblem * Problem = G.GetLinearProblem();
\end{verbatim}
Now, we define a Teuchos parameters' list, together with sublists for
SuperLU\_dist and MUMPS. We set some parameters.
\begin{verbatim}  
  Teuchos::ParameterList params;
  Teuchos::ParameterList & SluParamList=params.sublist("Superludist");
  Teuchos::ParameterList & MumpsParamList=params.sublist("mumps");

  params.setParameter("Redistribute",true);
  SluParamList.setParameter("MaxProcesses",8);
\end{verbatim}
At this point, we can create an Amesos\_BaseSolver object, depending on
the run-time choice (here hardcoded for the sake of simplicity as a
string):
\begin{verbatim}  
  Amesos_BaseSolver * Solver;
  Amesos_Factory A_Factory;
  string SolutionLib = "umfpack";

  if( SolutionLib == "klu" ) 
    Solver = A_Factory.Create(AMESOS_KLU, *Problem, params );
  else if( SolutionLib == "umfpack" )
    Solver = A_Factory.Create(AMESOS_UMFPACK, *Problem, params );
  else if( SolutionLib == "superludist" )
    Solver = A_Factory.Create(AMESOS_SUPERLUDIST, *Problem, params );
  else if( SolutionLib == "mumps" )
    Solver = A_Factory.Create(AMESOS_MUMPS, *Problem, params );
\end{verbatim}
We can solve the linear problem, taking some timing:
\begin{verbatim}  
  Epetra_Time Time(Comm);      
  Solver->SymbolicFactorization();
  double TimeForSymbolicFactorization = Time.ElapsedTime();
  
  Time.ResetStartTime();  
  Solver->NumericFactorization();
  double TimeForNumericFactorization = Time.ElapsedTime();
   
  Time.ResetStartTime();
  Solver->Solve();
  Solver->Solve();
  Solver->Solve();
  double TimeForSolve = Time.ElapsedTime();
\end{verbatim}
and finally delete the Amesos\_BaseSolver object, and exit the code.

\begin{verbatim}  
  delete Solver;
    
#ifdef HAVE_MPI
  MPI_Finalize();
#endif

  return( EXIT_SUCCESS );

}
\end{verbatim}

First, we compare (on a serial computer), the performances of
KLU and UMFPACK. Then, we analyze the performances of the two parallel
direct solver avaialable through Amesos: SuperLU\_dist 2.0 and MUMPS
4.3.1.


{\bf Compare different HB matrices?? Matrices from somewhere else?????}

{\bf suggestions for the names of the matrices???????}

{\bf maybe some MPSalsa comparisons...}

\bibliographystyle{plain}
\bibliography{AmesosBibliography}

\end{document}
