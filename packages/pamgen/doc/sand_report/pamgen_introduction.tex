\section{Introduction}
To overcome the challenge of producing multi{}-million finite element
meshes for simulations using more than 1000 processors a library has been 
developed (\textsc{pamgen}) that for several simple geometries produces each processor's mesh as an early step of the analysis execution. The specification for these meshes is provided by a block of terse instructions that may be placed in an input file. These instructions are passed to the library as a {}''C{}''-programming language character array. \textsc{pamgen} is also referred to as an ``in line'' mesh generator because the meshing instructions may be included in one of the analysis input decks.

The simple input format allows analysts to change the resolution of 
a simulation by altering a few parameters. It also allows them to execute their simulations on
different numbers of processors without requiring any pre-processing.

The mesh generation proceeds through steps of decomposition, local
element creation, and communication information generation. The final
product of the library is a data structure that can be queried using an API (Application Programming Interface) that is based on the NEMESIS and EXODUS APIs. Currently the library
is limited to generating meshes of domains with cylindrical, tubular,
and block shapes. Substantial control is allowed over the element
density within these shapes. Boundary condition application  
regions can be specified on the surfaces and interior of the mesh .

Development of this capability revealed that the parallel mesh
generation process can be reduced to answering a series of questions:
What elements are on this processor? What nodes are on this processor?
What is the connectivity of this element? What elements border this
element? What processor does this element reside on?... Resolving these
questions inductively, without resolution to communication, is
essential for preserving scalability. Once a framework
for posing and answering these questions for a particular geometry is
established, expanding the capability to support additional geometries
is straightforward.

