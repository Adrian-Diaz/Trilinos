Time integration is a central component for most transient simulations.
It coordinates many of the major parts of a simulation together, e.g.,
a residual calculation with a transient solver, solution with the
output, various operator-split physics, and forward and adjoint solutions
for inversion. Even though there is this variety in these transient
simulations, there is still a common set of algorithms and procedures
to progress transient solutions for ordinary-differential equations
(ODEs) and differential-alegbraic equations (DAEs). 

Rythmos is a collection of these algorithms that can be used for the
solution of transient simulations. It provides common time-integration
methods, such as Backward and Forward Euler, Explicit and Implicit
Runge-Kutta, and Backward-Difference Formulas. It can also provide
sensitivities, and adjoint components for transient simulations. Rythmos
is a package within Trilinos, and requires some other packages (e.g.,
Teuchos and Thrya) to provide basic time-integration capabilities.
It also can be coupled with several other Trilinos packages to provide
additional capabilities (e.g., AztecOO and Belos for linear solutions,
and NOX for non-linear solutions).

The documentation is broken down into three parts: Theory Manual,
User's Manual, and Developer's Guide. The Theory Manual contains the
basic theory of the time integrators, the nomenclature and mathematical
structure utilized within Rythmos, and verification results demonstrating
that the designed order of accuracy is achieved. The User's Manual
provides information on how to use the Rythmos, description of input
parameters through Teuchos Parameter Lists, and description of convergence
test examples. The Developer's Guide is a high-level discussion of
the design and structure of Rythmos to provide information to developers
for the continued development of capabilities. Details of individual
components can be found in the Doxygen webpages.

\textbf{Notice:} This document is a collection of notes gathered over
many years, however its development has been dormant for the past
several years due to the lack of funding. Pre-release copies of this
document have circulated to internal Sandia developers, who have found
it useful in their application development. Also external Sandia developers
have made inquiries for additional Rythmos documentation. To make
this documentation publicly available, we are releasing this documentation
in an ``\textbf{as-is}'' condition. We apologize for its incomplete
state and obvious lack of readability, however we hope that the information
contained in this document will still be helpful to users in their
application development.
