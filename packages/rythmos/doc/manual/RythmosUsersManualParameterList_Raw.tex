\begin{figure} 
\centering{} 
  \begin{tabular}{p{0.9\textwidth}}
\hspace*{0in} Integrator Base (Section~\ref{sec:Integrator Base})
    \index{Integrator Base} \\ 
\hspace*{0.2in} Integrator Settings (Section~\ref{sec:Integrator Settings})
    \index{Integrator Settings} \\ 
\hspace*{0.4in} Integrator Selection (Section~\ref{sec:Integrator Selection})
    \index{Integrator Selection} \\ 
\hspace*{0.6in} Default Integrator (Section~\ref{sec:Default Integrator})
    \index{Default Integrator} \\ 
\hspace*{0.8in} VerboseObject (Section~\ref{sec:VerboseObject})
    \index{VerboseObject} \\ 
\hspace*{0.2in} Integration Control Strategy Selection (Section~\ref{sec:Integration Control Strategy Selection})
    \index{Integration Control Strategy Selection} \\ 
\hspace*{0.4in} Simple Integration Control Strategy (Section~\ref{sec:Simple Integration Control Strategy})
    \index{Simple Integration Control Strategy} \\ 
\hspace*{0.4in} Ramping Integration Control Strategy (Section~\ref{sec:Ramping Integration Control Strategy})
    \index{Ramping Integration Control Strategy} \\ 
\hspace*{0.2in} Stepper Settings (Section~\ref{sec:Stepper Settings})
    \index{Stepper Settings} \\ 
\hspace*{0.4in} Stepper Selection (Section~\ref{sec:Stepper Selection})
    \index{Stepper Selection} \\ 
\hspace*{0.6in} Forward Euler (Section~\ref{sec:Forward Euler})
    \index{Forward Euler} \\ 
\hspace*{0.6in} Backward Euler (Section~\ref{sec:Backward Euler})
    \index{Backward Euler} \\ 
\hspace*{0.6in} Implicit BDF (Section~\ref{sec:Implicit BDF})
    \index{Implicit BDF} \\ 
\hspace*{0.6in} Explicit RK (Section~\ref{sec:Explicit RK})
    \index{Explicit RK} \\ 
\hspace*{0.6in} Implicit RK (Section~\ref{sec:Implicit RK})
    \index{Implicit RK} \\ 
\hspace*{0.4in} Step Control Settings (Section~\ref{sec:Step Control Settings})
    \index{Step Control Settings} \\ 
\hspace*{0.6in} Step Control Strategy Selection (Section~\ref{sec:Step Control Strategy Selection})
    \index{Step Control Strategy Selection} \\ 
\hspace*{0.8in} Fixed Step Control Strategy (Section~\ref{sec:Fixed Step Control Strategy})
    \index{Fixed Step Control Strategy} \\ 
\hspace*{0.8in} Simple Step Control Strategy (Section~\ref{sec:Simple Step Control Strategy})
    \index{Simple Step Control Strategy} \\ 
\hspace*{0.8in} First Order Error Step Control Strategy (Section~\ref{sec:First Order Error Step Control Strategy})
    \index{First Order Error Step Control Strategy} \\ 
\hspace*{0.8in} Implicit BDF Stepper Step Control Strategy (Section~\ref{sec:Implicit BDF Stepper Step Control Strategy})
    \index{Implicit BDF Stepper Step Control Strategy} \\ 
\hspace*{1in} magicNumbers (Section~\ref{sec:magicNumbers})
    \index{magicNumbers} \\ 
\hspace*{0.8in} Implicit BDF Stepper Ramping Step Control Strategy (Section~\ref{sec:Implicit BDF Stepper Ramping Step Control Strategy})
    \index{Implicit BDF Stepper Ramping Step Control Strategy} \\ 
\hspace*{0.6in} Error Weight Vector Calculator Selection (Section~\ref{sec:Error Weight Vector Calculator Selection})
    \index{Error Weight Vector Calculator Selection} \\ 
\hspace*{0.8in} Implicit BDF Stepper Error Weight Vector Calculator (Section~\ref{sec:Implicit BDF Stepper Error Weight Vector Calculator})
    \index{Implicit BDF Stepper Error Weight Vector Calculator} \\ 
\hspace*{0.4in} Interpolator Selection (Section~\ref{sec:Interpolator Selection})
    \index{Interpolator Selection} \\ 
\hspace*{0.6in} Linear Interpolator (Section~\ref{sec:Linear Interpolator})
    \index{Linear Interpolator} \\ 
\hspace*{0.6in} Hermite Interpolator (Section~\ref{sec:Hermite Interpolator})
    \index{Hermite Interpolator} \\ 
\hspace*{0.6in} Cubic Spline Interpolator (Section~\ref{sec:Cubic Spline Interpolator})
    \index{Cubic Spline Interpolator} \\ 
\hspace*{0.4in} Runge Kutta Butcher Tableau Selection (Section~\ref{sec:Runge Kutta Butcher Tableau Selection})
    \index{Runge Kutta Butcher Tableau Selection} \\ 
\hspace*{0.6in} Forward Euler (Section~\ref{sec:Forward Euler-Runge Kutta Butcher Tableau Selection})
    \index{Forward Euler} \\ 
\hspace*{0.6in} Explicit 2 Stage 2nd order by Runge (Section~\ref{sec:Explicit 2 Stage 2nd order by Runge})
    \index{Explicit 2 Stage 2nd order by Runge} \\ 
\hspace*{0.6in} Explicit Trapezoidal (Section~\ref{sec:Explicit Trapezoidal})
    \index{Explicit Trapezoidal} \\ 
\hspace*{0.6in} Explicit 3 Stage 3rd order (Section~\ref{sec:Explicit 3 Stage 3rd order})
    \index{Explicit 3 Stage 3rd order} \\ 
\hspace*{0.6in} Explicit 3 Stage 3rd order by Heun (Section~\ref{sec:Explicit 3 Stage 3rd order by Heun})
    \index{Explicit 3 Stage 3rd order by Heun} \\ 
\hspace*{0.6in}  ... \\ 
\hspace*{0.2in} Interpolation Buffer Settings (Section~\ref{sec:Interpolation Buffer Settings})
    \index{Interpolation Buffer Settings} \\ 
\hspace*{0.4in} Trailing Interpolation Buffer Selection (Section~\ref{sec:Trailing Interpolation Buffer Selection})
    \index{Trailing Interpolation Buffer Selection} \\ 
\hspace*{0.6in} Interpolation Buffer (Section~\ref{sec:Interpolation Buffer})
    \index{Interpolation Buffer} \\ 
\hspace*{0.4in} Interpolation Buffer Appender Selection (Section~\ref{sec:Interpolation Buffer Appender Selection})
    \index{Interpolation Buffer Appender Selection} \\ 
\hspace*{0.6in} Pointwise Interpolation Buffer Appender (Section~\ref{sec:Pointwise Interpolation Buffer Appender})
    \index{Pointwise Interpolation Buffer Appender} \\ 
\hspace*{0.4in} Interpolator Selection (Section~\ref{sec:Interpolator Selection-Interpolation Buffer Settings})
    \index{Interpolator Selection} \\ 
\hspace*{0.6in} Linear Interpolator (Section~\ref{sec:Linear Interpolator-Interpolator Selection})
    \index{Linear Interpolator} \\ 
\hspace*{0.6in} Hermite Interpolator (Section~\ref{sec:Hermite Interpolator-Interpolator Selection})
    \index{Hermite Interpolator} \\ 
\hspace*{0.6in} Cubic Spline Interpolator (Section~\ref{sec:Cubic Spline Interpolator-Interpolator Selection})
    \index{Cubic Spline Interpolator} \\ 
  \end{tabular}
\caption{Schematic of ParameterList hierarchy.} 
\label{fig:ParameterList-schematic} 
\end{figure} 
\newpage 

% ----------------------------------------------------------
\subsection{Integrator Base}
\label{sec:Integrator Base}
\index{Integrator Base}

\begin{list}{}
  {\setlength{\leftmargin}{1.0in}
   \setlength{\labelwidth}{0.75in}
   \setlength{\labelsep}{0.125in}}
  \item[Description:]
    The root ParameterList for the Integrator Builder.
  \item[Parent(s):]
   ROOT 
  \item[Child(ren):]
    Integrator Settings (Section~\ref{sec:Integrator Settings})
      \index{Integrator Settings} 
      \newline 
    Integration Control Strategy Selection (Section~\ref{sec:Integration Control Strategy Selection})
      \index{Integration Control Strategy Selection} 
      \newline 
    Stepper Settings (Section~\ref{sec:Stepper Settings})
      \index{Stepper Settings} 
      \newline 
    Interpolation Buffer Settings (Section~\ref{sec:Interpolation Buffer Settings})
      \index{Interpolation Buffer Settings} 
  \item[Parameters:]
    None. 
\end{list}

% ----------------------------------------------------------
\subsection{Integrator Settings}
\label{sec:Integrator Settings}
\index{Integrator Settings}

\begin{list}{}
  {\setlength{\leftmargin}{1.0in}
   \setlength{\labelwidth}{0.75in}
   \setlength{\labelsep}{0.125in}}
  \item[Description:]
    These parameters are used directly in setting up the Integrator.
  \item[Parent(s):]
    Integrator Base (Section~\ref{sec:Integrator Base})
      \index{Integrator Base} 
  \item[Child(ren):]
    Integrator Selection (Section~\ref{sec:Integrator Selection})
      \index{Integrator Selection} 
  \item[Parameters:]
    \begin{description}
      \item[Initial Time = 0] 
The initial time to start integration.
        \index{Integrator Settings!Initial Time}
        \index{Initial Time}
      \item[Final Time = 1] 
The final time to end integration.
        \index{Integrator Settings!Final Time}
        \index{Final Time}
      \item[Land On Final Time = 1] 
Exactly land on the final time; do not step past final time and interpolate.
        \index{Integrator Settings!Land On Final Time}
        \index{Land On Final Time}
\end{description}

\end{list}

% ----------------------------------------------------------
\subsection{Integrator Selection}
\label{sec:Integrator Selection}
\index{Integrator Selection}

\begin{list}{}
  {\setlength{\leftmargin}{1.0in}
   \setlength{\labelwidth}{0.75in}
   \setlength{\labelsep}{0.125in}}
  \item[Description:]
    Select the Integrator to be used.
  \item[Parent(s):]
    Integrator Settings (Section~\ref{sec:Integrator Settings})
      \index{Integrator Settings} 
  \item[Child(ren):]
    Default Integrator (Section~\ref{sec:Default Integrator})
      \index{Default Integrator} 
  \item[Parameters:]
    \begin{description}
      \item[Integrator Type = Default Integrator] 
Determines the type of Rythmos::Integrator object that will be built.
The parameters for each Integrator Type are specified in this sublist

  Valid std::string values:

      \begin{tabular}{lp{0.4\textwidth}}
      "None" & \\ 
      "Default Integrator" & \\ 
      \end{tabular}
        \index{Integrator Selection!Integrator Type}
        \index{Integrator Type}
\end{description}

\end{list}

% ----------------------------------------------------------
\subsection{Default Integrator}
\label{sec:Default Integrator}
\index{Default Integrator}

\begin{list}{}
  {\setlength{\leftmargin}{1.0in}
   \setlength{\labelwidth}{0.75in}
   \setlength{\labelsep}{0.125in}}
  \item[Description:]
    This Integrator will accept an IntergationControlStrategy, and can have an IntegrationObserver.  The client can specify the maximum number of time steps allowed.  The Integrator will loop over the Stepper until it reaches the requested time. For each step, the step size will be determined through a couple mechanisms/filters.  If an Integration Control Strategy has been specified, the step size and the step type (fixed or variable) will be determined by it.  Otherwise the step size will be set to the maximum real value and the step type will be variable.  Next if the step size is beyond the final time and the `Land On Final Time' is specified, the step size is adjusted to advance the state to the final time.  The Stepper is passed this step size and type to advance the state.  The DefaultIntegrator determines the step size and type taken by the Stepper, and if the step has failed.  If the IntegrationControlStrategy handles failures, it can suggest another step size and retry with the Stepper.  Otherwise, the Integrator will fall through with a failure.  With a successful step of the Stepper, the Integrator repeats the above until it reaches the requested time.  Multiple requested times can be passed to the Integrator.
  \item[Parent(s):]
    Integrator Selection (Section~\ref{sec:Integrator Selection})
      \index{Integrator Selection} 
  \item[Child(ren):]
    VerboseObject (Section~\ref{sec:VerboseObject})
      \index{VerboseObject} 
  \item[Parameters:]
    \begin{description}
      \item[Max Number Time Steps = 2147483647] 
Set the maximum number of integration time-steps allowed.
        \index{Default Integrator!Max Number Time Steps}
        \index{Max Number Time Steps}
\end{description}

\end{list}

% ----------------------------------------------------------
\subsection{VerboseObject}
\label{sec:VerboseObject}
\index{VerboseObject}

\begin{list}{}
  {\setlength{\leftmargin}{1.0in}
   \setlength{\labelwidth}{0.75in}
   \setlength{\labelsep}{0.125in}}
  \item[Description:]
  \item[Parent(s):]
    Default Integrator (Section~\ref{sec:Default Integrator})
      \index{Default Integrator} 
      \newline 
    Simple Integration Control Strategy (Section~\ref{sec:Simple Integration Control Strategy})
      \index{Simple Integration Control Strategy} 
      \newline 
    Ramping Integration Control Strategy (Section~\ref{sec:Ramping Integration Control Strategy})
      \index{Ramping Integration Control Strategy} 
      \newline 
    Forward Euler (Section~\ref{sec:Forward Euler})
      \index{Forward Euler} 
      \newline 
    Backward Euler (Section~\ref{sec:Backward Euler})
      \index{Backward Euler} 
      \newline 
    Implicit BDF (Section~\ref{sec:Implicit BDF})
      \index{Implicit BDF} 
      \newline 
    Explicit RK (Section~\ref{sec:Explicit RK})
      \index{Explicit RK} 
      \newline 
    Implicit RK (Section~\ref{sec:Implicit RK})
      \index{Implicit RK} 
      \newline 
    Fixed Step Control Strategy (Section~\ref{sec:Fixed Step Control Strategy})
      \index{Fixed Step Control Strategy} 
      \newline 
    Simple Step Control Strategy (Section~\ref{sec:Simple Step Control Strategy})
      \index{Simple Step Control Strategy} 
      \newline 
    First Order Error Step Control Strategy (Section~\ref{sec:First Order Error Step Control Strategy})
      \index{First Order Error Step Control Strategy} 
      \newline 
    magicNumbers (Section~\ref{sec:magicNumbers})
      \index{magicNumbers} 
      \newline 
    Implicit BDF Stepper Error Weight Vector Calculator (Section~\ref{sec:Implicit BDF Stepper Error Weight Vector Calculator})
      \index{Implicit BDF Stepper Error Weight Vector Calculator} 
      \newline 
    Linear Interpolator (Section~\ref{sec:Linear Interpolator})
      \index{Linear Interpolator} 
      \newline 
    Hermite Interpolator (Section~\ref{sec:Hermite Interpolator})
      \index{Hermite Interpolator} 
      \newline 
    Cubic Spline Interpolator (Section~\ref{sec:Cubic Spline Interpolator})
      \index{Cubic Spline Interpolator} 
      \newline 
    Forward Euler (Section~\ref{sec:Forward Euler-Runge Kutta Butcher Tableau Selection})
      \index{Forward Euler} 
      \newline 
    Explicit 2 Stage 2nd order by Runge (Section~\ref{sec:Explicit 2 Stage 2nd order by Runge})
      \index{Explicit 2 Stage 2nd order by Runge} 
      \newline 
    Explicit Trapezoidal (Section~\ref{sec:Explicit Trapezoidal})
      \index{Explicit Trapezoidal} 
      \newline 
    Explicit 3 Stage 3rd order (Section~\ref{sec:Explicit 3 Stage 3rd order})
      \index{Explicit 3 Stage 3rd order} 
      \newline 
    Explicit 3 Stage 3rd order by Heun (Section~\ref{sec:Explicit 3 Stage 3rd order by Heun})
      \index{Explicit 3 Stage 3rd order by Heun} 
      \newline 
    Explicit 3 Stage 3rd order TVD (Section~\ref{sec:Explicit 3 Stage 3rd order TVD})
      \index{Explicit 3 Stage 3rd order TVD} 
      \newline 
    Explicit 4 Stage 3rd order by Runge (Section~\ref{sec:Explicit 4 Stage 3rd order by Runge})
      \index{Explicit 4 Stage 3rd order by Runge} 
      \newline 
    Explicit 4 Stage (Section~\ref{sec:Explicit 4 Stage})
      \index{Explicit 4 Stage} 
      \newline 
    Explicit 3/8 Rule (Section~\ref{sec:Explicit 3/8 Rule})
      \index{Explicit 3/8 Rule} 
      \newline 
    Backward Euler (Section~\ref{sec:Backward Euler-Runge Kutta Butcher Tableau Selection})
      \index{Backward Euler} 
      \newline 
    IRK 1 Stage Theta Method (Section~\ref{sec:IRK 1 Stage Theta Method})
      \index{IRK 1 Stage Theta Method} 
      \newline 
    IRK 2 Stage Theta Method (Section~\ref{sec:IRK 2 Stage Theta Method})
      \index{IRK 2 Stage Theta Method} 
      \newline 
    Singly Diagonal IRK 2 Stage 2nd order (Section~\ref{sec:Singly Diagonal IRK 2 Stage 2nd order})
      \index{Singly Diagonal IRK 2 Stage 2nd order} 
      \newline 
    Singly Diagonal IRK 2 Stage 3rd order (Section~\ref{sec:Singly Diagonal IRK 2 Stage 3rd order})
      \index{Singly Diagonal IRK 2 Stage 3rd order} 
      \newline 
    Singly Diagonal IRK 3 Stage 4th order (Section~\ref{sec:Singly Diagonal IRK 3 Stage 4th order})
      \index{Singly Diagonal IRK 3 Stage 4th order} 
      \newline 
    Singly Diagonal IRK 5 Stage 4th order (Section~\ref{sec:Singly Diagonal IRK 5 Stage 4th order})
      \index{Singly Diagonal IRK 5 Stage 4th order} 
      \newline 
    Singly Diagonal IRK 5 Stage 5th order (Section~\ref{sec:Singly Diagonal IRK 5 Stage 5th order})
      \index{Singly Diagonal IRK 5 Stage 5th order} 
      \newline 
    Diagonal IRK 2 Stage 3rd order (Section~\ref{sec:Diagonal IRK 2 Stage 3rd order})
      \index{Diagonal IRK 2 Stage 3rd order} 
      \newline 
    Implicit 1 Stage 2nd order Gauss (Section~\ref{sec:Implicit 1 Stage 2nd order Gauss})
      \index{Implicit 1 Stage 2nd order Gauss} 
      \newline 
    Implicit 2 Stage 4th order Gauss (Section~\ref{sec:Implicit 2 Stage 4th order Gauss})
      \index{Implicit 2 Stage 4th order Gauss} 
      \newline 
    Implicit 3 Stage 6th order Gauss (Section~\ref{sec:Implicit 3 Stage 6th order Gauss})
      \index{Implicit 3 Stage 6th order Gauss} 
      \newline 
    Implicit 2 Stage 4th Order Hammer \& Hollingsworth (Section~\ref{sec:Implicit 2 Stage 4th Order Hammer and Hollingsworth})
      \index{Implicit 2 Stage 4th Order Hammer \& Hollingsworth} 
      \newline 
    Implicit 3 Stage 6th Order Kuntzmann \& Butcher (Section~\ref{sec:Implicit 3 Stage 6th Order Kuntzmann and Butcher})
      \index{Implicit 3 Stage 6th Order Kuntzmann \& Butcher} 
      \newline 
    Implicit 1 Stage 1st order Radau left (Section~\ref{sec:Implicit 1 Stage 1st order Radau left})
      \index{Implicit 1 Stage 1st order Radau left} 
      \newline 
    Implicit 2 Stage 3rd order Radau left (Section~\ref{sec:Implicit 2 Stage 3rd order Radau left})
      \index{Implicit 2 Stage 3rd order Radau left} 
      \newline 
    Implicit 3 Stage 5th order Radau left (Section~\ref{sec:Implicit 3 Stage 5th order Radau left})
      \index{Implicit 3 Stage 5th order Radau left} 
      \newline 
    Implicit 1 Stage 1st order Radau right (Section~\ref{sec:Implicit 1 Stage 1st order Radau right})
      \index{Implicit 1 Stage 1st order Radau right} 
      \newline 
    Implicit 2 Stage 3rd order Radau right (Section~\ref{sec:Implicit 2 Stage 3rd order Radau right})
      \index{Implicit 2 Stage 3rd order Radau right} 
      \newline 
    Implicit 3 Stage 5th order Radau right (Section~\ref{sec:Implicit 3 Stage 5th order Radau right})
      \index{Implicit 3 Stage 5th order Radau right} 
      \newline 
    Implicit 2 Stage 2nd order Lobatto A (Section~\ref{sec:Implicit 2 Stage 2nd order Lobatto A})
      \index{Implicit 2 Stage 2nd order Lobatto A} 
      \newline 
    Implicit 3 Stage 4th order Lobatto A (Section~\ref{sec:Implicit 3 Stage 4th order Lobatto A})
      \index{Implicit 3 Stage 4th order Lobatto A} 
      \newline 
    Implicit 4 Stage 6th order Lobatto A (Section~\ref{sec:Implicit 4 Stage 6th order Lobatto A})
      \index{Implicit 4 Stage 6th order Lobatto A} 
      \newline 
    Implicit 2 Stage 2nd order Lobatto B (Section~\ref{sec:Implicit 2 Stage 2nd order Lobatto B})
      \index{Implicit 2 Stage 2nd order Lobatto B} 
      \newline 
    Implicit 3 Stage 4th order Lobatto B (Section~\ref{sec:Implicit 3 Stage 4th order Lobatto B})
      \index{Implicit 3 Stage 4th order Lobatto B} 
      \newline 
    Implicit 4 Stage 6th order Lobatto B (Section~\ref{sec:Implicit 4 Stage 6th order Lobatto B})
      \index{Implicit 4 Stage 6th order Lobatto B} 
      \newline 
    Implicit 2 Stage 2nd order Lobatto C (Section~\ref{sec:Implicit 2 Stage 2nd order Lobatto C})
      \index{Implicit 2 Stage 2nd order Lobatto C} 
      \newline 
    Implicit 3 Stage 4th order Lobatto C (Section~\ref{sec:Implicit 3 Stage 4th order Lobatto C})
      \index{Implicit 3 Stage 4th order Lobatto C} 
      \newline 
    Implicit 4 Stage 6th order Lobatto C (Section~\ref{sec:Implicit 4 Stage 6th order Lobatto C})
      \index{Implicit 4 Stage 6th order Lobatto C} 
      \newline 
    Interpolation Buffer (Section~\ref{sec:Interpolation Buffer})
      \index{Interpolation Buffer} 
      \newline 
    Pointwise Interpolation Buffer Appender (Section~\ref{sec:Pointwise Interpolation Buffer Appender})
      \index{Pointwise Interpolation Buffer Appender} 
      \newline 
    Linear Interpolator (Section~\ref{sec:Linear Interpolator-Interpolator Selection})
      \index{Linear Interpolator} 
      \newline 
    Hermite Interpolator (Section~\ref{sec:Hermite Interpolator-Interpolator Selection})
      \index{Hermite Interpolator} 
      \newline 
    Cubic Spline Interpolator (Section~\ref{sec:Cubic Spline Interpolator-Interpolator Selection})
      \index{Cubic Spline Interpolator} 
  \item[Child(ren):]
    None. 
  \item[Parameters:]
    \begin{description}
      \item[Verbosity Level = default] 
The verbosity level to use to override whatever is set in code.
The value of "default" will allow the level set in code to be used.

  Valid std::string values:

      \begin{tabular}{lp{0.4\textwidth}}
      "default" & Use level set in code \\ 
      "none" & Produce no output \\ 
      "low" & Produce minimal output \\ 
      "medium" & Produce a little more output \\ 
      "high" & Produce a higher level of output \\ 
      "extreme" & Produce the highest level of output \\ 
      \end{tabular}
        \index{VerboseObject!Verbosity Level}
        \index{Verbosity Level}
      \item[Output File = none] 
The file to send output to.  If the value "none" is used, then
whatever is set in code will be used.  However, any other std::string value
will be used to create an std::ofstream object to a file with the given name.
Therefore, any valid file name is a valid std::string value for this parameter.
        \index{VerboseObject!Output File}
        \index{Output File}
\end{description}

\end{list}

% ----------------------------------------------------------
\subsection{Integration Control Strategy Selection}
\label{sec:Integration Control Strategy Selection}
\index{Integration Control Strategy Selection}

\begin{list}{}
  {\setlength{\leftmargin}{1.0in}
   \setlength{\labelwidth}{0.75in}
   \setlength{\labelsep}{0.125in}}
  \item[Description:]
    Note that some settings conflict between step control and integration control.  In general, the integration control decides which steps will be fixed or variable, not the stepper.  When the integration control decides to take variable steps, the step control is then responsible for choosing appropriate step-sizes.
  \item[Parent(s):]
    Integrator Base (Section~\ref{sec:Integrator Base})
      \index{Integrator Base} 
  \item[Child(ren):]
    Simple Integration Control Strategy (Section~\ref{sec:Simple Integration Control Strategy})
      \index{Simple Integration Control Strategy} 
      \newline 
    Ramping Integration Control Strategy (Section~\ref{sec:Ramping Integration Control Strategy})
      \index{Ramping Integration Control Strategy} 
  \item[Parameters:]
    \begin{description}
      \item[Integration Control Strategy Type = None] 
Determines the type of Rythmos::IntegrationControlStrategy object that will be built.
The parameters for each Integration Control Strategy Type are specified in this sublist

  Valid std::string values:

      \begin{tabular}{lp{0.4\textwidth}}
      "None" & \\ 
      "Simple Integration Control Strategy" & \\ 
      "Ramping Integration Control Strategy" & \\ 
      \end{tabular}
        \index{Integration Control Strategy Selection!Integration Control Strategy Type}
        \index{Integration Control Strategy Type}
\end{description}

\end{list}

% ----------------------------------------------------------
\subsection{Simple Integration Control Strategy}
\label{sec:Simple Integration Control Strategy}
\index{Simple Integration Control Strategy}

\begin{list}{}
  {\setlength{\leftmargin}{1.0in}
   \setlength{\labelwidth}{0.75in}
   \setlength{\labelsep}{0.125in}}
  \item[Description:]
    This Integration Control Strategy is meant to be simple with very few parameters controlling it.  Basically the client can select fixed step type (the Stepper can only take the requested step size) or variable step type (the Stepper can adjust the step size to meet accuracy, order, or other criteria).  For fixed step type, the client can specify the step size and number of steps. For variable step type, the client can set the maximum step size allowable.
  \item[Parent(s):]
    Integration Control Strategy Selection (Section~\ref{sec:Integration Control Strategy Selection})
      \index{Integration Control Strategy Selection} 
  \item[Child(ren):]
    None. 
  \item[Parameters:]
    \begin{description}
      \item[Take Variable Steps = 1] 
Take variable time steps or fixed time steps.
If set to false, then the parameter "Fixed dt"
or "Number of Time Steps" must be set!
        \index{Simple Integration Control Strategy!Take Variable Steps}
        \index{Take Variable Steps}
      \item[Max dt = 1.79769e+308] 
Gives the max size of the variable time steps.  This is only read and used if
"Take Variable Steps" is set to true.
        \index{Simple Integration Control Strategy!Max dt}
        \index{Max dt}
      \item[Number of Time Steps = -1] 
Gives the number of fixed time steps.  The actual step size gets computed
on the fly given the size of the time domain.
This is only read and used if "Take Variable Steps" is set to false
and "Fixed dt" is set to $<$ 0.0.
        \index{Simple Integration Control Strategy!Number of Time Steps}
        \index{Number of Time Steps}
      \item[Fixed dt = -1] 
Gives the size of the fixed time steps.  This is only read and used if
"Take Variable Steps" is set to false.
        \index{Simple Integration Control Strategy!Fixed dt}
        \index{Fixed dt}
\end{description}

\end{list}

% ----------------------------------------------------------
\subsection{Ramping Integration Control Strategy}
\label{sec:Ramping Integration Control Strategy}
\index{Ramping Integration Control Strategy}

\begin{list}{}
  {\setlength{\leftmargin}{1.0in}
   \setlength{\labelwidth}{0.75in}
   \setlength{\labelsep}{0.125in}}
  \item[Description:]
    This Integration Control Strategy is very similar to `Simple Integration Control Strategy' except for handling an initial constant-sized steps followed by a ramped-fixed-sized steps, and finally variable- or fixed-sized steps.  The client needs to additionally set the initial step size and the maximum number of step failures allowed.
  \item[Parent(s):]
    Integration Control Strategy Selection (Section~\ref{sec:Integration Control Strategy Selection})
      \index{Integration Control Strategy Selection} 
  \item[Child(ren):]
    None. 
  \item[Parameters:]
    \begin{description}
      \item[Take Variable Steps = 1] 
Take variable time steps after 'Number of Initial Constant Steps' plus 'Number of Ramping Steps' steps.  Variable time stepping allows the Stepper to adjust the time step through a StepControlStrategy after fixed time steps during initial constant steps and ramping steps.  If false, fixed-time steps are taken after ramping.  Fixed time stepping requires the Stepper to take the time step set by this IntegrationControlStrategy.
        \index{Ramping Integration Control Strategy!Take Variable Steps}
        \index{Take Variable Steps}
      \item[Number of Initial Constant Steps = 0] 
Number of initial constant steps to take before starting the ramping.
        \index{Ramping Integration Control Strategy!Number of Initial Constant Steps}
        \index{Number of Initial Constant Steps}
      \item[Number of Ramping Steps = 6] 
Number of ramping steps to take before handing control to variable stepper if 'Take Variable Steps' is set to true.  Otherwise take fixed-time steps.
        \index{Ramping Integration Control Strategy!Number of Ramping Steps}
        \index{Number of Ramping Steps}
      \item[Initial dt = -1] 
Initial time step.
        \index{Ramping Integration Control Strategy!Initial dt}
        \index{Initial dt}
      \item[Min dt = 2.22507e-308] 
Minimum time step.
        \index{Ramping Integration Control Strategy!Min dt}
        \index{Min dt}
      \item[Max dt = 1.79769e+308] 
Maximum time step.
        \index{Ramping Integration Control Strategy!Max dt}
        \index{Max dt}
      \item[Ramping Factor = 1] 
Time step growth factor used during ramping phase. dt\_{n+1} = (ramping factor) * dt\_n
        \index{Ramping Integration Control Strategy!Ramping Factor}
        \index{Ramping Factor}
      \item[Maximum Number of Step Failures = 100] 
The maximum number of step failures before exiting with error.
        \index{Ramping Integration Control Strategy!Maximum Number of Step Failures}
        \index{Maximum Number of Step Failures}
\end{description}

\end{list}

% ----------------------------------------------------------
\subsection{Stepper Settings}
\label{sec:Stepper Settings}
\index{Stepper Settings}

\begin{list}{}
  {\setlength{\leftmargin}{1.0in}
   \setlength{\labelwidth}{0.75in}
   \setlength{\labelsep}{0.125in}}
  \item[Description:]
    This parameter list sets various parameters for the Stepper.
  \item[Parent(s):]
    Integrator Base (Section~\ref{sec:Integrator Base})
      \index{Integrator Base} 
  \item[Child(ren):]
    Stepper Selection (Section~\ref{sec:Stepper Selection})
      \index{Stepper Selection} 
      \newline 
    Step Control Settings (Section~\ref{sec:Step Control Settings})
      \index{Step Control Settings} 
      \newline 
    Interpolator Selection (Section~\ref{sec:Interpolator Selection})
      \index{Interpolator Selection} 
      \newline 
    Runge Kutta Butcher Tableau Selection (Section~\ref{sec:Runge Kutta Butcher Tableau Selection})
      \index{Runge Kutta Butcher Tableau Selection} 
  \item[Parameters:]
    None. 
\end{list}

% ----------------------------------------------------------
\subsection{Stepper Selection}
\label{sec:Stepper Selection}
\index{Stepper Selection}

\begin{list}{}
  {\setlength{\leftmargin}{1.0in}
   \setlength{\labelwidth}{0.75in}
   \setlength{\labelsep}{0.125in}}
  \item[Description:]
    Selects the Stepper for the time integration.  It should be that some time integrators can be accessed through different Steppers, e.g., Backward Euler can be obtained through the `Backward Euler', a first-order `Implicit BDF', or a one-stage `Implicit RK' Stepper.Special note for `Implicit RK' Stepper:  If a fully implicit RK Butcher tableau is chosen, then the stepper will not be fully initialized unless a W factory object is set on the IntegratorBuilder through setWFactoryObject.
  \item[Parent(s):]
    Stepper Settings (Section~\ref{sec:Stepper Settings})
      \index{Stepper Settings} 
  \item[Child(ren):]
    Forward Euler (Section~\ref{sec:Forward Euler})
      \index{Forward Euler} 
      \newline 
    Backward Euler (Section~\ref{sec:Backward Euler})
      \index{Backward Euler} 
      \newline 
    Implicit BDF (Section~\ref{sec:Implicit BDF})
      \index{Implicit BDF} 
      \newline 
    Explicit RK (Section~\ref{sec:Explicit RK})
      \index{Explicit RK} 
      \newline 
    Implicit RK (Section~\ref{sec:Implicit RK})
      \index{Implicit RK} 
  \item[Parameters:]
    \begin{description}
      \item[Stepper Type = Backward Euler] 
Determines the type of Rythmos::Stepper object that will be built.
The parameters for each Stepper Type are specified in this sublist

  Valid std::string values:

      \begin{tabular}{lp{0.4\textwidth}}
      "None" & \\ 
      "Forward Euler" & \\ 
      "Backward Euler" & \\ 
      "Implicit BDF" & \\ 
      "Explicit RK" & \\ 
      "Implicit RK" & \\ 
      \end{tabular}
        \index{Stepper Selection!Stepper Type}
        \index{Stepper Type}
\end{description}

\end{list}

% ----------------------------------------------------------
\subsection{Forward Euler}
\label{sec:Forward Euler}
\index{Forward Euler}

\begin{list}{}
  {\setlength{\leftmargin}{1.0in}
   \setlength{\labelwidth}{0.75in}
   \setlength{\labelsep}{0.125in}}
  \item[Description:]
    This is the basic Forward Euler method: x\_n = x\_{n-1} + dt*x\_dot\_{n-1}
  \item[Parent(s):]
    Stepper Selection (Section~\ref{sec:Stepper Selection})
      \index{Stepper Selection} 
  \item[Child(ren):]
    None. 
  \item[Parameters:]
    None. 
\end{list}

% ----------------------------------------------------------
\subsection{Backward Euler}
\label{sec:Backward Euler}
\index{Backward Euler}

\begin{list}{}
  {\setlength{\leftmargin}{1.0in}
   \setlength{\labelwidth}{0.75in}
   \setlength{\labelsep}{0.125in}}
  \item[Description:]
    This is the basic Backward Euler method: x\_n = x\_{n-1} + dt*x\_dot\_n
  \item[Parent(s):]
    Stepper Selection (Section~\ref{sec:Stepper Selection})
      \index{Stepper Selection} 
  \item[Child(ren):]
    None. 
  \item[Parameters:]
    None. 
\end{list}

% ----------------------------------------------------------
\subsection{Implicit BDF}
\label{sec:Implicit BDF}
\index{Implicit BDF}

\begin{list}{}
  {\setlength{\leftmargin}{1.0in}
   \setlength{\labelwidth}{0.75in}
   \setlength{\labelsep}{0.125in}}
  \item[Description:]
    This Stepper provides a re-implementation of the algorithms in the LLNL Sundials code IDA. This is an implicit BDF integrator for DAEs which uses variable step-sizes and variable-orders first through fourth.
  \item[Parent(s):]
    Stepper Selection (Section~\ref{sec:Stepper Selection})
      \index{Stepper Selection} 
  \item[Child(ren):]
    None. 
  \item[Parameters:]
    None. 
\end{list}

% ----------------------------------------------------------
\subsection{Explicit RK}
\label{sec:Explicit RK}
\index{Explicit RK}

\begin{list}{}
  {\setlength{\leftmargin}{1.0in}
   \setlength{\labelwidth}{0.75in}
   \setlength{\labelsep}{0.125in}}
  \item[Description:]
    This Stepper has many explicit time-integrators using Runge-Kutta formulation and the Butcher Tableau specification.  See `Runge Kutta Butcher Tableau Selection' ParameterList for available options.
  \item[Parent(s):]
    Stepper Selection (Section~\ref{sec:Stepper Selection})
      \index{Stepper Selection} 
  \item[Child(ren):]
    None. 
  \item[Parameters:]
    None. 
\end{list}

% ----------------------------------------------------------
\subsection{Implicit RK}
\label{sec:Implicit RK}
\index{Implicit RK}

\begin{list}{}
  {\setlength{\leftmargin}{1.0in}
   \setlength{\labelwidth}{0.75in}
   \setlength{\labelsep}{0.125in}}
  \item[Description:]
    This Stepper has many implicit time-integrators using Runge-Kutta formulation and the Butcher Tableau specification.  See `Runge Kutta Butcher Tableau Selection' ParameterList for available options.
  \item[Parent(s):]
    Stepper Selection (Section~\ref{sec:Stepper Selection})
      \index{Stepper Selection} 
  \item[Child(ren):]
    None. 
  \item[Parameters:]
    None. 
\end{list}

% ----------------------------------------------------------
\subsection{Step Control Settings}
\label{sec:Step Control Settings}
\index{Step Control Settings}

\begin{list}{}
  {\setlength{\leftmargin}{1.0in}
   \setlength{\labelwidth}{0.75in}
   \setlength{\labelsep}{0.125in}}
  \item[Description:]
    Not all step control strategies are compatible with each stepper.  If the strategy has the name of a stepper in its name, then it only works with that stepper.
  \item[Parent(s):]
    Stepper Settings (Section~\ref{sec:Stepper Settings})
      \index{Stepper Settings} 
  \item[Child(ren):]
    Step Control Strategy Selection (Section~\ref{sec:Step Control Strategy Selection})
      \index{Step Control Strategy Selection} 
      \newline 
    Error Weight Vector Calculator Selection (Section~\ref{sec:Error Weight Vector Calculator Selection})
      \index{Error Weight Vector Calculator Selection} 
  \item[Parameters:]
    None. 
\end{list}

% ----------------------------------------------------------
\subsection{Step Control Strategy Selection}
\label{sec:Step Control Strategy Selection}
\index{Step Control Strategy Selection}

\begin{list}{}
  {\setlength{\leftmargin}{1.0in}
   \setlength{\labelwidth}{0.75in}
   \setlength{\labelsep}{0.125in}}
  \item[Description:]
    Used to select the Control Strategy for the stepper.
  \item[Parent(s):]
    Step Control Settings (Section~\ref{sec:Step Control Settings})
      \index{Step Control Settings} 
  \item[Child(ren):]
    Fixed Step Control Strategy (Section~\ref{sec:Fixed Step Control Strategy})
      \index{Fixed Step Control Strategy} 
      \newline 
    Simple Step Control Strategy (Section~\ref{sec:Simple Step Control Strategy})
      \index{Simple Step Control Strategy} 
      \newline 
    First Order Error Step Control Strategy (Section~\ref{sec:First Order Error Step Control Strategy})
      \index{First Order Error Step Control Strategy} 
      \newline 
    Implicit BDF Stepper Step Control Strategy (Section~\ref{sec:Implicit BDF Stepper Step Control Strategy})
      \index{Implicit BDF Stepper Step Control Strategy} 
      \newline 
    Implicit BDF Stepper Ramping Step Control Strategy (Section~\ref{sec:Implicit BDF Stepper Ramping Step Control Strategy})
      \index{Implicit BDF Stepper Ramping Step Control Strategy} 
  \item[Parameters:]
    \begin{description}
      \item[Step Control Strategy Type = None] 
Determines the type of Rythmos::StepControlStrategy object that will be built.
The parameters for each Step Control Strategy Type are specified in this sublist

  Valid std::string values:

      \begin{tabular}{lp{0.4\textwidth}}
      "None" & \\ 
      "Fixed Step Control Strategy" & \\ 
      "Simple Step Control Strategy" & \\ 
      "First Order Error Step Control Strategy" & \\ 
      "Implicit BDF Stepper Step Control Strategy" & \\ 
      "Implicit BDF Stepper Ramping Step Control Strategy" & \\ 
      \end{tabular}
        \index{Step Control Strategy Selection!Step Control Strategy Type}
        \index{Step Control Strategy Type}
\end{description}

\end{list}

% ----------------------------------------------------------
\subsection{Fixed Step Control Strategy}
\label{sec:Fixed Step Control Strategy}
\index{Fixed Step Control Strategy}

\begin{list}{}
  {\setlength{\leftmargin}{1.0in}
   \setlength{\labelwidth}{0.75in}
   \setlength{\labelsep}{0.125in}}
  \item[Description:]
    This Step Control Strategy can be used for Steppers setup for variable step type (a stepper that can adjust its step size based on accuracy, order or other criteria), but would like to make fixed step sizes or used fixed step size as its default.
  \item[Parent(s):]
    Step Control Strategy Selection (Section~\ref{sec:Step Control Strategy Selection})
      \index{Step Control Strategy Selection} 
  \item[Child(ren):]
    None. 
  \item[Parameters:]
    None. 
\end{list}

% ----------------------------------------------------------
\subsection{Simple Step Control Strategy}
\label{sec:Simple Step Control Strategy}
\index{Simple Step Control Strategy}

\begin{list}{}
  {\setlength{\leftmargin}{1.0in}
   \setlength{\labelwidth}{0.75in}
   \setlength{\labelsep}{0.125in}}
  \item[Description:]
    This Step Control Strategy starts with the initial step size, and simply increases or decreases the step size by the appropriate factor which is based on the change in the solution relative to the specified relative and absolute tolerances ($|$dx$|$ $<$ r*$|$x$|$ + a) and if solution status from the solver passes.  Additionally the step size is bounded by the minimum and maximum step size, and the stepper will fail if the step size fails more than the specified value.
  \item[Parent(s):]
    Step Control Strategy Selection (Section~\ref{sec:Step Control Strategy Selection})
      \index{Step Control Strategy Selection} 
  \item[Child(ren):]
    None. 
  \item[Parameters:]
    \begin{description}
      \item[Initial Step Size = 2.22507e-308] 
Initial step size.
        \index{Simple Step Control Strategy!Initial Step Size}
        \index{Initial Step Size}
      \item[Min Step Size = 2.22507e-308] 
Minimum step size.
        \index{Simple Step Control Strategy!Min Step Size}
        \index{Min Step Size}
      \item[Max Step Size = 1.79769e+308] 
Maximum step size.
        \index{Simple Step Control Strategy!Max Step Size}
        \index{Max Step Size}
      \item[Step Size Increase Factor = 1.5] 
Factor used to increase the step size after a successful step.
        \index{Simple Step Control Strategy!Step Size Increase Factor}
        \index{Step Size Increase Factor}
      \item[Step Size Decrease Factor = 0.5] 
Factor used to decrease the step size after a step failure.
        \index{Simple Step Control Strategy!Step Size Decrease Factor}
        \index{Step Size Decrease Factor}
      \item[Maximum Number of Step Failures = 100] 
The maximum number of step failures before exiting with an error.  The number of failure steps are carried between successful steps.
        \index{Simple Step Control Strategy!Maximum Number of Step Failures}
        \index{Maximum Number of Step Failures}
      \item[Solution Change Relative Tolerance = 1e-06] 
The allowable relative change in the solution for each step to pass.  The stepper solution status is also used to determine pass/fail.
        \index{Simple Step Control Strategy!Solution Change Relative Tolerance}
        \index{Solution Change Relative Tolerance}
      \item[Solution Change Absolute Tolerance = 1e-12] 
The allowable absolute change in the solution for each step to pass.  The stepper solution status is also used to determine pass/fail.
        \index{Simple Step Control Strategy!Solution Change Absolute Tolerance}
        \index{Solution Change Absolute Tolerance}
\end{description}

\end{list}

% ----------------------------------------------------------
\subsection{First Order Error Step Control Strategy}
\label{sec:First Order Error Step Control Strategy}
\index{First Order Error Step Control Strategy}

\begin{list}{}
  {\setlength{\leftmargin}{1.0in}
   \setlength{\labelwidth}{0.75in}
   \setlength{\labelsep}{0.125in}}
  \item[Description:]
    This Step Control Strategy produces a step size based on a first-order predictor (Forward Euler) and a first-order solution (Backward Euler) by by using a weight norm of the difference between the predicted and solution.  See Gresho and Sani, `Incompressible Flow and the Finite Element Method', Vol. 1, 1998, p. 268.
  \item[Parent(s):]
    Step Control Strategy Selection (Section~\ref{sec:Step Control Strategy Selection})
      \index{Step Control Strategy Selection} 
  \item[Child(ren):]
    None. 
  \item[Parameters:]
    \begin{description}
      \item[Initial Step Size = 1] 
Initial step size.
        \index{First Order Error Step Control Strategy!Initial Step Size}
        \index{Initial Step Size}
      \item[Min Step Size = 2.22507e-308] 
Minimum step size.
        \index{First Order Error Step Control Strategy!Min Step Size}
        \index{Min Step Size}
      \item[Max Step Size = 1.79769e+308] 
Maximum step size.
        \index{First Order Error Step Control Strategy!Max Step Size}
        \index{Max Step Size}
      \item[Max Step Size Increase Factor = 1.5] 
The maximum factor to increase the step size after a successful step.
        \index{First Order Error Step Control Strategy!Max Step Size Increase Factor}
        \index{Max Step Size Increase Factor}
      \item[Min Step Size Decrease Factor = 0.5] 
The minimum allowable factor to decrease the step size.  If the stepSizeFactor\_ is below this, the current step is considered a failure and retried with `Min Step Size Decrease Factor' applied.
        \index{First Order Error Step Control Strategy!Min Step Size Decrease Factor}
        \index{Min Step Size Decrease Factor}
      \item[Maximum Number of Step Failures = 100] 
The maximum number of step failures before exiting with an error.  The number of failure steps are carried between successful steps.
        \index{First Order Error Step Control Strategy!Maximum Number of Step Failures}
        \index{Maximum Number of Step Failures}
      \item[Error Relative Tolerance = 1e-06] 
The allowable relative change in the error (the difference between the solution at the end of the step and the predicted solution) for each step to pass.  The stepper solution status is also used to determine pass/fail.
        \index{First Order Error Step Control Strategy!Error Relative Tolerance}
        \index{Error Relative Tolerance}
      \item[Error Absolute Tolerance = 1e-12] 
The allowable absolute change in the error (the difference between the solution at the end of the step and the predicted solution) for each step to pass.  The stepper solution status is also used to determine pass/fail.
        \index{First Order Error Step Control Strategy!Error Absolute Tolerance}
        \index{Error Absolute Tolerance}
\end{description}

\end{list}

% ----------------------------------------------------------
\subsection{Implicit BDF Stepper Step Control Strategy}
\label{sec:Implicit BDF Stepper Step Control Strategy}
\index{Implicit BDF Stepper Step Control Strategy}

\begin{list}{}
  {\setlength{\leftmargin}{1.0in}
   \setlength{\labelwidth}{0.75in}
   \setlength{\labelsep}{0.125in}}
  \item[Description:]
    This Step Control Strategy is specifically for use with the `Implicit BDF' Stepper.  The parameters in this list and sublist are directly related to those available in SUNDIALS/IDA.  See Hindmarsh, `The PVODE and IDA Algorithms', 2000 for more details.
  \item[Parent(s):]
    Step Control Strategy Selection (Section~\ref{sec:Step Control Strategy Selection})
      \index{Step Control Strategy Selection} 
  \item[Child(ren):]
    magicNumbers (Section~\ref{sec:magicNumbers})
      \index{magicNumbers} 
  \item[Parameters:]
    \begin{description}
      \item[minOrder = 1] 
lower limit of order selection, guaranteed
        \index{Implicit BDF Stepper Step Control Strategy!minOrder}
        \index{minOrder}
      \item[maxOrder = 5] 
upper limit of order selection, does not guarantee this order
        \index{Implicit BDF Stepper Step Control Strategy!maxOrder}
        \index{maxOrder}
      \item[relErrTol = 0.0001] 
Relative tolerance value used in WRMS calculation.
        \index{Implicit BDF Stepper Step Control Strategy!relErrTol}
        \index{relErrTol}
      \item[absErrTol = 1e-06] 
Absolute tolerance value used in WRMS calculation.
        \index{Implicit BDF Stepper Step Control Strategy!absErrTol}
        \index{absErrTol}
      \item[constantStepSize = 0] 
Use constant (=0) or variable (=1) step sizes during BDF steps.
        \index{Implicit BDF Stepper Step Control Strategy!constantStepSize}
        \index{constantStepSize}
      \item[stopTime = 10] 
The final time for time integration.  This may be in conflict with the Integrator's final time.
        \index{Implicit BDF Stepper Step Control Strategy!stopTime}
        \index{stopTime}
      \item[failStepIfNonlinearSolveFails = 0] 
Power user command. Will force the function acceptStep() to return false even if the LET is acceptable.  Used to run with loose tolerances but enforce a correct nonlinear solution to the step.
        \index{Implicit BDF Stepper Step Control Strategy!failStepIfNonlinearSolveFails}
        \index{failStepIfNonlinearSolveFails}
\end{description}

\end{list}

% ----------------------------------------------------------
\subsection{magicNumbers}
\label{sec:magicNumbers}
\index{magicNumbers}

\begin{list}{}
  {\setlength{\leftmargin}{1.0in}
   \setlength{\labelwidth}{0.75in}
   \setlength{\labelsep}{0.125in}}
  \item[Description:]
    These are knobs in the algorithm that have been set to reasonable values using lots of testing and heuristics and some theory.
  \item[Parent(s):]
    Implicit BDF Stepper Step Control Strategy (Section~\ref{sec:Implicit BDF Stepper Step Control Strategy})
      \index{Implicit BDF Stepper Step Control Strategy} 
  \item[Child(ren):]
    None. 
  \item[Parameters:]
    \begin{description}
      \item[h0\_safety = 2] 
Safety factor on the initial step size for time-dependent DAEs.  Larger values will tend to reduce the initial step size.
        \index{magicNumbers!h0\_safety}
        \index{h0\_safety}
      \item[h0\_max\_factor = 0.001] 
Factor multiplier for the initial step size for non-time-dependent DAEs.
        \index{magicNumbers!h0\_max\_factor}
        \index{h0\_max\_factor}
      \item[h\_phase0\_incr = 2] 
Initial ramp-up in variable mode (stepSize multiplier).
        \index{magicNumbers!h\_phase0\_incr}
        \index{h\_phase0\_incr}
      \item[h\_max\_inv = 0] 
The inverse of the maximum time step (maxTimeStep). (Not used?)
        \index{magicNumbers!h\_max\_inv}
        \index{h\_max\_inv}
      \item[Tkm1\_Tk\_safety = 2] 
Used to help control the decrement of the order when the order is less than or equal to 2.  Larger values will tend to reduce the order from one step to the next.
        \index{magicNumbers!Tkm1\_Tk\_safety}
        \index{Tkm1\_Tk\_safety}
      \item[Tkp1\_Tk\_safety = 0.5] 
Used to control the increment of the order when the order is one.  Larger values tend to increase the order for the next step.
        \index{magicNumbers!Tkp1\_Tk\_safety}
        \index{Tkp1\_Tk\_safety}
      \item[r\_factor = 0.9] 
Used in rejectStep:  time step ratio multiplier.
        \index{magicNumbers!r\_factor}
        \index{r\_factor}
      \item[r\_safety = 2] 
Local error multiplier as part of time step ratio calculation.
        \index{magicNumbers!r\_safety}
        \index{r\_safety}
      \item[r\_fudge = 0.0001] 
Local error addition as part of time step ratio calculation.
        \index{magicNumbers!r\_fudge}
        \index{r\_fudge}
      \item[r\_min = 0.125] 
Used in rejectStep:  How much to cut step and lower bound for time step ratio.
        \index{magicNumbers!r\_min}
        \index{r\_min}
      \item[r\_max = 0.9] 
Upper bound for time step ratio.
        \index{magicNumbers!r\_max}
        \index{r\_max}
      \item[r\_hincr\_test = 2] 
Used in completeStep:  If time step ratio $>$ this then set time step ratio to r\_hincr.
        \index{magicNumbers!r\_hincr\_test}
        \index{r\_hincr\_test}
      \item[r\_hincr = 2] 
Used in completeStep:  Limit on time step ratio increases, not checked by r\_max.
        \index{magicNumbers!r\_hincr}
        \index{r\_hincr}
      \item[max\_LET\_fail = 15] 
Max number of rejected steps
        \index{magicNumbers!max\_LET\_fail}
        \index{max\_LET\_fail}
      \item[minTimeStep = 0] 
Bound on smallest time step in variable mode.
        \index{magicNumbers!minTimeStep}
        \index{minTimeStep}
      \item[maxTimeStep = 10] 
bound on largest time step in variable mode.
        \index{magicNumbers!maxTimeStep}
        \index{maxTimeStep}
\end{description}

\end{list}

% ----------------------------------------------------------
\subsection{Implicit BDF Stepper Ramping Step Control Strategy}
\label{sec:Implicit BDF Stepper Ramping Step Control Strategy}
\index{Implicit BDF Stepper Ramping Step Control Strategy}

\begin{list}{}
  {\setlength{\leftmargin}{1.0in}
   \setlength{\labelwidth}{0.75in}
   \setlength{\labelsep}{0.125in}}
  \item[Description:]
    This Step Control Strategy is specifically for use with the `Implicit BDF' Stepper, and has a two-phase approach: constant step sizes and followed by variable step sizes.  The step size is adjusted based on the WRMS, see Implicit BDF Stepper Error Weight Vector Calculator
  \item[Parent(s):]
    Step Control Strategy Selection (Section~\ref{sec:Step Control Strategy Selection})
      \index{Step Control Strategy Selection} 
  \item[Child(ren):]
    None. 
  \item[Parameters:]
    \begin{description}
      \item[Number of Constant First Order Steps = 10] 
Number of constant steps to take before handing control to variable stepper.
        \index{Implicit BDF Stepper Ramping Step Control Strategy!Number of Constant First Order Steps}
        \index{Number of Constant First Order Steps}
      \item[Initial Step Size = 0.001] 
Initial time step size and target step size to take during the initial constant step phase (could be reduced due to step failures).
        \index{Implicit BDF Stepper Ramping Step Control Strategy!Initial Step Size}
        \index{Initial Step Size}
      \item[Min Step Size = 1e-07] 
Minimum time step size.
        \index{Implicit BDF Stepper Ramping Step Control Strategy!Min Step Size}
        \index{Min Step Size}
      \item[Max Step Size = 1] 
Maximum time step size.
        \index{Implicit BDF Stepper Ramping Step Control Strategy!Max Step Size}
        \index{Max Step Size}
      \item[Step Size Increase Factor = 1.2] 
Time step growth factor used after a successful time step. dt\_{n+1} = (increase factor) * dt\_n
        \index{Implicit BDF Stepper Ramping Step Control Strategy!Step Size Increase Factor}
        \index{Step Size Increase Factor}
      \item[Step Size Decrease Factor = 0.5] 
Time step reduction factor used for a failed time step. dt\_{n+1} = (decrease factor) * dt\_n
        \index{Implicit BDF Stepper Ramping Step Control Strategy!Step Size Decrease Factor}
        \index{Step Size Decrease Factor}
      \item[Min Order = 1] 
Minimum order to run at.
        \index{Implicit BDF Stepper Ramping Step Control Strategy!Min Order}
        \index{Min Order}
      \item[Max Order = 5] 
Maximum order to run at.
        \index{Implicit BDF Stepper Ramping Step Control Strategy!Max Order}
        \index{Max Order}
      \item[Absolute Error Tolerance = 1e-05] 
abstol value used in WRMS calculation.
        \index{Implicit BDF Stepper Ramping Step Control Strategy!Absolute Error Tolerance}
        \index{Absolute Error Tolerance}
      \item[Relative Error Tolerance = 0.001] 
reltol value used in WRMS calculation.
        \index{Implicit BDF Stepper Ramping Step Control Strategy!Relative Error Tolerance}
        \index{Relative Error Tolerance}
      \item[Use LET To Determine Step Acceptance = FALSE] 
If set to TRUE, then acceptance of step depends on LET in addition to Nonlinear solver converging.

  Valid std::string values:

      \begin{tabular}{lp{0.4\textwidth}}
      "TRUE" & \\ 
      "FALSE" & \\ 
      \end{tabular}
        \index{Implicit BDF Stepper Ramping Step Control Strategy!Use LET To Determine Step Acceptance}
        \index{Use LET To Determine Step Acceptance}
\end{description}

\end{list}

% ----------------------------------------------------------
\subsection{Error Weight Vector Calculator Selection}
\label{sec:Error Weight Vector Calculator Selection}
\index{Error Weight Vector Calculator Selection}

\begin{list}{}
  {\setlength{\leftmargin}{1.0in}
   \setlength{\labelwidth}{0.75in}
   \setlength{\labelsep}{0.125in}}
  \item[Description:]
    Not all ErrWtVec calculators are compatible with each step control strategy.  If the calculator has the name of a stepper or another step control strategy in its name, then it only works with that step control strategy.
  \item[Parent(s):]
    Step Control Settings (Section~\ref{sec:Step Control Settings})
      \index{Step Control Settings} 
  \item[Child(ren):]
    Implicit BDF Stepper Error Weight Vector Calculator (Section~\ref{sec:Implicit BDF Stepper Error Weight Vector Calculator})
      \index{Implicit BDF Stepper Error Weight Vector Calculator} 
  \item[Parameters:]
    \begin{description}
      \item[Error Weight Vector Calculator Type = None] 
Determines the type of Rythmos::ErrWtVecCalc object that will be built.
The parameters for each Error Weight Vector Calculator Type are specified in this sublist

  Valid std::string values:

      \begin{tabular}{lp{0.4\textwidth}}
      "None" & \\ 
      "Implicit BDF Stepper Error Weight Vector Calculator" & \\ 
      \end{tabular}
        \index{Error Weight Vector Calculator Selection!Error Weight Vector Calculator Type}
        \index{Error Weight Vector Calculator Type}
\end{description}

\end{list}

% ----------------------------------------------------------
\subsection{Implicit BDF Stepper Error Weight Vector Calculator}
\label{sec:Implicit BDF Stepper Error Weight Vector Calculator}
\index{Implicit BDF Stepper Error Weight Vector Calculator}

\begin{list}{}
  {\setlength{\leftmargin}{1.0in}
   \setlength{\labelwidth}{0.75in}
   \setlength{\labelsep}{0.125in}}
  \item[Description:]
    This Error Weight Vector Calculator is specifically for use with the `Implicit BDF' Stepper.
  \item[Parent(s):]
    Error Weight Vector Calculator Selection (Section~\ref{sec:Error Weight Vector Calculator Selection})
      \index{Error Weight Vector Calculator Selection} 
  \item[Child(ren):]
    None. 
  \item[Parameters:]
    None. 
\end{list}

% ----------------------------------------------------------
\subsection{Interpolator Selection}
\label{sec:Interpolator Selection}
\index{Interpolator Selection}

\begin{list}{}
  {\setlength{\leftmargin}{1.0in}
   \setlength{\labelwidth}{0.75in}
   \setlength{\labelsep}{0.125in}}
  \item[Description:]
    Note all Steppers accept an interpolator.  Currently, only the BackwardEuler stepper does.
  \item[Parent(s):]
    Stepper Settings (Section~\ref{sec:Stepper Settings})
      \index{Stepper Settings} 
  \item[Child(ren):]
    Linear Interpolator (Section~\ref{sec:Linear Interpolator})
      \index{Linear Interpolator} 
      \newline 
    Hermite Interpolator (Section~\ref{sec:Hermite Interpolator})
      \index{Hermite Interpolator} 
      \newline 
    Cubic Spline Interpolator (Section~\ref{sec:Cubic Spline Interpolator})
      \index{Cubic Spline Interpolator} 
  \item[Parameters:]
    \begin{description}
      \item[Interpolator Type = None] 
Determines the type of Rythmos::Interpolator object that will be built.
The parameters for each Interpolator Type are specified in this sublist

  Valid std::string values:

      \begin{tabular}{lp{0.4\textwidth}}
      "None" & \\ 
      "Linear Interpolator" & \\ 
      "Hermite Interpolator" & \\ 
      "Cubic Spline Interpolator" & \\ 
      \end{tabular}
        \index{Interpolator Selection!Interpolator Type}
        \index{Interpolator Type}
\end{description}

\end{list}

% ----------------------------------------------------------
\subsection{Linear Interpolator}
\label{sec:Linear Interpolator}
\index{Linear Interpolator}

\begin{list}{}
  {\setlength{\leftmargin}{1.0in}
   \setlength{\labelwidth}{0.75in}
   \setlength{\labelsep}{0.125in}}
  \item[Description:]
    This provides a simple linear interpolation between time nodes.
  \item[Parent(s):]
    Interpolator Selection (Section~\ref{sec:Interpolator Selection})
      \index{Interpolator Selection} 
  \item[Child(ren):]
    None. 
  \item[Parameters:]
    None. 
\end{list}

% ----------------------------------------------------------
\subsection{Hermite Interpolator}
\label{sec:Hermite Interpolator}
\index{Hermite Interpolator}

\begin{list}{}
  {\setlength{\leftmargin}{1.0in}
   \setlength{\labelwidth}{0.75in}
   \setlength{\labelsep}{0.125in}}
  \item[Description:]
    This provides a piecewise cubic Hermite interpolation on each interval where the data is the solution and its time derivatives at the end points of the interval.  It will match 3rd degree polynomials exactly with both function values and derivatives.
  \item[Parent(s):]
    Interpolator Selection (Section~\ref{sec:Interpolator Selection})
      \index{Interpolator Selection} 
  \item[Child(ren):]
    None. 
  \item[Parameters:]
    None. 
\end{list}

% ----------------------------------------------------------
\subsection{Cubic Spline Interpolator}
\label{sec:Cubic Spline Interpolator}
\index{Cubic Spline Interpolator}

\begin{list}{}
  {\setlength{\leftmargin}{1.0in}
   \setlength{\labelwidth}{0.75in}
   \setlength{\labelsep}{0.125in}}
  \item[Description:]
    This provides a cubic spline interpolation between time nodes.
  \item[Parent(s):]
    Interpolator Selection (Section~\ref{sec:Interpolator Selection})
      \index{Interpolator Selection} 
  \item[Child(ren):]
    None. 
  \item[Parameters:]
    None. 
\end{list}

% ----------------------------------------------------------
\subsection{Runge Kutta Butcher Tableau Selection}
\label{sec:Runge Kutta Butcher Tableau Selection}
\index{Runge Kutta Butcher Tableau Selection}

\begin{list}{}
  {\setlength{\leftmargin}{1.0in}
   \setlength{\labelwidth}{0.75in}
   \setlength{\labelsep}{0.125in}}
  \item[Description:]
    Only the Explicit RK Stepper and the Implicit RK Stepper accept an RK Butcher Tableau.
  \item[Parent(s):]
    Stepper Settings (Section~\ref{sec:Stepper Settings})
      \index{Stepper Settings} 
  \item[Child(ren):]
    Forward Euler (Section~\ref{sec:Forward Euler-Runge Kutta Butcher Tableau Selection})
      \index{Forward Euler} 
      \newline 
    Explicit 2 Stage 2nd order by Runge (Section~\ref{sec:Explicit 2 Stage 2nd order by Runge})
      \index{Explicit 2 Stage 2nd order by Runge} 
      \newline 
    Explicit Trapezoidal (Section~\ref{sec:Explicit Trapezoidal})
      \index{Explicit Trapezoidal} 
      \newline 
    Explicit 3 Stage 3rd order (Section~\ref{sec:Explicit 3 Stage 3rd order})
      \index{Explicit 3 Stage 3rd order} 
      \newline 
    Explicit 3 Stage 3rd order by Heun (Section~\ref{sec:Explicit 3 Stage 3rd order by Heun})
      \index{Explicit 3 Stage 3rd order by Heun} 
      \newline 
    Explicit 3 Stage 3rd order TVD (Section~\ref{sec:Explicit 3 Stage 3rd order TVD})
      \index{Explicit 3 Stage 3rd order TVD} 
      \newline 
    Explicit 4 Stage 3rd order by Runge (Section~\ref{sec:Explicit 4 Stage 3rd order by Runge})
      \index{Explicit 4 Stage 3rd order by Runge} 
      \newline 
    Explicit 4 Stage (Section~\ref{sec:Explicit 4 Stage})
      \index{Explicit 4 Stage} 
      \newline 
    Explicit 3/8 Rule (Section~\ref{sec:Explicit 3/8 Rule})
      \index{Explicit 3/8 Rule} 
      \newline 
    Backward Euler (Section~\ref{sec:Backward Euler-Runge Kutta Butcher Tableau Selection})
      \index{Backward Euler} 
      \newline 
    IRK 1 Stage Theta Method (Section~\ref{sec:IRK 1 Stage Theta Method})
      \index{IRK 1 Stage Theta Method} 
      \newline 
    IRK 2 Stage Theta Method (Section~\ref{sec:IRK 2 Stage Theta Method})
      \index{IRK 2 Stage Theta Method} 
      \newline 
    Singly Diagonal IRK 2 Stage 2nd order (Section~\ref{sec:Singly Diagonal IRK 2 Stage 2nd order})
      \index{Singly Diagonal IRK 2 Stage 2nd order} 
      \newline 
    Singly Diagonal IRK 2 Stage 3rd order (Section~\ref{sec:Singly Diagonal IRK 2 Stage 3rd order})
      \index{Singly Diagonal IRK 2 Stage 3rd order} 
      \newline 
    Singly Diagonal IRK 3 Stage 4th order (Section~\ref{sec:Singly Diagonal IRK 3 Stage 4th order})
      \index{Singly Diagonal IRK 3 Stage 4th order} 
      \newline 
    Singly Diagonal IRK 5 Stage 4th order (Section~\ref{sec:Singly Diagonal IRK 5 Stage 4th order})
      \index{Singly Diagonal IRK 5 Stage 4th order} 
      \newline 
    Singly Diagonal IRK 5 Stage 5th order (Section~\ref{sec:Singly Diagonal IRK 5 Stage 5th order})
      \index{Singly Diagonal IRK 5 Stage 5th order} 
      \newline 
    Diagonal IRK 2 Stage 3rd order (Section~\ref{sec:Diagonal IRK 2 Stage 3rd order})
      \index{Diagonal IRK 2 Stage 3rd order} 
      \newline 
    Implicit 1 Stage 2nd order Gauss (Section~\ref{sec:Implicit 1 Stage 2nd order Gauss})
      \index{Implicit 1 Stage 2nd order Gauss} 
      \newline 
    Implicit 2 Stage 4th order Gauss (Section~\ref{sec:Implicit 2 Stage 4th order Gauss})
      \index{Implicit 2 Stage 4th order Gauss} 
      \newline 
    Implicit 3 Stage 6th order Gauss (Section~\ref{sec:Implicit 3 Stage 6th order Gauss})
      \index{Implicit 3 Stage 6th order Gauss} 
      \newline 
    Implicit 2 Stage 4th Order Hammer \& Hollingsworth (Section~\ref{sec:Implicit 2 Stage 4th Order Hammer and Hollingsworth})
      \index{Implicit 2 Stage 4th Order Hammer \& Hollingsworth} 
      \newline 
    Implicit 3 Stage 6th Order Kuntzmann \& Butcher (Section~\ref{sec:Implicit 3 Stage 6th Order Kuntzmann and Butcher})
      \index{Implicit 3 Stage 6th Order Kuntzmann \& Butcher} 
      \newline 
    Implicit 1 Stage 1st order Radau left (Section~\ref{sec:Implicit 1 Stage 1st order Radau left})
      \index{Implicit 1 Stage 1st order Radau left} 
      \newline 
    Implicit 2 Stage 3rd order Radau left (Section~\ref{sec:Implicit 2 Stage 3rd order Radau left})
      \index{Implicit 2 Stage 3rd order Radau left} 
      \newline 
    Implicit 3 Stage 5th order Radau left (Section~\ref{sec:Implicit 3 Stage 5th order Radau left})
      \index{Implicit 3 Stage 5th order Radau left} 
      \newline 
    Implicit 1 Stage 1st order Radau right (Section~\ref{sec:Implicit 1 Stage 1st order Radau right})
      \index{Implicit 1 Stage 1st order Radau right} 
      \newline 
    Implicit 2 Stage 3rd order Radau right (Section~\ref{sec:Implicit 2 Stage 3rd order Radau right})
      \index{Implicit 2 Stage 3rd order Radau right} 
      \newline 
    Implicit 3 Stage 5th order Radau right (Section~\ref{sec:Implicit 3 Stage 5th order Radau right})
      \index{Implicit 3 Stage 5th order Radau right} 
      \newline 
    Implicit 2 Stage 2nd order Lobatto A (Section~\ref{sec:Implicit 2 Stage 2nd order Lobatto A})
      \index{Implicit 2 Stage 2nd order Lobatto A} 
      \newline 
    Implicit 3 Stage 4th order Lobatto A (Section~\ref{sec:Implicit 3 Stage 4th order Lobatto A})
      \index{Implicit 3 Stage 4th order Lobatto A} 
      \newline 
    Implicit 4 Stage 6th order Lobatto A (Section~\ref{sec:Implicit 4 Stage 6th order Lobatto A})
      \index{Implicit 4 Stage 6th order Lobatto A} 
      \newline 
    Implicit 2 Stage 2nd order Lobatto B (Section~\ref{sec:Implicit 2 Stage 2nd order Lobatto B})
      \index{Implicit 2 Stage 2nd order Lobatto B} 
      \newline 
    Implicit 3 Stage 4th order Lobatto B (Section~\ref{sec:Implicit 3 Stage 4th order Lobatto B})
      \index{Implicit 3 Stage 4th order Lobatto B} 
      \newline 
    Implicit 4 Stage 6th order Lobatto B (Section~\ref{sec:Implicit 4 Stage 6th order Lobatto B})
      \index{Implicit 4 Stage 6th order Lobatto B} 
      \newline 
    Implicit 2 Stage 2nd order Lobatto C (Section~\ref{sec:Implicit 2 Stage 2nd order Lobatto C})
      \index{Implicit 2 Stage 2nd order Lobatto C} 
      \newline 
    Implicit 3 Stage 4th order Lobatto C (Section~\ref{sec:Implicit 3 Stage 4th order Lobatto C})
      \index{Implicit 3 Stage 4th order Lobatto C} 
      \newline 
    Implicit 4 Stage 6th order Lobatto C (Section~\ref{sec:Implicit 4 Stage 6th order Lobatto C})
      \index{Implicit 4 Stage 6th order Lobatto C} 
  \item[Parameters:]
    \begin{description}
      \item[Runge Kutta Butcher Tableau Type = None] 
Determines the type of Rythmos::RKButcherTableau object that will be built.
The parameters for each Runge Kutta Butcher Tableau Type are specified in this sublist

  Valid std::string values:

      \begin{tabular}{lp{0.4\textwidth}}
      "None" & \\ 
      "Forward Euler" & \\ 
      "Explicit 2 Stage 2nd order by Runge" & \\ 
      "Explicit Trapezoidal" & \\ 
      "Explicit 3 Stage 3rd order" & \\ 
      "Explicit 3 Stage 3rd order by Heun" & \\ 
      "Explicit 3 Stage 3rd order TVD" & \\ 
      "Explicit 4 Stage 3rd order by Runge" & \\ 
      "Explicit 4 Stage" & \\ 
      "Explicit 3/8 Rule" & \\ 
      "Backward Euler" & \\ 
      "IRK 1 Stage Theta Method" & \\ 
      "IRK 2 Stage Theta Method" & \\ 
      "Singly Diagonal IRK 2 Stage 2nd order" & \\ 
      "Singly Diagonal IRK 2 Stage 3rd order" & \\ 
      "Singly Diagonal IRK 3 Stage 4th order" & \\ 
      "Singly Diagonal IRK 5 Stage 4th order" & \\ 
      "Singly Diagonal IRK 5 Stage 5th order" & \\ 
      "Diagonal IRK 2 Stage 3rd order" & \\ 
      "Implicit 1 Stage 2nd order Gauss" & \\ 
      "Implicit 2 Stage 4th order Gauss" & \\ 
      "Implicit 3 Stage 6th order Gauss" & \\ 
      "Implicit 2 Stage 4th Order Hammer \& Hollingsworth" & \\ 
      "Implicit 3 Stage 6th Order Kuntzmann \& Butcher" & \\ 
      "Implicit 1 Stage 1st order Radau left" & \\ 
      "Implicit 2 Stage 3rd order Radau left" & \\ 
      "Implicit 3 Stage 5th order Radau left" & \\ 
      "Implicit 1 Stage 1st order Radau right" & \\ 
      "Implicit 2 Stage 3rd order Radau right" & \\ 
      "Implicit 3 Stage 5th order Radau right" & \\ 
      "Implicit 2 Stage 2nd order Lobatto A" & \\ 
      "Implicit 3 Stage 4th order Lobatto A" & \\ 
      "Implicit 4 Stage 6th order Lobatto A" & \\ 
      "Implicit 2 Stage 2nd order Lobatto B" & \\ 
      "Implicit 3 Stage 4th order Lobatto B" & \\ 
      "Implicit 4 Stage 6th order Lobatto B" & \\ 
      "Implicit 2 Stage 2nd order Lobatto C" & \\ 
      "Implicit 3 Stage 4th order Lobatto C" & \\ 
      "Implicit 4 Stage 6th order Lobatto C" & \\ 
      \end{tabular}
        \index{Runge Kutta Butcher Tableau Selection!Runge Kutta Butcher Tableau Type}
        \index{Runge Kutta Butcher Tableau Type}
\end{description}

\end{list}

% ----------------------------------------------------------
\subsection{Forward Euler}
\label{sec:Forward Euler-Runge Kutta Butcher Tableau Selection}
\index{Forward Euler}

\begin{list}{}
  {\setlength{\leftmargin}{1.0in}
   \setlength{\labelwidth}{0.75in}
   \setlength{\labelsep}{0.125in}}
  \item[Description:]
\begin{verbatim}
  Forward Euler
  c = [ 0 ]'
  A = [ 0 ]
  b = [ 1 ]'
\end{verbatim}
  \item[Parent(s):]
    Runge Kutta Butcher Tableau Selection (Section~\ref{sec:Runge Kutta Butcher Tableau Selection})
      \index{Runge Kutta Butcher Tableau Selection} 
  \item[Child(ren):]
    None. 
  \item[Parameters:]
    None. 
\end{list}

% ----------------------------------------------------------
\subsection{Explicit 2 Stage 2nd order by Runge}
\label{sec:Explicit 2 Stage 2nd order by Runge}
\index{Explicit 2 Stage 2nd order by Runge}

\begin{list}{}
  {\setlength{\leftmargin}{1.0in}
   \setlength{\labelwidth}{0.75in}
   \setlength{\labelsep}{0.125in}}
  \item[Description:]
\begin{verbatim}
  Explicit 2 Stage 2nd order by Runge
  Also known as Explicit Midpoint
  Solving Ordinary Differential Equations I:
  Nonstiff Problems, 2nd Revised Edition
  E. Hairer, S.P. Norsett, G. Wanner
  Table 1.1, pg 135
  c = [  0  1/2 ]'
  A = [  0      ]
      [ 1/2  0  ]
  b = [  0   1  ]'
\end{verbatim}
  \item[Parent(s):]
    Runge Kutta Butcher Tableau Selection (Section~\ref{sec:Runge Kutta Butcher Tableau Selection})
      \index{Runge Kutta Butcher Tableau Selection} 
  \item[Child(ren):]
    None. 
  \item[Parameters:]
    None. 
\end{list}

% ----------------------------------------------------------
\subsection{Explicit Trapezoidal}
\label{sec:Explicit Trapezoidal}
\index{Explicit Trapezoidal}

\begin{list}{}
  {\setlength{\leftmargin}{1.0in}
   \setlength{\labelwidth}{0.75in}
   \setlength{\labelsep}{0.125in}}
  \item[Description:]
\begin{verbatim}
  Explicit Trapezoidal
  c = [  0   1  ]'
  A = [  0      ]
      [  1   0  ]
  b = [ 1/2 1/2 ]'
\end{verbatim}
  \item[Parent(s):]
    Runge Kutta Butcher Tableau Selection (Section~\ref{sec:Runge Kutta Butcher Tableau Selection})
      \index{Runge Kutta Butcher Tableau Selection} 
  \item[Child(ren):]
    None. 
  \item[Parameters:]
    None. 
\end{list}

% ----------------------------------------------------------
\subsection{Explicit 3 Stage 3rd order}
\label{sec:Explicit 3 Stage 3rd order}
\index{Explicit 3 Stage 3rd order}

\begin{list}{}
  {\setlength{\leftmargin}{1.0in}
   \setlength{\labelwidth}{0.75in}
   \setlength{\labelsep}{0.125in}}
  \item[Description:]
\begin{verbatim}
  Explicit 3 Stage 3rd order
  c = [  0  1/2  1  ]'
  A = [  0          ]
      [ 1/2  0      ]
      [ -1   2   0  ]
  b = [ 1/6 4/6 1/6 ]'
\end{verbatim}
  \item[Parent(s):]
    Runge Kutta Butcher Tableau Selection (Section~\ref{sec:Runge Kutta Butcher Tableau Selection})
      \index{Runge Kutta Butcher Tableau Selection} 
  \item[Child(ren):]
    None. 
  \item[Parameters:]
    None. 
\end{list}

% ----------------------------------------------------------
\subsection{Explicit 3 Stage 3rd order by Heun}
\label{sec:Explicit 3 Stage 3rd order by Heun}
\index{Explicit 3 Stage 3rd order by Heun}

\begin{list}{}
  {\setlength{\leftmargin}{1.0in}
   \setlength{\labelwidth}{0.75in}
   \setlength{\labelsep}{0.125in}}
  \item[Description:]
\begin{verbatim}
  Explicit 3 Stage 3rd order by Heun
  Solving Ordinary Differential Equations I:
  Nonstiff Problems, 2nd Revised Edition
  E. Hairer, S.P. Norsett, G. Wanner
  Table 1.1, pg 135
  c = [  0  1/3 2/3 ]'
  A = [  0          ]
      [ 1/3  0      ]
      [  0  2/3  0  ]
  b = [ 1/4  0  3/4 ]'
\end{verbatim}
  \item[Parent(s):]
    Runge Kutta Butcher Tableau Selection (Section~\ref{sec:Runge Kutta Butcher Tableau Selection})
      \index{Runge Kutta Butcher Tableau Selection} 
  \item[Child(ren):]
    None. 
  \item[Parameters:]
    None. 
\end{list}

% ----------------------------------------------------------
\subsection{Explicit 3 Stage 3rd order TVD}
\label{sec:Explicit 3 Stage 3rd order TVD}
\index{Explicit 3 Stage 3rd order TVD}

\begin{list}{}
  {\setlength{\leftmargin}{1.0in}
   \setlength{\labelwidth}{0.75in}
   \setlength{\labelsep}{0.125in}}
  \item[Description:]
\begin{verbatim}
  Explicit 3 Stage 3rd order TVD
  Sigal Gottlieb and Chi-Wang Shu
  `Total Variation Diminishing Runge-Kutta Schemes'
  Mathematics of Computation
  Volume 67, Number 221, January 1998, pp. 73-85
  c = [  0   1  1/2 ]'
  A = [  0          ]
      [  1   0      ]
      [ 1/4 1/4  0  ]
  b = [ 1/6 1/6 4/6 ]'
  This is also written in the following set of updates.
  u1 = u^n + dt L(u^n)
  u2 = 3 u^n/4 + u1/4 + dt L(u1)/4
  u^(n+1) = u^n/3 + 2 u2/2 + 2 dt L(u2)/3
\end{verbatim}
  \item[Parent(s):]
    Runge Kutta Butcher Tableau Selection (Section~\ref{sec:Runge Kutta Butcher Tableau Selection})
      \index{Runge Kutta Butcher Tableau Selection} 
  \item[Child(ren):]
    None. 
  \item[Parameters:]
    None. 
\end{list}

% ----------------------------------------------------------
\subsection{Explicit 4 Stage 3rd order by Runge}
\label{sec:Explicit 4 Stage 3rd order by Runge}
\index{Explicit 4 Stage 3rd order by Runge}

\begin{list}{}
  {\setlength{\leftmargin}{1.0in}
   \setlength{\labelwidth}{0.75in}
   \setlength{\labelsep}{0.125in}}
  \item[Description:]
\begin{verbatim}
  Explicit 4 Stage 3rd order by Runge
  Solving Ordinary Differential Equations I:
  Nonstiff Problems, 2nd Revised Edition
  E. Hairer, S.P. Norsett, G. Wanner
  Table 1.1, pg 135
  c = [  0  1/2  1   1  ]'
  A = [  0              ]
      [ 1/2  0          ]
      [  0   1   0      ]
      [  0   0   1   0  ]
  b = [ 1/6 2/3  0  1/6 ]'
\end{verbatim}
  \item[Parent(s):]
    Runge Kutta Butcher Tableau Selection (Section~\ref{sec:Runge Kutta Butcher Tableau Selection})
      \index{Runge Kutta Butcher Tableau Selection} 
  \item[Child(ren):]
    None. 
  \item[Parameters:]
    None. 
\end{list}

% ----------------------------------------------------------
\subsection{Explicit 4 Stage}
\label{sec:Explicit 4 Stage}
\index{Explicit 4 Stage}

\begin{list}{}
  {\setlength{\leftmargin}{1.0in}
   \setlength{\labelwidth}{0.75in}
   \setlength{\labelsep}{0.125in}}
  \item[Description:]
\begin{verbatim}
  Explicit 4 Stage
  "The" Runge-Kutta Method (explicit):
  Solving Ordinary Differential Equations I:
  Nonstiff Problems, 2nd Revised Edition
  E. Hairer, S.P. Norsett, G. Wanner
  Table 1.2, pg 138
  c = [  0  1/2 1/2  1  ]'
  A = [  0              ]
      [ 1/2  0          ]
      [  0  1/2  0      ]
      [  0   0   1   0  ]
  b = [ 1/6 1/3 1/3 1/6 ]'
\end{verbatim}
  \item[Parent(s):]
    Runge Kutta Butcher Tableau Selection (Section~\ref{sec:Runge Kutta Butcher Tableau Selection})
      \index{Runge Kutta Butcher Tableau Selection} 
  \item[Child(ren):]
    None. 
  \item[Parameters:]
    None. 
\end{list}

% ----------------------------------------------------------
\subsection{Explicit 3/8 Rule}
\label{sec:Explicit 3/8 Rule}
\index{Explicit 3/8 Rule}

\begin{list}{}
  {\setlength{\leftmargin}{1.0in}
   \setlength{\labelwidth}{0.75in}
   \setlength{\labelsep}{0.125in}}
  \item[Description:]
\begin{verbatim}
  Explicit 3/8 Rule
  Solving Ordinary Differential Equations I:
  Nonstiff Problems, 2nd Revised Edition
  E. Hairer, S.P. Norsett, G. Wanner
  Table 1.2, pg 138
  c = [  0  1/3 2/3  1  ]'
  A = [  0              ]
      [ 1/3  0          ]
      [-1/3  1   0      ]
      [  1  -1   1   0  ]
  b = [ 1/8 3/8 3/8 1/8 ]'
\end{verbatim}
  \item[Parent(s):]
    Runge Kutta Butcher Tableau Selection (Section~\ref{sec:Runge Kutta Butcher Tableau Selection})
      \index{Runge Kutta Butcher Tableau Selection} 
  \item[Child(ren):]
    None. 
  \item[Parameters:]
    None. 
\end{list}

% ----------------------------------------------------------
\subsection{Backward Euler}
\label{sec:Backward Euler-Runge Kutta Butcher Tableau Selection}
\index{Backward Euler}

\begin{list}{}
  {\setlength{\leftmargin}{1.0in}
   \setlength{\labelwidth}{0.75in}
   \setlength{\labelsep}{0.125in}}
  \item[Description:]
\begin{verbatim}
  Backward Euler
  c = [ 1 ]'
  A = [ 1 ]
  b = [ 1 ]'
\end{verbatim}
  \item[Parent(s):]
    Runge Kutta Butcher Tableau Selection (Section~\ref{sec:Runge Kutta Butcher Tableau Selection})
      \index{Runge Kutta Butcher Tableau Selection} 
  \item[Child(ren):]
    None. 
  \item[Parameters:]
    None. 
\end{list}

% ----------------------------------------------------------
\subsection{IRK 1 Stage Theta Method}
\label{sec:IRK 1 Stage Theta Method}
\index{IRK 1 Stage Theta Method}

\begin{list}{}
  {\setlength{\leftmargin}{1.0in}
   \setlength{\labelwidth}{0.75in}
   \setlength{\labelsep}{0.125in}}
  \item[Description:]
\begin{verbatim}
  IRK 1 Stage Theta Method
  Non-standard finite-difference methods
  in dynamical systems, P. Kama,
  Dissertation, University of Pretoria, pg. 49.
  Comment:  Generalized Implicit Midpoint Method
  c = [ theta ]'
  A = [ theta ]
  b = [  1  ]'
\end{verbatim}
  \item[Parent(s):]
    Runge Kutta Butcher Tableau Selection (Section~\ref{sec:Runge Kutta Butcher Tableau Selection})
      \index{Runge Kutta Butcher Tableau Selection} 
  \item[Child(ren):]
    None. 
  \item[Parameters:]
    \begin{description}
      \item[theta = 0.5] 
Valid values are 0 $<$= theta $<$= 1, where theta = 0 implies Forward Euler, theta = 1/2 implies midpoint method, and theta = 1 implies Backward Euler. For theta != 1/2, this method is first-order accurate, and with theta = 1/2, it is second-order accurate.  This method is A-stable, but becomes L-stable with theta=1.
        \index{IRK 1 Stage Theta Method!theta}
        \index{theta}
\end{description}

\end{list}

% ----------------------------------------------------------
\subsection{IRK 2 Stage Theta Method}
\label{sec:IRK 2 Stage Theta Method}
\index{IRK 2 Stage Theta Method}

\begin{list}{}
  {\setlength{\leftmargin}{1.0in}
   \setlength{\labelwidth}{0.75in}
   \setlength{\labelsep}{0.125in}}
  \item[Description:]
\begin{verbatim}
  IRK 2 Stage Theta Method
  Computer Methods for ODEs and DAEs
  U. M. Ascher and L. R. Petzold
  p. 113
  c = [  0       1     ]'
  A = [  0       0     ]
      [ 1-theta  theta ]
  b = [ 1-theta  theta ]'
\end{verbatim}
  \item[Parent(s):]
    Runge Kutta Butcher Tableau Selection (Section~\ref{sec:Runge Kutta Butcher Tableau Selection})
      \index{Runge Kutta Butcher Tableau Selection} 
  \item[Child(ren):]
    None. 
  \item[Parameters:]
    \begin{description}
      \item[theta = 0.5] 
Valid values are 0 $<$ theta $<$= 1, where theta = 0 implies Forward Euler, theta = 1/2 implies trapezoidal method, and theta = 1 implies Backward Euler. For theta != 1/2, this method is first-order accurate, and with theta = 1/2, it is second-order accurate.  This method is A-stable, but becomes L-stable with theta=1.
        \index{IRK 2 Stage Theta Method!theta}
        \index{theta}
\end{description}

\end{list}

% ----------------------------------------------------------
\subsection{Singly Diagonal IRK 2 Stage 2nd order}
\label{sec:Singly Diagonal IRK 2 Stage 2nd order}
\index{Singly Diagonal IRK 2 Stage 2nd order}

\begin{list}{}
  {\setlength{\leftmargin}{1.0in}
   \setlength{\labelwidth}{0.75in}
   \setlength{\labelsep}{0.125in}}
  \item[Description:]
\begin{verbatim}
  Singly Diagonal IRK 2 Stage 2nd order
  Computer Methods for ODEs and DAEs
  U. M. Ascher and L. R. Petzold
  p. 106
  gamma = (2+-sqrt(2))/2
  c = [  gamma   1     ]'
  A = [  gamma   0     ]
      [ 1-gamma  gamma ]
  b = [ 1-gamma  gamma ]'
\end{verbatim}
  \item[Parent(s):]
    Runge Kutta Butcher Tableau Selection (Section~\ref{sec:Runge Kutta Butcher Tableau Selection})
      \index{Runge Kutta Butcher Tableau Selection} 
  \item[Child(ren):]
    None. 
  \item[Parameters:]
    \begin{description}
      \item[gamma = 0.292893] 
The default value is gamma = (2-sqrt(2))/2. This will produce an L-stable 2nd order method with the stage times within the timestep.  Other values of gamma will still produce an L-stable scheme, but will only be 1st order accurate.
        \index{Singly Diagonal IRK 2 Stage 2nd order!gamma}
        \index{gamma}
\end{description}

\end{list}

% ----------------------------------------------------------
\subsection{Singly Diagonal IRK 2 Stage 3rd order}
\label{sec:Singly Diagonal IRK 2 Stage 3rd order}
\index{Singly Diagonal IRK 2 Stage 3rd order}

\begin{list}{}
  {\setlength{\leftmargin}{1.0in}
   \setlength{\labelwidth}{0.75in}
   \setlength{\labelsep}{0.125in}}
  \item[Description:]
\begin{verbatim}
  Singly Diagonal IRK 2 Stage 3rd order
  Solving Ordinary Differential Equations I:
  Nonstiff Problems, 2nd Revised Edition
  E. Hairer, S. P. Norsett, and G. Wanner
  Table 7.2, pg 207
  gamma = (3+-sqrt(3))/6 -> 3rd order and A-stable
  gamma = (2+-sqrt(2))/2 -> 2nd order and L-stable
  c = [  gamma     1-gamma  ]'
  A = [  gamma     0        ]
      [ 1-2*gamma  gamma    ]
  b = [ 1/2        1/2      ]'
\end{verbatim}
  \item[Parent(s):]
    Runge Kutta Butcher Tableau Selection (Section~\ref{sec:Runge Kutta Butcher Tableau Selection})
      \index{Runge Kutta Butcher Tableau Selection} 
  \item[Child(ren):]
    None. 
  \item[Parameters:]
    \begin{description}
      \item[3rd Order A-stable = 1] 
If true, set gamma to gamma = (3+sqrt(3))/6 to obtain a 3rd order A-stable scheme. '3rd Order A-stable' and '2nd Order L-stable' can not both be true.
        \index{Singly Diagonal IRK 2 Stage 3rd order!3rd Order A-stable}
        \index{3rd Order A-stable}
      \item[2nd Order L-stable = 0] 
If true, set gamma to gamma = (2+sqrt(2))/2 to obtain a 2nd order L-stable scheme. '3rd Order A-stable' and '2nd Order L-stable' can not both be true.
        \index{Singly Diagonal IRK 2 Stage 3rd order!2nd Order L-stable}
        \index{2nd Order L-stable}
      \item[gamma = 0.788675] 
If both '3rd Order A-stable' and '2nd Order L-stable' are false, gamma will be used. The default value is the '3rd Order A-stable' gamma value, (3+sqrt(3))/6.
        \index{Singly Diagonal IRK 2 Stage 3rd order!gamma}
        \index{gamma}
\end{description}

\end{list}

% ----------------------------------------------------------
\subsection{Singly Diagonal IRK 3 Stage 4th order}
\label{sec:Singly Diagonal IRK 3 Stage 4th order}
\index{Singly Diagonal IRK 3 Stage 4th order}

\begin{list}{}
  {\setlength{\leftmargin}{1.0in}
   \setlength{\labelwidth}{0.75in}
   \setlength{\labelsep}{0.125in}}
  \item[Description:]
\begin{verbatim}
  Singly Diagonal IRK 3 Stage 4th order
  A-stable
  Solving Ordinary Differential Equations II:
  Stiff and Differential-Algebraic Problems,
  2nd Revised Edition
  E. Hairer and G. Wanner
  pg100
  gamma = (1/sqrt(3))*cos(pi/18)+1/2
  delta = 1/(6*(2*gamma-1)^2)
  c = [ gamma      1/2        1-gamma ]'
  A = [ gamma                         ]
      [ 1/2-gamma  gamma              ]
      [ 2*gamma    1-4*gamma  gamma   ]
  b = [ delta      1-2*delta  delta   ]'
\end{verbatim}
  \item[Parent(s):]
    Runge Kutta Butcher Tableau Selection (Section~\ref{sec:Runge Kutta Butcher Tableau Selection})
      \index{Runge Kutta Butcher Tableau Selection} 
  \item[Child(ren):]
    None. 
  \item[Parameters:]
    None. 
\end{list}

% ----------------------------------------------------------
\subsection{Singly Diagonal IRK 5 Stage 4th order}
\label{sec:Singly Diagonal IRK 5 Stage 4th order}
\index{Singly Diagonal IRK 5 Stage 4th order}

\begin{list}{}
  {\setlength{\leftmargin}{1.0in}
   \setlength{\labelwidth}{0.75in}
   \setlength{\labelsep}{0.125in}}
  \item[Description:]
\begin{verbatim}
  Singly Diagonal IRK 5 Stage 4th order
  L-stable
  Solving Ordinary Differential Equations II:
  Stiff and Differential-Algebraic Problems,
  2nd Revised Edition
  E. Hairer and G. Wanner
  pg100
  c  = [ 1/4       3/4        11/20   1/2     1   ]'
  A  = [ 1/4                                      ]
       [ 1/2       1/4                            ]
       [ 17/50     -1/25      1/4                 ]
       [ 371/1360  -137/2720  15/544  1/4         ]
       [ 25/24     -49/48     125/16  -85/12  1/4 ]
  b  = [ 25/24     -49/48     125/16  -85/12  1/4 ]'
  b' = [ 59/48     -17/96     225/32  -85/12  0   ]'
\end{verbatim}
  \item[Parent(s):]
    Runge Kutta Butcher Tableau Selection (Section~\ref{sec:Runge Kutta Butcher Tableau Selection})
      \index{Runge Kutta Butcher Tableau Selection} 
  \item[Child(ren):]
    None. 
  \item[Parameters:]
    None. 
\end{list}

% ----------------------------------------------------------
\subsection{Singly Diagonal IRK 5 Stage 5th order}
\label{sec:Singly Diagonal IRK 5 Stage 5th order}
\index{Singly Diagonal IRK 5 Stage 5th order}

\begin{list}{}
  {\setlength{\leftmargin}{1.0in}
   \setlength{\labelwidth}{0.75in}
   \setlength{\labelsep}{0.125in}}
  \item[Description:]
\begin{verbatim}
  Singly Diagonal IRK 5 Stage 5th order
  A-stable
  Solving Ordinary Differential Equations II:
  Stiff and Differential-Algebraic Problems,
  2nd Revised Edition
  E. Hairer and G. Wanner
  pg101
  c = [ (6-sqrt(6))/10   ]
      [ (6+9*sqrt(6))/35 ]
      [ 1                ]
      [ (4-sqrt(6))/10   ]
      [ (4+sqrt(6))/10   ]
  A = [ A1 A2 A3 A4 A5 ]
        A1 = [ (6-sqrt(6))/10               ]
             [ (-6+5*sqrt(6))/14            ]
             [ (888+607*sqrt(6))/2850       ]
             [ (3153-3082*sqrt(6))/14250    ]
             [ (-32583+14638*sqrt(6))/71250 ]
        A2 = [ 0                           ]
             [ (6-sqrt(6))/10              ]
             [ (126-161*sqrt(6))/1425      ]
             [ (3213+1148*sqrt(6))/28500   ]
             [ (-17199+364*sqrt(6))/142500 ]
        A3 = [ 0                       ]
             [ 0                       ]
             [ (6-sqrt(6))/10          ]
             [ (-267+88*sqrt(6))/500   ]
             [ (1329-544*sqrt(6))/2500 ]
        A4 = [ 0                     ]
             [ 0                     ]
             [ 0                     ]
             [ (6-sqrt(6))/10        ]
             [ (-96+131*sqrt(6))/625 ]
        A5 = [ 0              ]
             [ 0              ]
             [ 0              ]
             [ 0              ]
             [ (6-sqrt(6))/10 ]
  b = [               0 ]
      [               0 ]
      [             1/9 ]
      [ (16-sqrt(6))/36 ]
      [ (16+sqrt(6))/36 ]
\end{verbatim}
  \item[Parent(s):]
    Runge Kutta Butcher Tableau Selection (Section~\ref{sec:Runge Kutta Butcher Tableau Selection})
      \index{Runge Kutta Butcher Tableau Selection} 
  \item[Child(ren):]
    None. 
  \item[Parameters:]
    None. 
\end{list}

% ----------------------------------------------------------
\subsection{Diagonal IRK 2 Stage 3rd order}
\label{sec:Diagonal IRK 2 Stage 3rd order}
\index{Diagonal IRK 2 Stage 3rd order}

\begin{list}{}
  {\setlength{\leftmargin}{1.0in}
   \setlength{\labelwidth}{0.75in}
   \setlength{\labelsep}{0.125in}}
  \item[Description:]
\begin{verbatim}
  Diagonal IRK 2 Stage 3rd order
  Hammer & Hollingsworth method
  Solving Ordinary Differential Equations I:
  Nonstiff Problems, 2nd Revised Edition
  E. Hairer, S. P. Norsett, and G. Wanner
  Table 7.1, pg 205
  c = [  0   2/3 ]'
  A = [  0    0  ]
      [ 1/3  1/3 ]
  b = [ 1/4  3/4 ]'
\end{verbatim}
  \item[Parent(s):]
    Runge Kutta Butcher Tableau Selection (Section~\ref{sec:Runge Kutta Butcher Tableau Selection})
      \index{Runge Kutta Butcher Tableau Selection} 
  \item[Child(ren):]
    None. 
  \item[Parameters:]
    None. 
\end{list}

% ----------------------------------------------------------
\subsection{Implicit 1 Stage 2nd order Gauss}
\label{sec:Implicit 1 Stage 2nd order Gauss}
\index{Implicit 1 Stage 2nd order Gauss}

\begin{list}{}
  {\setlength{\leftmargin}{1.0in}
   \setlength{\labelwidth}{0.75in}
   \setlength{\labelsep}{0.125in}}
  \item[Description:]
\begin{verbatim}
  Implicit 1 Stage 2nd order Gauss
  A-stable
  Solving Ordinary Differential Equations II:
  Stiff and Differential-Algebraic Problems,
  2nd Revised Edition
  E. Hairer and G. Wanner
  Table 5.2, pg 72
  Also:  Implicit midpoint rule
  Solving Ordinary Differential Equations I:
  Nonstiff Problems, 2nd Revised Edition
  E. Hairer, S. P. Norsett, and G. Wanner
  Table 7.1, pg 205
  c = [ 1/2 ]'
  A = [ 1/2 ]
  b = [  1  ]'
\end{verbatim}
  \item[Parent(s):]
    Runge Kutta Butcher Tableau Selection (Section~\ref{sec:Runge Kutta Butcher Tableau Selection})
      \index{Runge Kutta Butcher Tableau Selection} 
  \item[Child(ren):]
    None. 
  \item[Parameters:]
    None. 
\end{list}

% ----------------------------------------------------------
\subsection{Implicit 2 Stage 4th order Gauss}
\label{sec:Implicit 2 Stage 4th order Gauss}
\index{Implicit 2 Stage 4th order Gauss}

\begin{list}{}
  {\setlength{\leftmargin}{1.0in}
   \setlength{\labelwidth}{0.75in}
   \setlength{\labelsep}{0.125in}}
  \item[Description:]
\begin{verbatim}
  Implicit 2 Stage 4th order Gauss
  A-stable
  Solving Ordinary Differential Equations II:
  Stiff and Differential-Algebraic Problems,
  2nd Revised Edition
  E. Hairer and G. Wanner
  Table 5.2, pg 72
  c = [ 1/2-sqrt(3)/6  1/2+sqrt(3)/6 ]'
  A = [ 1/4            1/4-sqrt(3)/6 ]
      [ 1/4+sqrt(3)/6  1/4           ]
  b = [ 1/2            1/2 ]'
\end{verbatim}
  \item[Parent(s):]
    Runge Kutta Butcher Tableau Selection (Section~\ref{sec:Runge Kutta Butcher Tableau Selection})
      \index{Runge Kutta Butcher Tableau Selection} 
  \item[Child(ren):]
    None. 
  \item[Parameters:]
    None. 
\end{list}

% ----------------------------------------------------------
\subsection{Implicit 3 Stage 6th order Gauss}
\label{sec:Implicit 3 Stage 6th order Gauss}
\index{Implicit 3 Stage 6th order Gauss}

\begin{list}{}
  {\setlength{\leftmargin}{1.0in}
   \setlength{\labelwidth}{0.75in}
   \setlength{\labelsep}{0.125in}}
  \item[Description:]
\begin{verbatim}
  Implicit 3 Stage 6th order Gauss
  A-stable
  Solving Ordinary Differential Equations II:
  Stiff and Differential-Algebraic Problems,
  2nd Revised Edition
  E. Hairer and G. Wanner
  Table 5.2, pg 72
  c = [ 1/2-sqrt(15)/10   1/2              1/2+sqrt(15)/10  ]'
  A = [ 5/36              2/9-sqrt(15)/15  5/36-sqrt(15)/30 ]
      [ 5/36+sqrt(15)/24  2/9              5/36-sqrt(15)/24 ]
      [ 5/36+sqrt(15)/30  2/9+sqrt(15)/15  5/36             ]
  b = [ 5/18              4/9              5/18             ]'
\end{verbatim}
  \item[Parent(s):]
    Runge Kutta Butcher Tableau Selection (Section~\ref{sec:Runge Kutta Butcher Tableau Selection})
      \index{Runge Kutta Butcher Tableau Selection} 
  \item[Child(ren):]
    None. 
  \item[Parameters:]
    None. 
\end{list}

% ----------------------------------------------------------
\subsection{Implicit 2 Stage 4th Order Hammer \& Hollingsworth}
\label{sec:Implicit 2 Stage 4th Order Hammer and Hollingsworth}
\index{Implicit 2 Stage 4th Order Hammer \& Hollingsworth}

\begin{list}{}
  {\setlength{\leftmargin}{1.0in}
   \setlength{\labelwidth}{0.75in}
   \setlength{\labelsep}{0.125in}}
  \item[Description:]
\begin{verbatim}
  Implicit 2 Stage 4th Order Hammer & Hollingsworth
  Hammer & Hollingsworth method
  Solving Ordinary Differential Equations I:
  Nonstiff Problems, 2nd Revised Edition
  E. Hairer, S. P. Norsett, and G. Wanner
  Table 7.3, pg 207
  c = [ 1/2-sqrt(3)/6  1/2+sqrt(3)/6 ]'
  A = [ 1/4            1/4-sqrt(3)/6 ]
      [ 1/4+sqrt(3)/6  1/4           ]
  b = [ 1/2            1/2           ]'
\end{verbatim}
  \item[Parent(s):]
    Runge Kutta Butcher Tableau Selection (Section~\ref{sec:Runge Kutta Butcher Tableau Selection})
      \index{Runge Kutta Butcher Tableau Selection} 
  \item[Child(ren):]
    None. 
  \item[Parameters:]
    None. 
\end{list}

% ----------------------------------------------------------
\subsection{Implicit 3 Stage 6th Order Kuntzmann \& Butcher}
\label{sec:Implicit 3 Stage 6th Order Kuntzmann and Butcher}
\index{Implicit 3 Stage 6th Order Kuntzmann \& Butcher}

\begin{list}{}
  {\setlength{\leftmargin}{1.0in}
   \setlength{\labelwidth}{0.75in}
   \setlength{\labelsep}{0.125in}}
  \item[Description:]
\begin{verbatim}
  Implicit 3 Stage 6th Order Kuntzmann & Butcher
  Kuntzmann & Butcher method
  Solving Ordinary Differential Equations I:
  Nonstiff Problems, 2nd Revised Edition
  E. Hairer, S. P. Norsett, and G. Wanner
  Table 7.4, pg 209
  c = [ 1/2-sqrt(15)/10   1/2              1/2-sqrt(15)/10  ]'
  A = [ 5/36              2/9-sqrt(15)/15  5/36-sqrt(15)/30 ]
      [ 5/36+sqrt(15)/24  2/9              5/36-sqrt(15)/24 ]
      [ 5/36+sqrt(15)/30  2/9+sqrt(15)/15  5/36             ]
  b = [ 5/18              4/9              5/18             ]'
\end{verbatim}
  \item[Parent(s):]
    Runge Kutta Butcher Tableau Selection (Section~\ref{sec:Runge Kutta Butcher Tableau Selection})
      \index{Runge Kutta Butcher Tableau Selection} 
  \item[Child(ren):]
    None. 
  \item[Parameters:]
    None. 
\end{list}

% ----------------------------------------------------------
\subsection{Implicit 1 Stage 1st order Radau left}
\label{sec:Implicit 1 Stage 1st order Radau left}
\index{Implicit 1 Stage 1st order Radau left}

\begin{list}{}
  {\setlength{\leftmargin}{1.0in}
   \setlength{\labelwidth}{0.75in}
   \setlength{\labelsep}{0.125in}}
  \item[Description:]
\begin{verbatim}
  Implicit 1 Stage 1st order Radau left
  A-stable
  Solving Ordinary Differential Equations II:
  Stiff and Differential-Algebraic Problems,
  2nd Revised Edition
  E. Hairer and G. Wanner
  Table 5.3, pg 73
  c = [ 0 ]'
  A = [ 1 ]
  b = [ 1 ]'
\end{verbatim}
  \item[Parent(s):]
    Runge Kutta Butcher Tableau Selection (Section~\ref{sec:Runge Kutta Butcher Tableau Selection})
      \index{Runge Kutta Butcher Tableau Selection} 
  \item[Child(ren):]
    None. 
  \item[Parameters:]
    None. 
\end{list}

% ----------------------------------------------------------
\subsection{Implicit 2 Stage 3rd order Radau left}
\label{sec:Implicit 2 Stage 3rd order Radau left}
\index{Implicit 2 Stage 3rd order Radau left}

\begin{list}{}
  {\setlength{\leftmargin}{1.0in}
   \setlength{\labelwidth}{0.75in}
   \setlength{\labelsep}{0.125in}}
  \item[Description:]
\begin{verbatim}
  Implicit 2 Stage 3rd order Radau left
  A-stable
  Solving Ordinary Differential Equations II:
  Stiff and Differential-Algebraic Problems,
  2nd Revised Edition
  E. Hairer and G. Wanner
  Table 5.3, pg 73
  c = [  0    2/3 ]'
  A = [ 1/4  -1/4 ]
      [ 1/4  5/12 ]
  b = [ 1/4  3/4  ]'
\end{verbatim}
  \item[Parent(s):]
    Runge Kutta Butcher Tableau Selection (Section~\ref{sec:Runge Kutta Butcher Tableau Selection})
      \index{Runge Kutta Butcher Tableau Selection} 
  \item[Child(ren):]
    None. 
  \item[Parameters:]
    None. 
\end{list}

% ----------------------------------------------------------
\subsection{Implicit 3 Stage 5th order Radau left}
\label{sec:Implicit 3 Stage 5th order Radau left}
\index{Implicit 3 Stage 5th order Radau left}

\begin{list}{}
  {\setlength{\leftmargin}{1.0in}
   \setlength{\labelwidth}{0.75in}
   \setlength{\labelsep}{0.125in}}
  \item[Description:]
\begin{verbatim}
  Implicit 3 Stage 5th order Radau left
  A-stable
  Solving Ordinary Differential Equations II:
  Stiff and Differential-Algebraic Problems,
  2nd Revised Edition
  E. Hairer and G. Wanner
  Table 5.4, pg 73
  c = [  0   (6-sqrt(6))/10       (6+sqrt(6))/10      ]'
  A = [ 1/9  (-1-sqrt(6))/18      (-1+sqrt(6))/18     ]
      [ 1/9  (88+7*sqrt(6))/360   (88-43*sqrt(6))/360 ]
      [ 1/9  (88+43*sqrt(6))/360  (88-7*sqrt(6))/360  ]
  b = [ 1/9  (16+sqrt(6))/36      (16-sqrt(6))/36     ]'
\end{verbatim}
  \item[Parent(s):]
    Runge Kutta Butcher Tableau Selection (Section~\ref{sec:Runge Kutta Butcher Tableau Selection})
      \index{Runge Kutta Butcher Tableau Selection} 
  \item[Child(ren):]
    None. 
  \item[Parameters:]
    None. 
\end{list}

% ----------------------------------------------------------
\subsection{Implicit 1 Stage 1st order Radau right}
\label{sec:Implicit 1 Stage 1st order Radau right}
\index{Implicit 1 Stage 1st order Radau right}

\begin{list}{}
  {\setlength{\leftmargin}{1.0in}
   \setlength{\labelwidth}{0.75in}
   \setlength{\labelsep}{0.125in}}
  \item[Description:]
\begin{verbatim}
  Implicit 1 Stage 1st order Radau right
  A-stable
  Solving Ordinary Differential Equations II:
  Stiff and Differential-Algebraic Problems,
  2nd Revised Edition
  E. Hairer and G. Wanner
  Table 5.5, pg 74
  c = [ 1 ]'
  A = [ 1 ]
  b = [ 1 ]'
\end{verbatim}
  \item[Parent(s):]
    Runge Kutta Butcher Tableau Selection (Section~\ref{sec:Runge Kutta Butcher Tableau Selection})
      \index{Runge Kutta Butcher Tableau Selection} 
  \item[Child(ren):]
    None. 
  \item[Parameters:]
    None. 
\end{list}

% ----------------------------------------------------------
\subsection{Implicit 2 Stage 3rd order Radau right}
\label{sec:Implicit 2 Stage 3rd order Radau right}
\index{Implicit 2 Stage 3rd order Radau right}

\begin{list}{}
  {\setlength{\leftmargin}{1.0in}
   \setlength{\labelwidth}{0.75in}
   \setlength{\labelsep}{0.125in}}
  \item[Description:]
\begin{verbatim}
  Implicit 2 Stage 3rd order Radau right
  A-stable
  Solving Ordinary Differential Equations II:
  Stiff and Differential-Algebraic Problems,
  2nd Revised Edition
  E. Hairer and G. Wanner
  Table 5.5, pg 74
  c = [ 1/3     1   ]'
  A = [ 5/12  -1/12 ]
      [ 3/4    1/4  ]
  b = [ 3/4    1/4  ]'
\end{verbatim}
  \item[Parent(s):]
    Runge Kutta Butcher Tableau Selection (Section~\ref{sec:Runge Kutta Butcher Tableau Selection})
      \index{Runge Kutta Butcher Tableau Selection} 
  \item[Child(ren):]
    None. 
  \item[Parameters:]
    None. 
\end{list}

% ----------------------------------------------------------
\subsection{Implicit 3 Stage 5th order Radau right}
\label{sec:Implicit 3 Stage 5th order Radau right}
\index{Implicit 3 Stage 5th order Radau right}

\begin{list}{}
  {\setlength{\leftmargin}{1.0in}
   \setlength{\labelwidth}{0.75in}
   \setlength{\labelsep}{0.125in}}
  \item[Description:]
\begin{verbatim}
  Implicit 3 Stage 5th order Radau right
  A-stable
  Solving Ordinary Differential Equations II:
  Stiff and Differential-Algebraic Problems,
  2nd Revised Edition
  E. Hairer and G. Wanner
  Table 5.6, pg 74
  c = [ (4-sqrt(6))/10          (4+sqrt(6))/10          1    ]'
  A = [ A1 A2 A3 ]
        A1 = [ (88-7*sqrt(6))/360     ]
             [ (296+169*sqrt(6))/1800 ]
             [ (16-sqrt(6))/36        ]
        A2 = [ (296-169*sqrt(6))/1800 ]
             [ (88+7*sqrt(6))/360     ]
             [ (16+sqrt(6))/36        ]
        A3 = [ (-2+3*sqrt(6))/225 ]
             [ (-2-3*sqrt(6))/225 ]
             [ 1/9                ]
  b = [ (16-sqrt(6))/36         (16+sqrt(6))/36         1/9 ]'
\end{verbatim}
  \item[Parent(s):]
    Runge Kutta Butcher Tableau Selection (Section~\ref{sec:Runge Kutta Butcher Tableau Selection})
      \index{Runge Kutta Butcher Tableau Selection} 
  \item[Child(ren):]
    None. 
  \item[Parameters:]
    None. 
\end{list}

% ----------------------------------------------------------
\subsection{Implicit 2 Stage 2nd order Lobatto A}
\label{sec:Implicit 2 Stage 2nd order Lobatto A}
\index{Implicit 2 Stage 2nd order Lobatto A}

\begin{list}{}
  {\setlength{\leftmargin}{1.0in}
   \setlength{\labelwidth}{0.75in}
   \setlength{\labelsep}{0.125in}}
  \item[Description:]
\begin{verbatim}
  Implicit 2 Stage 2nd order Lobatto A
  A-stable
  Solving Ordinary Differential Equations II:
  Stiff and Differential-Algebraic Problems,
  2nd Revised Edition
  E. Hairer and G. Wanner
  Table 5.7, pg 75
  c = [  0    1   ]'
  A = [  0    0   ]
      [ 1/2  1/2  ]
  b = [ 1/2  1/2  ]'
\end{verbatim}
  \item[Parent(s):]
    Runge Kutta Butcher Tableau Selection (Section~\ref{sec:Runge Kutta Butcher Tableau Selection})
      \index{Runge Kutta Butcher Tableau Selection} 
  \item[Child(ren):]
    None. 
  \item[Parameters:]
    None. 
\end{list}

% ----------------------------------------------------------
\subsection{Implicit 3 Stage 4th order Lobatto A}
\label{sec:Implicit 3 Stage 4th order Lobatto A}
\index{Implicit 3 Stage 4th order Lobatto A}

\begin{list}{}
  {\setlength{\leftmargin}{1.0in}
   \setlength{\labelwidth}{0.75in}
   \setlength{\labelsep}{0.125in}}
  \item[Description:]
\begin{verbatim}
  Implicit 3 Stage 4th order Lobatto A
  A-stable
  Solving Ordinary Differential Equations II:
  Stiff and Differential-Algebraic Problems,
  2nd Revised Edition
  E. Hairer and G. Wanner
  Table 5.7, pg 75
  c = [  0    1/2    1  ]'
  A = [  0     0     0   ]
      [ 5/24  1/3  -1/24  ]
      [ 1/6   2/3   1/6   ]
  b = [ 1/6   2/3   1/6   ]'
\end{verbatim}
  \item[Parent(s):]
    Runge Kutta Butcher Tableau Selection (Section~\ref{sec:Runge Kutta Butcher Tableau Selection})
      \index{Runge Kutta Butcher Tableau Selection} 
  \item[Child(ren):]
    None. 
  \item[Parameters:]
    None. 
\end{list}

% ----------------------------------------------------------
\subsection{Implicit 4 Stage 6th order Lobatto A}
\label{sec:Implicit 4 Stage 6th order Lobatto A}
\index{Implicit 4 Stage 6th order Lobatto A}

\begin{list}{}
  {\setlength{\leftmargin}{1.0in}
   \setlength{\labelwidth}{0.75in}
   \setlength{\labelsep}{0.125in}}
  \item[Description:]
\begin{verbatim}
  Implicit 4 Stage 6th order Lobatto A
  A-stable
  Solving Ordinary Differential Equations II:
  Stiff and Differential-Algebraic Problems,
  2nd Revised Edition
  E. Hairer and G. Wanner
  Table 5.8, pg 75
  c = [ 0  (5-sqrt(5))/10  (5+sqrt(5))/10  1 ]'
  A = [ A1  A2  A3  A4 ]
        A1 = [ 0               ]
             [ (11+sqrt(5)/120 ]
             [ (11-sqrt(5)/120 ]
             [ 1/12            ]
        A2 = [ 0                    ]
             [ (25-sqrt(5))/120     ]
             [ (25+13*sqrt(5))/120  ]
             [ 5/12                 ]
        A3 = [ 0                   ]
             [ (25-13*sqrt(5))/120 ]
             [ (25+sqrt(5))/120    ]
             [ 5/12                ]
        A4 = [ 0                ]
             [ (-1+sqrt(5))/120 ]
             [ (-1-sqrt(5))/120 ]
             [ 1/12             ]
  b = [ 1/12  5/12  5/12   1/12 ]'
\end{verbatim}
  \item[Parent(s):]
    Runge Kutta Butcher Tableau Selection (Section~\ref{sec:Runge Kutta Butcher Tableau Selection})
      \index{Runge Kutta Butcher Tableau Selection} 
  \item[Child(ren):]
    None. 
  \item[Parameters:]
    None. 
\end{list}

% ----------------------------------------------------------
\subsection{Implicit 2 Stage 2nd order Lobatto B}
\label{sec:Implicit 2 Stage 2nd order Lobatto B}
\index{Implicit 2 Stage 2nd order Lobatto B}

\begin{list}{}
  {\setlength{\leftmargin}{1.0in}
   \setlength{\labelwidth}{0.75in}
   \setlength{\labelsep}{0.125in}}
  \item[Description:]
\begin{verbatim}
  Implicit 2 Stage 2nd order Lobatto B
  A-stable
  Solving Ordinary Differential Equations II:
  Stiff and Differential-Algebraic Problems,
  2nd Revised Edition
  E. Hairer and G. Wanner
  Table 5.9, pg 76
  c = [  0    1   ]'
  A = [ 1/2   0   ]
      [ 1/2   0   ]
  b = [ 1/2  1/2  ]'
\end{verbatim}
  \item[Parent(s):]
    Runge Kutta Butcher Tableau Selection (Section~\ref{sec:Runge Kutta Butcher Tableau Selection})
      \index{Runge Kutta Butcher Tableau Selection} 
  \item[Child(ren):]
    None. 
  \item[Parameters:]
    None. 
\end{list}

% ----------------------------------------------------------
\subsection{Implicit 3 Stage 4th order Lobatto B}
\label{sec:Implicit 3 Stage 4th order Lobatto B}
\index{Implicit 3 Stage 4th order Lobatto B}

\begin{list}{}
  {\setlength{\leftmargin}{1.0in}
   \setlength{\labelwidth}{0.75in}
   \setlength{\labelsep}{0.125in}}
  \item[Description:]
\begin{verbatim}
  Implicit 3 Stage 4th order Lobatto B
  A-stable
  Solving Ordinary Differential Equations II:
  Stiff and Differential-Algebraic Problems,
  2nd Revised Edition
  E. Hairer and G. Wanner
  Table 5.9, pg 76
  c = [  0    1/2    1   ]'
  A = [ 1/6  -1/6    0   ]
      [ 1/6   1/3    0   ]
      [ 1/6   5/6    0   ]
  b = [ 1/6   2/3   1/6  ]'
\end{verbatim}
  \item[Parent(s):]
    Runge Kutta Butcher Tableau Selection (Section~\ref{sec:Runge Kutta Butcher Tableau Selection})
      \index{Runge Kutta Butcher Tableau Selection} 
  \item[Child(ren):]
    None. 
  \item[Parameters:]
    None. 
\end{list}

% ----------------------------------------------------------
\subsection{Implicit 4 Stage 6th order Lobatto B}
\label{sec:Implicit 4 Stage 6th order Lobatto B}
\index{Implicit 4 Stage 6th order Lobatto B}

\begin{list}{}
  {\setlength{\leftmargin}{1.0in}
   \setlength{\labelwidth}{0.75in}
   \setlength{\labelsep}{0.125in}}
  \item[Description:]
\begin{verbatim}
  Implicit 4 Stage 6th order Lobatto B
  A-stable
  Solving Ordinary Differential Equations II:
  Stiff and Differential-Algebraic Problems,
  2nd Revised Edition
  E. Hairer and G. Wanner
  Table 5.10, pg 76
  c = [ 0     (5-sqrt(5))/10       (5+sqrt(5))/10       1     ]'
  A = [ 1/12  (-1-sqrt(5))/24      (-1+sqrt(5))/24      0     ]
      [ 1/12  (25+sqrt(5))/120     (25-13*sqrt(5))/120  0     ]
      [ 1/12  (25+13*sqrt(5))/120  (25-sqrt(5))/120     0     ]
      [ 1/12  (11-sqrt(5))/24      (11+sqrt(5))/24      0     ]
  b = [ 1/12  5/12                 5/12                 1/12  ]'
\end{verbatim}
  \item[Parent(s):]
    Runge Kutta Butcher Tableau Selection (Section~\ref{sec:Runge Kutta Butcher Tableau Selection})
      \index{Runge Kutta Butcher Tableau Selection} 
  \item[Child(ren):]
    None. 
  \item[Parameters:]
    None. 
\end{list}

% ----------------------------------------------------------
\subsection{Implicit 2 Stage 2nd order Lobatto C}
\label{sec:Implicit 2 Stage 2nd order Lobatto C}
\index{Implicit 2 Stage 2nd order Lobatto C}

\begin{list}{}
  {\setlength{\leftmargin}{1.0in}
   \setlength{\labelwidth}{0.75in}
   \setlength{\labelsep}{0.125in}}
  \item[Description:]
\begin{verbatim}
  Implicit 2 Stage 2nd order Lobatto C
  A-stable
  Solving Ordinary Differential Equations II:
  Stiff and Differential-Algebraic Problems,
  2nd Revised Edition
  E. Hairer and G. Wanner
  Table 5.11, pg 76
  c = [  0    1   ]'
  A = [ 1/2 -1/2  ]
      [ 1/2  1/2  ]
  b = [ 1/2  1/2  ]'
\end{verbatim}
  \item[Parent(s):]
    Runge Kutta Butcher Tableau Selection (Section~\ref{sec:Runge Kutta Butcher Tableau Selection})
      \index{Runge Kutta Butcher Tableau Selection} 
  \item[Child(ren):]
    None. 
  \item[Parameters:]
    None. 
\end{list}

% ----------------------------------------------------------
\subsection{Implicit 3 Stage 4th order Lobatto C}
\label{sec:Implicit 3 Stage 4th order Lobatto C}
\index{Implicit 3 Stage 4th order Lobatto C}

\begin{list}{}
  {\setlength{\leftmargin}{1.0in}
   \setlength{\labelwidth}{0.75in}
   \setlength{\labelsep}{0.125in}}
  \item[Description:]
\begin{verbatim}
  Implicit 3 Stage 4th order Lobatto C
  A-stable
  Solving Ordinary Differential Equations II:
  Stiff and Differential-Algebraic Problems,
  2nd Revised Edition
  E. Hairer and G. Wanner
  Table 5.11, pg 76
  c = [  0    1/2    1   ]'
  A = [ 1/6  -1/3   1/6  ]
      [ 1/6   5/12 -1/12 ]
      [ 1/6   2/3   1/6  ]
  b = [ 1/6   2/3   1/6  ]'
\end{verbatim}
  \item[Parent(s):]
    Runge Kutta Butcher Tableau Selection (Section~\ref{sec:Runge Kutta Butcher Tableau Selection})
      \index{Runge Kutta Butcher Tableau Selection} 
  \item[Child(ren):]
    None. 
  \item[Parameters:]
    None. 
\end{list}

% ----------------------------------------------------------
\subsection{Implicit 4 Stage 6th order Lobatto C}
\label{sec:Implicit 4 Stage 6th order Lobatto C}
\index{Implicit 4 Stage 6th order Lobatto C}

\begin{list}{}
  {\setlength{\leftmargin}{1.0in}
   \setlength{\labelwidth}{0.75in}
   \setlength{\labelsep}{0.125in}}
  \item[Description:]
\begin{verbatim}
  Implicit 4 Stage 6th order Lobatto C
  A-stable
  Solving Ordinary Differential Equations II:
  Stiff and Differential-Algebraic Problems,
  2nd Revised Edition
  E. Hairer and G. Wanner
  Table 5.12, pg 76
  c = [ 0     (5-sqrt(5))/10       (5+sqrt(5))/10       1          ]'
  A = [ 1/12  -sqrt(5)/12          sqrt(5)/12          -1/12       ]
      [ 1/12  1/4                  (10-7*sqrt(5))/60   sqrt(5)/60  ]
      [ 1/12  (10+7*sqrt(5))/60    1/4                 -sqrt(5)/60 ]
      [ 1/12  5/12                 5/12                 1/12       ]
  b = [ 1/12  5/12                 5/12                 1/12       ]'
\end{verbatim}
  \item[Parent(s):]
    Runge Kutta Butcher Tableau Selection (Section~\ref{sec:Runge Kutta Butcher Tableau Selection})
      \index{Runge Kutta Butcher Tableau Selection} 
  \item[Child(ren):]
    None. 
  \item[Parameters:]
    None. 
\end{list}

% ----------------------------------------------------------
\subsection{Interpolation Buffer Settings}
\label{sec:Interpolation Buffer Settings}
\index{Interpolation Buffer Settings}

\begin{list}{}
  {\setlength{\leftmargin}{1.0in}
   \setlength{\labelwidth}{0.75in}
   \setlength{\labelsep}{0.125in}}
  \item[Description:]
    This parameter list sets various parameters for the InterpolationBuffer.
  \item[Parent(s):]
    Integrator Base (Section~\ref{sec:Integrator Base})
      \index{Integrator Base} 
  \item[Child(ren):]
    Trailing Interpolation Buffer Selection (Section~\ref{sec:Trailing Interpolation Buffer Selection})
      \index{Trailing Interpolation Buffer Selection} 
      \newline 
    Interpolation Buffer Appender Selection (Section~\ref{sec:Interpolation Buffer Appender Selection})
      \index{Interpolation Buffer Appender Selection} 
      \newline 
    Interpolator Selection (Section~\ref{sec:Interpolator Selection-Interpolation Buffer Settings})
      \index{Interpolator Selection} 
  \item[Parameters:]
    None. 
\end{list}

% ----------------------------------------------------------
\subsection{Trailing Interpolation Buffer Selection}
\label{sec:Trailing Interpolation Buffer Selection}
\index{Trailing Interpolation Buffer Selection}

\begin{list}{}
  {\setlength{\leftmargin}{1.0in}
   \setlength{\labelwidth}{0.75in}
   \setlength{\labelsep}{0.125in}}
  \item[Description:]
    Used to select the Interpolation Buffer.
  \item[Parent(s):]
    Interpolation Buffer Settings (Section~\ref{sec:Interpolation Buffer Settings})
      \index{Interpolation Buffer Settings} 
  \item[Child(ren):]
    Interpolation Buffer (Section~\ref{sec:Interpolation Buffer})
      \index{Interpolation Buffer} 
  \item[Parameters:]
    \begin{description}
      \item[Interpolation Buffer Type = None] 
Determines the type of Rythmos::InterpolationBuffer object that will be built.
The parameters for each Interpolation Buffer Type are specified in this sublist

  Valid std::string values:

      \begin{tabular}{lp{0.4\textwidth}}
      "None" & \\ 
      "Interpolation Buffer" & \\ 
      \end{tabular}
        \index{Trailing Interpolation Buffer Selection!Interpolation Buffer Type}
        \index{Interpolation Buffer Type}
\end{description}

\end{list}

% ----------------------------------------------------------
\subsection{Interpolation Buffer}
\label{sec:Interpolation Buffer}
\index{Interpolation Buffer}

\begin{list}{}
  {\setlength{\leftmargin}{1.0in}
   \setlength{\labelwidth}{0.75in}
   \setlength{\labelsep}{0.125in}}
  \item[Description:]
    Sets parameters for the Interpolation Buffer.
  \item[Parent(s):]
    Trailing Interpolation Buffer Selection (Section~\ref{sec:Trailing Interpolation Buffer Selection})
      \index{Trailing Interpolation Buffer Selection} 
  \item[Child(ren):]
    None. 
  \item[Parameters:]
    \begin{description}
      \item[InterpolationBufferPolicy = Keep Newest Policy] 
Interpolation Buffer Policy for when the maximum storage size is exceeded.  Static will throw an exception when the storage limit is exceeded.  Keep Newest will over-write the oldest data in the buffer when the storage limit is exceeded.

  Valid std::string values:

      \begin{tabular}{lp{0.4\textwidth}}
      "Invalid Policy" & \\ 
      "Static Policy" & \\ 
      "Keep Newest Policy" & \\ 
      \end{tabular}
        \index{Interpolation Buffer!InterpolationBufferPolicy}
        \index{InterpolationBufferPolicy}
      \item[StorageLimit = 0] 
Storage limit for the interpolation buffer.
        \index{Interpolation Buffer!StorageLimit}
        \index{StorageLimit}
\end{description}

\end{list}

% ----------------------------------------------------------
\subsection{Interpolation Buffer Appender Selection}
\label{sec:Interpolation Buffer Appender Selection}
\index{Interpolation Buffer Appender Selection}

\begin{list}{}
  {\setlength{\leftmargin}{1.0in}
   \setlength{\labelwidth}{0.75in}
   \setlength{\labelsep}{0.125in}}
  \item[Description:]
    Used to select the Interpolation Buffer Appender.
  \item[Parent(s):]
    Interpolation Buffer Settings (Section~\ref{sec:Interpolation Buffer Settings})
      \index{Interpolation Buffer Settings} 
  \item[Child(ren):]
    Pointwise Interpolation Buffer Appender (Section~\ref{sec:Pointwise Interpolation Buffer Appender})
      \index{Pointwise Interpolation Buffer Appender} 
  \item[Parameters:]
    \begin{description}
      \item[Interpolation Buffer Appender Type = None] 
Determines the type of Rythmos::InterpolationBufferAppender object that will be built.
The parameters for each Interpolation Buffer Appender Type are specified in this sublist

  Valid std::string values:

      \begin{tabular}{lp{0.4\textwidth}}
      "None" & \\ 
      "Pointwise Interpolation Buffer Appender" & \\ 
      \end{tabular}
        \index{Interpolation Buffer Appender Selection!Interpolation Buffer Appender Type}
        \index{Interpolation Buffer Appender Type}
\end{description}

\end{list}

% ----------------------------------------------------------
\subsection{Pointwise Interpolation Buffer Appender}
\label{sec:Pointwise Interpolation Buffer Appender}
\index{Pointwise Interpolation Buffer Appender}

\begin{list}{}
  {\setlength{\leftmargin}{1.0in}
   \setlength{\labelwidth}{0.75in}
   \setlength{\labelsep}{0.125in}}
  \item[Description:]
    Appender that just transfers nodes without any regard for accuracy or order.
  \item[Parent(s):]
    Interpolation Buffer Appender Selection (Section~\ref{sec:Interpolation Buffer Appender Selection})
      \index{Interpolation Buffer Appender Selection} 
  \item[Child(ren):]
    None. 
  \item[Parameters:]
    None. 
\end{list}

% ----------------------------------------------------------
\subsection{Interpolator Selection}
\label{sec:Interpolator Selection-Interpolation Buffer Settings}
\index{Interpolator Selection}

\begin{list}{}
  {\setlength{\leftmargin}{1.0in}
   \setlength{\labelwidth}{0.75in}
   \setlength{\labelsep}{0.125in}}
  \item[Description:]
    Choose the interpolator to use.
  \item[Parent(s):]
    Interpolation Buffer Settings (Section~\ref{sec:Interpolation Buffer Settings})
      \index{Interpolation Buffer Settings} 
  \item[Child(ren):]
    Linear Interpolator (Section~\ref{sec:Linear Interpolator-Interpolator Selection})
      \index{Linear Interpolator} 
      \newline 
    Hermite Interpolator (Section~\ref{sec:Hermite Interpolator-Interpolator Selection})
      \index{Hermite Interpolator} 
      \newline 
    Cubic Spline Interpolator (Section~\ref{sec:Cubic Spline Interpolator-Interpolator Selection})
      \index{Cubic Spline Interpolator} 
  \item[Parameters:]
    \begin{description}
      \item[Interpolator Type = None] 
Determines the type of Rythmos::Interpolator object that will be built.
The parameters for each Interpolator Type are specified in this sublist

  Valid std::string values:

      \begin{tabular}{lp{0.4\textwidth}}
      "None" & \\ 
      "Linear Interpolator" & \\ 
      "Hermite Interpolator" & \\ 
      "Cubic Spline Interpolator" & \\ 
      \end{tabular}
        \index{Interpolator Selection!Interpolator Type}
        \index{Interpolator Type}
\end{description}

\end{list}

% ----------------------------------------------------------
\subsection{Linear Interpolator}
\label{sec:Linear Interpolator-Interpolator Selection}
\index{Linear Interpolator}

\begin{list}{}
  {\setlength{\leftmargin}{1.0in}
   \setlength{\labelwidth}{0.75in}
   \setlength{\labelsep}{0.125in}}
  \item[Description:]
    This provides a simple linear interpolation between time nodes.
  \item[Parent(s):]
    Interpolator Selection (Section~\ref{sec:Interpolator Selection-Interpolation Buffer Settings})
      \index{Interpolator Selection} 
  \item[Child(ren):]
    None. 
  \item[Parameters:]
    None. 
\end{list}

% ----------------------------------------------------------
\subsection{Hermite Interpolator}
\label{sec:Hermite Interpolator-Interpolator Selection}
\index{Hermite Interpolator}

\begin{list}{}
  {\setlength{\leftmargin}{1.0in}
   \setlength{\labelwidth}{0.75in}
   \setlength{\labelsep}{0.125in}}
  \item[Description:]
    This provides a piecewise cubic Hermite interpolation on each interval where the data is the solution and its time derivatives at the end points of the interval.  It will match 3rd degree polynomials exactly with both function values and derivatives.
  \item[Parent(s):]
    Interpolator Selection (Section~\ref{sec:Interpolator Selection-Interpolation Buffer Settings})
      \index{Interpolator Selection} 
  \item[Child(ren):]
    None. 
  \item[Parameters:]
    None. 
\end{list}

% ----------------------------------------------------------
\subsection{Cubic Spline Interpolator}
\label{sec:Cubic Spline Interpolator-Interpolator Selection}
\index{Cubic Spline Interpolator}

\begin{list}{}
  {\setlength{\leftmargin}{1.0in}
   \setlength{\labelwidth}{0.75in}
   \setlength{\labelsep}{0.125in}}
  \item[Description:]
    This provides a cubic spline interpolation between time nodes.
  \item[Parent(s):]
    Interpolator Selection (Section~\ref{sec:Interpolator Selection-Interpolation Buffer Settings})
      \index{Interpolator Selection} 
  \item[Child(ren):]
    None. 
  \item[Parameters:]
    None. 
\end{list}
