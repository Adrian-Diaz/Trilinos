\noindent {\bf Remarks:}
\begin{itemize}
    \item \lcode{ThyraSolverCG} derives from \lcode{IterativeSolver}
  which, in turn, derives from \lcode{LinearSolverBase}.  If the
  solver is derived from  \lcode{LinearSolverBase} it can be attached
  to a linear operator using the \lcode{inverse} method.  The
  \lcode{IterativeSolver} class has several methods specific to
  iterative solvers, including methods to set and retrieve controls
  such as the tolerance value for convergence and the maximum number
  of iterations allowed.
    \item \lcode{ThyraSolverCG} also derives from \lcode{Printable}
  and \lcode{Describable} which are interfaces that provide a standard
  way for a class to describe itself to the outside world.
    \item verbosity
    \item \paramList\ is used to pass in a set of parameters in a
  standard way.  Basically, \paramList\ is a set of key-value pairs.
  It can be constructed using methods in the class itself or its
  information can be obtained from an \xml\ input file.  We'll show
  how to do this in the main code that follows.
\end{itemize}
