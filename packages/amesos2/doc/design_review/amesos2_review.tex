%\documentclass{beamer}
\documentclass[xcolor=dvipsnames]{beamer}

\usecolortheme[named=brown]{structure}
\useoutertheme{infolines}
%\usefonttheme{serif}
\usefonttheme[stillsansseriftext]{serif}
%\usetheme{Singapore}
\usetheme[height=7mm]{Madrid}
\setbeamertemplate{navigation symbols}{}

%-------------------------------------------------------------------------------
% get epsf.tex file, for encapsulated postscript files:
\input epsf
%-------------------------------------------------------------------------------
% macro for Postscript figures the easy way
% usage:  \postscript{file.ps}{scale}
% where scale is 1.0 for 100%, 0.5 for 50% reduction, etc.
%

\newcommand{\postscript}[2]
{\setlength{\epsfxsize}{#2\hsize}
\centerline{\epsfbox{#1}}}
%-------------------------------------------------------------------------------

\usepackage{graphics}
\usepackage{color}
\usepackage{epsfig}
\renewcommand{\thefootnote}{}

\title[Amesos2]{Amesos2: Common Interface to Direct Solvers}
\author[Bavier, Boman, Rajamanickam]{Eric Bavier and Erik Boman and Siva Rajamanickam}
\institute[]{
Sandia National Laboratories
}
\date[]{July 6, 2011}


\begin{document}

\begin{frame}[plain]
    \titlepage
\footnote{\tiny{Sandia is a multiprogram laboratory operated by Sandia Corporation, a wholly owned subsidiary of Lockheed Martin, for the United States Department of Energy'’s National Nuclear Security Administration under contract DE-AC04-94AL85000.}}
\end{frame}

\begin{frame}
    \frametitle{Amesos2 Motivation}

\begin{itemize}
\item Common Interface for direct solvers.
\medskip
\item Support 32-bit and 64-bit global ids.
\medskip
\item Support Epetra and Tpetra data structures and allow a design that can
      extend to other matrix types (for eg.: PetSc matrix, compressed column
       matrix)
\medskip
\item Revisit Amesos design choices and redesign to easily support more
      solvers and matrix types.
\end{itemize}

\end{frame}

\end{document}
