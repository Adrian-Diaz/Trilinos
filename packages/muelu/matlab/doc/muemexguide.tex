%
% $Id: SANDExampleReportNotstrict.tex,v 1.26 2009-05-01 20:59:19 rolf Exp $
%
% This is an example LaTeX file which uses the SANDreport class file.
% It shows how a SAND report should be formatted, what sections and
% elements it should contain, and how to use the SANDreport class.
% It uses the LaTeX report class, but not the strict option.
%
% Get the latest version of the class file and more at
%    http://www.cs.sandia.gov/~rolf/SANDreport
%
% This file and the SANDreport.cls file are based on information
% contained in "Guide to Preparing {SAND} Reports", Sand98-0730, edited
% by Tamara K. Locke, and the newer "Guide to Preparing SAND Reports and
% Other Communication Products", SAND2002-2068P.
% Please send corrections and suggestions for improvements to
% Rolf Riesen, Org. 9223, MS 1110, rolf@cs.sandia.gov
%
\documentclass[pdf,12pt,report]{SANDreport}
\usepackage{algpseudocode}
\usepackage{amsthm}
\usepackage{booktabs}
\usepackage{calc}
\usepackage{color}
\usepackage{eso-pic}
\usepackage{fancyhdr}
\usepackage{ifthen}
\usepackage{indentfirst}
\usepackage{geometry}
\usepackage{graphicx}
\usepackage[colorlinks, bookmarksopen, %pagebackref=true, backref=page,
             linkcolor={blue},
             anchorcolor={black},
             citecolor={blue},
             filecolor={magenta},
             menucolor={blue},
             pagecolor={red},
             plainpages=false,pdfpagelabels,
             pdfauthor={Brian Kelley, Chris Siefert, Ray Tuminaro},
             pdftitle={MueMex User's Guide},
             pdfkeywords={MueLu,AMG,multigrid,guide,user},
             urlcolor={blue}]{hyperref}
\usepackage{listings}
\usepackage{mathptmx}	% Use the Postscript Times font
\usepackage{multirow}
\usepackage{pifont}
\usepackage[FIGBOTCAP,normal,bf,tight]{subfigure}
\usepackage{tabularx}
\usepackage{verbatim}
\usepackage{xspace}
\usepackage{flowchart} % also loads tikz
\usepackage{algorithm}
\usetikzlibrary{arrows}

%\usepackage{draftwatermark}
%\SetWatermarkScale{.5}

\algrenewcommand{\algorithmiccomment}[1]{\hskip3em // #1}


% If you want to relax some of the SAND98-0730 requirements, use the "relax"
% option. It adds spaces and boldface in the table of contents, and does not
% force the page layout sizes.
% e.g. \documentclass[relax,12pt]{SANDreport}
%
% You can also use the "strict" option, which applies even more of the
% SAND98-0730 guidelines. It gets rid of section numbers which are often
% useful; e.g. \documentclass[strict]{SANDreport}



% ---------------------------------------------------------------------------- %
%
% Set the title, author, and date
%
\title{MueMex User's Guide}

\author{Brian Kelley\\
  Scalable Algorithms \\
  Sandia National Laboratories\\
  Mailstop 1318 \\
  P.O.~Box 5800 \\
  Albuquerque, NM 87185-1318\\
  bmkelle@sandia.gov\\
  \and
  Christopher M. Siefert\\
  Computational Multiphysics\\
  Sandia National Laboratories\\
  Mailstop 1322 \\
  P.O.~Box 5800 \\
  Albuquerque, NM 87185-1322\\
  csiefer@sandia.gov
}

% There is a "Printed" date on the title page of a SAND report, so
% the generic \date should generally be empty.
\date{}

\newcommand{\JG}[1]{\textcolor{JG: Red}{#1}}
\newcommand{\RST}[1]{\textcolor{RayBlue}{RST: #1}}
\newcommand{\JJH}[1]{\textcolor{jhuGreen}{JJH: #1}}
\newcommand{\CMS}[1]{\textcolor{cmsPurple}{CMS: #1}}

% For displaying class names, computer code, etc.
%\newcommand{\cc}[1]{{\lstinline!#1!}}
\newcommand{\cc}[1]{{\tt #1}}

% Package names.
\newcommand{\amesos}{{\sc Amesos}\xspace}
\newcommand{\anasazi}{{\sc Anasazi}\xspace}
\newcommand{\aztecoo}{{\sc AztecOO}\xspace}
\newcommand{\belos}{{\sc Belos}\xspace}
\newcommand{\epetra}{{\sc Epetra}\xspace}
\newcommand{\ifpack}{{\sc Ifpack}\xspace}
\newcommand{\isorropia}{{\sc Isorropia}\xspace}
\newcommand{\ml}{{\sc ML}\xspace}
\newcommand{\muelu}{{\sc \textsf{{MueLu}}}\xspace}
\newcommand{\muemat}{{\sc \textsl{MueMat}}\xspace}
\newcommand{\nox}{{\sc NOX}\xspace}
\newcommand{\teuchos}{{\sc Teuchos}\xspace}
\newcommand{\tifpack}{{\sc Ifpack2}\xspace}
\newcommand{\tpetra}{{\sc Tpetra}\xspace}
\newcommand{\trilinos}{{\sc Trilinos}\xspace}
\newcommand{\zoltan}{{\sc Zoltan}\xspace}
\newcommand{\xpetra}{{\sc Xpetra}\xspace}

% Miscellaneous.
\newcommand{\be}{\begin{enumerate}}
\newcommand{\ee}{\end{enumerate}}


\newtheorem*{mycomment}{\ding{42}}
\newtheoremstyle{plain}
  {\topsep}   % ABOVESPACE
  {\topsep}   % BELOWSPACE
  {\normalfont}  % BODYFONT
  {0pt}       % INDENT (empty value is the same as 0pt)
  {\bfseries} % HEADFONT
  {}         % HEADPUNCT
  {5pt plus 1pt minus 1pt} % HEADSPACE
  {}          % CUSTOM-HEAD-SPEC

% further declarations and additional commands
\definecolor{hellgelb}{rgb}{1,1,0.8}   % background color for C++ listings
\definecolor{darkgreen}{rgb}{0.0, 0.2, 0.13}
%\definecolor{hellrot}{HTML}{FFA4C2}    % background color for xml files

% settings for listings package
\lstset{
  backgroundcolor=\color{hellgelb},
  basicstyle=\ttfamily\small,
  breakautoindent=true,
  breaklines=true,
  captionpos=b,
  columns=flexible,
  commentstyle=\color{darkgreen},
  extendedchars=true,
  float=hbp,
  frame=single,
  identifierstyle=\color{black},
  keywordstyle=\color{blue},
  numbers=none,
  numberstyle=\tiny,
  showspaces=false,
  showstringspaces=false,
  stringstyle=\color{purple},
  tabsize=2,
}


% ---------------------------------------------------------------------------- %
% Set some things we need for SAND reports. These are mandatory
%
\SANDnum{SAND2015-XXXYY}
\SANDprintDate{SomeMonth 2015}
\SANDauthor{Brian Kelley, Christopher M. Siefert}


% ---------------------------------------------------------------------------- %
% Include the markings required for your SAND report. The default is "Unlimited
% Release". You may have to edit the file included here, or create your own
% (see the examples provided).
%
% \include{MarkUR} % Not needed for unlimted release reports


% ---------------------------------------------------------------------------- %
% The following definition does not have a default value and will not
% print anything, if not defined
%
%\SANDsupersed{SAND1901-0001}{January 1901}
%\input{MarkOUO}


% ---------------------------------------------------------------------------- %
%
% Start the document
%
\begin{document}
    \maketitle

    % ------------------------------------------------------------------------ %
    % An Abstract is required for SAND reports
    %
    \begin{abstract}
	The Software Utilities Package for the Engineering Sciences (SUPES) is a
collection of subprograms which perform frequently used non-numerical
services for the engineering applications programmer.  The three functional
categories of SUPES are: (1) input command parsing, (2) dynamic memory
management, and (3) system dependent utilities.  The subprograms in categories
one and two are written in standard FORTRAN-77, while the subprograms in
category three are written to provide a standardized FORTRAN interface to
several system dependent features. 

    \end{abstract}


    % ------------------------------------------------------------------------ %
    % An Acknowledgement section is optional but important, if someone made
    % contributions or helped beyond the normal part of a work assignment.
    % Use \section* since we don't want it in the table of context
    %
    \clearpage
    \chapter*{Acknowledgment}
	Many people have helped develop \muelu{} and/or provided valuable feedback, and
we would like to acknowledge their contributions here: Tom Benson, Julian
Cortial, Eric Cyr, Stefan Domino, Travis Fisher, Jeremie Gaidamour, Axel
Gerstenberger, Chetan Jhurani, Mark Hoemmen, Paul Lin, Eric Phipps, Siva
Rajamanickam, Nico Schl{\"o}mer, and Paul Tsuji.



    % ------------------------------------------------------------------------ %
    % The table of contents and list of figures and tables
    % Comment out \listoffigures and \listoftables if there are no
    % figures or tables. Make sure this starts on an odd numbered page
    %
    \cleardoublepage		% TOC needs to start on an odd page
    \tableofcontents
    \listoffigures
    \listoftables


    % ---------------------------------------------------------------------- %
    % An optional preface or Foreword
    %\clearpage
    %\chapter*{Preface}
    %\addcontentsline{toc}{chapter}{Preface}
	%    Although staff members usually have only limited experience
    with dry-erase markers, and many even dispute their existence,
    it is worthwhile to be open minded and explore the possibilities.



    % ---------------------------------------------------------------------- %
    % An optional executive summary
    %\clearpage
    %\chapter*{Summary}
    %\addcontentsline{toc}{chapter}{Summary}
	%    Once a certain level of mistrust and skepticism has been
    overcome, dry-erase markers find many uses in todays science and
    engineering. In this report we explain some of the fundamental
    properties, dangers, and benefits of dry-erase markers. We then
    conclude with a few examples on how they can be used in daily
    activities at national Laboratories.



    % ---------------------------------------------------------------------- %
    % An optional glossary. We don't want it to be numbered
    %\clearpage
    %\chapter*{Nomenclature}
    %\addcontentsline{toc}{chapter}{Nomenclature}
    %\begin{description}
	%\item[dry spell]
	%    using a dry erase marker to spell words
	%\item[dry wall]
	%    the writing on the wall
	%\item[dry humor]
	%    when people just do not understand
	%\item[DRY]
	%    Don't Repeat Yourself
    %\end{description}


    % ---------------------------------------------------------------------- %
    % This is where the body of the report begins; usually with an Introduction
    %
    \SANDmain		% Start the main part of the report

    %-----------------------------%
    \chapter{Users}\label{sec:users}
    %-----------------------------%
    Muemex - MueLu Interface for MATLAB

\subsection{Basic Instructions}

\begin{enumerate}
  \item Run "matlab" script in this directory to launch matlab with proper shared libraries
  \item Run "help muelu" from matlab to see detailed help for "muelu" function
  \item Basic setup for muelu is "problemID = muelu('setup', A);"
  \item Basic solve for muelu is "x = muelu(problemID, b);"
  \item Run "ctest" in this directory to run the experimental matlab tests for MueLu
\end{enumerate}

\subsection{Full Instructions}

Navigate to muelu/matlab/bin.\\
Run MATLAB through the "matlab" script.\\
Alternatively, find the value for {\tt \$LD\_PRELOAD} printed by the "./matlab" script and set it yourself.\\
Then MATLAB can be run directly with

\begin{lstlisting}[language=Matlab]
    matlab -nosplash -nojvm
\end{lstlisting}

The "muelu" function will only be available if "muelu/matlab/bin" is in MATLAB's path.
If you want to use it from a different directory, run

\begin{lstlisting}[language=Matlab]
    "addpath('path/to/matlab/bin');"
\end{lstlisting}

from within MATLAB.

With a sparse matrix A, set up a MueLu hierarchy:

\begin{lstlisting}[language=Matlab]
    problemID = muelu('setup', A);
\end{lstlisting}

Run

\begin{lstlisting}[language=Matlab]
    A = laplacianfun([50, 50]);
\end{lstlisting}

to get a 50x50 2D Laplace matrix.

Note: any number of problems can be set up at a time.
Optionally, pass A and fine level coordinates:

\begin{lstlisting}[language=Matlab]
    problemID = muelu('setup', A, coords);
\end{lstlisting}

Parameters for the "easy parameter list" are listed after A and coords:

\begin{lstlisting}[language=Matlab]
    problemID = muelu('setup', A, coords, 'coarse: max size', 50);
\end{lstlisting}

Sublists can be passed using MATLAB cell arrays.

\begin{lstlisting}[language=Matlab]
    problemID = muelu('setup', A, 'level 0', {'aggregation: drop tol', 0.03}, 'level 1', {'aggregation: drop tol', 0.01});
\end{lstlisting}

XML parameter lists can also be used.

\begin{lstlisting}[language=Matlab]
    problemID = muelu('setup', A, 'xml parameter file', 'myParams.xml');
\end{lstlisting}

The setup function can also return operator complexity:

\begin{lstlisting}[language=Matlab]
    [problemID, oc] = muelu('setup', A);
\end{lstlisting}

By default, muemex uses Tpetra objects to store the MueLu preconditioner.
This can be customized by setting the parameter "Linear Algebra".

\begin{lstlisting}[language=Matlab]
    problemID = muelu('setup', A, 'Linear Algebra', 'epetra');
    problemID = muelu('setup', A, 'Linear Algebra', 'tpetra');
\end{lstlisting}

Parameters can be strings, integers, booleans or arrays (full or sparse, real
or complex).

Solve a problem with b as the right-hand side vector(s):

\begin{lstlisting}[language=Matlab]
    x = muelu(problemID, b);
    x = muelu(problemID, A, b);
\end{lstlisting}

The solve function can also return the number of iterations taken by Belos:

\begin{lstlisting}[language=Matlab]
    [x, numIters] = muelu(problemID, b);
\end{lstlisting}

The first version uses the A that was passed when setting up the problem.
The second version uses a new A with the same hierarchy.

Parameters for Belos can be listed after the required parameters:

\begin{lstlisting}[language=Matlab]
    [x, numIters] = muelu(0, b, 'Maximum Iterations', 400);
\end{lstlisting}

Remove a specific problem and free memory associated with it:

\begin{lstlisting}[language=Matlab]
    muelu('cleanup', problemID);
\end{lstlisting}

or all problems:

\begin{lstlisting}[language=Matlab]
    muelu('cleanup');
\end{lstlisting}

List basic information about problem(s):

\begin{lstlisting}[language=Matlab]
    muelu('status', problemID);
    muelu('status');
\end{lstlisting}

Get data from a level:

\begin{lstlisting}[language=Matlab]
    data = muelu('get', problemID, levelID, dataName);
\end{lstlisting}

This works if dataName is A, P, R, Nullspace, Coordinates, or Aggregates.
Otherwise, the data type must also be specified.
    
\begin{lstlisting}[language=Matlab]
    data = muelu('get', problemID, levelID, dataName, dataType);
\end{lstlisting}

For example, to get the finest prolongator matrix from problem 0, do

\begin{lstlisting}[language=Matlab]
    P = muelu('get', 0, 1, 'P');
\end{lstlisting}

Or, to get non-standard data:
    
\begin{lstlisting}[language=Matlab]
    myList = muelu('get', 0, 0, 'SpecialNodeList', 'OrdinalVector');
\end{lstlisting}

Matrix, MultiVector, OrdinalVector, Int, Scalar, Double and Complex are the supported types.
Note that any data retrieved this way must have a keep flag set on it during hierarchy setup.
For example, "Aggregates" are not normally available since they are discarded after P is created.


    %-----------------------------%
    \chapter{Developers}\label{sec:developers}
    %-----------------------------%
    Muemex - MueLu interface for MATLAB
Developer Features

\subsection{Matlab Factories}

\begin{enumerate}
  \item MatlabSmoother, SingleLevelMatlabFactory and TwoLevelMatlabFactory are
implementations of SmootherPrototype, SingleLevelFactoryBase, and
TwoLevelFactoryBase that use matlab functions instead of C++ code
to generate data in MueLu levels.

Parameters: Set in XML parameter list in the factory instantiation.

MatlabSmoother: "Needs", "Setup Function", "Solve Function", "Number of Solver
Args"
  \item "Needs" is a comma-separated list of hierarchy data that the smoother will
request, and pass into setup function in order that they are listed.
  \item "Setup Function" is the matlab function to run to set up the smoothing. Must
take a sparse matrix (A) followed by the matlab types corresponding to "Needs"
  \item "Solve Function" is the matlab function to run the solve phase of the
smoother. Must take sparse matrix A, double array X, double array B, followed by
the outputs of "Setup Function" as arguments. Shouldn't return anything.
  \item "Number of Solver Args" (int) is the number of expected outputs of "Setup
Function" that will be stored until the solve phase, when they will be
passed in after A, x, b.

SingleLevelMatlabFactory: "Needs", "Provides", "Function"
  \item "Needs" is list of inputs to the matlab function that will be pulled from
level. "Level" is a special key for the Needs list, and will be passed to MATLAB with the
current level ID.
  \item "Provides" is what will be returned by the matlab function and added to the
level.
  \item "Function" is the name of the matlab function to run. Parameters/return
values must match Needs/Provides.

TwoLevelMatlabFactory: "Needs Fine", "Needs Coarse", "Provides", "Function"
  \item Just like SingleLevelMatlabFactory, but inputs come from both fine and coarse
levels.
\end{enumerate}

\subsection{Custom Variables}

Muemex also supports setting custom data in the hierarchy. To use this
feature, set the data you want as a parameter in a level sublist when
setting up the problem. For example, if you want a matrix "MyMatrix" to
be available to a matlab factory, set up with this command:

\begin{lstlisting}[language=Matlab]
muelu('setup', A, 'level 0', {'Matrix MyMatrix', MyMatrix}, 'xml parameter
file', 'myXMLParams.xml');
\end{lstlisting}

Notice the type specifier "Matrix" before the variable name. This is part
of the name, and has to be included in every mention of the variable. The
type is not case sensitive; 'MultiVector' and 'multivector' are equivalent.
The name itself can be any string without whitespace. Names have to unique
within their level - there can't be "Double d1" and "Int d1".

If an aggregates factory needs that custom matrix, the XML parameter list for
the problem might look like this:

\begin{lstlisting}[language=XML]
...
<Parameter name="factory" type="string" value="SingleLevelMatlabFactory"/>
<Parameter name="Provides" type="string" value="Aggregates"/>
<Parameter name="Needs" type="string" value="A, Matrix MyMatrix"/>
<Parameter name="Function" type="string" value="simpleAggregation"/>
...
\end{lstlisting}

This will send MyMatrix to simpleAggregation as the second argument.
There must always be exactly one space between the
type and the name. Leading and trailing spaces are always ignored.

The following custom variable types are supported:
\begin{enumerate}
\item Matrix (sparse, MxM, real or complex depending on MueLu context)
\item MultiVector
\item OrdinalVector (must be Mx1 column vector of int32)
\item Int (32 bit signed)
\item Scalar (double or complex depending on context)
\item Double
\item Complex
\end{enumerate}

Note: Custom variables added to the hierarchy either through muelu('setup') or
by a matlab factory are never removed from their level - a UserData keep
flag is set for them.

\subsection{Matlab Callbacks}

The "MueLu\_MatlabUtils" code contains the backend for the matlab factories.
There is a system for defining inputs and outputs for arbitrary matlab
functions. There are also functions for converting between matlab data types (mxArray*)
and C++/Xpetra data types.

\begin{lstlisting}[language=C++]
vector<RCP<MuemexArg>> callMatlab(string function, int numOutputs, vector<RCP<MuemexArg>> args);
\end{lstlisting}

This function is declared in "MueLu\_MatlabUtils\_decl.hpp". It calls the matlab function "function",
with "args" as arguments, expecting numOutputs outputs, and returning the list of returned values.
The MuemexArg class is a basic wrapper for the several types of data that can be passed to and from
matlab. It only stores a MUEMEX\_TYPE representing the underlying type of the data:

\begin{lstlisting}[language=C++]
enum MUEMEX_TYPE
{
  INT,
  DOUBLE,
  STRING,
  COMPLEX,
  XPETRA_ORDINAL_VECTOR,
  TPETRA_MULTIVECTOR_DOUBLE,
  TPETRA_MULTIVECTOR_COMPLEX,
  TPETRA_MATRIX_DOUBLE,
  TPETRA_MATRIX_COMPLEX,
  XPETRA_MATRIX_DOUBLE,
  XPETRA_MATRIX_COMPLEX,
  XPETRA_MULTIVECTOR_DOUBLE,
  XPETRA_MULTIVECTOR_COMPLEX,
  EPETRA_CRSMATRIX,
  EPETRA_MULTIVECTOR,
  AGGREGATES,
  AMALGAMATION_INFO
};

class MuemexArg
{
    MuemexArg(MUEMEX_TYPE dataType);
    MUEMEX_TYPE type;
};
\end{lstlisting}

MuemexData is a subclass of MuemexArg. It has one the types listed in MUEMEX\_TYPE as a field.

\begin{lstlisting}[language=C++]
template<typename T>
class MuemexData
{
    public:
        MuemexData(T& data);
        MuemexData(const mxArray* mxa);
        mxArray* convertToMatlab();
        T& getData();
    private:
        T data;
};
\end{lstlisting}

MuemexData can either be constructed using the actual underlying data object, or an mxArray*. Either
way, the data is copied and stored as a C++/Xpetra object. Note that MuemexData can only be used with Trilinos
types wrapped in Teuchos::RCP. Int, double, string and complex should not be wrapped.

Any RCP<MuemexData<T>> object can be converted to a pure RCP<MuemexArg> with

\begin{lstlisting}[language=C++]
    RCP<MuemexArg> arg = rcp_implicit_cast<MuemexArg>(mmData);
\end{lstlisting}

and back with

\begin{lstlisting}[language=C++]
    RCP<MuemexData<T>> mmData = rcp_static_cast<MuemexData<T>>(arg);
\end{lstlisting}

The type of the underlying MuemexData data is stored in MuemexArg::type.

\subsection{Matlab Callback Example}

This matlab function calculates the inverse tangent and also times the calculation.

\begin{lstlisting}[language=Matlab]
    function [angle, timeElapsed] = timedAtan(y, x)
        tic;
        angle = atan2(y, x);
        timeElapsed = toc;
    end
\end{lstlisting}

To use the function from C++, do:

\begin{lstlisting}[language=C++]
    double y = 5.423;
    double x = 42.6;
    vector<MuemexArg> inputs;
    inputs.push_back(rcp(new MuemexData<double>(y)));                                      //add the first argument (y)
    inputs.push_back(rcp(new MuemexData<double>(x)));                                      //add the second argument (x)
    /*
      Note that it is not necessary to rcp_implicit_cast the RCP<MuemexData> objects when adding them to MuemexArg vector - this happens implicitly
    */
    vector<MuemexArg> outputs = callMatlab("timedAtan", 2, inputs);                        //call the function, expect 2 outputs
    RCP<MuemexData<double>> angleOutput = rcp_static_cast<MuemexData<double>>(outputs[0]); //recover the outputs, knowing both have type "double"
    RCP<MuemexData<double>> timeOutput = rcp_static_cast<MuemexData<double>>(outputs[1]);
    printf("The arctan of %f/%f is %f, and the calculation took %f seconds.", y, x, angleOutput->getData(), timeOutput->getData());
\end{lstlisting}

\subsection{Direct Data Conversion}

It is also possible to directly convert between matlab arrays and C++ objects (without using a MuemexData wrapper).
To access a matlab array pointer "mxa" from C++, do:

\begin{lstlisting}[language=C++]
   int myInt = loadDataFromMatlab<int>(mxa);
\end{lstlisting}

To copy a C++ object to matlab, do:

\begin{lstlisting}[language=C++]
   mxArray* mxa = saveDataToMatlab(data);
\end{lstlisting}

The following types are supported by these functions ('Node' is the default Kokkos serial node type):

\begin{enumerate}
  \item {\tt int}
  \item {\tt double}
  \item {\tt std::complex<double>}
  \item {\tt std::string}
  \item {\tt RCP<Xpetra::Matrix<double, int, int, Node>>}
  \item {\tt RCP<Xpetra::Matrix<complex, int, int, Node>>}
  \item {\tt RCP<Xpetra::MultiVector<double, int, int, Node>>}
  \item {\tt RCP<Xpetra::MultiVector<complex, int, int, Node>>}
  \item {\tt RCP<Tpetra::CrsMatrix<double, int, int, Node>>}
  \item {\tt RCP<Tpetra::CrsMatrix<complex, int, int, Node>>}
  \item {\tt RCP<Tpetra::MultiVector<double, int, int, Node>>}
  \item {\tt RCP<Tpetra::MultiVector<complex, int, int, Node>>}
  \item {\tt RCP<Epetra\_CrsMatrix>}
  \item {\tt RCP<Epetra\_MultiVector>}
  \item {\tt RCP<Xpetra::Vector<int, int, int, Node>>}
  \item {\tt RCP<MueLu::Aggregates<int, int, Node>>}
\end{enumerate}

Note that the Aggregates structure in matlab must be constructed with the "constructAggregates.m" function:

\begin{lstlisting}[language=Matlab]
    function agg = constructAggregates(nVertices, nAggregates, vertexToAggID, rootNodes, aggSizes);

        % nVertices:     Total number of nodes in the problem (since all MueMex problems are serial)
        % nAggregates:   Total number of aggregates
        % vertexToAggID: Array of length nVertices, maps nodes to aggregate IDs
        % rootNodes:     Array of length nAggregates, contains node IDs of all the aggregate roots
        % aggSizes:      Array of length nAggregates, contains the number of nodes in each aggregate
\end{lstlisting}



    %\nocite{*}

    % ---------------------------------------------------------------------- %
    % References
    %
    \clearpage
    % If hyperref is included, then \phantomsection is already defined.
    % If not, we need to define it.
    \providecommand*{\phantomsection}{}
    \phantomsection
    \addcontentsline{toc}{chapter}{References}
    \bibliographystyle{plain}
    \bibliography{muemex}


    % ---------------------------------------------------------------------- %
    %
    \appendix
    %\chapter{Copyright and License}
    %\label{sec:license}
\begin{center}
MueLu: A package for multigrid based preconditioning

Copyright 2012 Sandia Corporation
\end{center}

\noindent
Under the terms of Contract DE--AC04--94AL85000 with Sandia Corporation,
the U.S. Government retains certain rights in this software.

\noindent
Redistribution and use in source and binary forms, with or without
modification, are permitted provided that the following conditions are
met:

\begin{enumerate}
  \item Redistributions of source code must retain the above copyright
    notice, this list of conditions and the following disclaimer.

\item Redistributions in binary form must reproduce the above copyright
  notice, this list of conditions and the following disclaimer in the
  documentation and/or other materials provided with the distribution.

\item Neither the name of the Corporation nor the names of the
  contributors may be used to endorse or promote products derived from
  this software without specific prior written permission.
\end{enumerate}

\noindent
THIS SOFTWARE IS PROVIDED BY SANDIA CORPORATION ``AS IS'' AND ANY
EXPRESS OR IMPLIED WARRANTIES, INCLUDING, BUT NOT LIMITED TO, THE
IMPLIED WARRANTIES OF MERCHANTABILITY AND FITNESS FOR A PARTICULAR
PURPOSE ARE DISCLAIMED\@. IN NO EVENT SHALL SANDIA CORPORATION OR THE
CONTRIBUTORS BE LIABLE FOR ANY DIRECT, INDIRECT, INCIDENTAL, SPECIAL,
EXEMPLARY, OR CONSEQUENTIAL DAMAGES (INCLUDING, BUT NOT LIMITED TO,
PROCUREMENT OF SUBSTITUTE GOODS OR SERVICES\@; LOSS OF USE, DATA, OR
PROFITS\@; OR BUSINESS INTERRUPTION) HOWEVER CAUSED AND ON ANY THEORY OF
LIABILITY, \\WHETHER IN CONTRACT, STRICT LIABILITY, OR TORT (INCLUDING
NEGLIGENCE OR OTHERWISE) ARISING IN ANY WAY OUT OF THE USE OF THIS
SOFTWARE, EVEN IF ADVISED OF THE POSSIBILITY OF SUCH DAMAGE\@.

    %\chapter{Historical Perspective}
	%    This is an example of an appendix.

    If we follow~\cite{Sand98-0730} strictly, we would have to have
    a separate bibliography section for each appendix.  The style
    file doesn't provide that, but it can be done using the {\tt
    bibunits} and {\tt chapterbib} packages.

    If there are many subsections in an appendix, it should also
    have its own table of contents. Again, the SAND report class
    file does not provide that functionality.

    \ifthenelse{\boolean{reportSAND}}   {
	\section{The Past a Long Time Ago}
    }{
	\subsection{The Past a Long Time Ago}
    }
	This is where we talk about things so old nobody can verify
	them. We are safe.

    \ifthenelse{\boolean{reportSAND}}   {
	\section{The Past More Recently}
    }{
	\subsection{The Past More Recently}
    }
	Now we have to be a little bit more careful, since records
	exist from that time, and some people still alive actually
	lived back then.



    %\chapter{Some Other Appendix}
	%    Just to show what a second Appendix would look like. It contains
    a table. Each appendix is supposed to be self-contained, so
    tables and figures are not supposed to show up in the main
    table of contents. There can be a separate table of contents
    for each appendix.

    \begin{table}[ht]
	\centering
	\caption{A small table}
	\bigskip

	\begin{tabular}{|c|c|}
	    \hline
		A & B  \\ \hline
		C & D  \\ \hline
	\end{tabular}
	\label{tab3}
    \end{table}

    \begin{figure}[ht]
	\centering
	\begin{picture}(50,50)(0,0)
	    \put(25,25){\circle{1}}
	    \put(25,25){\circle{5}}
	    \put(25,25){\circle{10}}
	    \put(25,25){\circle{15}}
	    \put(25,25){\circle{20}}
	    \put(25,25){\circle{25}}
	    \put(25,25){\circle{30}}
	    \put(25,25){\circle{35}}
	    \put(25,25){\circle{40}}
	    \put(25,25){\circle{45}}
	    \put(25,25){\circle{50}}
	\end{picture}
	\caption{Dizzy yet?}
	\label{fig4}
    \end{figure}


    % \printindex

    %
% This is an example of how to create the distribution page. Some
% distributions are required by Sandia; e.g. the housekeeping copies.
% Depending on the type of report; e.g. CRADA, Patent Caution, etc.
% additional distribution lines may have to be added. See the
% "Guide for Preparing SAND Reports"
%

{\bfseries DISTRIBUTION}

{\bfseries Email--External (encrypt for OUO)}\\[1ex]
\begin{tabular}{|p{4cm}|p{5cm}|p{4cm}|}
  \hline
  \multicolumn{1}{|>{\centering\arraybackslash}p{4cm}|}{\cellcolor{SANDgreen}\bfseries Name} & \multicolumn{1}{>{\centering\arraybackslash}p{5cm}|}{\cellcolor{SANDgreen}\bfseries Company Email Address} & \multicolumn{1}{>{\centering\arraybackslash}p{4cm}|}{\cellcolor{SANDgreen}\bfseries Company Name}\\
  \hline
  Matthias Mayr & matthias.mayr@unibw.de & University of the Bundeswehr Munich\\
  \hline
  Andrey Prokopenko & prokopenkoav@ornl.gov & Oak Ridge National Laboratory\\
  \hline
  Tobias Wiesner & tobias.wiesner@leica-geosystems.com & Leica Geosystems AG\\
  \hline
\end{tabular}

{\bfseries Email--Internal}\\[1ex]
\begin{tabular}{|p{5cm}|p{3cm}|p{5cm}|}
  \hline
  \multicolumn{1}{|c}{\cellcolor{SANDgreen}\bfseries Name} & \multicolumn{1}{|c}{\cellcolor{SANDgreen}\bfseries Org.} & \multicolumn{1}{|c|}{\cellcolor{SANDgreen}\bfseries Sandia Email Address}\\
  \hline
  Luc Berger-Vergiat & 1442 & lberge@sandia.gov\\
  \hline
  Christopher Siefert & 1465 & csiefer@sandia.gov\\
  \hline
  Christian Glusa & 1465 & caglusa@sandia.gov\\
  \hline
  Mark Hoemmen & 1541 & mhoemme@sandia.gov\\
  \hline
  Jonathan Hu & 1465 & jhu@sandia.gov\\
  \hline
  Paul Lin & 1422 & ptlin@sandia.gov\\
  \hline
  Ray Tuminaro & 1442 & rstumin@sandia.gov\\
  \hline
  Technical Library & 9536 & libref@sandia.gov\\
  \hline
\end{tabular}


\end{document}
