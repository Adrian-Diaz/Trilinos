\documentclass{article}[11pt]
\usepackage{fancyhdr, tabularx, verbatim, epsfig}
\usepackage{amssymb,psboxit}
\usepackage{rotating}
\usepackage[pdftex,
            pdfpagemode=none,
            pdfstartview=FitH]{hyperref}
\usepackage{xspace}

\usepackage{graphicx,type1cm,eso-pic,color}

% Lightly print `DRAFT' on every page of the document
\makeatletter
\AddToShipoutPicture{%
  \setlength{\@tempdimb}{.5\paperwidth}%
    \setlength{\@tempdimc}{.5\paperheight}%
    \setlength{\unitlength}{1pt}%
    \put(\strip@pt\@tempdimb,\strip@pt\@tempdimc){%
      \makebox(0,0){\rotatebox{45}{\textcolor[gray]{0.93}%
        {\fontsize{5cm}{5cm}\selectfont{DRAFT}}}}%
    }%
}
\makeatother


\hypersetup{
  pdfauthor  = {Jonathan J. Hu, Andrey Prokopenko},
  pdftitle   = {MueLu 0.01 User's Guide},
  colorlinks = {true},
  citecolor  = {blue},
}

\def\optionbox#1#2{\noindent$\hphantom{ii}${\parbox[t]{1.5in}{\it
#1}}{\parbox[t]{4.8in}{#2}} \\[1.1em]}

\def\choicebox#1#2{\noindent$\hphantom{th}$\parbox[t]{3.0in}{\sf
#1}\parbox[t]{3.35in}{#2}\\[0.8em]}

\def\structbox#1#2{\noindent$\hphantom{hix}${\parbox[t]{2.10in}{\it
#1}}{\parbox[t]{3.9in}{#2}} \\[.02cm]}

\def\protobox#1{\vspace{2em}{\flushleft{\bf Prototype}
\hrulefill}\flushleft{\fbox{\parbox[t]{6in}{\vspace{1em}{\sf
#1}\vspace{1em}}}}}


\setlength{\oddsidemargin} {0.1\oddsidemargin}
\setlength{\evensidemargin}{0.5\evensidemargin}
\setlength{\topmargin}     {0.0\topmargin}
\setlength{\textheight}    {1.16\textheight}
\setlength{\textwidth}     {1.35\textwidth}

\newcommand{\amesos}       {\textsc{Amesos}\xspace}
\newcommand{\anasazi}      {\textsc{Anasazi}\xspace}
\newcommand{\aztecoo}      {\textsc{AztecOO}\xspace}
\newcommand{\belos}        {\textsc{Belos}\xspace}
\newcommand{\epetra}       {\textsc{Epetra}\xspace}
\newcommand{\epetraext}    {\textsc{EpetraExt}\xspace}
\newcommand{\galeri}       {\textsc{Galeri}\xspace}
\newcommand{\ifpack}       {\textsc{Ifpack}\xspace}
\newcommand{\ifpacktwo}    {\textsc{Ifpack2}\xspace}
\newcommand{\loca}         {\textsc{Loca}\xspace}
\newcommand{\ml}           {\textsc{ML}\xspace}
\newcommand{\muelu}        {\textsc{MueLu}\xspace}
\newcommand{\nox}          {\textsc{NOX}\xspace}
\newcommand{\stratimikos}  {\textsc{Stratimikos}\xspace}
\newcommand{\teuchos}      {\textsc{Teuchos}\xspace}
\newcommand{\trilinos}     {\textsc{Trilinos}\xspace}
\newcommand{\zoltan}       {\textsc{Zoltan}\xspace}
\newcommand{\zoltantwo}    {\textsc{Zoltan2}\xspace}

\newcommand{\klu}          {\textsc{Klu}\xspace}
\newcommand{\metis}        {\textsc{Metis}\xspace}
\newcommand{\mumps}        {\textsc{Mumps}\xspace}
\newcommand{\umfpack}      {\textsc{Umfpack}\xspace}
\newcommand{\superlu}      {\textsc{SuperLU}\xspace}
\newcommand{\superludist}  {\textsc{SuperLU\_dist}\xspace}
\newcommand{\parmetis}     {\textsc{ParMetis}\xspace}
\newcommand{\paraview}     {\textsc{ParaView}\xspace}

\newcommand{\parameterlist}{\texttt{ParameterList}\xspace}

\newcommand \trilinosWeb   {trilinos.sandia.gov\xspace}

\newcommand{\cba}[3]{\choicebox{\texttt{#1}}{[{\texttt #2}] #3}}
\newcommand{\cbb}[4]{\choicebox{\texttt{#1}}{[{\texttt #2}] #4 {\bf Default:~}#3.}}
\newcommand{\cbc}[4]{\choicebox{\texttt{\color{red}#1}}{[{\texttt #2}] #4 {\bf Default:~}#3.}}

\newcommand{\comm}[2]{\bigskip
                      \begin{tabular}{|p{4.5in}|}\hline
                      \multicolumn{1}{|c|}{{\bf Comment by #1}}\\ \hline
                      #2\\ \hline
                      \end{tabular}\\
                      \bigskip
                     }

\begin{document}
% \bibliographystyle{siam}
\setcounter{page}{3}

\large

\begin{center}
SAND????-????
Unlimited Release \\
Printed Oct 2013
\end{center}

\vspace{0.2in}

\begin{center}
{\Large {\bf MueLu 0.01 User's Guide}}

\vspace*{0.8in}
Andrey Prokopenko \\
Computational Math \& Algorithms \\
Sandia National Laboratories\\
Mailstop 1318 \\
P.O.~Box 5800 \\
Albuquerque, NM 87185-1318\\[10pt]
Jonathan J. Hu \\
Computational Math \& Algorithms \\
Sandia National Laboratories\\
Mailstop 9159 \\
P.O.~Box 0969 \\
Livermore, CA 94551-0969


\vspace*{1in}

\end{center}

\begin{abstract}

Blah-blah-blah.

\end{abstract}

\clearpage
\newpage

\vfill
\begin{center}
(page intentionally left blank)
\end{center}
\clearpage
\newpage


\tableofcontents
\newpage

%-----------------------------%
\section{\muelu options}
%-----------------------------%
In this section, we report the complete list of input parameters. Input parameters are passed to \muelu in a single \parameterlist.

Some of the parameters that affect the preconditioner may in principle be different from level to level. By default, the parameter affects all levels
in the multigrid hierarchy. To change the behaviour on a particular level, say level $d$, one is to construct a sublist of a \parameterlist with the
name "level $d$".

Rather than implementing all possible algorithms, \muelu relies on other \trilinos packages. For instance, smoothers are provided by \ifpack or
\ifpacktwo. Many of those algorithms take a list of parameters specifying their behaviour. \muelu simply passes this list to the proper package.

\

\cbb{verbosity}                          {string}    {"high"}        {Controll of the amount of printed information. Possible values: "none", "low",
                                                                     "medium", "high", "extreme".}
\cbb{number of equations}                {int}       {1}             {Number of PDE equations at each gride node. Only constant block size is considered.}
% \cbc{explicit restriction}               {bool}      {true}          {Restrition operator is constructed as a separate matrix.}
\cbb{max levels}                         {int}       {10}            {Maximum number of levels.}
\cbb{cycle type}                         {string}    {"V"}           {Multigrid cycle type. Possible values: "V", "W".}

\cba{print}                              {\parameterlist}            {Saving a subset of the hierarchy data in a file. Currently, the list can contain any of three parameter names ("A",
                                                                      "P", "R") of type "string" and value "\{$<$levels separated by comma$>$\}". A
                                                                      matrix is saved in two files: a) data is saved in the MatrixMarket format in a
                                                                      file called "A\_$<$level$>$.mm", or similar; b) row map is saved in the
                                                                      MatrixMarket format in a file called "rowmap\_A\_$<$level$>$.mm", or similar.}

\cbb{multigrid algorithm}                {string}    {"sa"}          {Multigrid method. Possible values: "sa", "emin", "pg".}

\cbb{aggregation: type}                  {string}    {"uncoupled"}   {Aggregation scheme. Possible values: "uncoupled", "coupled".}
\cbc{aggregation: symmetrize}            {bool}      {false}         {Symmetrize connectivity graph for nonsymmetric matrices.}
\cbb{aggregation: drop scheme}           {string}    {"classical"}   {Aggregation connectivity dropping scheme. Possible values: "classical", "laplacian".}
\cbb{aggregation: drop tol}              {double}    {0.0}           {Aggregation dropping threshold.}
\cbb{aggregation: Dirichlet threshold}   {double}    {}              {}

\cbb{smoother: pre or post}              {string}    {"both"}        {Smoother combination. Possible values: "pre", "post", "both", "none".}
\cbb{smoother: type}                     {string}    {"gs"}          {Smoother type. Possible values: see Table~\ref{t:smoothers}.}
\cbb{smoother: pre type}                 {string}    {}              {Pre-smoother type. Possible values: see Table~\ref{t:smoothers}.}
\cbb{smoother: post type}                {string}    {}              {Post-smoother type. Possible values: see Table~\ref{t:smoothers}.}
\cba{smoother: params}                   {\parameterlist}            {Smoother parameters. For standard smoothers, \muelu passes them directly to \stratimikos.}
\cba{smoother: pre params}               {\parameterlist}            {Pre-smoother parameters. For standard smoothers, \muelu passes them directly to \stratimikos.}
\cba{smoother: post params}              {\parameterlist}            {Post-smoother parameters. For standard smoothers, \muelu passes them directly to \stratimikos.}
\cbc{smoother: sweeps}                   {int}       {2}             {Number of smoother sweeps for relaxation methods, or the order of polynomial for Chebyshev.}
\cbc{smoother: pre sweeps}               {int}       {2}             {Number of pre-smoother sweeps for relaxation methods, or the order of polynomial for Chebyshev.}
\cbc{smoother: post sweeps}              {int}       {2}             {Number of post-smoother sweeps for relaxation methods, or the order of polynomial for Chebyshev.}

\cbb{coarse: type}                       {string}    {"SuperLU"}     {Coarse solver. Possible values: see Table~\ref{t:coarse_solvers}.}
\cba{coarse: params}                     {\parameterlist}            {Coarse solver parameters. \muelu passes them directly to coarse solver.}
\cbb{coarse: max size}                   {int}       {2000}          {Maximum dimension of the coarse grid. \muelu will stop coarsening once it is achieved.}

\cbb{repartition: enable}                {bool}      {false}         {Repartitioning on/off switch.}
\cbb{repartition: partitioner}           {string}    {"zoltan2"}     {Partitioning package to use. Possible values: "zoltan", "zoltan2".}
\cba{repartition: params}                {\parameterlist}            {Partitioner parameters. \muelu passes them directly to partitioner.}
\cbb{repartition: start level}           {int}       {2}             {Minimum level to run partitioner. \muelu does not repartition for finer levels.}
\cbb{repartition: min rows per proc}     {int}       {800}           {Desired minimum number of rows per processor. If actual number if smaller, then
                                                                      repartitioning occurs.}
\cbb{repartition: max imbalance}         {double}    {1.2}           {Desired maximum nonzero imbalance ratio.}
\cbb{repartition: remap parts}           {bool}      {true}          {Postprocessing for partitioning to reduce data migration.}
\cbb{repartition: keep proc 0}           {bool}      {true}          {Postprocessing for partitioning to keep processor 0 from dropping out. The goal
                                                                      is to keep processor 0 of the original fine level communication even when we use
                                                                      level subcommunicators.}
\cbb{repartition: rebalance P and R}     {bool}      {true}          {Do rebalancing of R and P during the setup. This speeds up the solve, but slows
                                                                      down the setup phases.}

\cbb{sa: damping factor}                 {double}    {4/3}           {Damping factor for smoothed aggregation.}
\cbb{sa: use filtered matrix}            {bool}      {true}          {Matrix to use for smoothing the tentative prolongator. The two options are: to
                                                                      use the original matrix, and to use the filtered matrix with filtering based on
                                                                      filtered graph used for aggregation.}
\cbb{filtered matrix: use lumping}       {bool}      {true}          {During construction of a filtered matrix, we have an option to add dropped
                                                                      entries to the diagonal. This is useful for preserving constant nullspace
                                                                      for the Laplacian type matrix.}
\cbb{filtered matrix: reuse eigenvalue}  {bool}      {true}          {During construction of a filtered matrix, we have an option to get the
                                                                      eigenvalue estimate from the original matrix. This allows us to skip heavy
                                                                      computation.}

\cbb{emin: iterative method}             {string}    {"cg"}          {Iterative method to use for energy minimization of intial prolongator in
                                                                      energy-minimization. Possible values: "cg".}
\cbb{emin: num iterations}               {int}       {2}             {Number of iterations to minimize initial prolongator energy in
                                                                      energy-minimization.}
\cbb{emin: pattern}                      {string}    {"AkPtent"}     {Sparsity pattern to use for energy minization. Possible values: "AkPtent".}
\cbb{emin: pattern order}                {int}       {1}             {Matrix order for the "AkPtent" pattern.}

\end{document}
