%
% $Id: SANDExampleReportNotstrict.tex,v 1.26 2009-05-01 20:59:19 rolf Exp $
%
% This is an example LaTeX file which uses the SANDreport class file.
% It shows how a SAND report should be formatted, what sections and
% elements it should contain, and how to use the SANDreport class.
% It uses the LaTeX report class, but not the strict option.
%
% Get the latest version of the class file and more at
%    http://www.cs.sandia.gov/~rolf/SANDreport
%
% This file and the SANDreport.cls file are based on information
% contained in "Guide to Preparing {SAND} Reports", Sand98-0730, edited
% by Tamara K. Locke, and the newer "Guide to Preparing SAND Reports and
% Other Communication Products", SAND2002-2068P.
% Please send corrections and suggestions for improvements to
% Rolf Riesen, Org. 9223, MS 1110, rolf@cs.sandia.gov
%
\documentclass[pdf,12pt,report]{SANDreport}
\usepackage{algpseudocode}
\usepackage{amsthm}
\usepackage{booktabs}
\usepackage{calc}
\usepackage{color}
\usepackage{eso-pic}
\usepackage{fancyhdr}
\usepackage{ifthen}
\usepackage{indentfirst}
\usepackage{geometry}
\usepackage{graphicx}
\usepackage[colorlinks, bookmarksopen, %pagebackref=true, backref=page,
             linkcolor={blue},
             anchorcolor={black},
             citecolor={blue},
             filecolor={magenta},
             menucolor={blue},
             pagecolor={red},
             plainpages=false,pdfpagelabels,
             pdfauthor={Jonathan J. Hu, Andrey Prokopenko, Tobias Wiesner, Chris Siefert, Ray Tuminaro},
             pdftitle={MueLu User's Guide},
             pdfkeywords={MueLu,AMG,multigrid,guide,user},
             urlcolor={blue}]{hyperref}
\usepackage{listings}
\usepackage{mathptmx}	% Use the Postscript Times font
\usepackage{multirow}
\usepackage{pifont}
\usepackage[FIGBOTCAP,normal,bf,tight]{subfigure}
\usepackage{tabularx}
\usepackage{verbatim}
\usepackage{xspace}
\usepackage{flowchart} % also loads tikz
\usepackage{algorithm}
\usetikzlibrary{arrows}

%\usepackage{draftwatermark}
%\SetWatermarkScale{.5}

\algrenewcommand{\algorithmiccomment}[1]{\hskip3em // #1}


% If you want to relax some of the SAND98-0730 requirements, use the "relax"
% option. It adds spaces and boldface in the table of contents, and does not
% force the page layout sizes.
% e.g. \documentclass[relax,12pt]{SANDreport}
%
% You can also use the "strict" option, which applies even more of the
% SAND98-0730 guidelines. It gets rid of section numbers which are often
% useful; e.g. \documentclass[strict]{SANDreport}



% ---------------------------------------------------------------------------- %
%
% Set the title, author, and date
%
\title{MueLu User's Guide 1.0}

\author{Andrey Prokopenko \\
  Scalable Algorithms \\
  Sandia National Laboratories\\
  Mailstop 1318 \\
  P.O.~Box 5800 \\
  Albuquerque, NM 87185-1318\\
  aprokop@sandia.gov\\
  \and
  Tobias Wiesner \\
  Institute for Computational Mechanics \\
  Technische Universit\"at M\"unchen\\
  Boltzmanstra\ss e 15 \\
  85747 Garching, Germany\\
  wiesner@lnm.mw.tum.de\\
  \and
  Jonathan J. Hu \\
  Scalable Algorithms \\
  Sandia National Laboratories\\
  Mailstop 9159 \\
  P.O.~Box 0969 \\
  Livermore, CA 94551-0969\\
  jhu@sandia.gov
  \and
  Christopher M. Siefert\\
  Computational Multiphysics\\
  Sandia National Laboratories\\
  Mailstop 1322 \\
  P.O.~Box 5800 \\
  Albuquerque, NM 87185-1322\\
  csiefer@sandia.gov
  \and
  Raymond S. Tuminaro\\
  Computational Mathematics\\
  Sandia National Laboratories\\
  Mailstop 9159 \\
  P.O.~Box 0969 \\
  Livermore, CA 94551-0969\\
  rstumin@sandia.gov
}

% There is a "Printed" date on the title page of a SAND report, so
% the generic \date should generally be empty.
\date{}

\newcommand{\JG}[1]{\textcolor{JG: Red}{#1}}
\newcommand{\RST}[1]{\textcolor{RayBlue}{RST: #1}}
\newcommand{\JJH}[1]{\textcolor{jhuGreen}{JJH: #1}}
\newcommand{\CMS}[1]{\textcolor{cmsPurple}{CMS: #1}}

% For displaying class names, computer code, etc.
%\newcommand{\cc}[1]{{\lstinline!#1!}}
\newcommand{\cc}[1]{{\tt #1}}

% Package names.
\newcommand{\amesos}{{\sc Amesos}\xspace}
\newcommand{\anasazi}{{\sc Anasazi}\xspace}
\newcommand{\aztecoo}{{\sc AztecOO}\xspace}
\newcommand{\belos}{{\sc Belos}\xspace}
\newcommand{\epetra}{{\sc Epetra}\xspace}
\newcommand{\ifpack}{{\sc Ifpack}\xspace}
\newcommand{\isorropia}{{\sc Isorropia}\xspace}
\newcommand{\ml}{{\sc ML}\xspace}
\newcommand{\muelu}{{\sc \textsf{{MueLu}}}\xspace}
\newcommand{\muemat}{{\sc \textsl{MueMat}}\xspace}
\newcommand{\nox}{{\sc NOX}\xspace}
\newcommand{\teuchos}{{\sc Teuchos}\xspace}
\newcommand{\tifpack}{{\sc Ifpack2}\xspace}
\newcommand{\tpetra}{{\sc Tpetra}\xspace}
\newcommand{\trilinos}{{\sc Trilinos}\xspace}
\newcommand{\zoltan}{{\sc Zoltan}\xspace}
\newcommand{\xpetra}{{\sc Xpetra}\xspace}

% Miscellaneous.
\newcommand{\be}{\begin{enumerate}}
\newcommand{\ee}{\end{enumerate}}


\newtheorem*{mycomment}{\ding{42}}
\newtheoremstyle{plain}
  {\topsep}   % ABOVESPACE
  {\topsep}   % BELOWSPACE
  {\normalfont}  % BODYFONT
  {0pt}       % INDENT (empty value is the same as 0pt)
  {\bfseries} % HEADFONT
  {}         % HEADPUNCT
  {5pt plus 1pt minus 1pt} % HEADSPACE
  {}          % CUSTOM-HEAD-SPEC

% further declarations and additional commands
\definecolor{hellgelb}{rgb}{1,1,0.8}   % background color for C++ listings
\definecolor{darkgreen}{rgb}{0.0, 0.2, 0.13}
%\definecolor{hellrot}{HTML}{FFA4C2}    % background color for xml files

% settings for listings package
\lstset{
  backgroundcolor=\color{hellgelb},
  basicstyle=\ttfamily\small,
  breakautoindent=true,
  breaklines=true,
  captionpos=b,
  columns=flexible,
  commentstyle=\color{darkgreen},
  extendedchars=true,
  float=hbp,
  frame=single,
  identifierstyle=\color{black},
  keywordstyle=\color{blue},
  numbers=none,
  numberstyle=\tiny,
  showspaces=false,
  showstringspaces=false,
  stringstyle=\color{purple},
  tabsize=2,
}


% ---------------------------------------------------------------------------- %
% Set some things we need for SAND reports. These are mandatory
%
\SANDnum{SAND2014-18874}
\SANDprintDate{October 2014}
\SANDauthor{Andrey Prokopenko, Jonathan J. Hu, Tobias A. Wiesner, Christopher M. Siefert, Raymond S. Tuminaro}


% ---------------------------------------------------------------------------- %
% Include the markings required for your SAND report. The default is "Unlimited
% Release". You may have to edit the file included here, or create your own
% (see the examples provided).
%
% \include{MarkUR} % Not needed for unlimted release reports


% ---------------------------------------------------------------------------- %
% The following definition does not have a default value and will not
% print anything, if not defined
%
%\SANDsupersed{SAND1901-0001}{January 1901}
%\input{MarkOUO}


% ---------------------------------------------------------------------------- %
%
% Start the document
%
\begin{document}
    \maketitle

    % ------------------------------------------------------------------------ %
    % An Abstract is required for SAND reports
    %
    \begin{abstract}
	The Software Utilities Package for the Engineering Sciences (SUPES) is a
collection of subprograms which perform frequently used non-numerical
services for the engineering applications programmer.  The three functional
categories of SUPES are: (1) input command parsing, (2) dynamic memory
management, and (3) system dependent utilities.  The subprograms in categories
one and two are written in standard FORTRAN-77, while the subprograms in
category three are written to provide a standardized FORTRAN interface to
several system dependent features. 

    \end{abstract}


    % ------------------------------------------------------------------------ %
    % An Acknowledgement section is optional but important, if someone made
    % contributions or helped beyond the normal part of a work assignment.
    % Use \section* since we don't want it in the table of context
    %
    \clearpage
    \chapter*{Acknowledgment}
	Many people have helped develop \muelu{} and/or provided valuable feedback, and
we would like to acknowledge their contributions here: Tom Benson, Julian
Cortial, Eric Cyr, Stefan Domino, Travis Fisher, Jeremie Gaidamour, Axel
Gerstenberger, Chetan Jhurani, Mark Hoemmen, Paul Lin, Eric Phipps, Siva
Rajamanickam, Nico Schl{\"o}mer, and Paul Tsuji.



    % ------------------------------------------------------------------------ %
    % The table of contents and list of figures and tables
    % Comment out \listoffigures and \listoftables if there are no
    % figures or tables. Make sure this starts on an odd numbered page
    %
    \cleardoublepage		% TOC needs to start on an odd page
    \tableofcontents
    \listoffigures
    \listoftables


    % ---------------------------------------------------------------------- %
    % An optional preface or Foreword
    %\clearpage
    %\chapter*{Preface}
    %\addcontentsline{toc}{chapter}{Preface}
	%    Although staff members usually have only limited experience
    with dry-erase markers, and many even dispute their existence,
    it is worthwhile to be open minded and explore the possibilities.



    % ---------------------------------------------------------------------- %
    % An optional executive summary
    %\clearpage
    %\chapter*{Summary}
    %\addcontentsline{toc}{chapter}{Summary}
	%    Once a certain level of mistrust and skepticism has been
    overcome, dry-erase markers find many uses in todays science and
    engineering. In this report we explain some of the fundamental
    properties, dangers, and benefits of dry-erase markers. We then
    conclude with a few examples on how they can be used in daily
    activities at national Laboratories.



    % ---------------------------------------------------------------------- %
    % An optional glossary. We don't want it to be numbered
    %\clearpage
    %\chapter*{Nomenclature}
    %\addcontentsline{toc}{chapter}{Nomenclature}
    %\begin{description}
	%\item[dry spell]
	%    using a dry erase marker to spell words
	%\item[dry wall]
	%    the writing on the wall
	%\item[dry humor]
	%    when people just do not understand
	%\item[DRY]
	%    Don't Repeat Yourself
    %\end{description}


    % ---------------------------------------------------------------------- %
    % This is where the body of the report begins; usually with an Introduction
    %
    \SANDmain		% Start the main part of the report

    %-----------------------------%
    \chapter{Introduction}\label{sec:introduction}
    %-----------------------------%
    %@HEADER
% ************************************************************************
% 
%          Trilinos: An Object-Oriented Solver Framework
%              Copyright (2001) Sandia Corporation
% 
% Under terms of Contract DE-AC04-94AL85000, there is a non-exclusive
% license for use of this work by or on behalf of the U.S. Government.
% 
% This program is free software; you can redistribute it and/or modify
% it under the terms of the GNU General Public License as published by
% the Free Software Foundation; either version 2, or (at your option)
% any later version.
%   
% This program is distributed in the hope that it will be useful, but
% WITHOUT ANY WARRANTY; without even the implied warranty of
% MERCHANTABILITY or FITNESS FOR A PARTICULAR PURPOSE.  See the GNU
% General Public License for more details.
%   
% You should have received a copy of the GNU General Public License
% along with this program; if not, write to the Free Software
% Foundation, Inc., 675 Mass Ave, Cambridge, MA 02139, USA.
% 
% Questions? Contact Michael A. Heroux (maherou@sandia.gov)
% 
% ************************************************************************
%@HEADER

\section{Introduction}

The Trilinos Project is an effort to facilitate the design, development,
integration and ongoing support of mathematical software libraries.
Goal of the Trilinos Project is develop parallel solver algorithms and
libraries within an object-oriented software framework for the solution
of large-scale, complex multiphysics engineering and scientific
applications. The emphasis is on developing robust, scalable algorithm
in a software framework, using abstract interfaces for flexible
interoperability of components while providing a full-featured set of
concrete classes that implement all abstract interfaces.

%%%
%%%
%%%

\subsection{Getting Started with Trilinos}
\label{sec:getting}

The Trilinos Project uses a two-level software structure designed around
collections of packages. A Trilinos package is an integral unit, usually
developed to solve a specific task, by a (relatively) small group of
expert of the field.  Packages exist underneath the Trilinos top level,
which provides a common look-and-feel. Each package has its own
structure, documentation and set of examples. In principle, Trilinos
packages can live independently. However, each package is even more
valuable when combined with other Trilinos packages.

\smallskip

Trilinos is a large software project, and currently about twenty
packages are included. Fully understanding all the functionalities of
the Trilinos packages requires time. The entire set of packages covers a
wide range of numerical methods for large scale computing. Some packages
are focused on the development of computational schemes, like for
instance the solution of linear and nonlinear systems, to the definition
of parallel preconditioners for Krylov methods, eigenvalue computation.
Other packages are more focused on implementation issues (like
definition of matrices and vectors, abstract classes for linear
operators). The first Chapters of this tutorial will be focused on
implementation issues, while the last Chapters will have a more
``mathematical'' taste.

Each package offers sophisticated features, that cannot be ``unleashed''
at a very first usage. For each package, we will outline only the basic
features, and we refer to the documentation of each package for a more
involved usage. Our goal is to present enough material so that the
reader can successfully use the described packages.  In fact, for new
users, it is neither easy, nor necessary, to manage all the Trilinos
functionalities. At the beginning, it is more important for them to
understand how to manage the basic classes, such as vector, matrix and
linear system classes. However, it is clear that for a fine-tuning, the
reader will have to look through each package's documentation and
examples.

\medskip

Although all packages have the same importance in the Trilinos
structure, a typical user will probably --- at least at the beginning
--- make use of the following packages:
\begin{itemize} 
\item {\bf Epetra}. This package defines the basic classes for
  distributed matrices and vectors, linear operators and linear
  problems. Epetra classes are the common language spoken by all the
  Trilinos packages (even if some of them can ``speak'' other
  languages). Each Trilinos package is able to accept in input Epetra
  objects. This allows powerful combinations among the various Trilinos
  functionalities.
\item {\bf AztecOO}. This is a linear solve package based on
  preconditioned Krylov methods. It supports all the Aztec interfaces
  and functionality, but also provides significant new functionality.
\item {\bf IFPACK}. This is a package to perform various incomplete
  factorizations, and it is here used in conjunction with AztecOO.
\item {\bf ML}. This is an algebraic multilevel preconditioner package, which
  provided scalable preconditioning capabilities for a variety of
  problem classes. It is here used in conjunction with AztecOO.
\item {\bf Amesos}. This package provides a common interface to various
  direct solvers (generally available outside the Trilinos framework),
  both sequential and parallel.
\item {\bf NOX}. This is a collection of nonlinear solvers, designed to
  be easily integrated into an application and used with many different
  linear solvers.
\item {\bf Triutils}. This is a collection of various utilities, that
  can be extremely useful in some phases of software development.
\end{itemize}

Table~\ref{tab:tripackages} gives a partial overview of what can be
accomplished using Trilinos.
\begin{table}[htbp]
  \centering
  \begin{tabular}{| p{10cm} | p{3cm} |}
    \hline
    {\bf Task} & {\bf Package} \\
    \hline
    Light-weight interface to BLAS and LAPACK: & Epetra, Teuchos$^\star$ \\\hline
    Definition of serial dense or sparse matrices: & Epetra \\\hline
    Definition of distributed sparse matrices:& Epetra \\\hline
    solve a linear system with preconditioned Krylov accelerators, like
    CG, GMRES, Bi-CGSTAB, TFQMR:& AztecOO, Belos$^\star$ \\\hline
    Definition of incomplete factorizations:& AztecOO, \newline IFPACK \\\hline
    Definition of a multilevel preconditioner:& ML \\\hline
    Definition of a one-level Schwarz preconditioner (overlapping domain
    decomposition):& AztecOO, \newline IFPACK \\\hline
    Definition a two-level Schwarz preconditioner, with coarse grid based on
    aggregation:& AztecOO+ML \\\hline
    Solution of  systems of nonlinear equations:& NOX \\\hline
    interface with various direct solvers, as UMFPACK, MUMPS, SuperLU
    and others :& Amesos \\\hline
    Computation of eigenvalue of large, sparse matrices:& Anasazi$^\star$
    \\\hline
    Solution of complex linear equations (using equivalent real formulation):&
    Komplex$^\star$ \\\hline
    Definition of segregated preconditioners and block preconditioners (for
    instance, for the incompressible Navier-Stokes equations):&
    Meros$^\star$ \\\hline
    Templated interface to BLAS and LAPACK, arbitrary-precision
    arithmetic, parameter lists:& Teuchos$^\star$ \\\hline
    Definition of abstract interfaces to vectors, linear operators, and solvers:& TSF$^\star$, TSFCore$^\star$, TSFExtended$^\star$    \\
    \hline
  \end{tabular}
  \caption{Partial overview of what can be done with Trilinos. $\star$:
    not covered in this tutorial.}
  \label{tab:tripackages}
\end{table}

This tutorial is divided into 10 chapters:
\begin{itemize}
\item Chapter \ref{chap:epetra_vec} describes the Epetra\_Vector class;
\item Chapter \ref{chap:epetra_mat} introduces the Epetra\_Matrix
  class; 
\item Chapter \ref{chap:epetra_others} briefly describes some other
  Epetra classes;
\item Chapter \ref{chap:aztecoo} shows how to solve linear systems with
  AztecOO;
\item Chapter \ref{chap:ifpack} presents the basic usage of IFPACK;
\item Chapter \ref{chap:ml} introduces multilevel preconditioners based
  on ML;
\item Chapter \ref{chap:amesos} introduces the Amesos package;
\item Chapter \ref{chap:nox} outlines the main features of the Trilinos
  nonlinear solver package, NOX.
\item Chapter \ref{chap:triutils} presents some tools provided with the
  Triutils package. 
\end{itemize}

\begin{remark}
  As already pointed out, Epetra objects are meant to be the ``common
  language'' spoken by all the Trilinos packages, and therefore the new
  user must become familiar with those objects. Therefore we suggest to
  read Chapters \ref{chap:epetra_vec}-\ref{chap:epetra_others} before
  considering other Trilinos packages. Also, Chapter~\ref{chap:aztecoo}
  should be read before Chapters~\ref{chap:ifpack} and~\ref{chap:ml}
  (even if both IFPACK and ML can be compiled and run without AztecOO).
\end{remark}

This tutorial assume a basic background in numerical methods for PDEs,
and in iterative linear and nonlinear solvers. Although not strictly
necessary, the reader is suppose to have a certain familiarity with
distributed memory computing and, to a minor extent, with MPI.

\smallskip

Note that this tutorial is not a substitute ofr individual packages
documentation. Also, for an overview of all the Trilinos packages, the
Trilinos philosophy, and a description of the packages provided by
Trilinos, the reader is referred to \cite{Trilinos-Overview}.
Developers should also consider the Trilinos Developers' Guide, which
addresses many topics, including the development tools used by Trilinos'
developers, and how to include a new package\footnote{ Trilinos provides
  a variety of services to a developer wanting to integrate a package
  into Trilinos.  They include Autoconf~\cite{Autoconf},
  Automake~\cite{Automake} and Libtool~\cite{Libtool}. Those tools
  provide a robust, full-featured set of tools for building software
  across a broad set of platforms.  Although these tools are not
  official standards, they are widely used.  All existing Trilinos
  packages use Autoconf and Automake.  Libtool support will be added in
  future releases.}.

%%%
%%%
%%%

\subsection{Installing Trilinos}
\label{sec:installing}

To obtain Trilinos, please refers to the instructions reported at the
following web site:
\begin{verbatim}
http://software.sandia.gov/Trilinos
\end{verbatim}

Trilinos has been compiled on a variety of architectures, including
Linux, Sun Solaris, SGI Irix, DEC, and many others. Trilinos has been
designed to support parallel applications. However, it can be compiled
and run on serial computer.  Detailed comments on the installation, and
an exhaustive list of FAQs, can be found at the web pages:
\begin{verbatim}
http://software.sandia.gov/Trilinos/installing_manual.html
http://software.sandia.gov/Trilinos/faq.html
\end{verbatim}


Before using Trilinos, users might decide to set the environmental
variables \verb!TRILINOS_HOME!, indicating the full path of the Trilinos
directory, \verb!TRILINOS_LIB!, indicating the location of the compiled
Trilinos library, and \verb!TRILINOS_ARCH!, containing the architecture
and the communicator currently used.  For example, using the BASH shell,
command lines of the form
\begin{verbatim}
export TRILINOS_HOME=/home/msala/Trilinos
export TRILINOS_ARCH=LINUX.MPI
export TRILINOS_LIB=${TRILINOS_HOME}/${TRILINOS_ARCH}
\end{verbatim}
can be places in the users' \verb!.bashrc! file.

\smallskip

Here, we briefly report the procedure one should follow in order to
compile Trilinos as required by the examples reported in the following
chapters \ref{chap:epetra_vec}-\ref{chap:triutils}\footnote{Amesos can
  be more difficult to compile for the unexperienced user, as it
  required some information about the packages to interface. Suggestions
  about the configuration of Amesos are reported in
  Chapter~\ref{chap:amesos}. More details about the installation of
  Trilinos can be found in \cite{Trilinos-Users-Guide}.}.  Suppose we
want to compile under LINUX with MPI. The installation procedure can be
are reported below. (\verb!$! indicates the shell prompt.)
\begin{verbatim}
$ cd ${TRILINOS_HOME}
$ mkdir ${TRILINOS_ARCH}
$ cd ${TRILINOS_ARCH}
$ ../configure --prefix="${TRILINOS_HOME}/${TRILINOS_ARCH}" \
  --enable-mpi --with-mpi-compilers \
  --enable-triutils --enable-aztecoo \
  --enable-ifpack \
  --enable-ml --enable-nox | tee configure_${TRILINOS_ARCH}.log
$ make | tee make_${TRILINOS_ARCH}.log
$ make install | tee make_install_${TRILINOS_ARCH}.log
\end{verbatim}

\begin{remark}
  All Trilinos packages can be build to run with or without MPI. If MPI
  is enabled (using \verb!--enable-mpi!), the users must know the
  procedure for beginning MPI jobs on their computer system(s). In some
  cases, options must be set on the configure line to specify the
  location of MPI include files and libraries.
\end{remark}

%%%
%%%
%%%

\subsection{Compiling and Linking a program using Trilinos}
\label{sec:intro_compiling}

In order to compile and link (part of) the Trilinos library, the use can
decide to use a Makefile as reported below. This Makefile refers to one
of the examples, reported in the NOX subdirectory of this tutorial.
\begin{verbatim}
 1: TRILINOS_HOME = /home/msala/Trilinos/
 2: TRILINOS_ARCH - LINUX_MPI
 3: TRILINOS_LIB = $(TRILINOS_HOME)$(TRILINOS_ARCH)
 4: 
 5: include $(TRILINOS_HOME)/build/makefile.$(TRILINOS_ARCH)
 6: 
 7: MY_COMPILER_FLAGS = -DHAVE_CONFIG_H $(CXXFLAGS) -c -g\
 8:                    -I$(TRILINOS_LIB)/include/
 9:
10: MY_LINKER_FLAGS = $(LDFLAGS) $(TEST_C_OBJ) \
11:         -L$(TRILINOS_LIB)/lib/ \
12:         -lnoxepetra -lnox -lifpack \
13:         -laztecoo -lepetra -llapack -lblas $(ARCH_LIBS)
14:
15: ex1: ex1.cpp
16:         $(CXX)     ex1.cpp $(MY_COMPILER_FLAGS)
17:         $(LINKER)  ex1.o   $(MY_LINKER_FLAGS)    -o ex1.exe
\end{verbatim}

Line number have been reported for  reader's convenience. 

The lines 1-3 can be omitted, see Section \ref{sec:installing}.  Line 5
includes basic definitions of Trilinos. (Note that, on some
architectures, one may need to use \verb!gmake! instead of \verb!make!.)
In line 7, the variable \verb!HAVE_CONFIG_H! is defined. Linker flags of
lines 10-13 defines the library to link (location of BLAS and LAPACK can
change on different platforms). The variable \verb!ARCH_LIBS! is defined
in line 5.

To run the compiled example in a sequential environment, simply type
\begin{verbatim}
$ ./ex1.exe
\end{verbatim}
In a MPI environment, the user might have to
use an instruction of type
\begin{verbatim}
$ mpirun -np 2 ./ex1.exe
\end{verbatim}
Please check the local MPI documentation for more details. 

%%%
%%%
%%%

\subsection{Copyright and Licensing of Trilinos}
\label{sec:copyright}

Trilinos is released under the Lesser GPL GNU Licence.

Trilinos is copyrighted by Sandia Corporation. Under the terms of
Contract DE-AC04-94AL85000, there is a non-exclusive license for use of
this work by or on behalf of the U.S. Government.  Export of this
program may require a license from the United States Government.

NOTICE: The United States Government is granted for itself and others
acting on its behalf a paid-up, nonexclusive, irrevocable worldwide
license in ths data to reproduce, prepare derivative works, and perform
publicly and display publicly.  Beginning five (5) years from July 25,
2001, the United States Government is granted for itself and others
acting on its behalf a paid-up, nonexclusive, irrevocable worldwide
license in this data to reproduce, prepare derivative works, distribute
copies to the public, perform publicly and display publicly, and to
permit others to do so.

NEITHER THE UNITED STATES GOVERNMENT, NOR THE UNITED STATES DEPARTMENT
OF ENERGY, NOR SANDIA CORPORATION, NOR ANY OF THEIR EMPLOYEES, MAKES ANY
WARRANTY, EXPRESS OR IMPLIED, OR ASSUMES ANY LEGAL LIABILITY OR
RESPONSIBILITY FOR THE ACCURACY, COMPLETENESS, OR USEFULNESS OF ANY
INFORMATION, APPARATUS, PRODUCT, OR PROCESS DISCLOSED, OR REPRESENTS
THAT ITS USE WOULD NOT INFRINGE PRIVATELY OWNED RIGHTS.

\medskip

Some parts of Trilinos are dependent on a third party code. Each third
party code comes with its own copyright and/or licensing requirements.
It is responsibility of the user to understand these requirements.

%%%
%%%
%%%

\subsection{Programming Language Used in this Tutorial}
\label{sec:language}

Trilinos is written in C++ (for most packages), and in C. Some
interfaces are provided to FORTRAN code (mainly BLAS and LAPACK
routines). Even if a limited support is included for C programs (and a
more limited for FORTRAN code), to unleashed the full power of Trilinos
we suggest to use C++. All the example programs contained in this
tutorial will be in C++; some packages contains examples in C.

%%%
%%%
%%%

\subsection{Referencing Trilinos}
\label{sec:referencing}

The Trilinos project can be referenced by using the following BiBTeX
citation information:
\begin{verbatim}
@techreport{Trilinos-Overview,
title = "{An Overview of Trilinos}",
author = "Michael Heroux and Roscoe Bartlett and Vicki Howle
Robert Hoekstra and Jonathan Hu and Tamara Kolda and
Richard Lehoucq and Kevin Long and Roger Pawlowski and
Eric Phipps and Andrew Salinger and Heidi Thornquist and
Ray Tuminaro and James Willenbring and Alan Williams ",
institution = "Sandia National Laboratories",
number = "SAND2003-2927",
year = 2003}

@techreport{Trilinos-Dev-Guide,
title = "{Trilinos Developers Guide}",
author = "Michael A. Heroux and James M. Willenbring and Robert Heaphy",
institution = "Sandia National Laboratories",
number = "SAND2003-1898",
year = 2003}

@techreport{Trilinos-Dev-Guide-II,
title = "{Trilinos Developers Guide Part II: ASCI Software Quality
Engineering Practices Version 1.0}",
author = "Michael A. Heroux and James M. Willenbring and Robert Heaphy",
institution = "Sandia National Laboratories",
number = "SAND2003-1899",
year = 2003}

@techreport{Trilinos-Users-Guide,
title = "{Trilinos Users Guide}",
author = "Michael A. Heroux and James M. Willenbring",
institution = "Sandia National Laboratories",
number = "SAND2003-2952",
year = 2003}
\end{verbatim}
These BiBTeX information can be downloaded from the web page

\begin{verb}
http://software.sandia.gov/Trilinos/citing.html
\end{verb}

%%%
%%%
%%%

\subsection{A Note on Directory Structure}
\label{sec:into_note}

Each Trilinos package in contained in the subdirectory
\begin{verbatim}
${TRILINOS_HOME}/packages
\end{verbatim}
The structure of all packages is quite similar (although not exactly
equal). As a general line, source files are in
\begin{verbatim}
${TRILINOS_HOME}/packages/<package-name>/src
\end{verbatim}
Example files are reported in \begin{verbatim}
${TRILINOS_HOME}/packages/<package-name>/examples
\end{verbatim}
and test files in
\begin{verbatim}
${TRILINOS_HOME}/packages/<package-name>/test
\end{verbatim}
The documentation is reported
\begin{verbatim}
${TRILINOS_HOME}/packages/<package-name>/doc
\end{verbatim}
Often, Trilinos developers use Doxygen\footnote{Copyright \copyright
  1997-2003 by Dimitri van Heesch. More information can by found at the
  web address {\tt http://www.stack.nl/~dimitri/doxygen/}.}. For
instance, to create the documentation for Epetra, we use can type
\begin{verbatim}
$ cd ${TRILINOS_HOME}/packages/epetra/doc
$ doxygen Doxyfile
\end{verbatim}
and then browse it using an HTML reader, or compiling the \LaTeX file
using
\begin{verbatim}
$ cd ${TRILINOS_HOME}/packages/epetra/doc/latex
$ make
\end{verbatim}

%%%
%%%
%%%

\subsection{List of Trilinos Developers}
\label{sec:intro_incomplete}

A list of the Trilinos' developers, updated to December 2003, would
include the following names (in alphabetical order):

Roscoe A. Bartlett,
Jason A. Cross,
David M. Day,
Robert Heaphy,
Michael A. Heroux (project leader),
Russell Hooper,
Vicki E. Howle,
Robert J. Hoekstra,
Jonathan J. Hu,
Tamara G. Kolda,
Richard B. Lehoucq,
Paul Lin,
Kevin R. Long,
Roger P. Pawlowski,
Michael N. Phenow,
Eric T. Phipps,
Andrew J. Rothfuss,
Marzio Sala,
Andrew G. Salinger,
Paul M. Sexton,
Kendall S. Stanley,
Heidi K. Thornquist,
Ray S. Tuminaro,
James M. Willenbring,
Alan Williams.



    %-----------------------------%
    \chapter{Multigrid background}\label{sec:multigrid}
    %-----------------------------%
    \label{sec:multigrid intro}
Here we provide a brief multigrid introduction (see~\cite{MGTutorial}
or~\cite{OwlBook} for more information). A multigrid solver tries to approximate
the original problem of interest with a sequence of smaller (\textit{coarser})
problems. The solutions from the coarser problems are combined in order to
accelerate convergence of the original (\textit{fine}) problem on the finest
grid. A simple multilevel iteration is illustrated in
Algorithm~\ref{multigrid_code}.

\begin{algorithm}
\centering
\begin{algorithmic}[0]
  \State{$A_0 = A$}
  \Function{Multilevel}{$A_k$, $b$, $u$, $k$}
    \State{// Solve $A_k$ u = b (k is current grid level)}
    \State $ u = S^{1}_m (A_k, b, u)$
      \If{$(k \ne {\bf N-1})$}
        \State{$P_k = $ determine\_interpolant( $A_k$ )}
        \State{$R_k = $ determine\_restrictor( $A_k$ )}
        \State{$\widehat{r}_{k+1} = R_k (b - A_k u )$}
        \State{$A_{k+1} = R_k A_k P_k$}
        \State{$v = 0$}
        \State{}\Call{Multilevel}{$\widehat{A}_{k+1}$, $\widehat{r}_{k+1}$, $v$, $k+1$}
        \State{$ u = u + P_{k} v$}
        \State{$ u = S^{2}_m (A_k, b, u )$}
      \EndIf
  \EndFunction
\end{algorithmic}
\caption{V-cycle multigrid with $N$ levels to solve $Ax=b$.}
\label{multigrid_code}
\end{algorithm}

In the multigrid iteration in Algorithm~\ref{multigrid_code}, the $S^{1}_m()$'s
and $S^{2}_m()$'s are called \textit{pre-smoothers} and \textit{post-smoothers}.
They are approximate solvers (e.g., symmetric Gauss-Seidel), with the subscript
$m$ denoting the number of applications of the approximate solution method. The
purpose of a smoother is to quickly reduce certain error modes in the
approximate solution on a level $k$. For symmetric problems, the pre-
and post-smoothers should be chosen to maintain symmetry (e.g., forward
Gauss-Seidel for the pre-smoother and backward Gauss-Seidel for the
post-smoother). The $P_k$'s are \textit{interpolation} matrices that transfer
solutions from coarse levels to finer levels. The $R_k$'s are
\textit{restriction} matrices that restrict a fine level solution to a coarser
level. In a geometric multigrid, $P_k$'s and $R_k$'s are determined
by the application, whereas in an algebraic multigrid they are automatically
generated. For symmetric problems, typically $R_k=P_k^T$. For nonsymmetric
problems, this is not necessarily true. The $A_k$'s are the coarse level
problems, and are generated through a Galerkin (triple matrix) product.

Please note that the algebraic multigrid algorithms implemented in \muelu{}
generate the grid transfers $P_k$ automatically and the coarse problems $A_k$
via a sparse triple matrix product. \trilinos{} provides a wide selection of
smoothers and direct solvers for use in \muelu through the \ifpack,
\ifpacktwo, \amesos, and \amesostwo packages (see \S\ref{sec:options}).



    %-----------------------------%
    \chapter{Getting Started}\label{sec:getting started}
    %-----------------------------%
    This section is meant to get you using \ifpacktwo{} as quickly as possible.
\S\ref{sec:overview} gives a brief overview of \ifpacktwo{}.
\S\ref{sec:configuration_and_build} lists \ifpacktwo{}'s dependencies on other
\trilinos{} libraries and provides a sample cmake configuration line. Finally,
some examples of code are given in~\S\ref{sec:examples in code}.

\section{Overview of \ifpacktwo{}}
\label{sec:overview}
\ifpacktwo{} is a C++ linear solver library in the \trilinos{} project~\cite{Heroux2012}.
It originally began as a migration of \ifpack{} package capabilities to a new linear
algebra stack. While it retains some commonalities with the original package, it
has since diverged significantly from it and should be treated as completely
independent package.

\ifpacktwo{} only works with \tpetra{}~\cite{TpetraURL} matrix,
vector, and map types. Like \tpetra{}, it allows for different ordinal
(index) and scalar types. \ifpacktwo{} was designed to be efficient on a wide
range of computer architectures, from workstations to supercomputers~\cite{Lin2014}.
It relies on the ``MPI+X" principle, where ``X'' can be threading or
CUDA\@. The ``X'' portion, node-level parallelism, is controlled by a node
template type. Users should refer to \tpetra{}'s documentation for information
about node and device types.

\ifpacktwo provides a number of different solvers, including
\begin{itemize}
  \item Jacobi, Gauss-Seidel, polynomial, distributed relaxation;
  \item domain decomposition solvers;
  \item incomplete factorizations.
\end{itemize}
This list of solvers is not exhaustive. Instead, references for further
information are provided throughout the text. There are many excellent
references for iterative methods, including~\cite{Saad2003}.

Complete information on available capabilities and options can be found
in~\S\ref{sec:options}.

\section{Configuration and Build}\label{sec:configuration_and_build}

\ifpacktwo{} requires a C++11 compatible compiler for compilation. The
minimum required version of compilers are GCC (4.7.2 and later),
Intel (13 and later), and clang (3.5 and later).

\subsection{Dependencies}

Table~\ref{tab:dependencies} enumerates the dependencies of \ifpacktwo. Certain
dependencies are optional, whereas others are required.  Furthermore,
\ifpacktwo's tests depend on certain libraries that are not required if you only
want to link against the \ifpacktwo library and do not want to compile its
tests. Additionally, some functionality in \ifpacktwo{} may depend on other
Trilinos packages (for instance, \amesostwo{}) that may require additional
dependencies. We refer to the documentation of those packages for a full list of
dependencies.

\begin{table}[ht]
  \centering
  \begin{tabular}{p{3.5cm} c c c c}
    \toprule
    \multirow{2}{*}{Dependency} & \multicolumn{2}{c}{Library} & \multicolumn{2}{c}{Testing} \\
    \cmidrule(r){2-3} \cmidrule(l){4-5} & Required & Optional & Required & Optional  \\
    \midrule
    % \belos                       & $\times$ &          & $\times$ & \\
    \teuchos                     & $\times$ &          & $\times$ & \\
    \tpetra                      & $\times$ &          & $\times$ & \\
    \tpetrakernels               & $\times$ &          &          & \\
    \amesostwo                   &          & $\times$ &          & $\times$  \\
    \galeri                      &          &          &          & $\times$  \\
    \xpetra                      &          & $\times$ &          & $\times$  \\
    \zoltantwo                   &          & $\times$ &          & $\times$  \\
    \textsc{ThyraTpetraAdapters} &          & $\times$ &          & \\
    \textsc{ShyLUBasker}         &          & $\times$ &          & $\times$ \\
    \textsc{ShyLUHTS}            &          & $\times$ &          & $\times$ \\
    \midrule
    % BLAS                         & $\times$ &          & $\times$ & \\
    % LAPACK                       & $\times$ &          & $\times$ & \\
    MPI                          &          & $\times$ &          & $\times$  \\
    % Cholmod                      &          & $\times$ &          & $\times$  \\
    % SuperLU 4.3                  &          & $\times$ &          & $\times$  \\
    % QD                           &          & $\times$ &          & $\times$  \\
    \bottomrule
  \end{tabular}
  \caption{\label{tab:dependencies}\ifpacktwo{}'s required and optional dependencies,
    subdivided by whether a dependency is that of the \ifpacktwo{}{} library itself
    (\textit{Library}), or of some \ifpacktwo{}{} test (\textit{Testing}). }
\end{table}

\amesostwo and \superlu are necessary if you want to use either a sparse direct
solve or ILUTP as a subdomain solve in processor-based domain decomposition.
\zoltantwo and \xpetra are necessary if you want to reorder a matrix (e.g.,
reverse Cuthill McKee).

\subsection{Configuration}
The preferred way to configure and build \ifpacktwo{} is to do that outside of the source directory.
Here we provide a sample configure script that will enable \ifpacktwo{} and all of its optional dependencies:
\begin{lstlisting}
  export TRILINOS_HOME=/path/to/your/Trilinos/source/directory
  cmake -D BUILD_SHARED_LIBS:BOOL=ON \
        -D CMAKE_BUILD_TYPE:STRING="RELEASE" \
        -D CMAKE_CXX_FLAGS:STRING="-g" \
        -D Trilinos_ENABLE_EXPLICIT_INSTANTIATION:BOOL=ON \
        -D Trilinos_ENABLE_TESTS:BOOL=OFF \
        -D Trilinos_ENABLE_EXAMPLES:BOOL=OFF \
        -D Trilinos_ENABLE_Ifpack2:BOOL=ON \
        -D Ifpack2_ENABLE_TESTS:STRING=ON \
        -D Ifpack2_ENABLE_EXAMPLES:STRING=ON \
        -D TPL_ENABLE_BLAS:BOOL=ON \
        -D TPL_ENABLE_MPI:BOOL=ON \
        ${TRILINOS_HOME}
\end{lstlisting}

\noindent
More configure examples can be found in \texttt{Trilinos/sampleScripts}.
For more information on configuring, see the \trilinos Cmake Quickstart guide \cite{TrilinosCmakeQuickStart}.

\section{Interface to \ifpacktwo{} methods}
All \ifpacktwo operators inherit from the base class
\texttt{Ifpack2::Preconditioner}. This in turn inherits from
\texttt{Tpetra::Operator}. Thus, you may use an \ifpacktwo operator anywhere
that a \texttt{Tpetra::Operator} works. For example, you may use \ifpacktwo operators
directly as preconditioners in \trilinos' \belos package of iterative solvers.

You may either create an \ifpacktwo operator directly, by using the class and
options that you want, or by using \texttt{Ifpack2::Factory}. Some of
\ifpacktwo preconditioners only accept a \texttt{Tpetra::\\CrsMatrix} instance as
input, while others also may accept a \texttt{Tpetra::RowMatrix} (the base class
of \texttt{Tpetra::CrsMatrix}). They will decide at run time whether the input
\texttt{Tpetra::RowMatrix} is an instance of the right subclass.

\texttt{Ifpack2::Preconditioner} includes the following methods:
\begin{itemize}
  \item \texttt{initialize()}

    Performs all operations based on the graph of the matrix (without
    considering the numerical values).

  \item \texttt{compute()}

    Computes everything required to apply the preconditioner, using the matrix's
    values.

  \item \texttt{apply()}

    Applies or ``solves with'' the preconditioner.
\end{itemize}
Every time that \texttt{initialize()} is called, the object destroys all the
previously allocated information, and reinitializes the preconditioner. Every
time \texttt{compute()} is called, the object recomputes the actual values of the
preconditioner.

An \ifpacktwo preconditioner may also inherit from
\texttt{Ifpack2::CanChangeMatrix} class in order to express that users can
change its matrix (the matrix that it preconditions) after construction using a
\texttt{setMatrix} method.  Changing the matrix puts the preconditioner back in
an ``pre-initialized'' state.  You must first call \texttt{initialize()}, then
\texttt{compute()}, before you may call \texttt{apply()} on this preconditioner.
Depending on the implementation, it may be legal to set the matrix to null. In
that case, you may not call \texttt{initialize()} or \texttt{compute()} until
you have subsequently set a nonnull matrix.

\textbf{Warning.} If you are familiar with the \ifpack package, please be aware
that the behaviour of the \ifpacktwo preconditioner is different from \ifpack.
In \ifpack, the \texttt{ApplyInverse()} method applies or ``solves with'' the
preconditioner $M^{-1}$, and the \texttt{Apply()} method ``applies'' the
preconditioner $M$. In \ifpacktwo, the \texttt{apply()} method applies or
``solves with'' the preconditioner $M^{-1}$. \ifpacktwo has no method comparable
to \ifpack's \texttt{Apply()}.

\section{Example: \ifpacktwo preconditioner within \belos}\label{sec:examples in code}

The most commonly used scenario involving \ifpacktwo{} is using one of its
preconditioners preconditioners inside an iterative linear solver. In
\trilinos{}, the \belos{} package provides important Krylov subspace methods (such
as preconditioned CG and GMRES).

At this point, we assume that the reader is comfortable with \teuchos{} referenced-counted
pointers (RCPs) for memory management (an introduction to RCPs can be found
in~\cite{RCP2010}) and the \parameterlist class~\cite{TeuchosURL}.

First, we create an \ifpacktwo{} preconditioner using a provided \parameterlist
\begin{lstlisting}[language=C++]
 typedef Tpetra::CrsMatrix<Scalar, LocalOrdinal, GlobalOrdinal, Node>
   Tpetra_Operator;

 Teuchos::RCP<Tpetra_Operator> A;
 // create A here ...
 Teuchos::ParameterList paramList;
 paramList.set( "chebyshev: degree", 1 );
 paramList.set( "chebyshev: min eigenvalue", 0.5 );
 paramList.set( "chebyshev: max eigenvalue", 2.0 );
 // ...
 Ifpack2::Factory factory;
 RCP<Ifpack2::Ifpack2Preconditioner<> > ifpack2Preconditioner;
 ifpack2Preconditioner = factory.create( "CHEBYSHEV", A )
 ifpack2Preconditioner->setParameters( paramList );
 ifpack2Preconditioner->initialize();
 ifpack2Preconditioner->compute();
\end{lstlisting}

Besides the linear operator $A$, we also need an initial guess vector for the
solution $X$ and a right hand side vector $B$ for solving a linear system.
\begin{lstlisting}[language=C++]
 typedef Tpetra::Map<LocalOrdinal, GlobalOrdinal, Node> Tpetra_Map;
 typedef Tpetra::MultiVector<Scalar, LocalOrdinal, GlobalOrdinal, Node>
   Tpetra_MultiVector;

 Teuchos::RCP<const Tpetra_Map> map = A->getDomainMap();

 // create initial vector
 Teuchos::RCP<Tpetra_MultiVector> X =
   Teuchos::rcp( new Tpetra_MultiVector(map, numrhs) );

 // create right-hand side
 X->randomize();
 Teuchos::RCP<Tpetra_MultiVector> B =
   Teuchos::rcp( new Tpetra_MultiVector(map, numrhs) );
 A->apply( *X, *B );
 X->putScalar( 0.0 );
\end{lstlisting}
To generate a dummy example, the above code first declares two vectors. Then, a
right hand side vector is calculated as the matrix-vector product of a random vector
with the operator $A$. Finally, an initial guess is initialized with zeros.

Then, one can define a \texttt{Belos::LinearProblem} object where the
\texttt{ifpack2Preconditioner} is used for left preconditioning.
\begin{lstlisting}[language=C++]
 typedef Belos::LinearProblem<Scalar, Tpetra_MultiVector, Tpetra_Operator>
   Belos_LinearProblem;

 Teuchos::RCP<Belos_LinearProblem> problem =
   Teuchos::rcp( new Belos_LinearProblem( A, X, B ) );
 problem->setLeftPrec( mueLuPreconditioner );
 bool set = problem.setProblem();
\end{lstlisting}

Next, we set up a \belos{} solver using some basic parameters.
\begin{lstlisting}[language=C++]
 Teuchos::RCP<Teuchos::ParameterList> belosList =
   Teuchos::rcp(new Teuchos::ParameterList);
 belosList->set( "Block Size", 1 );
 belosList->set( "Maximum Iterations", 100 );
 belosList->set( "Convergence Tolerance", 1e-10 );
 belosList->set( "Output Frequency", 1 );
 belosList->set( "Verbosity", Belos::TimingDetails + Belos::FinalSummary );

 Belos::SolverFactory<Scalar, Tpetra_MultiVector, Tpetra_Operator> solverFactory;
 Teuchos::RCP<Belos::SolverManager<Scalar, Tpetra_MultiVector, Tpetra_Operator> >
   solver = solverFactory.create( "Block CG", belosList );
 solver->setProblem( problem );
\end{lstlisting}

Finally, we solve the system.
\begin{lstlisting}[language=C++]
 Belos::ReturnType ret = solver.solve();
\end{lstlisting}

It is often more convenient to specify the parameters as part of an XML-formatted options file.
Look in the subdirectory {\tt Trilinos/packages/ifpack2/test/belos} for examples of this.

This section is only meant to give a brief introduction on how to use
\ifpacktwo{} as a preconditioner within the \trilinos{} packages for iterative
solvers. There are other, more complicated, ways to use to work with
\ifpacktwo{}. For more information on these topics, the reader may refer to the
examples and tests in the \ifpacktwo{} source directory
(\texttt{Trilinos/packages/ifpack2}).


    %-----------------------------%
    \chapter{Performance tips}\label{sec:performance}
    %-----------------------------%
    In practice, it can be very challenging to find an appropriate set of multigrid
parameters for a specific problem, especially if few details are known about the
underlying linear system. In this Chapter, we provide some advice for improving
multigrid performance.

\begin{mycomment}
For optimizing multigrid parameters, it is highly recommended to set the
verbosity to \verb|high| or \verb|extreme| for \muelu{} to output more
information (e.g., for the effect of the chosen parameters to the aggregation
and coarsening process).
\end{mycomment}

Some general advice:
\begin{itemize}
  \item
    Choose appropriate iterative linear solver (e.g., GMRES for non-symmetric problems).
    If available, set options to perform as few all-reduces as possible.
    (E.g. \texttt{Use Single Reduction} in \belos.)

  \item
    Start with the recommended settings for particular problem types. See
    Table~\ref{t:problem_types}.

  \item
    Choose reasonable basic multigrid parameters
    (see~\S\ref{sec:options_general}), including maximum number of multigrid
    levels (\texttt{max levels}) and maximum allowed coarse size of the problem
    (\texttt{coarse: max size}). Take fine level problem size and sparsity
    pattern into account for a reasonable choice of these parameters.

  \item
    Select an appropriate transfer operator strategy
    (see~\S\ref{sec:options_mg}). For symmetric problems, you should start with smoothed
    aggregation multigrid. For non-symmetric problems, a Petrov-Galerkin smoothed
    aggregation method is a good starting point, though smoothed aggregation may
    also perform well.

  \item
    Try \texttt{unsmoothed} operators instead of smoothed aggregation (\texttt{sa}).
    Scalability in terms of iterations performed will suffer from this,
    but solution times might go down since the operators are less dense,
    and less communication is performed.

  \item
    Enable implicit restrictor construction (\texttt{transpose:} \texttt{use implicit}) for symmetric
    problems.

  \item
    Enable triple matrix products instead of two matrix-matrix products for
    the construction of coarse operators (\texttt{rap: triple product}).
    This is beneficial as long as the involved operators are not too dense.
    For \texttt{unsmoothed} hierarchies, it is always faster.

  \item
    Find good level smoothers (see~\S\ref{sec:options_smoothing}). If a problem
    is symmetric positive definite, choose a smoother with a matrix-vector
    computational kernel, such as the Chebyshev polynomial smoother. If you are
    using relaxation smoothers, we recommend starting with optimizing the
    damping parameter. Once you have found a good damping parameter for your
    problem, you can increase the number of smoothing iterations.

  \item
    Adjust aggregation parameters if you experience bad coarsening ratios
    (see~\S\ref{sec:options_aggregation}). Particularly, try adjusting the
    minimum (\texttt{aggregation:} \texttt{min agg size}) and maximum
    (\texttt{aggregation:} \texttt{max agg size}) aggregation parameters. For a
    2D (3D) isotropic problem on a regular mesh, the aggregate size should be
    about 9 (27) nodes per aggregate.

  \item
    Replace a direct solver with an iterative method (\texttt{coarse: type}) if
    your coarse level solution becomes too expensive (see~\S\ref{sec:options_smoothing}).

  \item
    If on-node parallelism is required, make sure to enable the \kokkos code path (\texttt{use kokkos refactor}).
    If Gauss-Seidel smoothing is used, switch to multi-threaded Gauss-Seidel (see~\S\ref{sec:options_smoothing}).

\end{itemize}

Some advice for parallel runs include:
\begin{enumerate}
  \item
    Enable matrix rebalancing when running in parallel (\texttt{repartition:}
    \texttt{enable}).

  \item
    Use smoothers invariant to the number of processors, such as
    polynomial smoothing, if possible.

  \item
    Use \texttt{uncoupled} aggregation instead of \texttt{coupled}, as the latter
    requires significantly more communication.

  \item
    Adjust rebalancing parameters (see~\S\ref{sec:options_rebalancing}). Try
    choosing rebalancing parameters so that you end up with one processor on the
    coarsest level for the direct solver (this avoids additional communication).

  \item
    If the \texttt{multijagged} algorithm from \zoltan2 is used, try setting the premigration option.

  \item
    Enable implicit rebalancing of prolongators and restrictors
    (\texttt{repartition: rebalance P and R}).
\end{enumerate}

%%% Local Variables:
%%% mode: latex
%%% TeX-master: "mueluguide"
%%% End:


    %-----------------------------%
    \chapter{\muelu{} options} \label{sec:options}
    %-----------------------------%
    \label{sec:options}
In this section, we report the complete list of input parameters. Input
parameters are passed to \ifpacktwo in a single \parameterlist.

In some cases, the parameter types may depend on runtime template parameters.
In such cases, we will follow the conventions in Table~\ref{tab:conventions}.

\begin{table}[htbp]
  \centering
  \begin{tabular}{p{13.3cm} p{2.5cm}}
    \toprule
    \verb!MatrixType::local_ordinal_type!                                  & \verb!local_ordinal! \\
    \verb!MatrixType::global_ordinal_type!                                 & \verb!global_ordinal! \\
    \verb!MatrixType::scalar_type!                                         & \verb!scalar! \\
    \verb!MatrixType::node_type!                                           & \verb!node! \\
    \verb!Tpetra::Vector<scalar,local_ordinal,global_ordinal,node>!        & \verb!vector!\\
    \verb!Tpetra::MultiVector<scalar,local_ordinal,global_ordinal,node>!   & \verb!multi_vector!\\
    \verb!vector::mag_type!                                                & \verb!magnitude! \\
    \bottomrule
  \end{tabular}
  \caption{\label{tab:conventions}Conventions for option types that depend on templates.}
\end{table}

\noindent\textbf{Note:} if \verb!scalar! is \texttt{double}, then \verb!magnitude! is also \texttt{double}.

\section{Point relaxation}\label{s:relaxation}

\textbf{Preconditioner type:} ``RELAXATION''.

\ifpacktwo{} implements the following classical relaxation methods: Jacobi (with
optional damping), Gauss-Seidel, Successive Over-Relaxation (SOR), symmetric
version of Gauss-Seidel and SOR. \ifpacktwo{} calls both Gauss-Seidel and SOR
"Gauss-Seidel". The algorithmic details can be found in~\cite{Saad2003}.

Besides the classical relaxation methods, \ifpacktwo{} also implements $l_1$
variants of Jacobi and Gauss-Seidel methods proposed in~\cite{Baker2011}, which
lead to a better performance in parallel applications.

\noindent{\bf Note:} if a user provides a \texttt{Tpetra::BlockCrsMatrix}, the point relaxation
methods become block relaxation methods, such as block Jacobi or block
Gauss-Seidel.

The following parameters are used in the point relaxation methods:

\ccc{relaxation: type}
    {string}
    {``Jacobi''}
    {Relaxation method to use. Accepted values: ``Jacobi'',
     ``Gauss-Seidel'', ``Symmetric Gauss-Seidel''.}
\ccc{relaxation: sweeps}
    {int}
    {1}
    {Number of sweeps of the relaxation.}
\ccc{relaxation: damping factor}
    {scalar}
    {1.0}
    {The value of the damping factor $\omega$ for the relaxation.}
\ccc{relaxation: backward mode}
    {bool}
    {\false}
    {Governs whether Gauss-Seidel is done in forward-mode (\false) or
     backward-mode (\true). Only valid for ``Gauss-Seidel'' type.}
\ccc{relaxation: use l1}
    {bool}
    {\false}
    {Use the $l_1$ variant of Jacobi or Gauss-Seidel.}
\ccc{relaxation: l1 eta}
    {magnitude}
    {1.5}
    {$\eta$ parameter for $l_1$ variant of Gauss-Seidel. Only used if
     {\tt "relaxation: use l1"} is \true.}
\ccc{relaxation: zero starting solution}
    {bool}
    {\true}
    {Governs whether or not \ifpacktwo{} uses existing values in the left hand
     side vector. If true, \ifpacktwo{} fill it with zeros before applying
     relaxation sweeps which may make the first sweep more efficient.}
\ccc{relaxation: fix tiny diagonal entries}
    {bool}
    {\false}
    {If true, the compute() method will do extra work (computation only, no MPI
     communication) to fix diagonal entries. Specifically, the diagonal values
     with a magnitude smaller than the magnitude of the threshold \texttt{relaxation: min
     diagonal value} are increased to threshold for the diagonal inversion. The
     matrix is not modified, instead the updated diagonal values are stored. If the
     threshold is zero, only the diagonal entries that are exactly zero are replaced
     with a small nonzero value (machine precision).}
\ccc{relaxation: min diagonal value}
    {scalar}
    {0.0}
    {The threshold value used in {\tt "relaxation: fix tiny diagonal entries"}.
     Only used if {\tt "relaxation: fix tiny diagonal entries"} is \true.}
\ccc{relaxation: check diagonal entries}
    {bool}
    {\false}
    {If true, the \texttt{compute()} method will do extra work (both computation
     and communication) to count diagonal entries that are zero, have negative
     real part, or are small in magnitude. This information can be later shown
     in the description.}
\ccc{relaxation: local smoothing indices}
    {Teuchos::ArrayRCP<local\_ordinal>}
    {empty}
%Teuchos::ArrayRCP MatrixType::local_ordinal_type}{\texttt{Teuchos::null}}
    {A given method will only relax on the local indices listed in the
     \texttt{ArrayRCP}, in the order that they are listed. This can be used to
     reorder the relaxation, or to only relax on a subset of ids.}

\section{Block relaxation}\label{s:block_relaxation}

\textbf{Preconditioner type:} ``BLOCK\_RELAXATION''.

% \info[inline]{AP}{ILUTP cannot be constructed through {\tt Ifpack2::Factory},
% only through additive Schwarz}

\ifpacktwo{} supports block relaxation methods. Each block corresponds to a set
of degrees of freedom within a local subdomain. The blocks can be
non-overlapping or overlapping. Block relaxation can be considered as domain
decomposition within an MPI process, and should not be confused with additive
Schwarz preconditioners (see~\ref{s:schwarz}) which implement domain
decomposition across MPI processes.

There are several ways the blocks are constructed:
\begin{itemize}
  \item Linear partitioning of unknowns

    The unknowns are divided equally among a specified number of
    partitions $L$ defined by {\tt "partitioner: local parts"}. In other words,
    assuming number of unknowns $n$ is divisible by $L$, unknown $i$ will belong
    to block number $\lfloor iL/n \rfloor$.

  \item Line partitioning of unknowns

    The unknowns are grouped based on a geometric criteria which tries to
    identify degrees of freedom that form an approximate geometric line.
    Current approach uses a local line detection inspired by the work of
    Mavriplis~\cite{Mavriplis1999} for convection-diffusion. \ifpacktwo uses
    coordinate information provided by {\tt "partitioner: coordinates"} to pick
    "close" points if they are sufficiently far away from the "far" points. It
    also makes sure the line can never double back on itself.

    These "line" partitions were found to be very beneficent to problems on
    highly anisotropic geometries such as ice-sheet simulations.

  \item User partitioning of unknowns

    The unknowns are grouped according to a user provided partition. A user
    may provide a non-overlapping partition {\tt "partitioner: map"} or an
    overlapping one {\tt "partitioner: parts"}.

    A particular example of a smoother using this approach is a Vanka
    smoother~\cite{Vanka1986}, where a user may in {\tt "partition: parts"} pressure
    degrees of freedom, and request a overlap of one thus constructing Vanka
    blocks.
\end{itemize}
The original partitioning may be further modified with {\tt "partitioner: overlap"}
parameter which will use the local matrix graph to construct overlapping
partitions.

The blocks are applied in the order they were constructed. This means that in
the case of overlap the entries in the solution vector relaxed by one block may
later be overwritten by relaxing another block.

The following parameters are used in the block relaxation methods:

\cccc{relaxation: type}
    {See~\ref{s:relaxation}.}
\cccc{relaxation: container}
    {string}
    {``TriDi''}
    {Containers are used to store and solve block matrices. These container
     types are always available: ``Dense'', ``TriDi''
     (equivalent to ``Tridiagonal''), ``Banded'' and ``SparseILUT''.
     ``Dense'', ``TriDi'' and ``Banded'' block matrices are
     solved exactly LAPACK routines, and ``SparseILUT'' blocks are solved approximately
     using an incomplete LU factorization with thresholding.

     If Amesos2 is enabled, ``SparseAmesos'' (equivalent to ``SparseAmesos2'') is available.
     The default Amesos2 sparse solver is KLU2, but this can be configured by setting
     ``Amesos2 solver name'' (see the Amesos2 documentation for all available solvers).

     If experimental kokkos-kernels features are enabled (true by default), the ``BlockTriDi''
     container (equivalent to ``Block Tridiagonal'') is available. This container's solver is the damped Jacobi method, using
     block tridiagonal matrices as the diagonal D.
     For a block size of 1, this is equivalent to standard damped Jacobi.
     This container is designed for high performance on KNL and GPU.}
\cccc{relaxation: sweeps}
    {See~\ref{s:relaxation}.}
\cccc{relaxation: damping factor}
    {See~\ref{s:relaxation}.}
\cccc{relaxation: zero starting solution}
    {See~\ref{s:relaxation}.}
\cccc{relaxation: backward mode}
    {See~\ref{s:relaxation}. Currently has no effect. }
\ccc{partitioner: type}
    {string}
    {``linear''}
    {The partitioner to use for defining the blocks.  This can be either
     ``linear'', ``line'' or ``user''.}
\ccc{partitioner: overlap}
    {int}
    {0}
    {The amount of overlap between partitions (0 corresponds to no overlap).
     Only valid for ``Jacobi'' relaxation.}
\ccc{partitioner: local parts}
    {int}
    {1}
    {Number of local partitions (1 corresponds to one local partition, which
     means "do not partition locally"). Only valid for ``linear'' partitioner
     type.}
\ccc{partitioner: map}
    {Teuchos::ArrayRCP<local\_ordinal>}
    {empty}
    {An array containing the partition number for each element.
     The $i$th entry in the \texttt{ArrayRCP} is the part (block) number that
     row $i$ belongs to. Use this option if the parts (blocks) do not
     overlap. Only valid for ``user'' partitioner type.}
\ccc{partitioner: parts}
    {Teuchos::Array<Teuchos::ArrayRCP\\<local\_ordinal>>}
    {empty}
    {Use this option if the parts (blocks) overlap. The $i$th entry in the
     \texttt{Array} is an \texttt{ArrayRCP} that contains all the rows in part
     (block) $i$. Only valid for ``user'' partitioner type.}
\ccc{partitioner: line detection threshold}
    {magnitude}
    {0.0}
    {Threshold used in line detection. If the distance between two connected
     points $i$ and $j$ is within the threshold times maximum distance of all
     points connected to $i$, then point $j$ is considered close enough to line
     smooth. Only valid for ``line'' partition type.}
\ccc{partitioner: PDE equations}
    {int}
    {1}
    {Number of equations per node. Only valid for ``line'' partition type.}
\ccc{partitioner: coordinates}
    {Teuchos::RCP<multi\_vector>}
    {null}
    {Coordinates of local nodes. Only valid for ``line'' partitioner type.}
\ccc{partitioner: maintain sparsity}
    {bool}
    {\false}
    {For OverlappingPartitioner, whether to sort the entries in each partition.}

\section{Chebyshev}\label{s:Chebyshev}

\textbf{Preconditioner type:} ``CHEBYSHEV''.

% Mark Hoemmen (2016/05/31):
%   The "textbook version" of Chebyshev doesn't really
%   work; we need to get rid of it.

\ifpacktwo{} implements a variant of Chebyshev iterative method following
\ifpack{}'s implementation.  \ifpack{} has a special-case modification of the
eigenvalue bounds for the case where the maximum eigenvalue estimate is close to
one. Experiments show that the \ifpack{} imitation is much less sensitive to the
eigenvalue bounds than the textbook version.

\ifpacktwo{} uses the diagonal of the matrix to precondition the linear system on the
left. Diagonal elements less than machine precision are replaced with machine
precision.

\ifpacktwo{} requires can take any matrix $A$ but can only guarantee convergence
for real valued symmetric positive definite matrices.
\iffalse
If users could provide the ellipse parameters ($d$ and $c$ in the literature,
where $d$ is the real-valued center of the ellipse, and $d-c$ and $d+c$ the two
foci), the iteration itself would work fine with nonsymmetric real-valued $A$,
as long as the eigenvalues of $A$ can be bounded in an ellipse that is entirely
to the right of the origin.
\unsure[inline]{AP}{Really unsure about Chebyshev nonsymmetric matrices. There does not
seem anything in the code to work with ellipse. I need to ask Mark Hoemmen
about this.}
\fi

The following parameters are used in the Chebyshev method:

\ccc{chebyshev: degree}
    {int}
    {1}
    {Degree of the Chebyshev polynomial, or the number of iterations. This
     overrides parameters {\tt "relaxation: sweeps"} and {\tt "smoother: sweeps"}.}
\cccc{relaxation: sweeps}
    {Same as {\tt "chebyshev: degree"}, for compatibility with \ifpack{}.}
\cccc{smoother: sweeps}
    {Same as {\tt "chebyshev: degree"}, for compatibility with \ml{}.}
\ccc{chebyshev: max eigenvalue}
    {scalar|double}
    {computed}
    {An upper bound of the matrix eigenvalues. If not provided, the value will
     be computed by power method (see parameters {\tt "eigen-analysis: type"} and
     {\tt "chebyshev: eigenvalue max iterations"}).}
\ccc{chebyshev: min eigenvalue}
    {scalar|double}
    {computed}
    {A lower bound of the matrix eigenvalues.  If not provided, \ifpacktwo{}
     will provide an estimate based on the maximum eigenvalue and the ratio.}
\ccc{chebyshev: ratio eigenvalue}
    {scalar|double}
    {30.0}
    {The ratio of the maximum and minimum estimates of the matrix
     eigenvalues.}
\cccc{smoother: Chebyshev alpha}
    {Same as {\tt "chebyshev: ratio eigenvalue"}, for compatibility with \ml{}.}
% \ccc{chebyshev: textbook algorithm}
    % {bool}
    % {\false}
    % {If true, use the textbook variant; otherwise, use the \ifpack{} variant.}
\ccc{chebyshev: compute max residual norm}
    {bool}
    {\false}
    {The \texttt{apply} call will optionally return the norm of the residual.}
\ccc{eigen-analysis: type}
    {string}
    {"power-method"}
    {The algorithm for estimating the max eigenvalue. Currently only supports
     power method ("power-method" or "power method"). The cost of the procedure is
     roughly equal to several matrix-vector multiplications.}
\ccc{chebyshev: eigenvalue max iterations}
    {int}
    {10}
    {Number of iterations to be used in calculating the estimate for the maximum
     eigenvalue, if it is not provided by the user.}
\cccc{eigen-analysis: iterations}
    {Same as {\tt "chebyshev: eigenvalue max iterations"}, for compatibility with \ml{}.}
\ccc{chebyshev: min diagonal value}
    {scalar}
    {0.0}
    {Values on the diagonal smaller than this value are increased to this value
     for the diagonal inversion.}
\ccc{chebyshev: boost factor}
    {double}
    {1.1}
    {Factor used to increase the estimate of matrix maximum eigenvalue to ensure
    the high-energy modes are not magnified by a smoother.}
\ccc{chebyshev: assume matrix does not change}
    {bool}
    {\false}
    {Whether \texttt{compute()} should assume that the matrix has not changed
     since the last call to \texttt{compute()}. If true, \texttt{compute()}
     will not recompute inverse diagonal or eigenvalue estimates.}
\ccc{chebyshev: operator inv diagonal}
    {Teuchos::RCP<const vector>|\\Teuchos::RCP<vector>|const vector*|\\vector}
    {Teuchos::null}
    {If nonnull, a deep copy of this vector will be used as the inverse
     diagonal of the matrix, instead of computing it. Expert use only.}
\ccc{chebyshev: min diagonal value}
    {scalar}
    {machine precision}
    {If any entry of the matrix diagonal is less that this in magnitude, it will
     be replaced with this value in the inverse diagonal used for left scaling.}
\cccc{chebyshev: zero starting solution}
    {See {\tt "relaxation: zero starting solution"}.}

\section{Incomplete factorizations}

\subsection{ILU($k$)}\label{s:ILU}

\textbf{Preconditioner type:} ``RILUK''.

\ifpacktwo{} implements a standard and modified (MILU) variants of the
ILU($k$) factorization~\cite{Saad2003}. In addition, it also provides an
optional \textit{a priori} modification of the diagonal entries of a matrix to
improve the stability of the factorization.

The following parameters are used in the ILU($k$) method:

\ccc{fact: iluk level-of-fill}
    {int|global\_ordinal|magnitude|double}
    {0}
    {Level-of-fill of the factorization.}
\ccc{fact: relax value}
    {magnitude|double}
    {0.0}
    {MILU diagonal compensation value. Entries dropped during factorization
     times this factor are added to diagonal entries.}
\ccc{fact: absolute threshold}
    {magnitude|double}
    {0.0}
    {Prior to the factorization, each diagonal entry is updated by adding
     this value (with the sign of the actual diagonal entry). Can be combined
     with {\tt "fact: relative threshold"}. The matrix remains unchanged.}
\ccc{fact: relative threshold}
    {magnitude|double}
    {1.0}
    {Prior to the factorization, each diagonal element is scaled by this factor
     (not including contribution specified by {\tt "fact: absolute
     threshold"}). Can be combined with {\tt "fact: absolute threshold"}.
     The matrix remains unchanged.}
% All overlap-related code was removed by M. Hoemmen in
%
% commit 162f64572fbf93e2cac73e3034d76a3db918a494
% Author: Mark Hoemmen <mhoemme@sandia.gov>
% Date:   Fri Jan 24 17:16:19 2014 -0700
%
%     Ifpack2: RILUK: Removed all overlap-related code.
%
%     Overlap never had a correct implementation in RILUK.  Furthermore,
%     AdditiveSchwarz is the proper place for overlap to be implemented, not
%     RILUK.  Ifpack2's incomplete factorizations are local (per MPI
%     process) solvers and don't need to know anything about overlap across
%     processes.  Thus, this commit removes all overlap-related code from
%     RILUK.
%
% So, older parameter "fact: iluk level-of-overlap" is no longer valid and is ignored.

\subsection{ILUT}\label{s:ILUT}

\textbf{Preconditioner type:} ``ILUT''.

\ifpacktwo{} implements a slightly modified variant of the standard ILU factorization with specified fill and
drop tolerance ILUT($p,\tau$)~\cite{Saad1994}. The modifications follow the \aztecoo implementation.
The main difference between the \ifpacktwo implementation and the algorithm in \cite{Saad1994} is the definition of
\texttt{fact: ilut level-of-fill}.

The following parameters are used in the ILUT method:

\ccc{fact: ilut level-of-fill}
    {int|magnitude|double}
    {1}
    {Maximum number of entries to keep in each row of $L$ and $U$. Each row of
     $L$ ($U$) will have at most $\lceil\frac{(\mbox{\small\tt
     level-of-fill}-1)nnz(A)}{2n}\rceil$ nonzero entries, where $nnz(A)$ is the
     number of nonzero entries in the matrix, and $n$ is the number of rows.
     ILUT always keeps the diagonal entry in the current row, regardless of the
     drop tolerance or fill level. \textbf{Note:} \textit{This is
     different from the $p$ in the classic algorithm in~\cite{Saad1994}.}}
\ccc{fact: drop tolerance}
    {magnitude|double}
    {0.0}
    {A threshold for dropping entries ($\tau$ in~\cite{Saad1994}).}
\cccc{fact: absolute threshold}
    {See~\ref{s:ILU}.}
\cccc{fact: relative threshold}
    {See~\ref{s:ILU}.}
\cccc{fact: relax value}
    {Currently has no effect. For backwards compatibility only.}

\subsection{ILUTP}\label{s:ILUTP}

\textbf{Preconditioner type:} ``AMESOS2''.

% \info[inline]{AP}{ILUTP cannot be constructed through {\tt Ifpack2::Factory},
% only through additive Schwarz}

\ifpacktwo{} implements a standard ILUTP factorization~\cite{Saad2003}. This is
done through is through the \amesostwo interface to SuperLU~\cite{Li2011}. We
reproduce the \amesostwo options here for convenience. {\em You should consider
the \href{http://trilinos.org/docs/dev/packages/amesos2/doc/html/group__amesos2__solver__parameters.html#superlu_parameters}{\amesostwo
documentation} to be the final authority.}

The following parameters are used in the ILUTP method:

\ccc{ILU\_DropTol}
    {double}
    {1e-4}
    {ILUT drop tolerance.}
\ccc{ILU\_FillFactor}
    {double}
    {10.0}
    {ILUT fill factor.}
\ccc{ILU\_Norm}
    {string}
    {``INF\_NORM''}
    {Norm to be used in factorization. Accepted values: ``ONE\_NORM'', ``TWO\_NORM'', or ``INF\_NORM''.}
\ccc{ILU\_MILU}
    {string}
    {``SILU''}
    {Type of modified ILU to use. Accepted values: ``SILU'', ``SMILU\_1'', ``SMILU\_2'', or ``SMILU\_3''.}

\subsection{ShyLU FastILU}\label{s:FastILU}
\ifpacktwo{} provides an interface to the FastILU family of factorizations provided by ShyLU. There are
three values of ``Preconditioner type:'' that use FastILU:

\begin{table}[h!]
\centering
\begin{tabular}{|l|l|}
\hline
``Preconditioner type:'' & Factorization   \\ \hline \hline
FAST\_ILU                & Incomplete LU       \\ \hline
FAST\_IC                 & Incomplete Cholesky \\ \hline
FAST\_ILDL               & Incomplete LDL*     \\ \hline
\end{tabular}
\end{table}

FAST\_ILU, FAST\_IC, and FAST\_ILDL all accept the same set of parameters:

\ccc{sweeps}
    {int}
    {5}
    {Number of applications of the FastILU/IC/ILDL algorithm.}
\ccc{triangular solve iterations}
    {int}
    {1}
    {Number of iterations of the block Jacobi triangular solver.}
\ccc{level}
    {int}
    {0}
    {Level of fill.}
\ccc{damping factor}
    {double}
    {0.5}
    {Damping factor $\omega$ for the Jacobi triangular solver. $0 < \omega \leq 1$. A lower $\omega$ slows convergence but improves stability.}
\ccc{shift}
    {double}
    {0}
    {Manteuffel shifting parameter $\alpha$.}
\ccc{guess}
    {bool}
    {true}
    {Whether to run a few sweeps of FastILU with a lower level of fill to create initial guess (only has an effect if level of fill $> 0$).} 
\ccc{block size}
    {int}
    {1}
    {Block size for the block Jacobi solver.}

\section{Additive Schwarz}\label{s:schwarz}

\textbf{Preconditioner type:} ``SCHWARZ''.

\ifpacktwo{} implements additive Schwarz domain decomposition with optional
overlap. Each subdomain corresponds to exactly one MPI process in the given
matrix's MPI communication. For domain decomposition within an
MPI process see~\ref{s:block_relaxation}.

One-level overlapping domain decomposition preconditioners use local solvers of
Dirichlet type. This means that the inverse of the local matrix (possibly with
overlap) is applied to the residual to be preconditioned. The preconditioner can
be written as:
$$ P_{AS}^{-1} = \sum_{i=1}^M P_i A_i^{-1} R_i, $$
where $M$ is the number of subdomains (in this case, the number of (MPI)
processes in the computation), $R_i$ is an operator that restricts the global
vector to the vector lying on subdomain $i$, $P_i$ is the prolongator
operator, and $A_i = R_i A P_i$.

Constructing a Schwarz preconditioner requires defining two components.

{\bf Definition of the restriction and prolongation operators.}
Users may control how the data is combined with existing data by setting {\tt
"combine mode"} parameter. Table~\ref{t:combine_mode} contains a list of modes to
combine overlapped entries. The default mode is ``ZERO'' which is equivalent to
using a restricted additive Schwarz~\cite{Cai1999} method.

\begin{table}[htbp]
  \centering
  \begin{tabular}{p{3.5cm} p{12.0cm}}
    \toprule
    Combine mode name & Description \\
    \midrule
    ``ADD''           & Sum values into existing values \\
    ``ZERO''          & Replace old values with zero \\
    ``INSERT''        & Insert new values that don't currently exist \\
    ``REPLACE''       & Replace existing values with new values \\
    ``ABSMAX''        & Replace old values with maximum of magnitudes of old and new values \\
    \bottomrule
  \end{tabular}
  \caption{\label{t:combine_mode}Combine mode descriptions.}
\end{table}

{\bf Definition of a solver for subdomain linear system.}
Some preconditioners may benefit from local modifications to the subdomain
matrix. It can be filtered to eliminate singletons and/or reordered.
Reordering will often improve performance during incomplete factorization setup,
and improve the convergence. The matrix reordering algorithms specified in {\tt
"schwarz: reordering list"} are provided by \zoltantwo.  At the present time,
the only available reordering algorithm is RCM (reverse Cuthill-McKee). Other
orderings will be supported by the Zoltan2 package in the future.

To solve linear systems involving $A_i$ on each subdomain, a user can specify
the inner solver by setting {\tt "inner preconditioner name"} parameter (or any
of its aliases) which allows to use any \ifpacktwo preconditioner. These include
but are not necessarily limited to the preconditioners in
Table~\ref{t:schwarz_inner}.

\begin{table}[htbp]
  \centering
  \begin{tabular}{p{5.0cm} p{10.5cm}}
    \toprule
    Inner solver type       & Description \\
    \midrule
    ``DIAGONAL''            & Diagonal scaling \\
    ``RELAXATION''          & Point relaxation (see~\ref{s:relaxation}) \\
    ``BLOCK\_RELAXATION''   & Block relaxation (see~\ref{s:block_relaxation}) \\
    ``CHEBYSHEV''           & Chebyshev iteration (see~\ref{s:Chebyshev}) \\
    ``RILUK''               & ILU($k$) (see~\ref{s:ILU}) \\
    ``ILUT''                & ILUT (see~\ref{s:ILUT}) \\
    ``FastILU''             & FastILU (see~\ref{s:FastILU}) \\
    ``FastIC''              & FastIC (see~\ref{s:FastIC}) \\
    ``FastILDL''            & FastILDL(see~\ref{s:FastILDL}) \\
    ``AMESOS2''             & \amesostwo's interface to sparse direct solvers \\
    ``DENSE'' or ``LAPACK'' & LAPACK's LU factorization for a dense representation of a subdomain matrix \\
    ``CUSTOM''              & User provided inner solver \\
    % ``RBILUK''
    \bottomrule
  \end{tabular}
  \caption{\label{t:schwarz_inner}Additive Schwarz solver preconditioner types.}
\end{table}

The following parameters are used in the Schwarz method:

\ccc{schwarz: inner preconditioner name}
    {string}
    {none}
    {The name of the subdomain solver.}
\cccc{inner preconditioner name}
    {Same as {\tt "schwarz: inner preconditioner name"}.}
\cccc{schwarz: subdomain solver name}
    {Same as {\tt "schwarz: inner preconditioner name"}.}
\cccc{subdomain solver name}
    {Same as {\tt "schwarz: inner preconditioner name"}.}
\ccc{schwarz: inner preconditioner parameters}
    {\parameterlist}
    {empty}
    {Parameters for the subdomain solver. If not provided, the subdomain solver
     will use its specific default parameters.}
\cccc{inner preconditioner parameters}
    {Same as {\tt "schwarz: inner preconditioner parameters"}.}
\cccc{schwarz: subdomain solver parameters}
    {Same as {\tt "schwarz: inner preconditioner parameters"}.}
\cccc{subdomain solver parameters}
    {Same as {\tt "schwarz: inner preconditioner parameters"}.}
\ccc{schwarz: combine mode}
    {string}
    {``ZERO''}
    {The rule for combining incoming data with existing data in overlap regions.
     Accepted values: see Table~\ref{t:combine_mode}.}
\ccc{schwarz: overlap level}
    {int}
    {0}
    {The level of overlap (0 corresponds to no overlap).}
\ccc{schwarz: num iterations}
    {int}
    {1}
    {Number of iterations to perform.}
\ccc{schwarz: use reordering}
    {bool}
    {\false}
    {If true, local matrix is reordered before computing subdomain solver. \trilinos must have been built with
     \zoltantwo and \xpetra enabled.}
\ccc{schwarz: reordering list}
    {\parameterlist}
    {empty}
    {Specify options for a \zoltantwo reordering algorithm to use. See {\tt
     "order\_method"}. {\em You should consider the
     \href{http://trilinos.org/docs/dev/packages/zoltan2/doc/html/z2_parameters.html}{\zoltantwo
     documentation} to be the final authority.}}
\ccc{order\_method}
    {string}
    {``rcm''}
    {Reordering algorithm. Accepted values: ``rcm'', ``minimum\_degree'',
     ``natural'', ``random'', or ``sorted\_degree''. Only used in {\tt
     "schwarz: reordering list"} sublist.}
\cccc{schwarz: zero starting solution}
    {See {\tt "relaxation: zero starting solution"}.}
\ccc{schwarz: filter singletons}
    {bool}
    {\false}
    {If true, exclude rows with just a single entry on the calling process.}
\cccc{schwarz: subdomain id}
    {Currently has no effect.}
\cccc{schwarz: compute condest}
    {Currently has no effect. For backwards compatibility only.}

\section{Hiptmair}

\ifpacktwo{} implements Hiptmair algorithm of~\cite{Hiptmair1997}. The method
operates on two spaces: a primary space and an auxiliary space. This situation
arises, for instance,  when preconditioning Maxwell's equations discretized by
edge elements. It is used in \muelu~\cite{MueLu} ``RefMaxwell''
solver~\cite{RefMaxwell}.

Hiptmair's algorithm does not use \texttt{Ifpack2::Factory} interface for
construction.  Instead, a user must explicitly call the constructor
\begin{lstlisting}[language=C++]
  Teuchos::RCP<Tpetra::CrsMatrix<> > A, Aaux, P;
  // create A, Aaux, P here ...
  Teuchos::ParameterList paramList;
  paramList.set("hiptmair: smoother type 1", "CHEBYSHEV");
  // ...
  RCP<Ifpack2::Ifpack2Preconditioner<> > ifpack2Preconditioner =
    Teuchos::rcp(new Ifpack2::Hiptmair(A, Aaux, P);
  ifpack2Preconditioner->setParameters(paramList);
\end{lstlisting}
\noindent Here, $A$ is a matrix in the primary space, $Aaux$ is a matrix in
auxiliary space, and $P$ is a prolongator/restrictor between the two spaces.

The following parameters are used in the Hiptmair method:

\ccc{hiptmair: smoother type 1}
    {string}
    {"CHEBYSHEV"}
    {Smoother type for smoothing the primary space.}
\ccc{hiptmair: smoother list 1}
    {\parameterlist}
    {empty}
    {Smoother parameters for smoothing the primary space.}
\ccc{hiptmair: smoother type 2}
    {string}
    {"CHEBYSHEV"}
    {Smoother type for smoothing the auxiliary space.}
\ccc{hiptmair: smoother list 2}
    {\parameterlist}
    {empty}
    {Smoother parameters for smoothing the auxiliary space.}
\ccc{hiptmair: pre or post}
    {string}
    {``both''}
    {\ifpacktwo{} always relaxes on the auxiliary space. ``pre'' (``post'') means
     that it relaxes on the primary space before (after) the relaxation on the
     auxiliary space. ``both'' means that we do both ``pre'' and ``post''.}
\cccc{hiptmair: zero starting solution}
    {See {\tt "relaxation: zero starting solution"}.}


    %-----------------------------%
    \chapter{\muemex: The MATLAB Interface for \muelu} \label{sec:muemex}
    %-----------------------------%
    %%%%%%%%%%%%%%%%%%%%%%%%%%%%%%%%%%%%%%%%%%%%%%%%%%%%%%%%%%%%%%%%%%%
\muemex is \muelu's interface to the MATLAB environment. It allows access
to a limited set of routines either \muelu as a preconditioner,
Belos as a solver and Epetra or Tpetra for data structures.
It is designed to provide access to \muelu's aggregation and
solver routines from MATLAB and does little else. \muemex allows users to
setup and solve arbitrarily many problems, so long as memory suffices.
More than one problem can be set up simultaneously.

\section{Cmake Configure and Make}\label{sec:muemex:cmake}
To use \muemex, Trilinos must be configured with (at least) the
following options:

\begin{lstlisting}
  export TRILINOS_HOME=/path/to/your/Trilinos/source/directory
  cmake \
      -D Trilinos_ENABLE_Amesos:BOOL=ON \
      -D Trilinos_ENABLE_Amesos2:BOOL=ON \
      -D Amesos2_ENABLE_KLU2:BOOL=ON \
      -D Trilinos_ENABLE_AztecOO:BOOL=ON \
      -D Trilinos_ENABLE_Epetra:BOOL=ON \
      -D Trilinos_ENABLE_EpetraExt:BOOL=ON \
      -D Trilinos_ENABLE_Fortran:BOOL=OFF \
      -D Trilinos_ENABLE_Ifpack:BOOL=ON \
      -D Trilinos_ENABLE_Ifpack:BOOL=ON \
      -D Trilinos_ENABLE_MueLu:BOOL=ON \
      -D Trilinos_ENABLE_Teuchos:BOOL=ON \
      -D Trilinos_ENABLE_Tpetra:BOOL=ON \
      -D TPL_ENABLE_MPI:BOOL=OFF \
      -D TPL_ENABLE_MATLAB:BOOL=ON \
      -D MATLAB_ROOT:STRING="<my matlab root>" \
      -D MATLAB_ARCH:STRING="<my matlab os string>" \
  ${TRILINOS_HOME}
\end{lstlisting}

Since \muemex supports both the Epetra and Tpetra linear algebra
libraries, you have to have both enabled in order to build \muemex.
Most additional options can be specified as well.  It is important to
note that \muemex does not work properly with MPI, hence MPI must be
disabled in order to compile \muemex.  The MATLAB\_ARCH option is new to
the cmake build system, and involves the MATLAB-specific architecture
code for your system.  There is currently no automatic way to extract
this, so it must be user-specified.  As of MATLAB 7.9 (R2009b), common
arch codes are:
\begin{center}
\begin{tabular}{l|l}
Code& OS\\
\hline
glnx86& 32-bit Linux (intel/amd)\\
glnxa64& 64-bit Linux (intel/amd)\\
maci64& 64-bit MacOS\\
maci& 32-bit MacOS\\
\end{tabular}
\end{center}

On 64-bit Intel/AMD architectures, Trilinos and all relevant TPLs
(note: this includes BLAS and LAPACK)
must be compiled with the \texttt{-fPIC} option.  This necessitates adding:
\begin{lstlisting}
    -D CMAKE_CXX_FLAGS:STRING="-fPIC" \
    -D CMAKE_C_FLAGS:STRING="-fPIC" \
    -D CMAKE_Fortran_FLAGS:STRING="-fPIC" \
\end{lstlisting}
to the cmake configure line.

\subsection{BLAS \& LAPACK Option \#1: Static Builds}
Trilinos does not play nicely with MATLAB's default LAPACK and BLAS on
64-bit machines.
If \muemex randomly crashes when you run with any Krylov method that
has orthogonalization, chances are \muemex is finding the wrong
BLAS/LAPACK libraries.
This leaves you
with one of two options.  The first is to build them both \textit{statically}
and then specify them as follows:
\begin{lstlisting}
    -D LAPACK_LIBRARY_DIRS:STRING="<path to my lapack.a>" \
    -D BLAS_LIBRARY_DIRS:STRING="<path to my blas.a>" \
\end{lstlisting}
Using static linking for LAPACK and BLAS prevents MATLAB's default libraries to take precedence.

\subsection{BLAS \& LAPACK Option \#2: LD$\_$PRELOAD}
The second option is to use \textsf{LD\_PRELOAD} to tell MATLAB exactly
which libraries to use.  For this option, you can use the dynamic
libraries installed on your system.
Before starting MATLAB, set
LD\_PRELOAD to the paths of libstdc++.so corresponding to the version of GCC used
to build Trilinos, and the paths of libblas.so and liblapack.so on your local system.

For example, if you use bash, you'd do something like this
\begin{lstlisting}
  export LD_PRELOAD=<path>/libstdc++.so:<path>/libblas.so:<path>/liblapack.so
  \end{lstlisting}

For csh / tcsh, do this
\begin{lstlisting}
  setenv LD_PRELOAD <path>/libstdc++.so:<path>/libblas.so:<path>/liblapack.so
\end{lstlisting}

\section{Using \muemex}\label{sec:muemex:usage}
\muemex is designed to be interfaced with via the MATLAB script
\texttt{muelu.m}.  There are five modes in which \muemex can be run:
\begin{enumerate}
\item Setup Mode --- Performs the problem setup for \muelu.
  Depending on whether or not the \texttt{Linear Algebra} option is
  used, \muemex creates either an unpreconditioned Epetra problem,
  an Epetra problem with \muelu, or a Tpetra problem with \muelu.
  The default is \texttt{epetra} for real matrices and \texttt{tpetra}
  for complex matrices. The \texttt{tpetra} option
  supports both real and complex and will infer the scalar type
  from the matrix passed during setup.  This call returns a problem
  handle used to reference the problem in the future, and (optionally)
  the operator complexity, if a preconditioner is being used.
\item Solve Mode --- Given a problem handle and a right-hand side, \muemex
  solves the problem specified.  Setup mode must be called before
  solve mode.
\item Cleanup Mode --- Frees the memory allocated to internal \muelu,
  Epetra and Tpetra objects.  This can be called with a particular
  problem handle, in which case it frees that problem, or without one,
  in which case all \muemex memory is freed.
\item Status Mode --- Prints out status information on problems which
  have been set up.  Like cleanup, it can be called with or without a
  particular problem handle.
\end{enumerate}
All of these modes, with the exception of status and cleanup take
option lists which will be directly converted into
\texttt{Teuchos::ParameterList} objects by \muemex, as key-value pairs.
Options passed during setup will apply to the \muelu preconditioner, and
options passed during a solve will apply to Belos.

\subsection{Setup Mode}
Setup mode is called as follows:
\begin{lstlisting}[language=Matlab]
  >> [h, oc] = muelu('setup', A[, 'parameter', value,...])
\end{lstlisting}
The parameter \texttt{A} represents the sparse matrix to perform aggregation on
and the parameter/value pairs represent standard \muelu options.

The routine returns a problem handle, \texttt{h}, and the operator
complexity \texttt{oc} for the operator.  In addition to the standard
options, setup mode has one unique option of its own:

\choicebox{\tt Linear Algebra}{[{\tt string}] Whether to use
  'epetra unprec', 'epetra', or 'tpetra'. Default is 'epetra' for
  real matrix and 'tpetra' for complex matrix.}

\subsection{Solve Mode}
Solve mode is called as follows:
\begin{lstlisting}[language=Matlab]
  >> [x, its] = muelu(h[, A], b[, 'parameter', value,...])
\end{lstlisting}
The parameter \texttt{h} is a problem handle returned by the
setup mode call, \texttt{A} is the sparse matrix with which to
solve and \texttt{b} is the right-hand side.  Parameter/value pairs
to configure the Belos solver are listed as above. If A is not supplied,
the matrix provided when setting up the problem will be used. \texttt{x} is
the solution multivector with the same dimensions as \texttt{b}, and \texttt{its}
is the number of iterations Belos needed to solve the problem.

All of these options are taken directly from Belos, so consult its
manual for more information. Belos output style and verbosity settings
are implemented as enums, but can be set as strings in \muemex. For example:

\begin{lstlisting}[language=Matlab]
  >> x = muelu(0, b, 'Verbosity', 'Warnings + IterationDetails', ...
                       'Output Style', 'Brief');
\end{lstlisting}

Verbosity settings can be separated by spaces, '+' or ','. Belos::Brief
is the default output style.

\subsection{Cleanup Mode}
Cleanup mode is called as follows:
\begin{lstlisting}[language=Matlab]
  >> muelu('cleanup'[, h])
\end{lstlisting}
The parameter \texttt{h} is a problem handle returned by the
setup mode call and is optional.  If \texttt{h} is provided, that
problem is cleaned up.  If the option is not provided all currently
set up problems are cleaned up.

\subsection{Status Mode}
Status mode is called as follows:
\begin{lstlisting}[language=Matlab]
  >> muelu('status'[, h])
\end{lstlisting}
The parameter \texttt{h} is a problem handle returned by the
setup mode call and is optional.  If \texttt{h} is provided, status
information for that problem is printed.  If the option is not provided all currently
set up problems have status information printed.

\subsection{Tips and Tricks }\label{sec:muemex:tips}

Internally, MATLAB represents all data as doubles unless you go
through efforts to do otherwise.  \muemex detects integer parameters by
a relative error test, seeing if the relative difference between the
value from MATLAB and the value of the \texttt{int}-typecast value are
less than 1e-15.  Unfortunately, this means that \muemex will choose the
incorrect type for parameters which are doubles that happen to have an
integer value (a good example of where this might happen would be the parameter
`smoother Chebyshev: alpha', which defaults to 30.0).  Since \muemex does no
internal typechecking of
parameters (it uses \muelu's internal checks), it has no way of detecting
this conflict.  From the user's perspective, avoiding this is as
simple as adding a small perturbation (greater than a relative 1e-15)
to the parameter that makes it non-integer valued.


    %-----------------------------%
    \chapter{YAML Parameter Lists}\label{sec:yaml}
    %-----------------------------%
    YAML is a human-readable data serialization format. MueLu provides a
YAML parameter list interpreter. It produces Teuchos::ParameterList
objects equivalent to those produced by the Teuchos XML helper functions.

Here is a simple example XML parameter list:
\begin{verbatim}
<ParameterList>
  <ParameterList Input>
    <Parameter name="values" type="Array(double)" value="{54.3 -4.5 2.0}"/>
    <Parameter name="myfunc" type="string" value="
def func(a, b):
  return a * 2 - b"/>
  </ParameterList>
  <ParameterList Solver>
    <Parameter name="iterations" type="int" value="5"/>
    <Parameter name="tolerance" type="double" value="1e-7"/>
    <Parameter name="do output" type="bool" value="true"/>
    <Parameter name="output file" type="string" value="output.txt"/>
  </ParameterList>
</ParameterList>
\end{verbatim}

Here is an equivalent YAML parameter list:
\begin{verbatim}
%YAML 1.1
---
ANONYMOUS:
  Input:
    values: [54.3, -4.5, 2.0]
    myfunc: |-

      def func(a, b):
        return a * 2 - b
  Solver:
    iterations: 5
    tolerance: 1e-7
    do output: yes
    output file: output.txt
...
\end{verbatim}

The nested structure and key-value pairs of these two lists are identical.
To a program querying them for settings, they are indistinguishable.

These are the general rules for creating a YAML parameter list:
\begin{itemize}
\item First line must be ``\%YAML 1.1'', second must be ``---'', and last must be ``...''
\item Nested map structure is determined by indentation. SPACES ONLY, NO TABS!
\item As with the above example, for a top-level anonymous parameter list, ``ANONYMOUS:'' must be explicit
\item Type is inferred. 5 is an int, 5.0 is a double, and '5.0' is a string
\item Quotation marks (single or double) are optional for strings, but required for strings with special characters: \verb.:{}[],&*#?|-<>=!%@\.
\item Quotation marks also turn non-string types into strings: '3' is a string
\item As with XML parameter lists, keys are regular strings
\item yes, no, y, n, true, false, on, off are all booleans. false, False and FALSE are all equivalent
\item Arrays of int, double and string supported. exampleArray: {[}hello, world, goodbye{]}
\item {[}3, 4, 5{]} is an int array, {[}3, 4, 5.0{]} is a double array, and {[}3, '4', 5.0{]} is a string array
\item For multi-line strings, place ``$|-$'' after the ``key:'' and then indent the string one level deeper than the key
\end{itemize}


    %\nocite{*}

    % ---------------------------------------------------------------------- %
    % References
    %
    \clearpage
    % If hyperref is included, then \phantomsection is already defined.
    % If not, we need to define it.
    \providecommand*{\phantomsection}{}
    \phantomsection
    \addcontentsline{toc}{chapter}{References}
    \bibliographystyle{plain}
    \bibliography{mueluguide}


    % ---------------------------------------------------------------------- %
    %
    \appendix
    \chapter{Copyright and License}
    \label{sec:license}
\begin{center}
MueLu: A package for multigrid based preconditioning

Copyright 2012 Sandia Corporation
\end{center}

\noindent
Under the terms of Contract DE--AC04--94AL85000 with Sandia Corporation,
the U.S. Government retains certain rights in this software.

\noindent
Redistribution and use in source and binary forms, with or without
modification, are permitted provided that the following conditions are
met:

\begin{enumerate}
  \item Redistributions of source code must retain the above copyright
    notice, this list of conditions and the following disclaimer.

\item Redistributions in binary form must reproduce the above copyright
  notice, this list of conditions and the following disclaimer in the
  documentation and/or other materials provided with the distribution.

\item Neither the name of the Corporation nor the names of the
  contributors may be used to endorse or promote products derived from
  this software without specific prior written permission.
\end{enumerate}

\noindent
THIS SOFTWARE IS PROVIDED BY SANDIA CORPORATION ``AS IS'' AND ANY
EXPRESS OR IMPLIED WARRANTIES, INCLUDING, BUT NOT LIMITED TO, THE
IMPLIED WARRANTIES OF MERCHANTABILITY AND FITNESS FOR A PARTICULAR
PURPOSE ARE DISCLAIMED\@. IN NO EVENT SHALL SANDIA CORPORATION OR THE
CONTRIBUTORS BE LIABLE FOR ANY DIRECT, INDIRECT, INCIDENTAL, SPECIAL,
EXEMPLARY, OR CONSEQUENTIAL DAMAGES (INCLUDING, BUT NOT LIMITED TO,
PROCUREMENT OF SUBSTITUTE GOODS OR SERVICES\@; LOSS OF USE, DATA, OR
PROFITS\@; OR BUSINESS INTERRUPTION) HOWEVER CAUSED AND ON ANY THEORY OF
LIABILITY, \\WHETHER IN CONTRACT, STRICT LIABILITY, OR TORT (INCLUDING
NEGLIGENCE OR OTHERWISE) ARISING IN ANY WAY OUT OF THE USE OF THIS
SOFTWARE, EVEN IF ADVISED OF THE POSSIBILITY OF SUCH DAMAGE\@.

    %\chapter{Historical Perspective}
	%    This is an example of an appendix.

    If we follow~\cite{Sand98-0730} strictly, we would have to have
    a separate bibliography section for each appendix.  The style
    file doesn't provide that, but it can be done using the {\tt
    bibunits} and {\tt chapterbib} packages.

    If there are many subsections in an appendix, it should also
    have its own table of contents. Again, the SAND report class
    file does not provide that functionality.

    \ifthenelse{\boolean{reportSAND}}   {
	\section{The Past a Long Time Ago}
    }{
	\subsection{The Past a Long Time Ago}
    }
	This is where we talk about things so old nobody can verify
	them. We are safe.

    \ifthenelse{\boolean{reportSAND}}   {
	\section{The Past More Recently}
    }{
	\subsection{The Past More Recently}
    }
	Now we have to be a little bit more careful, since records
	exist from that time, and some people still alive actually
	lived back then.


    \chapter{ML compatibility}
     
\label{sec:ml_options}
\muelu provides a basic compatibility layer for \ml parameter lists. This allows \ml users to quickly perform some experiments with \muelu. 

\textbf{First and most important: } Long term, we would like to have users use the new \muelu interface, as that is where most of new features will be made accessible. One should make note of the fact that it may not be possible to make ML deck do exactly same things in \ml and \muelu, as internally \ml implicitly makes some decision that we have no control over and which could be different from \muelu.

\noindent There are basically two distinct ways to use \ml input parameters with \muelu:
\begin{description}
\item[MLParameterListInterpreter:] This class is the pendant of the \texttt{ParameterListInterpreter} class for the \muelu parameters. It accepts parameter lists or XML files with \ml parameters and generates a \muelu multigrid hierarchy. It supports only a well-defined subset of \ml parameters which have a equivalent parameter in \muelu.
\item[ML2MueLuParameterTranslator:] This class is a simple wrapper class which translates \ml parameters to the corresponding \muelu parameters. It has to be used in combination with the \muelu \texttt{ParameterListInterpreter} class to generate a \muelu multigrid hierarchy. It is also meant to be used in combination with the \texttt{CreateEpetraPreconditioner} and \texttt{CreateTpetraPreconditioner} routines (see \S\ref{sec:examples in code}). It supports only a small subset of the \ml parameters.
\end{description}

\section{Usage of \ml parameter lists with \muelu}

\subsection{MLParameterListInterpreter}

The \texttt{MLParameterListInterpreter} directly accepts a \texttt{ParameterList} containing \ml parameters. It also interprets the \texttt{null space: vectors} and the \texttt{null space: dimension} \ml parameters. However, it is recommended to provide the near null space vectors directly in the \muelu way as shown in the following code snippet.

\begin{lstlisting}[language=C++]
    Teuchos::RCP<Tpetra::CrsMatrix<> > A;
    // create A here ...
    
    // XML file containing ML parameters
    std::string xmlFile = "mlParameters.xml"
    Teuchos::ParameterList paramList;
    Teuchos::updateParametersFromXmlFileAndBroadcast(xmlFile, Teuchos::Ptr<Teuchos::ParameterList>(&paramList), *comm);
       
    // use ParameterListInterpreter with MueLu parameters as input
    Teuchos::RCP<HierarchyManager> mueluFactory = Teuchos::rcp(new MLParameterListInterpreter(*paramList));
    
    RCP<Hierarchy> H = mueluFactory->CreateHierarchy();
    H->GetLevel(0)->Set<RCP<Matrix> >("A", A);
    H->GetLevel(0)->Set("Nullspace", nullspace);
    H->GetLevel(0)->Set("Coordinates", coordinates);
    mueluFactory->SetupHierarchy(*H);
\end{lstlisting}

Note that the \texttt{MLParameterListInterpreter} only supports a basic set of \ml parameters allowing to build smoothed aggregation transfer operaotrs (see \S\ref{sec:compatiblemlparameters} for a list of compatible \ml parameters).

\subsection{ML2MueLuParameterTranslator}

The \texttt{Ml2MueLuParameterTranslator} class is a simple wrapper translating \ml parameters to the corresponding \muelu parameters. This allows the usage of the simple \texttt{CreateEpetraPreconditioner} and \texttt{CreateTpetraPreconditioner} interface with \ml parameters:

\begin{lstlisting}[language=C++]
    Teuchos::RCP<Tpetra::CrsMatrix<> > A;
    // create A here ...
    
    // XML file containing ML parameters
    std::string xmlFile = "mlParameters.xml"
    Teuchos::ParameterList paramList;
    Teuchos::updateParametersFromXmlFileAndBroadcast(xmlFile, Teuchos::Ptr<Teuchos::ParameterList>(&paramList), *comm);

    // translate ML parameters to MueLu parameters
    RCP<ParameterList> mueluParamList = Teuchos::getParametersFromXmlString(MueLu::ML2MueLuParameterTranslator::translate(paramList,"SA"));

    Teuchos::RCP<MueLu::TpetraOperator> mueLuPreconditioner =
       MueLu::CreateTpetraPreconditioner(A, mueluParamList);
\end{lstlisting}

In a similar way, \ml input parameters can be used with the standard \muelu parameter list interpreter class. Note that the near null space vectors have to be provided in the \muelu way.

\begin{lstlisting}[language=C++]
    Teuchos::RCP<Tpetra::CrsMatrix<> > A;
    // create A here ...
    
    // XML file containing ML parameters
    std::string xmlFile = "mlParameters.xml"
    Teuchos::ParameterList paramList;
    Teuchos::updateParametersFromXmlFileAndBroadcast(xmlFile, Teuchos::Ptr<Teuchos::ParameterList>(&paramList), *comm);
    
    // translate ML parameters to MueLu parameters
    RCP<ParameterList> mueluParamList = Teuchos::getParametersFromXmlString(MueLu::ML2MueLuParameterTranslator::translate(paramList,"SA"));
    
    // use ParameterListInterpreter with MueLu parameters as input
    Teuchos::RCP<HierarchyManager> mueluFactory = Teuchos::rcp(new ParameterListInterpreter(*mueluParamList));
    
    RCP<Hierarchy> H = mueluFactory->CreateHierarchy();
    H->GetLevel(0)->Set<RCP<Matrix> >("A", A);
    H->GetLevel(0)->Set("Nullspace", nullspace);
    H->GetLevel(0)->Set("Coordinates", coordinates);
    mueluFactory->SetupHierarchy(*H);
\end{lstlisting}

Note that the set of supported \ml parameters is very limited. Please refer to \S\ref{sec:compatiblemlparameters} for a list of all compatible \ml parameters.

\section{Compatible \ml parameters}
\label{sec:compatiblemlparameters}
\subsection{General \ml options}

\mlcbb{ML output}{int}{0}{MLParameterListInterpreter, ML2MueLuParameterTranslator}{Control of the amount of printed information. Possible values: 0-10 with 0=no output and 10=maximum verbosity.}   
      
\mlcbb{PDE equations}{int}{1}{MLParameterListInterpreter, ML2MueLuParameterTranslator}{Number of PDE equations at each grid node. Only constant block size is considered.}   
      
\mlcbb{max levels}{int}{10}{MLParameterListInterpreter, ML2MueLuParameterTranslator}{Maximum number of levels in a hierarchy.}   
      
\mlcbb{prec type}{string}{"MGV"}{MLParameterListInterpreter, ML2MueLuParameterTranslator}{Multigrid cycle type. Possible values: "MGV", "MGW". Other values are NOT supported by MueLu.}   
      

\subsection{Smoothing and coarse solver options}

\mlcbb{smoother: type}{string}{"symmetric Gauss-Seidel"}{MLParameterListInterpreter, ML2MueLuParameterTranslator}{Smoother type for fine- and intermedium multigrid levels. Possible values: "Jacobi", "Gauss-Seidel", "symmetric Gauss-Seidel", "Chebyshev", "ILU".}

\mlcbb{smoother: sweeps}{int}{2}{MLParameterListInterpreter, ML2MueLuParameterTranslator}{Number of smoother sweeps for relaxation based level smoothers. In case of Chebyshev smoother it denotes the polynomial degree.}

\mlcbb{smoother: damping factor}{double}{1.0}{MLParameterListInterpreter, ML2MueLuParameterTranslator}{Damping factor for relaxation based level smoothers.}

\mlcbb{smoother: Chebyshev alpha}{double}{20}{MLParameterListInterpreter, ML2MueLuParameterTranslator}{Eigenvalue ratio for Chebyshev level smoother.}


\mlcbb{smoother: pre or post}{string}{"both"}{MLParameterListInterpreter, ML2MueLuParameterTranslator}{Pre- and post-smoother combination. Possible values: "pre" (only pre-smoother), "post" (only post-smoother), "both" (both pre-and post-smoothers).}

\mlcbb{max size}{int}{128}{MLParameterListInterpreter, ML2MueLuParameterTranslator}{Maximum dimension of a coarse grid. \ml will stop coarsening once it is achieved.}


\mlcbb{coarse: type}{string}{"Amesos-KLU"}{MLParameterListInterpreter, ML2MueLuParameterTranslator}{Solver for coarsest level. Possible values: "Amesos-KLU", "Amesos-Superlu" (depending on \muelu installation).}


\subsection{Transfer operator options}

\mlcbb{energy minimization: enable}{int}{0}{MLParameterListInterpreter, ML2MueLuParameterTranslator}{Enable energy minimization transfer operators (using Petrov-Galerkin approach).}
          
\mlcbb{aggregation: damping factor}{double}{1.33}{MLParameterListInterpreter, ML2MueLuParameterTranslator}{Damping factor for smoothed aggregation.}
          

\subsection{Rebalancing options}

\mlcbb{repartition: enable}{int}{0}{MLParameterListInterpreter}{Rebalancing on/off switch. Only limited support for repartitioning. Does not use provided node coordinates.}
          
\mlcbb{repartition: start level}{int}{1}{MLParameterListInterpreter}{Minimum level to run partitioner. \muelu does not rebalance levels finer than this one.}
          
\mlcbb{repartition: min per proc}{int}{512}{MLParameterListInterpreter}{Minimum number of rows per processor. If the actual number if smaller, then rebalancing occurs.}
          
\mlcbb{repartition: max min ratio}{double}{1.3}{MLParameterListInterpreter}{Maximum nonzero imbalance ratio. If the actual number is larger, the rebalancing occurs.}
          



    %\chapter{Some Other Appendix}
	%    Just to show what a second Appendix would look like. It contains
    a table. Each appendix is supposed to be self-contained, so
    tables and figures are not supposed to show up in the main
    table of contents. There can be a separate table of contents
    for each appendix.

    \begin{table}[ht]
	\centering
	\caption{A small table}
	\bigskip

	\begin{tabular}{|c|c|}
	    \hline
		A & B  \\ \hline
		C & D  \\ \hline
	\end{tabular}
	\label{tab3}
    \end{table}

    \begin{figure}[ht]
	\centering
	\begin{picture}(50,50)(0,0)
	    \put(25,25){\circle{1}}
	    \put(25,25){\circle{5}}
	    \put(25,25){\circle{10}}
	    \put(25,25){\circle{15}}
	    \put(25,25){\circle{20}}
	    \put(25,25){\circle{25}}
	    \put(25,25){\circle{30}}
	    \put(25,25){\circle{35}}
	    \put(25,25){\circle{40}}
	    \put(25,25){\circle{45}}
	    \put(25,25){\circle{50}}
	\end{picture}
	\caption{Dizzy yet?}
	\label{fig4}
    \end{figure}


    % \printindex

    %
% This is an example of how to create the distribution page. Some
% distributions are required by Sandia; e.g. the housekeeping copies.
% Depending on the type of report; e.g. CRADA, Patent Caution, etc.
% additional distribution lines may have to be added. See the
% "Guide for Preparing SAND Reports"
%

{\bfseries DISTRIBUTION}

{\bfseries Email--External (encrypt for OUO)}\\[1ex]
\begin{tabular}{|p{4cm}|p{5cm}|p{4cm}|}
  \hline
  \multicolumn{1}{|>{\centering\arraybackslash}p{4cm}|}{\cellcolor{SANDgreen}\bfseries Name} & \multicolumn{1}{>{\centering\arraybackslash}p{5cm}|}{\cellcolor{SANDgreen}\bfseries Company Email Address} & \multicolumn{1}{>{\centering\arraybackslash}p{4cm}|}{\cellcolor{SANDgreen}\bfseries Company Name}\\
  \hline
  Matthias Mayr & matthias.mayr@unibw.de & University of the Bundeswehr Munich\\
  \hline
  Andrey Prokopenko & prokopenkoav@ornl.gov & Oak Ridge National Laboratory\\
  \hline
  Tobias Wiesner & tobias.wiesner@leica-geosystems.com & Leica Geosystems AG\\
  \hline
\end{tabular}

{\bfseries Email--Internal}\\[1ex]
\begin{tabular}{|p{5cm}|p{3cm}|p{5cm}|}
  \hline
  \multicolumn{1}{|c}{\cellcolor{SANDgreen}\bfseries Name} & \multicolumn{1}{|c}{\cellcolor{SANDgreen}\bfseries Org.} & \multicolumn{1}{|c|}{\cellcolor{SANDgreen}\bfseries Sandia Email Address}\\
  \hline
  Luc Berger-Vergiat & 1442 & lberge@sandia.gov\\
  \hline
  Christopher Siefert & 1465 & csiefer@sandia.gov\\
  \hline
  Christian Glusa & 1465 & caglusa@sandia.gov\\
  \hline
  Mark Hoemmen & 1541 & mhoemme@sandia.gov\\
  \hline
  Jonathan Hu & 1465 & jhu@sandia.gov\\
  \hline
  Paul Lin & 1422 & ptlin@sandia.gov\\
  \hline
  Ray Tuminaro & 1442 & rstumin@sandia.gov\\
  \hline
  Technical Library & 9536 & libref@sandia.gov\\
  \hline
\end{tabular}


\end{document}
