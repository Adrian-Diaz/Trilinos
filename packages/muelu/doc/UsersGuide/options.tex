\label{sec:muelu_options}

In this section, we report the complete list of \muelu{} input parameters.  It
is important to notice, however, that \muelu{} relies on other \trilinos{}
packages to provide support for some of its algorithms. For instance,
\ifpack{}/\ifpacktwo{} provide standard smoothers like Jacobi, Gauss-Seidel or
Chebyshev, while \amesos{}/\amesostwo{} provide access to direct solvers. The
parameters affecting the behavior of the algorithms in those packages are
simply passed by \muelu{} to a routine from the corresponding library. Please
consult corresponding packages' documentation for a full list of supported
algorithms and corresponding parameters.

\section{Using parameters on individual levels}
Some of the parameters that affect the preconditioner can in principle be
different from level to level. By default, parameters affect all levels in
a multigrid hierarchy.

The settings on a particular level can be changed by using level sublists.
A level sublist is a \parameterlist{} sublist with the name ``level XX'', where XX is the level number. The
parameter names in the sublist do not require any modifications. For example,
the following fragment of code
\begin{lstlisting}[language=XML]
  <ParameterList name="level 2">
    <Parameter name="smoother: type" type="string" value="CHEBYSHEV"/>
  </ParameterList>
\end{lstlisting}
changes the smoother for level 2 to be a Chebyshev-type polynomial smoother.

\section{Parameter validation}
By default, \muelu{} validates the input parameter list. A parameter that is
misspelled is unknown. A parameter with an incorrect value type is also treated as invalid.
Both cases will cause an exception to be
thrown and execution to halt.

\begin{mycomment}
Spaces are important within a parameter's name. Please separate words
by just one space, and make sure there are no leading or trailing spaces.
\end{mycomment}

The option \verb|print initial parameters| prints the initial list given to the
interpreter. The option \verb|print unused parameters| prints the list of unused
parameters.

% ==================== GENERAL ====================
\section{General options}
\label{sec:options_general}

\begin{table}[h!]
  \begin{center}
    \begin{tabular}{p{3cm} p{12cm}}
      \toprule
      Verbosity level           & Description \\
      \midrule
      \verb!none!               & No output \\
      \verb!low!                & Errors, important warnings, and some statistics \\
      \verb!medium!             & Same as \verb!low!, but with more statistics \\
      \verb!high!               & Errors, all warnings, and all statistics \\
      \verb!extreme!            & Same as \verb!high!, but also includes output from other packages (\textit{i.e.}, \zoltan{}) \\
      \bottomrule
    \end{tabular}
    \caption{Verbosity levels.}
\label{t:verbosity_types}
  \end{center}
\end{table}

\begin{table}[h!]
  \begin{center}
    \begin{tabular}{p{4.3cm} p{4.3cm} c p{4.5cm}}
      \toprule
      Problem type                 & Multigrid algorithm    & Block size  & Smoother \\
      \midrule
      \verb!unknown!               & --                     & --          & -- \\
      \verb!Poisson-2D!            & Smoothed aggregation   & 1           & Chebyshev \\
      \verb!Poisson-3D!            & Smoothed aggregation   & 1           & Chebyshev \\
      \verb!Elasticity-2D!         & Smoothed aggregation   & 2           & Chebyshev \\
      \verb!Elasticity-3D!         & Smoothed aggregation   & 3           & Chebyshev \\
      \verb!Poisson-2D-complex!    & Smoothed aggregation   & 1           & Symmetric Gauss-Seidel \\
      \verb!Poisson-3D-complex!    & Smoothed aggregation   & 1           & Symmetric Gauss-Seidel \\
      \verb!Elasticity-2D-complex! & Smoothed aggregation   & 2           & Symmetric Gauss-Seidel \\
      \verb!Elasticity-3D-complex! & Smoothed aggregation   & 3           & Symmetric Gauss-Seidel \\
      \verb!ConvectionDiffusion!   & Petrov-Galerkin  AMG   & 1           & Gauss-Seidel \\
      \verb!MHD!                   & Unsmoothed aggregation & --          & Additive Schwarz method with one level of overlap and ILU(0) as a subdomain solver \\
      \bottomrule
    \end{tabular}
    \caption{Supported problem types (``--'' means not set).}
\label{t:problem_types}
  \end{center}
\end{table}


\cbb{problem: type}{string}{"unknown"}{Type of problem to be solved. Possible values: see Table~\ref{t:problem_types}.}
          
\cbb{verbosity}{string}{"high"}{Control of the amount of printed information. Possible values: see Table~\ref{t:verbosity_types}.}
          
\cbb{number of equations}{int}{1}{Number of PDE equations at each grid node. Only constant block size is considered.}
          
\cbb{max levels}{int}{10}{Maximum number of levels in a hierarchy.}
          
\cbb{cycle type}{string}{"V"}{Multigrid cycle type. Possible values: "V", "W".}
          
\cbb{problem: symmetric}{bool}{true}{Symmetry of a problem. This setting affects the construction of a restrictor. If set to true, the restrictor is set to be the transpose of a prolongator. If set to false, underlying multigrid algorithm makes the decision.}
          

% ==================== SMOOTHERS ====================
\section{Smoothing and coarse solver options}
\label{sec:options_smoothing}

\muelu{} relies on other \trilinos{} packages to provide level smoothers and
coarse solvers. \ifpack{} and \ifpacktwo{} provide standard smoothers (see
Table~\ref{tab:smoothers}), and \amesos{} and \amesostwo{} provide direct
solvers (see Table~\ref{tab:coarse_solvers}). Note that it is completely possible to use
any level smoother as a direct solver.

\muelu{} checks parameters \verb|smoother: * type| and \verb|coarse: type| to
determine:
\begin{itemize}
  \item what package to use (i.e., is it a smoother or a direct solver);
  \item (possibly) transform a smoother type

    \ding{42} \ifpack{} and \ifpacktwo{} use different smoother type names,
    e.g., ``point relaxation stand-alone'' vs ``RELAXATION''.  \muelu{} tries to follow
    \ifpacktwo{} notation for smoother types. Please consult \ifpacktwo{}
    documentation~\cite{Ifpack2} for more information.
\end{itemize}
The parameter lists \verb|smoother: * params| and \verb|coarse: params| are
passed directly to the corresponding package without any examination of their
content. Please consult the documentation of the corresponding packages for a list of
possible values.

By default, \muelu{} uses one sweep of symmetric Gauss-Seidel for both pre- and
post-smoothing, and SuperLU for coarse system solver.

\begin{table}[tbh]
  \begin{center}
    \begin{tabular}{p{4.0cm} p{10cm}}
      \toprule
      \texttt{smoother: type}           & \\
      \midrule
      \verb|RELAXATION|                 & Point relaxation smoothers, including
                                          Jacobi, Gauss-Seidel, symmetric Gauss-Seidel,
                                          multithreaded (coloring-based) Gauss-Seidel, etc. The exact
                                          smoother is chosen by specifying \texttt{relaxation: type} parameter in
                                          the \texttt{smoother: params} sublist. \\
      \verb|CHEBYSHEV|                  & Chebyshev polynomial smoother. \\
      \verb|ILUT|, \verb|RILUK|         & Local (processor-based) incomplete factorization methods. \\
      \bottomrule
    \end{tabular}
    \caption{Commonly used smoothers provided by \ifpack{}/\ifpacktwo{}. Note
    that these smoothers can also be used as coarse grid solvers.}
\label{tab:smoothers}
  \end{center}
\end{table}

\begin{table}[tbh]
  \begin{center}
    \begin{tabular}{p{4.0cm} c c p{7cm}}
      \toprule
      \texttt{coarse: type}             & \amesos{} & \amesostwo{} &  \\
      \midrule
      \verb|KLU|                        & x & & Default \amesos{} solver~\cite{klu}. \\
      \verb|KLU2|                       & & x & Default \amesostwo{} solver~\cite{amesos2_belos}. \\
      \verb|SuperLU|                    & x & x & Third-party serial sparse direct solver~\cite{Li2011}. \\
      \verb|SuperLU_dist|               & x & x & Third-party parallel sparse direct solver~\cite{Li2011}. \\
      \verb|Umfpack|                    & x & & Third-party solver~\cite{umfpack}. \\
      \verb|Mumps|                      & x & & Third-party solver~\cite{mumps}. \\
      \bottomrule
    \end{tabular}
    \caption{Commonly used direct solvers provided by \amesos{}/\amesostwo{}}
\label{tab:coarse_solvers}
  \end{center}
\end{table}

In certain cases, the user may want to do no smoothing on a particular level, or do no solve on the coarsest level.
\begin{itemize}
  \item To skip smoothing, use the option \verb!smoother: pre or post! with value \verb!none!.
  \item To skip the coarse grid solve, use the option \verb!coarse: type! with value \verb!none!.
\end{itemize}

When a problem can be solved using structured aggregation algorithms it is also possible to use the structured line detection factory,
 this will allow \muelu{} to pass additional information to \ifpack2{} enabling it to perform line smoothing.
An example of line smoothing is provided in \texttt{packages/trilinoscouplings/examples/scaling/muelu\_Ifpack2\_line\_detection.xml}.


\cbb{smoother: pre or post}{string}{"both"}{Pre- and post-smoother combination. Possible values: "pre" (only pre-smoother), "post" (only post-smoother), "both" (both pre-and post-smoothers), "none" (no smoothing).}

\cba{smoother: type}{string}{Smoother type. Possible values: see Table~\ref{tab:smoothers}.}

\cba{smoother: pre type}{string}{Pre-smoother type. Possible values: see Table~\ref{tab:smoothers}.}

\cba{smoother: post type}{string}{Post-smoother type. Possible values: see Table~\ref{tab:smoothers}.}

\cba{smoother: params}{\parameterlist}{Smoother parameters. For standard smoothers, \muelu passes them directly to the appropriate package library.}

\cba{smoother: pre params}{\parameterlist}{Pre-smoother parameters. For standard smoothers, \muelu passes them directly to the appropriate package library.}

\cba{smoother: post params}{\parameterlist}{Post-smoother parameters. For standard smoothers, \muelu passes them directly to the appropriate package library.}

\cbb{smoother: overlap}{int}{0}{Smoother subdomain overlap.}

\cbb{smoother: pre overlap}{int}{0}{Pre-smoother subdomain overlap.}

\cbb{smoother: post overlap}{int}{0}{Post-smoother subdomain overlap.}

\cbb{coarse: max size}{int}{2000}{Maximum dimension of a coarse grid. \muelu will stop coarsening once it is achieved.}

\cbb{coarse: type}{string}{"SuperLU"}{Coarse solver. Possible values: see Table~\ref{tab:coarse}.}

\cba{coarse: params}{\parameterlist}{Coarse solver parameters. \muelu passes them directly to the appropriate package library.}

\cbb{coarse: overlap}{int}{0}{Coarse solver subdomain overlap.}


% ==================== AGGREGATION ====================
\section{Aggregation options}
\label{sec:options_aggregation}

\begin{table}[H]
  \begin{center}
    \begin{tabular}{p{5.0cm} p{10cm}}
      \toprule
      \verb!structured!   & Attempts to construct hexahedral aggregates on a structured
                            mesh using a default coarsening rate of $3$ in each spatial
                            dimension.\\
      \verb!hybrid!       & This option takes in a user parameter that varies on each
                            rank and that specifies whether the local aggregation
                            scheme should be \verb!structured! or \verb!unstructured!.\\
      \verb!uncoupled!    & Attempts to construct aggregates of optimal size ($3^d$
                            nodes in $d$ dimensions). Each process works independently, and
                            aggregates cannot span multiple processes.\\
      \verb!coupled!      & Attempts to construct aggregates of optimal size ($3^d$
                            nodes in $d$ dimensions). Aggregates are allowed to
                            cross processor boundaries. \textbf{Use carefully}. If
                            unsure, use \verb!uncoupled! instead.\\
      \verb!brick!        & Attempts to construct rectangular aggregates \\
      %\verb!METIS!     & Use graph partitioning algorithm to create aggregates,
      %                   working process-wise. Number of nodes in each aggregate
      %                   is specified with option \texttt{aggregation: max agg
      %                   size}. \\
      % \verb!ParMETIS!  & As \verb!METIS!, but partition global graph. Aggregates
                         % can span arbitrary number of processes. Specify global
                         % number of aggregates with {\tt aggregation: global
                         % number}. \\
      \bottomrule
    \end{tabular}
    \caption{Available coarsening schemes. }
\label{t:aggregation}
  \end{center}
\end{table}


\cbb{aggregation: type}{string}{"uncoupled"}{Aggregation scheme. Possible values: see Table~\ref{t:aggregation}.}
          
\cbb{aggregation: ordering}{string}{"natural"}{Node ordering strategy. Possible values: "natural" (local index order), "graph" (filtered graph breadth-first order), "random" (random local index order).}
          
\cbb{aggregation: drop scheme}{string}{"classical"}{Connectivity dropping scheme for a graph used in aggregation. Possible values: "classical", "distance laplacian".}
          
\cbb{aggregation: drop tol}{double}{0.0}{Connectivity dropping threshold for a graph used in aggregation.}
          
\cbb{aggregation: min agg size}{int}{2}{Minimum size of an aggregate.}
          
\cbb{aggregation: max agg size}{int}{-1}{Maximum size of an aggregate (-1 means unlimited).}
          
\cbb{aggregation: brick x size}{int}{2}{Number of points for x axis in "brick" aggregation (limited to 3).}
          
\cbb{aggregation: brick y size}{int}{2}{Number of points for y axis in "brick" aggregation (limited to 3).}
          
\cbb{aggregation: brick z size}{int}{2}{Number of points for z axis in "brick" aggregation (limited to 3).}
          
\cbb{aggregation: Dirichlet threshold}{double}{0.0}{Threshold for determining whether entries are zero during Dirichlet row detection.}
          
\cbb{aggregation: export visualization data}{bool}{false}{Export data for visualization post-processing.}
          
\cbb{aggregation: output filename}{string}{""}{Filename to write VTK visualization data to.}
          
\cbb{aggregation: output file: time step}{int}{0}{Time step ID for non-linear problems.}
          
\cbb{aggregation: output file: iter}{int}{0}{Iteration for non-linear problems.}
          
\cbb{aggregation: output file: agg style}{string}{Point Cloud}{Style of aggregate visualization.}
          
<<<<<<< HEAD
\cbb{aggregation: output file: graph edges}{bool}{false}{Whether to draw all node connections along with the aggregates.}
=======
\cbb{aggregation: output file: fine graph edges}{bool}{false}{Whether to draw all fine node connections along with the aggregates.}
          
\cbb{aggregation: output file: coarse graph edges}{bool}{false}{Whether to draw all coarse node connections along with the aggregates.}
          
\cbb{aggregation: output file: build colormap}{bool}{false}{Whether to output a random colormap in a separate XML file.}
>>>>>>> MueLu: Viz and matlab tests
          

% ==================== REBALANCING ====================
\section{Rebalancing options}
\label{sec:options_rebalancing}


\cbb{repartition: enable}{bool}{false}{Rebalancing on/off switch.}
          
\cbb{repartition: partitioner}{string}{"zoltan2"}{Partitioning package to use. Possible values: "zoltan" (\zoltan{} library), "zoltan2" (\zoltantwo{} library).}
          
\cba{repartition: params}{\parameterlist}{Partitioner parameters. \muelu passes them directly to the appropriate package library.}
          
\cbb{repartition: start level}{int}{2}{Minimum level to run partitioner. \muelu does not rebalance levels finer than this one.}
          
\cbb{repartition: min rows per proc}{int}{800}{Minimum number of rows per MPI process. If the actual number if smaller, then rebalancing occurs.}
          
\cbb{repartition: target rows per proc}{int}{0}{Target number of rows per MPI process after rebalancing. If the value is set to 0, it will use the value of "repartition: min rows per proc"}
          
\cbb{repartition: max imbalance}{double}{1.2}{Maximum nonzero imbalance ratio. If the actual number is larger, the rebalancing occurs.}
          
\cbb{repartition: remap parts}{bool}{true}{Postprocessing for partitioning to reduce data migration.}
          
\cbb{repartition: rebalance P and R}{bool}{false}{Explicit rebalancing of R and P during the setup. This speeds up the solve, but slows down the setup phases.}
          

% ==================== MULTIGRID ====================
\section{Multigrid algorithm options}
\label{sec:options_mg}

\begin{table}[H]
  \begin{center}
    \begin{tabular}{p{3.5cm} p{11cm}}
      \toprule
      \verb!sa!          & Classic smoothed aggregation~\cite{VMB1996} \\
      \verb!unsmoothed!  & Aggregation-based, same as \verb!sa! but without damped Jacobi prolongator improvement step \\
      \verb!pg!          & Prolongator smoothing using $A$, restriction smoothing using $A^T$, local damping factors~\cite{ST2008} \\
      \verb!emin!        & Constrained minimization of energy in basis functions of grid transfer operator~\cite{WTWG2014,OST2011} \\
      \verb!interp!      & Interpolation based grid transfer operator, using piece-wise constant or linear interpolation from coarse nodes to fine nodes. This requires the use of structured aggregation.\\
      \verb!semicoarsen! & Semicoarsening grid transfer operator used to reduce a n-dimensional problem into a (n-1)-dimensional problem by coarsening fully in one of the spacial dimensions~\cite{TPSTP2015}.\\
      \verb!pcoarsen!    & \\
      \bottomrule
    \end{tabular}
    \caption{Available multigrid algorithms for generating grid transfer matrices. }
\label{t:mgs}
  \end{center}
\end{table}


\cbb{multigrid algorithm}{string}{"sa"}{Multigrid method. Possible values: see Table~\ref{t:mgs}.}
          
\cbb{semicoarsen: coarsen rate}{int}{3}{Rate at which to coarsen unknowns in the z direction.}
          
\cbb{sa: damping factor}{double}{1.33}{Damping factor for smoothed aggregation.}
          
\cbb{sa: use filtered matrix}{bool}{true}{Matrix to use for smoothing the tentative prolongator. The two options are: to use the original matrix, and to use the filtered matrix with filtering based on filtered graph used for aggregation.}
          
\cbb{interp: interpolation order}{int}{1}{Interpolation order used to interpolate values from coarse points to fine points. Possible values are 0 for piece-wise constant interpolation and 1 for piece-wise linear interpolation. This parameter is set to 1 by default.}
          
\cbb{interp: build coarse coordinates}{bool}{true}{If false, skip the calculation of coarse coordinates.}
          
\cbb{filtered matrix: use lumping}{bool}{true}{Lump (add to diagonal) dropped entries during the construction of a filtered matrix. This allows user to preserve constant nullspace.}
          
\cbb{filtered matrix: use root stencil}{bool}{false}{Use root-node based sparsification of the filtered
      matrix. This usually reduces operator complexity in the case of small aggregates.}
          
\cbb{filtered matrix: Dirichlet threshold}{double}{-1.0}{Dirichlet threshold to use for detecting zero
      diagonals in the filtered matrix (which get replaced with a one).  Any negative number disables
      the thresholding.}
          
\cbb{filtered matrix: reuse eigenvalue}{bool}{true}{Skip eigenvalue calculation during the construction of a filtered matrix by reusing eigenvalue estimate from the original matrix. This allows us to skip heavy computation, but may lead to poorer convergence.}
          
\cbb{emin: iterative method}{string}{"cg"}{Iterative method to use for energy minimization of initial prolongator in energy-minimization. Possible values: "cg" (conjugate gradient), "gmres" (generalized minimum residual), "sd" (steepest descent).}
          
\cbb{emin: num iterations}{int}{2}{Number of iterations to minimize initial prolongator energy in energy-minimization.}
          
\cbb{emin: num reuse iterations}{int}{1}{Number of iterations to minimize the reused prolongator energy in energy-minimization.}
          
\cbb{emin: pattern}{string}{"AkPtent"}{Sparsity pattern to use for energy minimization. Possible values: "AkPtent".}
          
\cbb{emin: pattern order}{int}{1}{Matrix order for the "AkPtent" pattern.}
          
\cbb{emin: use filtered matrix}{bool}{true}{Matrix to use for smoothing for energy minimization. The two options are: to use the original matrix, and to use the filtered matrix with filtering based on filtered graph used for aggregation.}
          

% ==================== REUSE ====================
\section{Reuse options}
\label{sec:options_reuse}

Reuse options are a way for a user to speed up the setup stage of multigrid.
The main requirement to use reuse is that the matrix' graph structure does not
change. Only matrix values are allowed to change.

The reuse options control the degree to which the multigrid hierarchy is preserved
for a subsequent setup call.

In addition, please note that not all combinations of multigrid algorithms and
reuse options are valid, or even make sense. For instance, the "emin" reuse
option should only be used with the "emin" multigrid algorithm.

Table~\ref{t:reuse_types} contains the information about different reuse
options. The options are ordered in increasing number of reuse components, from
the no reuse to the full reuse ("full").

\begin{table}[H]
  \begin{center}
    \begin{tabular}{p{3.0cm} p{12cm}}
      \toprule
      \verb!none!   & No reuse \\
      \verb!S!      & Reuse only the symbolic information of the level smoothers. \\
      \verb!tP!     & Reuse tentative prolongator. The graphs of smoothed
                      prolongator and matrices in Galerkin product are reused
                      only if filtering is not being used ({\it i.e.}, either
                      \verb!sa: use filtered matrix! or \verb!aggregation: drop tol! is false) \\
      \verb!emin!   & Reuse old prolongator as an initial guess to energy
                      minimization, and reuse the prolongator pattern \\
      \verb!RP!     & Reuse smoothed prolongator and restrictor. Smoothers are
                      recomputed.  \ding{42} \verb!RP! should reuse matrix graphs for
                      matrix-matrix product, but currently that is disabled as only \epetra{}
                      supports it. \\
      \verb!RAP!    & Recompute only the finest level smoothers, reuse all other operators \\
      \verb!full!   & Reuse everything \\
      \bottomrule
    \end{tabular}
    \caption{Available reuse options.}
\label{t:reuse_types}
  \end{center}
\end{table}


\cbb{reuse: type}{string}{"none"}{Reuse options for consecutive hierarchy construction. This speeds up the setup phase, but may lead to poorer convergence. Possible values: see Table~\ref{t:reuse_types}.}
          

% ==================== MISCELLANEOUS ====================
\section{Miscellaneous options}


\cba{export data}{\parameterlist}{Writes a subset of the hierarchy data to files.
      Valid \texttt{string} parameters are 
      ``A'', ``P'', ``R'', ``Nullspace'', and ``Coordinates''.
      The string values are comma separated nonnegative integers within curly braces, e.g.,
      ``\{0,1,3\}''.  Valid \texttt{int} options are \texttt{frequency} and \texttt{total}.
      Setting \texttt{frequency} to a positive integer value \texttt{n} means data is written
      to file on every \texttt{n}th Hierarchy construction.  If \texttt{total} is nonnegative,
      the data will be written at most \texttt{total} times.  If \texttt{total} is negative,
      there is no limit to the number of times the data is written.
      The only valid \texttt{bool} option is \texttt{export first system}.  If true, then writing
      will begin with the first linear system.
      A matrix/multivector with a name ``X'' on level ``L'' is saved in two or three
      three MatrixMarket files: a) data is saved in
      \textit{X\_L.mm}; b) its row map is saved in
      \textit{rowmap\_X\_L.mm}; c) its column map (for matrices) is saved in
      \textit{colmap\_X\_L.mm}.}
          
\cbb{print initial parameters}{bool}{true}{Print parameters provided for a hierarchy construction.}
          
\cbb{print unused parameters}{bool}{true}{Print parameters unused during a hierarchy construction.}
          
\cbb{transpose: use implicit}{bool}{false}{Use implicit transpose for the restriction operator.}
          


% ==================== MAXWELL ====================
\section{Maxwell solver options}


\cbb{refmaxwell: mode}{string}{"additive"}{Specifying the order of solve of the block system. Allowed values are: "additive" (default), "121", "212", "1", "2"}
          
\cbb{refmaxwell: disable addon}{bool}{true}{Specifing whether the addon should be built for stabilization}
          
\cba{refmaxwell: 11list}{\parameterlist}{Specifies the multigrid solver for the 11 block}
          
\cba{refmaxwell: 22list}{\parameterlist}{Specifies the multigrid solver for the 22 block}
          
\cbb{refmaxwell: use as preconditioner}{bool}{false}{Assume zero initial guess}
          
\cbb{refmaxwell: dump matrices}{bool}{false}{Dump matrices to disk.}
          
\cbb{refmaxwell: subsolves on subcommunicators}{bool}{false}{Redistribute the two subsolves to disjoint sub-communicators (so that the additive solve can occur in parallel).}
          
\cbb{refmaxwell: ratio AH / A22 subcommunicators}{double}{1.0}{Ratio for the split into sub-communicators.}
          

%%% Local Variables:
%%% mode: latex
%%% TeX-master: "mueluguide"
%%% End:
