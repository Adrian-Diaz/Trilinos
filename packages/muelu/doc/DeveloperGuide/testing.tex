\muelu supports both unit and broader testing.  Unit testing relies on the \teuchos unit-testing infrastructure\footnote{As of
October, 2010, there
doesn't seem to be much documentation for this.  Your best bet is to look at a package already using it, such as \ifpack2.}.
Implementing new unit tests is straightforward.
\be
  \item Copy the file \cc{test/unit\_tests/Level.cpp} to a new filename of your choosing, say, \cc{foo.cpp}.
  \item Edit \cc{foo.cpp} and put in your test.
  \item Add \cc{foo.cpp} to the list of \cc{SOURCES} in file \cc{test/unit\_tests/CMakeLists.txt}.
  \item Rebuild, run the unit test executable, and you should see the new test run.
\ee

To see the list of available options, type \cc{MueLu\_unit\_tests.exe --help}.  Two of the more useful options
are \cc{--group} and \cc{--test}.  The former lets you run just one group of tests, e.g., just the
\cc{MueLu::Level} tests.
%
\begin{verbatim}
  MueLu_unit_tests.exe --group="Level"
\end{verbatim}
%
The latter lets you run a single test.
\begin{verbatim}
  MueLu_unit_tests.exe --test="FillHierarchy1"
\end{verbatim}
