\documentclass{article}[12pt]
\usepackage{fancyhdr, fancybox, tabularx, verbatim, epsfig}
%\usepackage{fancyhdr, graphicx, fancybox, wrapfig, epic, ecltree, tabularx,
%  verbatim, alltt, ifthen, boxedminipage, epsfig}

\includeonly{unadvertised}

%\setlength{\oddsidemargin}{0.4\oddsidemargin}
%\setlength{\evensidemargin}{0.4\evensidemargin}
%\setlength{\topmargin}{0.0\topmargin}
%\setlength{\textheight}{1.16\textheight}
%\setlength{\textwidth}{1.29\textwidth}
\setlength{\oddsidemargin}{0.3\oddsidemargin}
\setlength{\evensidemargin}{0.3\evensidemargin}
\setlength{\topmargin}{0.0\topmargin}
\setlength{\textheight}{1.16\textheight}
\setlength{\textwidth}{1.39\textwidth}

%
% macros for formatting symbols for reals, integers, etc.
%

\newcommand{\Az}  {{\bf Aztec}}
\newcommand\R     {{\rm \bf R}}
\newcommand\I     {{\rm \bf I}}
\newcommand\C     {{\rm \bf C}}

%
% define boxes for describing variables, etc
%

\def\optionbox#1#2{\noindent$\hphantom{hix}${\parbox[t]{2.10in}{\it
#1}}{\parbox[t]{3.9in}{#2}} \\[1.1em]}

\def\choicebox#1#2{\noindent$\hphantom{hixthere}$\parbox[t]{2.10in}{\sf
#1}\parbox[t]{3.5in}{#2}\\[0.8em]}

\def\structbox#1#2{\noindent$\hphantom{hix}${\parbox[t]{2.10in}{\it
#1}}{\parbox[t]{3.9in}{#2}} \\[.02cm]}

\def\in{\hskip .2in \=}
\def\hsp{\hskip .4in \=}
\def\sp{\hskip .18in \=}
\def\sh{\hskip .18in }
\def\bb{\hskip .034in }
\def\lil{\hskip .1in }


\def\protobox#1{\vspace{2em}{\flushleft{\bf Prototype}
\hrulefill}\flushleft{\fbox{\parbox[t]{6in}{\vspace{1em}{\sf
#1}\vspace{1em}}}}}

%--- redefine this for the tabularx environment

\newcolumntype{Y}{>{\raggedright\arraybackslash}X}
\newcolumntype{Z}{>{\small\raggedright\arraybackslash}X}

%
% ***********************************************************************
% * 02 July 1993: McCorkle                                              *
% * Define a macro that will lightly print the word `DRAFT' diagonally  *
% * across each page of the document. This macro was obtained from the  *
% * NMSU math department                                                *
% *                                                                     *
% * Usage: \draft                                                       *
% ***********************************************************************
%
\def\draft{%
\special{!userdict begin /bop-hook{gsave
200 30 translate 65 rotate
/Times-Roman findfont 216 scalefont setfont
0 0 moveto 0.9 setgray (DRAFT) show grestore}def end}
}

\renewcommand\baselinestretch{0.9}

\begin{document}

\large
\pagenumbering{roman}

%\draft                                   % Lightly print `DRAFT' on every
                                         % page of the document

%
% Stuff for SAND report style from B.A. Hendrickson, SNL, 1422
%
\hspace{2.22in}
SAND--internal document
\hfill
Distribution

\hspace{2.07in}
Unlimited Release
\hfill
Category UC--405
\begin{center}
Printed Nov 1999
\end{center}

\vspace{0.8in}

\begin{center}
%{\Large{\bf \Az{} User's Guide$^\ast$ \\ Version 2.1}}
  {\Large{\bf Sandia Only \Az{} User's Addendum\footnote{This work was supported by 
	the
        Applied Mathematical Sciences program, U.S. Department of Energy,
        Office of Energy Research, and was performed at Sandia National
        Laboratories, operated for the U.S. Department of Energy under contract
        No. DE-AC04-94AL85000. The \Az{} software package was developed by the
        authors at Sandia National Laboratories and is under copyright
        protection} \\ Version 2.1}}

\vspace*{0.4in}

Ray S. Tuminaro\footnote{Applied \& Numerical Mathematics Department;
  tuminaro@cs.sandia.gov; (925) 294-2564}
\,\,\,\,Mike Heroux\footnote{Applied \& Numerical Mathematics Department;
  maherou@cs.sandia.gov; (320) 845-7695}
\,\,\,\,Scott A. Hutchinson\footnote{Parallel Computational Sciences Department;
  sahutch@cs.sandia.gov; (505) 845-7996}
\,\,\,\, John N. Shadid\footnote{Parallel Computational Sciences Department;
  jnshadi@cs.sandia.gov; (505) 845-7876}
\\

Massively Parallel Computing Research Laboratory \\
Sandia National Laboratories \\
Albuquerque, NM \, 87185

\vspace*{.9in}

\end{center}

{\centering\large Abstract \\[1em]}

This manual describes unadvertised features of \Az{} that are only 
available to Sandia users. Please see the \Az{} User's Guide for 
a more complete description of \Az{}.

\vfill

\newpage

%--- heading stuff

\pagestyle{fancyplain}
\addtolength{\headwidth}{\marginparsep}
%\addtolength{\headwidth}{\marginparwidth}
%--- remember section title
\renewcommand{\sectionmark}[1]{\markboth{#1}{}}
%\renewcommand{\subsectionmark}[1]{\markright{\thesubsection\ #1}}
%\lhead[\fancyplain{}{\bfseries\thepage}]%
%      {\fancyplain{}{\bfseries\rightmark}}
\rhead[\fancyplain{}{\bfseries\leftmark}]%
      {\fancyplain{}{\bfseries\thepage}}
\cfoot{}
%\rfoot{\leftmark\\\rightmark}

\large

\newpage

{\flushleft {\bf Notation Conventions} \hrulefill
\\[0.5em]

Different fonts are used to indicate program fragments, keys words, variables,
or parameters in order to clarify the presentation.  The table below describes
the meaning denoted by these different fonts.\\[3em]

\begin{tabularx}{\textwidth}{lX} \hline \\
{\bf Convention} & {\bf Meaning} \\[1.25em]

\tt typewriter & File names, code examples and code
fragments. \\

{\sf sans serif} & C language elements such as function names and constants
when they appear embedded in text or in function definition syntax
lines. \\

{\it italics\/} & Parameter and variable names when they appear embedded in
text or function definition syntax lines. \\

{\bf AZ\_ } & C language elements such as function names and constants which
are supplied by the \Az{} library. \\[1em] \hline
\end{tabularx}
%
\vskip 3.0em
\flushleft {\bf Code Distribution} \hrulefill
\\[0.5em]

\Az{} is publicly available for research purposes and may be licensed for
commercial application.  The code is distributed along with technical
documentation, example C and Fortran driver routines and sample input files via
the internet.  It may be obtained by contacting one of the authors listed on
page i of this report or from the \Az{} web site at
\tt http://www.cs.sandia.gov/CRF/aztec1.html.}

\newpage
\pagenumbering{arabic}

\section{{\protect \bf Aztec}: Unadvertised use of Aztec Options\label{unadvertised options}}

The integer array {\it options\/} of length {\sf AZ\_OPTIONS\_SIZE} is set by 
the user. It is used (but not altered) by the function {\sf AZ\_solve} and 
{\sf AZ\_iterate} to choose between iterative solvers, preconditioners, etc.  
Default values for this array (as well as for {\it params}) are set by invoking 
{\sf AZ\_defaults()}.
Below we discuss some options not described in the standard \Az{} manual. 

\vspace{2em}
{\flushleft{\bf Specifications} \hrulefill}
\nopagebreak \\[0.5em]
%
\optionbox{options[{\sf AZ\_solver}]}{Specifies solution
  algorithm. DEFAULT: \sf AZ\_gmres.}
\choicebox{AZ\_GMRESR}{A recursive GMRES algorithm similar to flexible GMRES
                       in that it allows preconditioners which change at 
                       each iteration to be used. Note: recursive GMRES requires
                       approximately twice the storage as GMRES.} 
\choicebox{AZ\_fixed\_pt}{This is a fixed point iterative solver. For the most
                         part, this solver is used for debugging purposes.
                         It occasionally might have some use when a very good
                         preconditioner is supplied (e.g. multigrid). It can
                         also be used to perform one \Az{} preconditioning
                         iteration (set {\it options[ {\sf AZ\_max\_iter} ]} = 1
                         and {\it options[{\sf AZ\_init\_guess}]} = 
                         {\sf AZ\_ZERO}).  }
                       
\choicebox{AZ\_symmlq}{Currently not compiled. This is a SYMMLQ solver that is
                       suitable for systems that are symmetric indefinite
                       systems.} 
%
\optionbox{options[{\sf AZ\_scaling}]}{Specifies scaling algorithm.
  The entire matrix is scaled (overwriting the old
  matrix). Additionally, the right hand side, the initial guess and
  the final computed solution are scaled if necessary. For 
  symmetric scaling, this transforms $ A x = b$ into
  $ S A S y = S b $ as opposed to $ S A x = S b $ when symmetric
  scaling is not used. NOTE: The residual within \Az{} is now 
  given by $ S (b - A x) $. Thus, residual printing and convergence
  checking are effected by scaling. {\bf However}, see {\sf AZ\_ignore\_scaling}
  for ways around this.  DEFAULT: \sf AZ\_none.}
%
\optionbox{options[{\sf AZ\_precond}]}{Specifies preconditioner.
  DEFAULT: \sf AZ\_none. \\[5pt] 1) Multigrid preconditioning can be done
  via the ML package. At present, there is no complete
  manual for ML. Please ask \Az{} authors for
  more information.\\[2pt] 2) Any Aztec solver can be used as a 
  preconditioner. Pick a solver to be used
  as a preconditioner via a second {\it options} array (e.g.
     {\it options2}) and perform:\\[5pt]
handle = AZ\_set\_solver\_parameters(params, \\
\phantom{handle = AZ\_set\_solver} 
options2,Amat,precond,scaling);\\
     options[AZ\_precond] = handle\\
}

\optionbox{options[{\sf AZ\_subdomain\_solve}]}{Specifies the solver
  to use on each subdomain when {\it options}[{\sf AZ\_precond}] is set
  to {\sf AZ\_dom\_decomp} DEFAULT: \sf AZ\_ilut.}
\choicebox{AZ\_bilu\_ifp}{Mike, I think it is okay to reference another
                          document. Could you also say something about 
                          the complex solves? Once again, a pointer
                          to another document is okay.}
\choicebox{handle}{Any Aztec solver can be used as a subdomain
                   solver. Pick a solver to be used on a subdomain via a 
                   second {\it options} array (e.g.  {\it options2}) and 
                   perform:\\[5pt]
handle = AZ\_set\_solver\_parameters(params, \\
\phantom{abfghijklmno} options2, Amat, precond, scaling); \\
     options[AZ\_subdomain\_solve] = handle
}

%
\optionbox{options[{\sf AZ\_conv}]}{Determines the residual expression used
  in convergence checks and printing.  DEFAULT: {\sf AZ\_r0}.
  The iterative solver terminates if the corresponding residual expression
  is less than {\it params}[{\sf AZ\_tol}]:}
\choicebox{AZ\_expected\_values}{$\|r\|_{Wmax} $
  where $\| \cdot\|_{Wmax} = \max_{i=1}^{n} (r_i/w_i)$, 
%= \max_{i=1}^n (r_i/w_i)$,
  $n$ is the total number of unknowns, $w$ is a weight
  vector with $w_i = \sum_{i=1}^n | a_{i,j} x_j |$ using
  for $x$ the initial guess supplied by the user.}
%
%
\optionbox{options[{\sf AZ\_keep\_kvecs}]}{When using the solver {\sf AZ\_cg}, this integer determines how many Krylov vectors are stored. Default: 0.}
%
%
\optionbox{options[{\sf AZ\_orth\_kvecs}]}{When set to {\sf AZ\_TRUE}, 
using the solver {\sf AZ\_cg}, 
and with {\it options[{\sf AZ\_keep\_kvecs}]} $> 0 $, conjugate gradient
will explicitly A-orthogonalize new Krylov vectors against kept Krylov
vectors. Default: {\sf AZ\_false}.}
%
%
\optionbox{options[{\sf AZ\_apply\_kvecs}]}{When set to {\sf AZ\_TRUE}
and using the solver {\sf AZ\_cg}, the initial guess will be improved 
using previously kept Krylov vectors (obtained by setting
{\it options[{\sf AZ\_keep\_info}]} to `1' and {\it 
options[{\sf AZ\_keep\_kvecs}] } $> 0 $ on a previous solve) before running 
the conjugate gradient algorithm. Default: {\sf AZ\_false}.}
%
%
\optionbox{options[{\sf AZ\_ignore\_scaling}]}
{When set to {\sf AZ\_TRUE} and when scaling is also requested, 
\Az{} performs scaling as usual, however, convergence and the 
printed residual correspond to the unscaled residual.
{\bf IMPORTANT NOTE:} When used in conjunction with multiple solves
of the same matrix, an identical scaling object must be passed
into {\sf AZ\_iterate()} each time: \\[2pt]
\phantom{hi} struct AZ\_SCALING $*$scaling;  \\
\phantom{hi} scaling = AZ\_scaling\_create(); \\
\phantom{hi} AZ\_iterate( ..., scaling);    \\
\phantom{hi} AZ\_iterate( ..., scaling);    \\
\phantom{hi} AZ\_scaling\_destroy(\&scaling);  
}
%
%
\optionbox{options[{\sf AZ\_check\_update\_size}]}{
When set to {\sf AZ\_TRUE}, convergence will be declared only if both
the convergence test set via {\it options[{\sf AZ\_conv}]} is satisfied
and the 
change in the solution at the current iteration, $k$ is small.
Specifically, \\
\phantom{hi}
  $ || \delta x ||_2 < $
{\it params [{\sf AZ\_update\_reduction}]} $ || x^{(k)} ||_2 $ \\
where \\
\phantom{hi}
$x^{(k)} = x^{(k-1)} + \delta x $. 
}
%
%
%
\subsection{\Az{} parameters\label{optionD}}

The double precision array {\it params\/} set by the user and normally of
length {\sf AZ\_PARAMS\_SIZE}. However, when a weight vector is needed for the
convergence check (i.e. {\it options}[{\sf AZ\_conv}] = {\sf AZ\_weighted}), it
is embedded in {\it params\/} whose length must now be {\sf AZ\_PARAMS\_SIZE} +
\# of elements updated on this processor.  In either case, the contents of {\it
  params\/} are used (but not altered) by the function {\sf AZ\_solve} to
control the behavior of the iterative methods.  The array elements are
specified as follows: \vspace{2em}
{\flushleft{\bf Specifications} \hrulefill} \nopagebreak \\[0.5em]
%
\optionbox{params[{\sf AZ\_update\_reduction}]}{See 
{\it options[{\sf AZ\_check\_update\_size}]}. }

%%% Local Variables:
%%% mode: latex
%%% TeX-master: "az_ug_20"
%%% End:




%%%%%%%%%%%%%%%%%%%%%%%%%%%%%%%%%%%%%%%%%%%%%%%%%%%%%%%%%%%%%%%%%%%%%%%%%%%%%%%

%\vspace{2em}
%
%\section{Compiling and Linking\label{comp_link}}
%
%The krylov solver library \Az{} uses code from a few publicly
%available linear algebra packages.  These packages are 1) the basic
%linear algebra subroutine library - BLAS  2) the linear algebra
%package - LAPACK and 3) a sparse direct lu solver Y12M. These packages
%can be obtained form netlib as described below.
%
%The following packages are needed by \Az{}:
%\begin{itemize}
%\item Y12M (used for LU factorizations)
%\item LAPACK routines
%\item BLAS   routines
%\end{itemize}
%These routines can be obtained via netlib.
%Send email to netlib@ornl.gov with the following information
%in the message:
%\vskip .5in
%\begin{verbatim}
%send help
%send index for y12m
%send index for lapack
%send index for blas
%\end{verbatim}
%\vskip .5in
%
%Current information on compiling and linking \Az{}
%for supported machines can be found in the file \verb'README'
%in the source code distribution directory.
%
%To obtain the \Az{} distribution send an email request to
%Ray S. Tuminaro (tuminaro@cs.sandia.gov).

\newpage
\bibliographystyle{plain}
\bibliography{aztec_guide}


\end{document}
